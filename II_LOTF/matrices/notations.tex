\section{Preliminaries}
\label{not}

We introduce a few notations, definitions and background notions that are used in this chapter. We illustrate some of these with classical examples, such as \emph{Reed-Solomon} codes. However,
our goal is not to provide a detailed exposition on coding theory, and we refer the reader to any good textbook such as \cite{vanlint} for a thorough treatment.

\subsection{Notations}
We write `$\defas$' equality by definition.
We note $\Fq$ an arbitrary finite field and \Ftwom the finite field with $2^m$ elements. We often consider $\Fst$, and implicitly use this specific field if not mentioned otherwise.
W.l.o.g. we use the representation
$\Fst \cong \mathbf{F}_2[\alpha]/(\alpha^4 + \alpha + 1)$ when a specific one is needed. We freely use ``integer representation'' for elements of $\Ftwom \cong \Ftwom[\alpha]/I(\alpha)$
by writing $n \in \{0\ldots2^n-1\} = \sum_{i = 0}^{n-1} a_i 2^i$ to represent the element
$x \in \Ftwom = \sum_{i = 0}^{n-1} a_i \alpha^i$. In the remainder of this section, and especially in \autoref{sec:ag}, we usually consider an arbitrary algebraically closed field, written $k$
(it will usually be clear from the context whether $k$ is a field or the dimension of a linear code, see \autoref{def:lincode} below).

Bold variables denote vectors (in the sense of elements of a vector space), and subscripts are used to denote their $i^\text{th}$ coordinate, starting from zero. For instance,
let $\veci \defas (1, 2, 7)$, then  $\veci_2 = 7$.
If $\mat$ is a matrix of $n$ columns, we call $\mat_i \defas (\mat_{i,j},~j = 0\ldots n - 1)$ the row vector formed from the coefficients of
its $i^\text{th}$ row.
We use angle brackets ``$\langle$'' and ``$\rangle$'' to write ordered sets.

Arrays, or tables, (in the sense of software data structures) are denoted by regular variables such as $x$ or $T$, and their elements are accessed by using square brackets.
For instance, $T[i]$ is the $i^\text{th}$ element of the table $T$, starting from zero.

% We may use the term ``vector'' as well to denote such variables,
% but the notation should make it clear if we mean a logical array or an element of a vector space; sometimes both notions might coincide.

%Other notations or specific variables are introduced when they are needed, and keep their meaning through the remainder of the text if not otherwise stated.
%Some background notions might not be redefined if they are not deemed to be essential for the understanding of the paper.

\subsection{Linear codes}

\begin{defi}[Hamming weight, Hamming distance]
Let $\veci$ be a vector of $k^n$. Its \emph{Hamming weight} $\weight(\veci) \in [0,\ldots,n]$ is $\#\{\veci_i, i = 0,\ldots n-1| \veci_i \neq 0\}$,
the number of its coordinates which are non-zero.
The \emph{Hamming distance} $\hdist(\veci,\veco)$ between two vectors is defined as $\weight(\veci - \veco)$.
\end{defi}


\begin{defi}[Linear code]
\label{def:lincode}
A \emph{linear code} of \emph{length} $n$, \emph{dimension} $k$ and \emph{minimal distance} $d$ over the alphabet $\Fq$ is a $k$-dimensional linear subspace of $\Fq^n$ such that $\veci, \veco \in \code \Rightarrow \hdist(\veci, \veco) \geq d$
and $\exists \veci, \veco \in \code,  \hdist(\veci, \veco) = d$.
The last conditions can equivalently be expressed as $\veci \neq \nullvec \in \code \Rightarrow \weight(\veci) \geq d$ and $\exists \veci \in \code, \weight(\veci) = d$.
We use the usual ``NKD'' notation to characterize codes: an $[n, k, d]_{\Fq}$ code $\code$ is a code of ength $n$, dimension $k$ and minimum distance $d$
with symbols in \Fq.
\end{defi}

We only consider linear codes in this manuscript and we will therefore omit this qualifier in the remainder of the text. 
We usually view a code as a set, and call \emph{codewords} its elements. By an abuse of terminology, we may call \emph{messages} the elements of $\Fq^k$.

\begin{defi}[Dual code]
Let $\code$ be an $[n, k, d]_{\Fq}$ code over $\Fq$ equipped with a scalar product $\langle\cdot,\cdot\rangle$. The \emph{dual}
$\code^\bot$ of $\code$ is defined as $\{\veci \in \Fq | \forall \veco \in \code, \langle\veci,\veco\rangle = \nullvec\}$.
\end{defi}

A code $\code$ equal to its dual $\code^\bot$ is called \emph{self-dual}.


\begin{defi}[Generator matrix, systematic form]
A \emph{generator matrix} $G$ of an $[n, k, d]_{\Fq}$ code $\code$ is a matrix of $\matspace_{k,n}(\Fq)$ such that
$\code = \{\veci G, \veci \in \Fq^k\}$. It is in \emph{systematic form} if it is of the form $(I_k~A)$ with
$I_k$ the identity matrix of dimension $k$. 
\end{defi}

If $(I_k~A)$ is a generator matrix for a code $\code$, then $(-A^t~I_{n-k})$ (or equivalently $(I_{n-k}~-A^t)$)
is a generator matrix for its dual $\code^\bot$. It follows that if $\code$ is self-dual, $A$ is symmetric; it is
also orthogonal.
% TODO why not just A.A^t diagonal?

\begin{defi}[MDS code, MDS matrix]
An $[n,k,d]_{\Fq}$ code $\code$ is called \emph{maximum distance separable}, or simply \emph{MDS}, if $d = n - k + 1$.
By abuse of definition, if $n = 2k$ and $(I_k~A)$ is a generator matrix of $\code$, we call $A$ an \emph{MDS matrix}.
\end{defi}

It is easy to see that this is the highest possible minimum distance that can be achieved by a code with a
systematic generator matrix, as all rows of such a matrix have weight at most $n - k + 1$.

A useful alternative characterization of MDS matrices is given by the following theorem.

\begin{thm}[\cite{mdsConj}]
\label{thm:mds_minors}
A matrix $M$ is MDS if and only if all of its minors are non-zero (\ie all the square sub-matrices of $M$ are invertible).
\end{thm}

A consequence of this is that the dual of an MDS code is also MDS.
We also have the following:

\begin{conj}[MDS conjecture]
There is no MDS code with symbols in $\Fq$ of length $n > \#\Fq$.
\end{conj}

The next definition rephrases and generalizes some of the above concepts in a way that is more suitable to cryptographic applications.
We assume in this case that $\Fq = \Ftwom$ for some $m$.

\begin{defi}[Branch number~\cite{aes}]
Let $A$ be the matrix of a linear mapping over $\Fu$.
The \emph{differential branch number} of $A$
is equal to $\min_{\veci \neq 0}(\weight(\veci) + \weight(\veci A))$,
and the \emph{linear branch number} of $A$ is equal to $\min_{\veci \neq 0}(\weight(\veci) + \weight(\veci A^t))$.
\end{defi}

If $A$ is such that $(I_k~A)$ is a generator matrix of a code of minimum distance $d$ which dual has minimum distance $d'$,
then $A$ has a differential (resp. linear) branch number of $d$ (resp. $d'$).

\subsubsection{Evaluation codes}

The algebraic-geometry codes that we use in this chapter are conveniently defined as instantiations of the general framework of \emph{evaluation codes}, for which
we give a brief overview.
We start with a general definition:

\begin{defi}[Evaluation code]
Let $\funcspace$ be an $\Fq$-vector space of dimension $k$ of functions $\fun : \dom \rightarrow \Fq$ (the \emph{function space}). Let $\points \defas \langle P_0,\ldots, P_{n-1}\rangle$
be an ordered subset of $\dom$ of cardinality $n$ (the \emph{evaluation domain}). Let $\evmap_\points : \funcspace \rightarrow \Fq^n$ be the \emph{evaluation map}
defined as $\evmap_\points(\fun) \defas (\fun(P_0),\ldots,\fun(P_{n-1}))$.
The \emph{evaluation code} $\evcode(\funcspace,\points)$ for an injective evaluation map $\evmap_\points$ is defined as $\{\evmap_\points(\fun)~|~\fun \in \funcspace\}$.
It is an $[n,k,\ast]_{\Fq}$ code.
\end{defi}

A generator matrix $G$ of an evaluation code can easily be constructed by evaluating a basis for $\funcspace$ on $\points$. Call $(\fun_0,\ldots,\fun_{k-1})$
such a basis, $G = (\evmap_\points(\fun_0),\ldots,\evmap_\points(\fun_{k-1}))$.
If the code admits a systematic generator matrix, it can simply be obtained by computing the reduced row echelon form of $G$.

We use ordered sets for the evaluation domain $\points$ in this definition. Although the order that is chosen does not impact the parameters of the code (hence we
may sometimes abuse our definition and talk about \emph{the} code for any of the codes based on the same $\funcspace$ and $\dom$), it may have
an influence on the performance of explicit instantiations, through \eg properties of the generator matrices. Most of our work is actually based on this fact,
as it will become clear later in this chapter. 

The probably best-known evaluation codes are the \emph{Reed-Solomon codes} (``RS codes''):

\begin{defi}[Reed-Solomon codes]
An $[n,k,\ast]_{\Fq}$ \emph{Reed-Solomon} code is an evaluation code obtained by taking $\funcspace$ to be the polynomials $\Fq[x]$
of degree less than $k-1$ and $\points$ any ordered subset of $\Fq$. 
\end{defi}

As we must have $n \geq k$ by definition of a code, all polynomials of $\funcspace$
can be uniquely interpolated on any $n$ points, hence $\evmap_\points$ is injective and Reed-Solomon codes are well-defined.
Furthermore, for any $\points$, any non-zero polynomial of $\funcspace$ is zero on at most $k - 1$ positions. The minimal distance of a Reed-Solomon code is thus $n - (k - 1)$, which makes them MDS codes.


% => section AG codes

% Affine and projective spaces
% Algebraic curves
% Divisors
% Riemann-Roch
% AG codes
