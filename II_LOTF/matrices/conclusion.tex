\section{Conclusion}

We revisited the \shark{} structure by replacing the MDS matrix of its linear diffusion layer by matrices built from an algebraic geometry code.
Although this code is not MDS, it still has a very high minimum distance, all the while being quite long.
This allowed us to define an efficient full-state diffusion layer for a 64-bit block cipher, operating on 4-bit values.

We studied algorithms suitable for a vector implementation of the multiplication by this matrix, and how to find matrices that are most efficiently implemented with
those algorithms.
Finally, we gave performance figures for assembly implementations of hypothetical \shark-like ciphers using this matrix as a linear layer.

This work provided the first generalisation of \shark{} that are not vulnerable to timing attacks as is the
original cipher, and also the first generalisation to 128-bit blocks. It also showed that even if not the fastest, such potential design could be implemented efficiently in software.

As a future work, it would be interesting to investigate how to use the full automorphism group of the code to design matrices with a lower cost. In that case, we would not restrict
ourselves to derive the rows from a single row and the powers of a \emph{single} automorphism, but could use several independent automorphisms instead. 
