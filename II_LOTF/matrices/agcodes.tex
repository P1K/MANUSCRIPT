\section{Algebraic geometry codes}
\label{sec:ag}

We now present \emph{algebraic geometry codes} (``AG codes''), which are the main object used in this chapter,
and we show how they can give rise to diffusion matrices with interesting parameters.
We introduce a few geometry notions along the way in order to be able to faithfully describe the codes we will
be using. 
We refer the reader to the relevant chapters of \cite{vanlint,tvn,stichtenoth,fulton} for a much more detailed and rigorous treatment of the subject. 

\paragraph{AG codes as a generalisation of RS codes.}
We first give an intuition of the construction used in AG codes. We have seen that RS codes are MDS
and that they can be instantiated for many lengths and dimensions. One drawback of the construction however
is that the maximal length of the code is limited by the cardinality of the alphabet $\Fq$: we
cannot evaluate the functions on more points than there are elements in $\Fq$ without introducing some
repetition; we can still slightly increase this maximal length by one, and very rarely two, by defining an \emph{extended} RS code, which is still MDS.
However, by the MDS
conjecture, we do not expect to be able to do better than that.
Under this light, it may seem natural to desire some sort of
tradeoff between the maximal length of a code construction and its \emph{designed minimal distance} $d^* \leq d$.

A natural way to increase the maximal length of an evaluation code is to consider functions over domains
of higher dimension than the ``line'' used in RS codes. For instance, one could take $\dom$ to be the
$m$-dimensional space $\Fq^m$, and $\funcspace$ to be the $m$-variate polynomials of $\Fq[x_0,\ldots,x_{n-1}]$
less than a certain degree $r$. This defines \emph{Reed-Muller codes} (``RM codes'') of order $r$.
Although this construction successfully yields codes longer than RS codes, they are not asymptotically good, in
the sense that construction does not allow to have code parameters where neither of $d/n$ and $k/n$ tends to
zero when $n$ goes to infinity.

An alternative approach to construct longer variants of RS codes is the one followed by AG codes. Seeing
RS codes as working with a line of genus zero, the idea is to take as evaluation domains
the points of plane curves of higher genera,
which can be assimilated to a subset of $\Fq^2$. A geometrically
meaningful way to choose the function spaces $\funcspace$ is then to consider certain subspaces of
the function fields of the curves. We will see that for well-chosen $\funcspace$, this construction
leads to codes of parameters $[n, k, d \geq n - k + 1 - g]_{\Fq}$ for curves of genus $g$. As the
maximal number of points on a curve is roughly increasing with the genus, this construction gives us the
sort of tradeoff we were searching for, and also yields asymptotically good codes.

\subsection{Selected algebraic geometry notions}

We will be needing a few tools from algebraic geometry to make our description of AG codes explicit. We only introduce
here the notions that are relevant to this specific application,
largely ignoring the wider picture. Note that unlike the remainder of this chapter, the underlying fields used in the definitions
are not necessarily finite.

We will describe AG codes as evaluation codes, which means that we need to be able first to describe the evaluation domains,
and second to
define the function spaces $\funcspace$ associated with given evaluation domains. Both of these need to be defined explicitly,
if we want the codes to be useful in practice; in particular, this means that we need to be able
to compute the bases of the function spaces used in the constructions.

\medskip

The construction of the examples of AG codes that we later consider can eventually be described in a purely affine way. However, in order to
justify them, we need to be able to reason about points at infinity. Thus we start with the following:

\begin{defi}[Affine space, projective space]
\label{def:affiprojec}
The $n$-dimensional \emph{affine space} $\affi^n(k)$ over a field $k$ is a set of \emph{points} formed by $n$-tuples over $k$,
$P \defas (x_0, \ldots, x_{n-1})$, $x_i \in k$.
The $n$-dimensional \emph{projective space} $\projec^n(k)$ over $k$ is a set of points formed by equivalence classes of $(n+1)$-tuples
$P \defas (x_0 : \ldots : x_n)$, $x_i \in k$, where not all $x_i$s are zero, and the equivalence relation $\sim$ is
defined by $(x_0 : \ldots : x_n) \sim (\lambda x_0 : \ldots : \lambda x_n)$, $\lambda \in k^*$. In the usual case where $n = 2$, we let
$x \defas x_0$, $y \defas x_1$ and $z \defas x_2$.

\noindent
The space $\affi^n(k)$ can be embedded into $\projec^n(k)$ in a natural way as $(x_0, \ldots, x_{n-1}) \mapsto (x_0 : \ldots : x_{n-1} : 1)$.
The set of projective points $(x_0 : \ldots : x_{n-1} : 0)$ is called the \emph{hyperplane at infinity}.
\end{defi}

We recall the following definitions about ideals.

\begin{defi}[Ideal, prime ideal, maximal ideal, principal ideal]
An \emph{ideal} $\mathcal{I}$ of a ring $\mathcal{R}$ is a subset of $\mathcal{R}$ closed by addition and absorbing by
multiplication: $a, b \in \mathcal{I} \Rightarrow a + b \in \mathcal{I}$; $a \in \mathcal{I}, b \in \mathcal{R} \Rightarrow
a b \in \mathcal{I}$.

\noindent
An ideal $\mathcal{I}$ is \emph{prime} if for any $a$, $b$ in $\mathcal{R}$ such that $a b \in \mathcal{I}$, then
$a \in \mathcal{I}$ or $b \in \mathcal{I}$.

\noindent
An ideal $\mathcal{I} \subsetneq \mathcal{R}$ is \emph{maximal} if there is no ideal $\mathcal{J}$ such that
$\mathcal{I} \subsetneq \mathcal{J} \subsetneq \mathcal{R}$.

\noindent
An ideal $\mathcal{I}$ is \emph{principal} if it is generated by a single element: $\exists\,a \in \mathcal{I} \text{ s.t. }
\mathcal{I} = a\mathcal{R} = \{a r,~r \in \mathcal{R}\}$.
\end{defi}


We may now define projective varieties.

\begin{defi}[Projective variety]
A \emph{projective variety} in the projective space $\projec^n(k)$ over an algebraically closed field $k$ is the set of common zeros
of a prime ideal $\mathcal{I} \defas \langle f_0, \ldots, f_m \rangle$ of homogeneous polynomials of $k[x_0,\ldots,x_n]$, \ie polynomials whose
monomials all have the same degree.
\end{defi}

A \emph{projective curve} is a projective variety of dimension one, where the dimension $n$ of a variety $V$ is defined by
the maximal length of the inclusion chain $V = V_0 \supsetneq \ldots \supsetneq V_n \neq \emptyset$, with $V_i$ closed in $V_{i-1}$, for the \emph{Zariski topology}
for all $i \in \{1,\ldots,n\}$. In practice, we will only consider curves in $\projec^2$, \ie plane curves,
which will then be defined by a single homogeneous polynomial $E$.

\paragraph{Remarks.}
\begin{enumerate}
\item The condition that $k$ be algebraically closed is not as restrictive as it may seem. Say that we wish to work in $\Fq$, with a
curve $\curve$ defined by an equation $E$ over $\Fq$ or one of its subfields. Then, 
although we define $\curve$ over the algebraic closure of $\Fq$, we
will only be interested in $\curve(\Fq)$, the points of $\curve$ that lie in $\Fq$: its \emph{$\Fq$-rational points}, or simply rational points.
For instance, we may wish to only consider curves which have a large number of rational points.
\item For a given projective variety, say $\curve$ defined by $E(x,y,z)$, one can consider the corresponding affine variety defined as the zeros of ``$E(x,y,1)$,
$E \in k[x,y]$'', the ``dehomogenised'' of $E$. It will have the same points as $\curve$, in the sense of the embedding of \autoref{def:affiprojec}, minus its potential points at infinity.
Similarly, an affine variety can be extended to a projective closure.
\end{enumerate}

Our first step towards defining suitable function spaces $\funcspace$ is to define the function field of a variety.
We give two definitions, one in the affine and one in the projective case.

\begin{defi}[Coordinate ring, function field of an affine variety]
Let $\curve$ be an affine variety in $\affi^n(k)$ defined by the ideal $\mathcal{I}$. The \emph{coordinate ring} $k[\curve]$ of $\curve$ is $k[x_0, \ldots, x_{n-1}]/\mathcal{I}$.
The \emph{function field} $k(\curve)$ of $\curve$ is the quotient field of $k[\curve]$.
\end{defi}

\begin{defi}[Function field of a projective variety]
Let $\curve$ be an affine variety in $\projec^n(k)$ defined by the ideal $\mathcal{I}$. Let $F$, $G$ be homogeneous polynomials of equal degrees in $k[x_0, \ldots, x_n]$
with $G \notin \mathcal{I}$. The \emph{function field} $k(\curve)$ of $\curve$ is the set of \emph{rational functions} $F/G$ modulo $\mathcal{I'}$,
where $\mathcal{I'} = \{F/G~|~ F \in \mathcal{I}\}$.

\noindent
A rational function $F/G$ is \emph{regular} at $P$, point of $\curve$, if $G(P) \neq 0$.
\end{defi}

We will need to be able to talk about the behaviour of functions at specific points of a curve. For this, we first use the following:

\begin{defi}[Local ring of a point]
The \emph{local ring} $\locring_P$ of a point $P$ of a variety $\curve$ is the set of rational functions regular at $P$.
It has a unique maximal ideal $\maxi_P \defas \{f \in \locring_P ~|~ f(P) = 0\}$. 
\end{defi}

The local ring and its maximal ideal are in particular useful to define the \emph{tangent space} at a point $P$, which in turn allows to define smooth and singular points,
depending on its dimension.
We will not give the general definition here, mentioning instead  the following convenient alternative characterization in the specific case of curves:

\begin{defi}[Smooth and singular points of a curve, local parameter]
A point $P$ of a curve $\curve$ is \emph{smooth} or \emph{non-singular} iff $\maxi_P$ is a principal ideal. Any generator of the ideal $t_P \in \locring_P$ s.t. $\maxi_P = t_P\locring_P$
is called a \emph{local parameter} at $P$.

\noindent
A point that is not smooth is called \emph{singular}.
\end{defi}

To build an AG code from a curve $\curve$, we will require that it is smooth, \ie that none of its points are singular.
% (actually, we will
%only require this for $\curve(\Fq)$ for some $\Fq$).
We only consider smooth curves until \autoref{sec:nonsingmod}.

The local parameters are useful in defining the order of zeros and poles of arbitrary rational functions at a given point.
Let $f \neq 0$ be a rational function regular at $P$ and $t_P$ be a local parameter at $P$. The \emph{order} of $f$ at $P$ is:
\[
\ord_P(f) \defas \max \{k~ |~ f \in t_P^k\locring_P,~ f \notin t_P^{k+1}\locring_P\}.
\]
This can be extended to arbitrary rational functions by writing them as quotients of functions of $\locring_P$:
\[
\ord_P(f/g) = \ord_P(f) - \ord_P(g),~f,g\in\locring_P.
\]
For a function $f$, if $\ord_P(f) \geq 0$, we call this value the \emph{zero order} of $f$ at $P$; if $\ord_P(f) < 0$
we call $-\ord_P(f)$ the \emph{pole order} of $f$ at $P$.

We note in particular the two useful properties of $\ord$:
1) $\ord(fg) = \ord(f) + \ord(g)$; 2) $\ord(1/f) = -\ord(f)$.

%We will shortly see that being able to compute the order of a function at some desired points is essential in the definition of $\funcspace$.
In effect, $\ord_P$ defines a \emph{discrete valuation} of functions of $k(\curve)$, associated with the point $P$, \ie $\locring_P$ is a \emph{valuation ring}:

\begin{defi}[Valuation ring]
A ring $\locring$ is a \emph{valuation ring} of $k(\curve)$ if $k \subsetneq \locring \subsetneq k(\curve)$  and for every $f \in k(\curve)$,
$f \in \locring$ or $f^{-1} \in \locring$.
\end{defi}

We will see shortly that the ability to define valuations at any point is essential
in the definition of our function space $\funcspace$; this is why we require the curve to be non-singular, as a local ring is a valuation ring iff $P$ is smooth.

\medskip

The last main objects that we introduce are the divisors on a curve.

\begin{defi}[Divisor]
A \emph{divisor} on a smooth curve $\curve$ is a formal finite sum of points $P_i$ of $\curve$: $D \defas \sum a_iP_i$, $a_i \in \mathbf{Z},~P_i \in \curve$.
The \emph{support} of a divisor is the set of points $P_i$ with $a_i \neq 0$.

\noindent
The set $\cdiv(\curve)$ of divisors on $\curve$ with formal addition and negation of divisors forms a group, \ie $\sum a_iP_i + \sum b_iP_i \defas \sum (a_i+b_i)P_i$;
$-\sum a_iP_i \defas \sum -a_iP_i$.

\noindent
A divisor $D \defas \sum a_iP_i$ is called \emph{effective} ($D \geq 0$) if $\forall i,~a_i \geq 0$.

\noindent
The \emph{degree} $\deg(D)$ of $D = \sum a_iP_i$ is defined as $\sum a_i$.
\end{defi}

\begin{defi}[Divisor of a rational function (principal divisor)]
The divisor $(f)$ of a rational function $f \in k(\curve)$, $f \neq 0$, is defined as $\sum \ord_P(f)P$.
A divisor $D$ equal to $(f)$ for some function $f$ is called a \emph{principal divisor}.
\end{defi}

Principal divisors form a subgroup of $\cdiv(\curve)$. It can be shown that they are always of degree zero, which
is equivalent to saying that a rational function always has the same number of poles and zeros, counted with
multiplicity.

\begin{defi}[Space associated with a divisor]
Let $D \in \cdiv(\curve)$. We define $\rrs(D)$, the \emph{space associated with $D$} (or \emph{Riemann-Roch space}) as:
\[
\rrs(D) = \{f \in k(\curve)^*~|~(f) + D \geq 0\} \cup \{0\}.
\]
This forms a $k$-vector space; we write $\ell(D)$ its dimension.
\end{defi}

This is finally the kind of function space we will be using to define AG codes. For a divisor
$D \defas \sum a_iP_i - \sum b_jP_j$, $a_i, b_j \geq 0$, a simple reformulation of $\rrs(D)$ is to say that it consists of the zero function and of all the functions
which may have poles only in the $P_i$s of order at most $a_i$, and which must have zeros in all the $P_j$s of order
at least $b_j$.

It can be shown that for any $D$, $\rrs(D)$ is finite-dimensional. In fact, this is a consequence of two major theorems
the first from Riemann and the second being an extension due to Roch, which allow in particular to compute the explicit value $\ell(D)$
for a given divisor $D$~; this is expressed by the following corollary:

\begin{cor}[Corollary of Riemann and Roch]
Let $D$ be a divisor on a curve of \emph{genus} $g$, then $\ell(D) \geq \deg(D) + 1 - g$, with equality
when $\deg(D) > 2g - 2$.
\end{cor}

The formulation of this theorem uses the notion of \emph{genus} of a curve, that we have not defined yet. A possible, purely algebraic definition is precisely to take
$g(\curve) \defas \max \{\deg(D) - \ell(D) + 1,~D \in \cdiv(\curve)\}$. Although this is not very satisfying, we will not provide a better definition here, and computing
the genus in this way is not as hard as it may seem; in practice, the genus
of the curves we use will be known beforehand.

\subsubsection{Working with singular curves}
\label{sec:nonsingmod}

In the last few paragraphs, we have assumed that we were working with a smooth curve. We justified this by the need of being able to define a valuation for functions
at any point of the curve, which allowed us to define divisors and their associated vector spaces.
Unfortunately, some of the curves we may wish to use in practice are not smooth. We briefly and informally sketch here how it is possible to nonetheless work with some
of these curves. We refer the reader to \eg Fulton~\cite[Chap. 7]{fulton} for a serious treatment of the matter.

A first remark is that what effectively matters in the definition of the function space is not so much the points on the curves as their associated valuation
rings. It is in fact possible to define entirely what we have presented above in a purely algebraic way, as is done for instance by Stichtenoth~\cite{stichtenoth}.
In this case, the notion of points is replaced by the one of \emph{places} of a function field, written $F/k$, not necessarily associated with a curve, where a place is simply the necessarily unique maximal
ideal of a valuation ring of the function field:

\begin{defi}[Place of a function field]
A place $P$ of a function field $F/k$ is the maximal ideal of a valuation ring $\locring$ of $F/k$. 
\end{defi}

Divisors can then be redefined as sums of places instead of sums of points.

For a smooth curve $\curve$, there is a bijection between its points and the valuation rings (the places) of $k(\curve)$, and the theory as presented above is sufficient.
On the other hand, if $\curve$ is singular, we would like to be able to find a way of additionally defining places at its singularities.
This can be done by constructing a \emph{non-singular model} $\nscurve$ of $\curve$,
and taking as places of $\curve$ the points of $\nscurve$. A useful theorem is that for a projective curve $\curve$, there is a unique (up to isomorphism) non-singular curve
$\nscurve$ that is birationally equivalent to it, which means that there are \emph{rational maps} from $\curve$ to $\nscurve$ and vice-versa, that compose to the identity wherever they are
defined,
a rational map $\curve \subseteq \projec^m \rightarrow \projec^n$ being an $n+1$-tuple of homogeneous polynomials s.t. the polynomials do not jointly
vanish at any point of the curve. Furthermore, the function fields $k(\curve)$ and $k(\nscurve)$ are the same, up to isomorphism, so defining the places of $\curve$ in this
way is meaningful.
We will not address here the problem of computing these non-singular models \emph{per se}.

The map $\curve \rightarrow \nscurve$ may not be defined on every point of $\curve$, but the number of points where it is not defined is finite; this makes sense, as a curve
has at most finitely many singular points. We say that a place is \emph{centered} at a point $P$ if $P$ is its image by the map $\nscurve \rightarrow \curve$.
Every non-singular
point of $\curve$ has exactly one place centered at it, while finitely many places may be centered at singular points.

We conclude by defining curves that are said to be in \emph{special position}, cf. \eg \cite{DBLP:journals/tit/SaintsH95}, which are convenient to use for several reasons.

\begin{defi}[Curve in special position]
A projective plane curve $\curve$ is in \emph{special position}
if\footnote{We recall our notation: a point $P \in \projec^2(k)$ is written $(x : y : z)$, and is
at infinity iff $z = 0$.}:
\begin{enumerate}
\item it has exactly one point $P_\infty$ at infinity;
\item there is exactly one place centered at $P_\infty$;
\item the affine curve $\curve_A \defas \curve\backslash P_\infty$ is smooth;
\item the pole order of $x/z$ and $y/z$ at $P_\infty$ are not equal.
\end{enumerate}
\end{defi}

Curves in special positions are useful as there is a bijection between their places and their (potentially singular) points; thus they can always be considered
``as if smooth'' w.r.t. to the discussions of this section. Additionally, if one considers the space $\rrs(aP_\infty)$, $a > 0$, of functions having poles only at the point at infinity,
its elements can be mapped to polynomials of the coordinate ring of $\curve_A$. In other words, one can perform all the computations in $\rrs(aP_\infty)$ with the affine curve
only; we discuss how to explicitly compute bases for such spaces next. The main curve used in this chapter is a singular curve in special position.

\subsubsection{Computation of a basis of $\rrs(aP_\infty)$ for a curve in special position}

We conclude this brief presentation of algebraic curves and their function fields by showing how to compute a basis of $\rrs(aP_\infty$), $a > 0$
for a plane curve in special position.

Consider $\curve$ defined by a homogeneous equation $E(x,y,z)$. From a local parameterisation at the place $P_\infty$, we can compute the pole order of
$f \defas x/z$ and $g \defas y/z$ at $P_\infty$, which is enough to determine the pole
order of any fraction of $x$, $y$ and $z$.
An element of $\rrs(aP_\infty)$ is then a linear combination of monomials of the form $f^{\alpha}g^{\beta}$ such that
we always have
$\alpha\ord_{P_\infty}(f) + \beta\ord_{P_\infty}(g) \leq a$, \ie the pole order at $P_\infty$ of the monomials is less
than $a$.
A generating family for $\rrs(aP_\infty)$ is thus
$\{f^ig^j~|~i\times\ord_{P_\infty}(f) + j\times\ord_{P_\infty}(g) \leq a\}$; if we keep only a single monomial
with any given pole order, it forms a basis.
As $P_\infty$ is the only place at infinity, we can switch to an affine view for all computations by considering points in affine coordinates and dehomogenising $E$ and the basis functions; notably, the latter effectively become polynomials rather
than rational fractions. Indeed, a generating family
for $\rrs(aP_\infty) \subseteq k[\curve]$ is formed by $\{x^iy^j~|~i\times\ord_{P_\infty}(f) + j\times\ord_{P_\infty}(g) \leq a\}$.

\medskip

We conclude by presenting a simple technique to compute the pole order of $x/z$ and $y/z$ at the place at infinity for a curve in special position,
in a way that does not require to explicitly find a local parameter at $P_\infty$.
In this setting, we know by definition that the valuation $\ord_{P_\infty}$ is well-defined and that it is negative and distinct for $x/z$ and $y/z$.
Thus $\exists\, \alpha,\beta \in \mathbf{N}^*$ s.t. $(x/z)^\beta(y/z)^{-\alpha}$ is regular and non-zero at $P_\infty$. The smallest such $\alpha$
and $\beta$ are the pole order of $x/z$ and $y/z$ respectively.

We give two examples of a computation with this technique, corresponding to the first two curves presented in \autoref{sec:constr_ag}.

\begin{example}[$x^2z + xz^2 = y^3+yz^2$]
\label{ex:order_gen1}
We consider the curve of equation $E(x,y,z) := x^2z + xz^2 = y^3+yz^2$ over a field of characteristic two. It has a unique place at infinity $P_\infty \defas [1:0:0]$.
From the equation of the curve, we may guess 3 and 2 to be the orders of $x/z$ and $y/z$, as the fraction $x^2z^3/y^3z^2$ seems to be the one of least degree
that can be simplified using $E$.

We may then rewrite $x^2z^3$ in the four following ways:
\begin{enumerate}
\item $z^2(x^2z)$;
\item $z^2(y^3 + yz^2 + xz^2)$;
\item $xz(xz^2)$;
\item $xz(y^3 + yz^2 + x^2z)$.
\end{enumerate}
Similarly, $y^3z^2$ can be rewritten as:
\renewcommand{\theenumi}{\roman{enumi}}
\begin{enumerate}
\item $z^2(y^3)$;
\item $z^2(x^2z + xz^2 + yz^2)$;
\item $y^2(yz^2)$;
\item $y^2(y^3 + x^2z + xz^2$).
\end{enumerate}
We can then notice that ``1/ii'' = $z^2(x^2z)/z^2(x^2z + xz^2 + yz^2)$ simplifies to $x^2z/(x^2z + xz^2 + yz^2)$ = $x^2/(x^2 + xz + yz)$,
which evaluates to 1 at $[1:0:0]$. Hence we indeed have $\ord_{P_\infty}(x/z) = 3$ and $\ord_{P_\infty}(y/z) = 2$.
\end{example}

\begin{example}[$x^5 = y^2z^3 + yz^4$]
\label{ex:order_gen2}
We consider the curve of equation $E(x,y,z) := x^5 = y^2z^3 + yz^4$ over a field of characteristic two. It has a unique place at infinity $P_\infty \defas [0:1:0]$.
As in \autoref{ex:order_gen1}, we may guess that $x/z$ and $y/z$ have order 2 and 5 respectively. We thus consider the fraction $x^5z^2/y^2z^5$.
This can be rewritten as $(y^2z^5 + yz^6)/y^2z^5$ which readily simplifies to $(y^2 +yz)/y^2$ = $(y+z)/y$ which evaluates to 1 at $[0:1:0]$.
Hence we indeed have $\ord_{P_\infty}(x/z) = 2$ and $\ord_{P_\infty}(y/z) = 5$.
\end{example}

\subsection{Construction of AG codes}
\label{sec:constr_ag}

We are now ready to define the family of AG codes that we will use in the remainder of this work. We refer
to~\cite[Chapter 4]{tvn} for the description of other ways of defining codes from algebraic geometry.

Let $\curve$ be a smooth projective curve defined by an equation in $\Fq$ such that the set of its rational places $\curve(\Fq)$ is non-empty.
Let $D \in \cdiv(\curve)$ be a divisor on $\curve$ and $\points \defas (P_0, \ldots, P_{n-1}) \subsetneq \curve(\Fq)$ be $n$ places not in the support of $D$.
This defines a code $\evcode(\rrs(D), \points)$ through the evaluation map $\evmap_\points(\fun) \defas (\fun(P_0),\ldots,\fun(P_{n-1}))$, $\fun \in \rrs(D)$.

For the definition to be valid, the evaluation map from $\rrs(D)$ to $\points$ has to be injective, \ie its kernel must be equal to $\{0\}$.
This condition can easily be expressed in terms of divisors; if we call $S \defas P_0 + \ldots + P_{n-1}$ the divisor of the
evaluation support, we need $\rrs(D - S)$ (\ie functions of $\rrs(D)$ with zeros in all of the $P_i$s, \ie the kernel of $\evmap$) to be reduced to the zero function.
One can see that this will always be the case as long as $\deg(D) < \deg(S)$.

\medskip

Assuming that we defined a code using this construction, one of our main concerns would be to determine its parameters $k$ and $d$, or at least to provide an estimate
for them.
By definition of the minimal distance, $n - d$ is the maximal number of zeros that a function of $\rrs(D)$ can have
over $\points$; we already showed that the latter is upper-bounded by $\deg(D)$, so we have $d \geq n - \deg(D)$. This lower-bound
$n - \deg(D)$ for $d$ is the designed minimum distance $d^*$ for the code.

The dimension of the code is simply $\ell(D)$, which can be computed from the Riemann-Roch theorem. In particular, for $\deg(D) > 2g -2$, $g$ the genus of
$\curve$, we have $k = \ell(D) = \deg(D) + 1 - g$. We can then rephrase the lower-bound on $d$ in terms of $k$ as $d \geq n - k + 1 - g$, which is indeed
what we promised at the beginning of the section.

\medskip

Deriving a lower bound on the parameters of an algebraic geometry code as we just did is quite straightforward, but the overall theory is much richer.
A useful result in particular is that the dual of an AG code from the family we described
is also an AG code whose parameters are related through the same equation $d \geq n - k + 1 - g$~\cite[Chapter 4]{tvn}.
This is an interesting property in our context, as it means that diffusion matrices derived from AG codes have identical differential and linear branch
numbers.


\subsubsection{Examples}

We give three concrete examples of AG codes, the second of which being the code used to define the diffusion matrices presented in this chapter.
All of the examples are built from curves with a maximal number of $\Fst$-rational points given their respective genus.

\begin{example}[An AG code from a curve of genus 1]
\label{ex:genus1}
Let $\curve$ be the plane projective curve of genus 1 of equation $x^2z + xz^2 = y^3 + yz^2$ defined over $\Fst$. It is smooth, in special position,
with a unique point at infinity $P_\infty \defas [1:0:0]$, and it has 24 $\Fst$-rational affine points. The pole order
of $x/z$ and $y/z$ at $P_\infty$ is 3 and 2 respectively, as computed in \autoref{ex:order_gen1}.

Let $D \defas 12P_\infty$. From the Riemann-Roch theorem, the dimension $\ell(D)$ of $\rrs(D)$ is $12 + 1 - g = 12$. We can thus
define a $[24, 12]_{\Fst}$ AG code $\evcode(\rrs(D), \points)$ with $\points$ the affine rational points of $\curve$
put in any order.

This code and its dual have minimum distance at least 12.
Either code can thus be used to define a diffusion matrix of dimension 12 over $\Fst$, which diffuses over $12\times 4 = 48$ bits and
has differential and linear branch number at least 12 (in fact exactly 12). 
\end{example}

\begin{example}[An AG code from a curve of genus 2]
\label{ex:genus2}
Let $\curve$ be the plane projective curve of genus 2 of equation $x^5 = y^2z^3 + yz^4$ defined over $\Fst$. It is singular, in special position,
with a unique point at infinity $P_\infty \defas [0:1:0]$ at which a unique place is centered. It has 32 $\Fst$-rational affine points. The pole order
of $x/z$ and $y/z$ at $P_\infty$ is 2 and 5 respectively, as computed in \autoref{ex:order_gen2}.

Let $D \defas 17P_\infty$. From the Riemann-Roch theorem, the dimension $\ell(D)$ of $\rrs(D)$ is $17 + 1 - g = 16$. We can thus
define a $[32, 16]_{\Fst}$ AG code $\evcode(\rrs(D), \points)$ with $\points$ the affine rational points of $\curve$
put in any order.

This code is self-dual and has minimum distance at least 15.
It can thus be used to define an orthogonal diffusion matrix of dimension 16 over $\Fst$, which diffuses over $16\times 4 = 64$ bits and
has differential and linear branch number at least 15 (in fact exactly 15).
\end{example}

\begin{example}[An AG code from a curve of genus 6]
\label{ex:genus6}
Let $\curve$ be the plane projective curve of genus 6 of equation $x^5 = y^4z + yz^4$ defined over $\Fst$. It is smooth, in special position,
with a unique point at infinity $P_\infty \defas [0:1:0]$ at which a unique place is centered. It has 64 $\Fst$-rational affine points. The pole order
of $x/z$ and $y/z$ at $P_\infty$ is 4 and 5 respectively.

Let $D \defas 37P_\infty$. From the Riemann-Roch theorem, the dimension $\ell(D)$ of $\rrs(D)$ is $37 + 1 - g = 32$. We can thus
define a $[64, 32]_{\Fst}$ AG code $\evcode(\rrs(D), \points)$ with $\points$ the affine rational points of $\curve$
put in any order.

This code is self-dual and has minimum distance at least 27.
It can thus be used to define an orthogonal diffusion matrix of dimension 32 over $\Fst$, which diffuses over $32\times 4 = 128$ bits and
has differential and linear branch number at least 27 (in fact exactly 27).
\end{example}
