\section{Applications and performance}
\label{sec:appli}
% TODO
% - perfs of table implems
% - perfs of circulant MDS on F_256

This last section presents the performance of straightforward assembly implementations of both of our
algorithms when applied to a fast encoder of the code $\cagh$ from \autoref{matt}.
In all of the remainder, $\matdiff$ denotes the matrix from \autoref{fig:matrand}; it is
of dimension 16 over $\Fst$ and has a differential and linear branch number of 15.
We use this matrix in the design of the \samneric couple of toy block ciphers; \sam
is a 64-bit block cipher that uses $\matdiff$ \emph{as is} with a four-bit S-box,
and \eric uses $\matdiff$ over $\Fth$ together with an eight-bit S-box to define
a 128-bit block cipher.

We do not actually specify complete block ciphers, in particular we do not
give key schedules; our objective is rather to evaluate the performance
of round functions following these structures, in order to assess the utility
of large diffusion mappings for block cipher design. Consequently, our security
analysis of \samneric is minimal, and only serves to give a realistic estimate
of how many rounds are needed to achieve good security. 

%We do this study in the context of block ciphers, by assuming that $\matdiff$ is used
%as the linear mapping of two ciphers with a \shark{} structure: one with 4-bit S-Boxes
%and a 64-bit block, and one with 8-bit S-Boxes and a 128-bit block.
%What we wish to
%measure in both cases is the speed in cycles per byte of such hypothetical ciphers, so as to
%be able to gauge the efficiency of this linear mapping and of the resulting ciphers.
%In order to do this, we need to estimate how many rounds would be needed for
%the ciphers to be secure.
%% We do this with the following basic analyses of
%% linear and differential properties of such algorithms.

\subsection{The \sam block cipher}
The round function of \sam is taken to be $\matdiff \circ \sapp \circ \adkey$, where $\adkey$ adds
a 64-bit round key to its input and $\sapp$ is the parallel application of sixteen identical
4-bit S-boxes, taken to be any optimal S-box of this size, see \eg \cite{4bit1,class4bit}.

We use standard $\emph{wide-trail}$ considerations to study differential and linear properties of a \sam instance~\cite{aes}.
This is very easy to do thanks to the simple structure of the cipher.
The branch number of $\matdiff$ is 15, which means that at least 15 S-boxes are active in any two rounds of
a differential path or linear characteristic, while the best 4-bit S-boxes have a maximum differential probability
of $2^{-2}$ and a bias of at most $2^{-2}$ for any linear approximation; we will discuss such notions in more details
in the next chapter. By using
such S-boxes, one can upper-bound the probability of a single differential path (respectively the bias of a single linear characteristic)
for $2 n$ rounds by $2^{-2\cdot 15 n}$ (respectively $2^{15n - 1}2^{-2\cdot 15 n}$) . This is smaller than $2^{-64}$ (respectively
$2^{-32}$)
as soon as $n > 2$. Hence we conjecture that 6 to 8 rounds are enough to make a cipher resistant
to standard statistical attacks.

The most powerful attacks against \sam may in fact not be statistical; algebraic attacks may be more
efficient in exploiting the fact that the round function is defined at a granularity of four bits,
and that the degree of an invertible 4-bit S-box is at most three. Applying the upper-bound
from Boura \etal~\cite[Theorem 2]{permdegree}, we get that seven rounds are necessary to
obtain a permutation of maximal degree. Thus, eight full rounds plus an additional half-round
consisting of only $\sapp$ would probably be necessary to avoid attacks based on the low degree
of the cipher.

On the other hand, we would expect \sam to be more resistant than most ciphers to structural attacks such as
impossible-differential or integral attacks, as its round function diffuses over the full block.

We give software performance figures for an 8-round version of \sam in \autoref{tbl:perf_sam}.
We include data both for a strict SSSE3 implementation and for one using AVX extensions, which
can be seen to bring a considerable benefit. Note that the last round is complete and includes the linear mapping, unlike \eg{}
\AES{}.
As the parallel application of the S-Boxes of $\sapp$ can be implemented very efficiently with
a single \pshufb{}, this part of the round function has virtually no impact on the performance, and one
could for instance add an extra half-round essentially for free.

\begin{table}
\caption[Performance of software implementations of \sam, given in cycles per byte.]{Performance of software implementations of \sam, given in cycles per byte (cpb),
for implementations using \autoref{alg:broadcast} and \autoref{alg:shuffle}. Figures in parentheses are for an AVX implementation when applicable.
\label{tbl:perf_sam}}
\begin{center}
\begin{tabularx}{\textwidth}{@{\extracolsep{2mm} } l c  X  X}
\toprule
Processor type & \# rounds &  cpb (Alg. 6.1) &  cpb (Alg. 6.2)\\
\midrule
Intel Xeon E5-2650 @ 2.00GHz & 8 &  66.5 (60.2) & 44.5 (31.9)\\
\midrule
Intel Xeon E5-2609 @ 2.40GHz & 8 &  95.3 (84.7)  & 63.3 (45.6)\\
\midrule
Intel Xeon E5649 @ 2.53GHz & 8 & 111.3 & 59.8 \\
\bottomrule
\end{tabularx}
\end{center}
\end{table}

\subsection{The \eric block cipher}
% In fact, could use circulant MDS w/o cost penalty?
Although the code $\cagh$ from which the matrix $\matdiff$ is built was initially defined with $\Fst$ as an alphabet, this latter
can be replaced by an algebraic extension of $\Fst$ such as $\Fth$, to yield a code $\cagh'$ with the same parameters,
namely a $[32, 16, 15]_{\Fth}$ code. One way of seeing this is that by constructing $\Fth$ as an extension of $\Fst$,
for instance by letting $\Fst \cong \mathbf{F}_2[\alpha]/(\alpha^4 + \alpha^3 + \alpha^2 + \alpha + 1)$ and
$\Fth \cong \Fst[t]/(t^2 + t + \alpha)$; an
element of $\Fth$ is represented as a degree-one polynomial $at + b$ over $\Fst$.
It follows that the minimum weight of a codeword $w \defas (a_i t + b_i),\,i \in \{0\ldots31\}$ of $\cagh'$ is at least equal to the
minimum weight of words $(a_i),\,i = 0\ldots31$ and $(b_i),\,i \in \{0\ldots31\}$.
If those are taken among codewords
of $\cagh$, their minimum weight is 15, and thus so is the one of $w$.

An efficient encoder can also be built for $\cagh'$, with only small changes from an encoder for $\cagh$.
If we consider finite field multiplications as performed by linear feedback shift registers (LFSRs), the multiplication by
a constant is done by summing shifts of the multiplicand and possibly shifted copies of the feedback polynomial. Such a
multiplication, say on eight bits can obviously be tabulated in a table of 256 8-bit entries, but smaller tables can also
be used, exploiting the linearity. For instance, one can compute the multiplication from two tables of 16 8-bit entries which store the partial
multiplication of the constant by the four most and least significant bits respectively; performing the complete multiplication
simply consists in accessing these tables with the right inputs and summing their results. We give an example of such
an approach
in \autoref{lst:mulex}.

\begin{figure}[!htb]
\begin{center}
\begin{minted}[breaklines]{c}
uint8_t m71(uint8_t x)
{
   uint8_t lo[16] = {0x00,0x71,0xE2,0x93,0xDF,0xAE,0x3D,0x4C
                    ,0xA5,0xD4,0x47,0x36,0x7A,0x0B,0x98,0xE9};
   uint8_t hi[16] = {0x00,0x51,0xA2,0xF3,0x5F,0x0E,0xFD,0xAC
                    ,0xBE,0xEF,0x1C,0x4D,0xE1,0xB0,0x43,0x12};

   return (lo[x & 0xF] ^ hi[x >> 4]);
}
\end{minted}
\end{center}
\caption{Multiplication by \texttt{0x71} in $\mathbf{F}_{2}[x]/<x^8+x^4+x^3+x+1>$\label{lst:mulex}}
\end{figure}

Table lookups from four to eight bits can be performed efficiently with \pshufb; multiplication
by a constant in $\Fth$ is thus twice as expensive as multiplication in $\Fst$, but it operates
on twice the amount of data. An implementation of a diffusion matrix
$\matdiff'$ from $\cagh'$ is thus comparatively as efficient as for the original code. 

\medskip 

The round function of \eric is taken to be $\matdiff' \circ \sapp' \circ \adkey'$, where $\adkey'$ adds a
128-bit round key to its input and $\sapp'$ is the parallel application of sixteen identical cryptographically
strong 8-bit S-boxes.

From wide-trail considerations, the resistance of an \eric cipher to statistical attacks is comparatively better than \sam,
when an appropriate 8-bit S-Box is used.
For instance, the $\AES$ S-box has a maximal differential probability of $2^{-6}$, and it cannot be approximated
linearly with a bias more than $2^{-4}$~\cite{aes}.
This implies that the probability (respectively bias) of a single differential path (respectively linear characteristic)
for $2n$ rounds is upper-bounded by $2^{-6\cdot15 n}$ (respectively $2^{15n -1}2^{-4\cdot 15n}$), which is already much smaller than
$2^{-128}$ (respectively $2^{-64}$) as soon as $n > 1$.
\eric is also likely to be more resistant to algebraic attacks; at least five rounds are necessary to reach
full degree for an S-box of degree seven, which is two less than \sam.
Thus, eight and a half rounds of an \eric cipher should bring adequate security, and six and a half rounds might be enough.
We provide performance figures for SSSE3 and an AVX implementations in \autoref{tbl:perf_eric}, for both six and eight full rounds.
Note that in the case of 8-bit S-Boxes,
the S-Box application is usually rather more complex and expensive a step than with 4-bit S-Boxes. In these test programs, we used the efficient vector
implementation of the \AES{} S-Box from Hamburg~\cite{vpaes}; better performance may be attained by using cryptographically
weaker but more efficient S-boxes, such as \whirlpool's second S-box \cite{whirlpool}.

\begin{table}
\caption[Performance of software implementations of \eric, in cycles per byte.]{Performance of software implementations of \eric, in cycles per byte (cpb),
for implementations using \autoref{alg:broadcast} and \autoref{alg:shuffle}. Figures in parentheses are for an AVX implementation when applicable.
\label{tbl:perf_eric}}
\begin{center}
\begin{tabularx}{\textwidth}{@{\extracolsep{2mm} } l c  X  X}
\toprule
%\cline{3-6}
\multicolumn{2}{c}{} & \multicolumn{2}{c}{128-bit Block} \\
\cmidrule(lr){3-4}
Processor type & \# rounds & cpb (Alg. 6.1) & cpb (Alg. 6.2)\\
\midrule
 \multirow{2}{*}{Intel Xeon E5-2650 @ 2.00GHz} & 6 & 58 (52.3) &  32.7 (26.5) \\
 %\cline{2-6}
 & 8 &  76.8 (69.6) &  43.8 (35.7) \\
\midrule

 \multirow{2}{*}{Intel Xeon E5-2609 @ 2.40GHz } & 6 & 79.8 (75.6) &  47.1 (36.8) \\
% \cline{2-6}
 & 8 & 106.6 (97.1)  &  62.1 (50.3) \\
\midrule

 \multirow{2}{*}{Intel Xeon E5649 @ 2.53GHz} & 6 & 84.5 & 47 \\
% \cline{2-6}
 & 8 & 111 & 61.9 \\
\bottomrule
\end{tabularx}
\end{center}
\end{table}


\subsection{Discussion}
The performance figures given in \autoref{tbl:perf_sam}, \autoref{tbl:perf_eric} are average for a block cipher. For instance, it compares favourably with
the optimized vector implementations of
64-bit ciphers \led and \piccolo in sequential mode from~\cite{sac2013}, which run at speeds between 70 and 90 cpb., depending on the CPU.
It is however slower than Hamburg's vector implementation of \AES{}, with reported speeds of 6 to 22 cpb. (9 to 25 for the inverse cipher)~\cite{vpaes,twine}.

Yet, we conjecture that the strong structure of the round function makes \samneric more suitable to speed-ups from derivative constructions. For instance,
the \emph{leak extraction} approach used in the stream cipher \lex allowed a 2.5-time speed-up from the \AES by leaking a quarter of the state of the cipher as keystream every round~\cite{DBLP:conf/sacrypt/Biryukov06}.
Although \lex was later broken~\cite{DBLP:conf/asiacrypt/DunkelmanK08a}, we believe that the stronger diffusion in \samneric may protect
leak extraction variants against similar attacks; these could for instance lead to two-time speed-ups for an eight-round \eric with leaks of
the same amount as the ones of \lex.
