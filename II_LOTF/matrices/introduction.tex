\section{Introduction}

The use of \emph{MDS} matrices over finite fields as linear mappings in block cipher design is an old trend, followed by many prominent algorithms such as the \AES/\rijndael{} family~\cite{aes} and its predecessors~\cite{shark,square}.
These matrices are called MDS as they are derived from \emph{maximum distance separable} linear error-correcting codes, which achieve the highest minimum distance possible
for a given length and dimension. This notion of minimum distance coincides with the one of \emph{branch number} of a mapping~\cite{phddaemen}, which is a measure of the effectiveness
of a diffusion layer.
MDS matrices thus have an optimal diffusion, in a cryptographic sense, which makes them attractive for cipher design.

The good security properties that can be derived from MDS matrices are often counter-balanced by the cost of their computation. The standard matrix-vector product is
quadratic in the dimension of the vector, and finite field operations are not always efficient.
For that reason, there is often a focus on finding matrices allowing efficient implementations. For instance, the \AES{} matrix is circulant and has small coefficients.
More recently, the \photon{} hash function~\cite{photon} introduced the use of matrices that can be obtained as the power of a companion matrix, whose sparsity may be useful in
lightweight hardware implementations. The topic of finding such so-called recursive diffusion layers has been quite active in the past years, and led to a series
of papers investigating some of their various aspects~\cite{recursive1,recursive2,recursive3}. One recent development shows how to systematically construct some of these matrices
from BCH codes~\cite{recursive4}. This allows in particular to construct very large recursive MDS matrices, for instance of dimension sixteen over $\Fth$. This defines a linear mapping over
a full 128-bit block with excellent diffusion properties, at a moderate hardware implementation cost.

As interesting as it may be in hardware, the cost in software of a large linear mapping tends to make these designs rather less attractive than more balanced solutions.
An early attempt to use a large matrix was the block cipher \shark{}, a \rijndael{} predecessor~\cite{shark}; the same kind of design
was also later used for \khazad~\cite{khazad}. Both are  64-bit ciphers which use an MDS matrix of dimension
eight over $\Fth$ for their linear diffusion. The usual technique for implementing such a mapping in software is to rely on a table of precomputed multiples of
the matrix rows. However, table-based implementations now tend to be frown upon as they may lead to \emph{timing attacks}~\cite{timinattacks}, and this could
leave ciphers with a structure similar to \shark{}'s without reasonable software implementations when resistance to these attacks is required.
Yet, such designs also have advantages of their own; their diffusion acts on the whole state at every round, and therefore makes structural attacks harder,
while also ensuring that many S-Boxes are kept active.
Additionally, the simplicity of the structure makes it arguably easier to analyse than in the case of most ciphers.

\paragraph{Our contributions.}

In this chapter, we revisit the use of a \emph{\shark{} structure} for block cipher design and endeavour to find good matrices and appropriate algorithms
to achieve both a linear mapping with very good diffusion and efficient software implementations that are not prone to timing attacks.
To be more specific on this latter point, we target software running on 32 or 64-bit CPUs featuring an SIMD vector unit.

A possible way of making a mapping more efficient is to decrease the size of the field from $\Fth$ to $\Fst$. However, according to the \emph{MDS conjecture},
there is no MDS code over $\Fst$ of length greater than seventeen~\cite{mdsConj}. Because a diffusion matrix of dimension $n$ is typically
obtained from a code of length $2n$, MDS matrices over $\Fst$ are therefore restricted to dimensions less than eight. Hence, the prospect
of finding an MDS matrix over $\Fst$ diffusing on more than $8\times4 = 32$ bits is hopeless.
Obviously,
32 bits is not enough for a large mapping \emph{à la} \shark{}. We must therefore search for codes with a slightly smaller minimum distance
in the hope that they can be made longer.

Our proposed solution to this problem is to use \emph{algebraic geometry codes}, as they precisely offer this tradeoff.
One way of defining these codes is as evaluation codes on algebraic curves; thus our proposal brings a nice connection between these objects and
symmetric cryptography: although elliptic and hyperelliptic curves are now commonplace in public-key cryptography, we show an unusual application of
a hyperelliptic curve to the design of block ciphers. We present a specific code of length 32 and dimension 16 over $\Fst$ with minimum distance 15, which is only
two less than what an MDS code would achieve; this is also much better than a random code for these alphabet, length and dimension, as such a code
would typically have an expected minimum distance of 11.
This lets us deriving a very good diffusion matrix on $16\times4 = 64$ bits in a straightforward way. Interestingly, this matrix can also be applied
to vectors over an extension of $\Fst$ such as $\Fth$, while keeping the same good diffusion properties.
This allows for instance to increase the diffusion to $16\times8 = 128$ bits.

We also study two simple yet efficient algorithms for implementing the matrix-vector multiplication needed in a \shark{}
structure, when a \emph{vector permute} instruction is available. From one of these, we derive conditions on the matrix to make the matrix-vector product
faster to compute, in the form of a cost function; we then search for matrices with a low cost, both randomly, and by using automorphisms of the code and of the hyperelliptic
curve on which it is based. The use of codes automorphisms to derive efficient encoders is not new~\cite{sysgrob,grobarch},
but it is not generally applied to the architecture and dimensions that we consider in our case.

We conclude the chapter by presenting examples of performance figures of assembly implementations of our algorithms when used as the linear mapping of
a block cipher.
