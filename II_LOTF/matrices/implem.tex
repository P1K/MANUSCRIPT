\section{Efficient algorithms for matrix-vector multiplication}
\label{implem}

We now turn to implementation considerations, and present two algorithms
for matrix-vector multiplication in $\matspace_n(\Fst)$. We let the dimension
$n$ be equal to 16 in this presentation, but the algorithms apply equally well
to other (close) values.

%This section presents software algorithms for matrix-vector multiplication over $\Fst$.
%%We specifically discuss these when the field is $\Fst$, but they can be generalized
%%at least to some extent.
%We focus on square matrices of dimension 16.
%This naturally defines linear operations on 64 bits, which can also be extended to 128 bits,
%as it will be made clear in \autoref{sec:appli}. Both cases are a common block size for block ciphers.


\paragraph{Targeted architecture.}
The algorithms in this section target CPUs featuring vector instructions, including in particular a
\emph{vector shuffle} instruction such as Intel's \pshufb{} from the SSSE3 instruction set extension~\cite{ia64}. These instructions are
now widespread and have already been used successfully in fast cryptographic implementations, see \eg{}~\cite{vpaes,twine,sac2013}.
We mostly considered SSSE3 when designing the algorithms, but other processor architectures
do feature vector instructions. This is for instance the case of ARM's NEON extensions, which may also
yield efficient implementations, see \eg{}~\cite{Bernstein}. We do not consider these explicitly in this chapter, however.

Because it plays an important role in our algorithms, we briefly recall the semantics of \pshufb{}.
This instruction takes two 128-bit inputs\footnote{It can actually also be used on 64-bit operands, but we do not
consider this possibility here.}. The first (the destination operand) is an \xmm{} SSE vector register which logically represents
a vector of sixteen bytes. The second (the source operand) is either a similar \xmm{} register, or a 128-bit memory location. The result of calling
$\pshufb~x~y$ is to overwrite the input $x$ with the vector $x'$ defined by:
\[
x'[i] = \left\{
				\begin{array}{ll}
				x[\lfloor y[i]\rfloor_4]  & \text{if the most significant bit of $y[i]$ is not set}\\
				0 & \text{otherwise}
				\end{array}
	    \right.
\]
where $\lfloor\cdot\rfloor_4$ denotes truncation to the four least significant bits.
This instruction allows to arbitrarily \emph{shuffle} a vector according to a mask, with possible repetition and omission of some of the vector
values\footnote{We will use the word \emph{shuffle} with this precise meaning in the remainder of this chapter.}.
Notice that this instruction can also be used to perform sixteen parallel 4-to-8-bit table lookups: let us call $T$ this table; take
as first operand to \pshufb{} the vector $x \defas (T[i],~i = 0,\ldots,15)$, as second operand the vector
$y \defas (a, b, c, d, \ldots)$
on which to perform the lookup; then we see that the first byte of the result is $x[y[0]] = T[a]$, the second is $x[y[1]] = T[b]$,
etc.

Finally, let us mention that there is a three-operand variant of this instruction in the more recent AVX instruction set and onward~\cite{ia64},
which allows not to overwrite the first operand.


\paragraph{Targeted properties.} In this chapter we focus solely on algorithms that can easily be implemented in a way that makes
them immune to timing-attacks (see \eg \cite{timinattacks}). Specifically, we consider the matrix as a known constant but the vector as a secret, and we wish to perform
the multiplication without secret-dependent branches or memory accesses. It might not always be important to be immune (or even partially resistant) to this type
of attacks, but we consider that it should be important for any cryptographic primitive or structure to \emph{possibly} be implemented in such a way. Hence we try
to find efficient such implementations for the \shark{} structure and therefore for dense matrix-vector multiplications.

\bigskip

We now go on to describe the algorithms.
In all of the remainder of this section, $\veci$ and $\veco$ are two (column) vectors of $\Fst^{16}$, and $\mat$ a matrix
of $\mappi_{16}(\Fst)$.
%We first briefly discuss two unsatisfactory straightforward ways of implementing matrix-vector multiplication.
We first briefly recall the principle of table implementations which are generic and efficient but unsatisfactory when timing attacks are taken into account.

\subsection{Table implementation}


% \subsubsection{Explicit multiplication.} The first obvious way to compute the multiplication $\veco = \mat\cdot\veci$ is to explicitly
% compute the vector $\veco = (\veco_i), i \in \{0\ldots15\}$, with $\veco_i = \sum_{j=0}^{15} \mat_{i,j} \cdot \veci_j$.
% This requires to perform about $16^2 = 256$ multiplications in $\Fst$ for a dense matrix.
% These multiplications can in turn be performed by table lookups, logarithm tables, or with an ``xtime'' operation (which
% is equivalent to using a linear-feedback shift register
% (LFSR)). Whatever the technique used, the number of multiplications required makes this option rather inefficient.
% In addition, only the in-itself quite slow LFSR multiplication can easily be made insensitive to potential timing attacks, which makes this
% algorithm even less attractive when these are a concern.

We wish to compute $\veco = \mat\veci$. The idea of table implementations is to use table lookups to perform the equivalent multiplication
$\veco^t = \veci^t\mat^t$, \ie $\veco^t = \sum_{i = 0}^{15} \veci_i\mat^t_i$.
This can be computed efficiently by tabulating beforehand the products $\lambda\mat^t_i, \lambda\in\Fst$ (resulting in
sixteen tables, each of sixteen entries of 64 bits), and then for each
multiplication by accessing the table for $\mat^t_i$ at the index $\veci_i$ and summing all the retrieved table entries together.
This only requires sixteen table lookups per multiplication. However, the memory accesses depend on the value of $\veci$,
which makes this algorithm inherently vulnerable to timing attacks.

Note that there is a more memory-efficient alternative implementation of this algorithm which consists in explicitly computing the multiplications for each term $\lambda\mat^t_i$.
This can be done with a single \pshufb instruction per matrix row, taking as first operand the table of multiplication by $\lambda$ and as second operand a register representing $\mat^t_i$.
In that case, only the sixteen multiplication tables need to be stored, but their accesses
still depend on the secret value $\veci$.


\subsection{A generic constant-time algorithm}
\label{bbased}
We now describe our first algorithm, which can be seen as a variant of table multiplication that is immune to timing attacks.
The idea consists again in computing the right multiplication $\veco^t = \veci^t\mat^t$, \ie{} $\veco^t = \sum_{i = 0}^{15} \veci_i\mat^t_i$. However, instead of tabulating
the results of the scalar multiplication of the matrix rows $\mat^t_i$, those are always recomputed, in a way that does not explicitly depend on the value of the scalar.
%We first present how to achieve this in a somehow abstract way, and then give an explicit sequence of instructions
%performing this operation.

\subsubsection{Description of \autoref{alg:broadcast}.}
%We give the full description of \autoref{alg:1} in appendix \autoref{app:algo}, and focus here on the intuition.
We start with a high-level description of the algorithm.
Assume we want to perform the scalar multiplication $\lambda\vect$ for an unknown scalar $\lambda$ and a known, constant vector $\vect \in \Fst^{16}$.
Let us write $\lambda$ as the polynomial
$\lambda_3\cdot \alpha^3 + \lambda_2\cdot \alpha^2 + \lambda_1\cdot \alpha + \lambda_0$ with coefficients in $\mathbf{F}_2$.
Then, the result of $\lambda\cdot\vect$ is simply $\lambda_3\cdot(\alpha^3\cdot\vect) + \lambda_2\cdot(\alpha^2\cdot\vect) + \lambda_1\cdot(\alpha\cdot\vect) + \lambda_0\cdot\vect$.
Thus we just need to precompute the products $\alpha^i\cdot\vect$, select the right ones with respect to the binary representation
of $\lambda$, and add these together.
This can easily be achieved thanks to a \emph{broadcast} function defined as:
\[
\broad(x,i)_n = \left\{
				\begin{array}{ll}
				\mathbf{1}_n  & \text{if the $i^\text{th}$ bit of $x$ is set}\\
				\mathbf{0}_n & \text{otherwise}
				\end{array}
	    \right.,
\]
The full algorithm then just consists in using this
scalar-vector multiplication sixteen times, one for each row of the matrix.

This results in \autoref{alg:broadcast}. Note that a similar algorithm was used by Käsper and Schwabe for binary
matrices of dimension 128~\cite{DBLP:conf/ches/KasperS09}.

\begin{algorithm}[!htb]
  \caption{Broadcast-based matrix-vector multiplication}
\label{alg:broadcast}
  \KwIn{ $\veci \in \Fst^{16}$, $\mat \in \mappi_{16}(\Fst)$}
  \KwOut{  $\veco = \veci^t\cdot\mat^t$}
  Offline phase\\
  \For{$i \in \{0\ldots15\}$}
  {
  $8\mat_i \defas \alpha^3\cdot\mat_i$\\
  $4\mat_i \defas \alpha^2\cdot\mat_i$\\
  $2\mat_i \defas \alpha\cdot\mat_i$\\
  }
  Online phase\\
  $\veco \defas \mathbf{0}_{64}$\\
  \For{$i \in \{0\ldots15\}$}
  {
  $\upsilon_i^8 \defas 8\mat_i \wedge_{64} \broad(\veci_i,3)_{64}$\\
  $\upsilon_i^4 \defas 4\mat_i \wedge_{64} \broad(\veci_i,2)_{64}$\\
  $\upsilon_i^2 \defas 2\mat_i \wedge_{64} \broad(\veci_i,1)_{64}$\\
  $\upsilon_i^1 \defas  \mat_i \wedge_{64} \broad(\veci_i,0)_{64}$\\
  $\upsilon_i \defas \upsilon_i^8 + \upsilon_i^4 + \upsilon_i^2 + \upsilon_i^1$\\
  $\veco \defas \veco + \upsilon_i$\\
  }
  \Return{$\veco$}
\end{algorithm}


\subsubsection{Implementation of \autoref{alg:broadcast} with SSSE3 instructions.}
We now consider how to efficiently implement \autoref{alg:broadcast} in practice. The only
non-trivial operation that is used is the $\broad$ function, and we show that it can be performed
with only one or two \pshufb{} instructions.

To compute $\broad(\lambda,i)_{64}$, with $\lambda$ a 4-bit value,
we can use a single \pshufb{} with first operand
$x$, such that $x[j] = 11111111_2$ if the $i^\text{th}$ bit of $j$ is set and 0 otherwise, and with second operand $y \defas (\lambda, \lambda, \ldots)$.
The result of \pshufb{}~$x~y$ is indeed $(x[\lambda], x[\lambda], \ldots)$ which is $\mathbf{1}_{64}$ if the $i^\text{th}$
bit of $\lambda$ is set and $\mathbf{0}_{64}$ otherwise, that is $\broad(\lambda, i)_{64}$\footnote{The quantities actually
computed this way are rather $\mathbf{0}_{128}$ and $\mathbf{1}_{128}$. Adapting this to $\mathbf{0}_{64}$
and $\mathbf{1}_{64}$ is however simple to do.}.

In practice, the vector $x$ can conveniently be constructed offline and stored in memory, but the vector $y$ might not be readily available before
performing this computation\footnote{And because it depends on what we assume to be a secret value, it cannot either be fetched from memory.}.
However, it can easily be computed thanks to an additional \pshufb{}. Alternatively, if the
above computation is done with a vector $y \defas (\lambda, ?, ?, \ldots)$ instead (with $?$ denoting unknown values) and call $z$ its result $(x[\lambda], ?, ?, \ldots)$,
then we have $\broad(\lambda, i)_n = \pshufb~z~ (0, 0, \ldots)$.

\medskip

In the specific case of matrices of dimension sixteen over $\Fst$, one can take advantage of the 128-bit
wide \xmm{} registers by interleaving, say, $8\veci$ with $4\veci$, and $2\veci$ with $\veci$, and by computing a slightly more
complex version of the $\broad$ function $\broad(x,i,j)_{2n}$ which interleaves $\broad(x,i)_n$ with $\broad(x,j)_n$.
In that case, an implementation of one step of \autoref{alg:broadcast} only requires two $\broad$ calls, two logical \emph{and}, folding back the interleaved vectors
(which only needs a couple of logical shift and exclusive or), and adding the folded vectors together.
We give a snippet of such an implementation in \autoref{app:gene}.

\paragraph{Scalar implementation of $\broad$.}
On an architecture without access to \pshufb, the $\broad$ function can still be implemented rather efficiently with a few arithmetic and logical instructions.
Let $x$ be an $n$-bit value, $n = 64$ (w.l.o.g.) represented by an unsigned integer, then the following sequence of C instructions computes $\broad(x,i)_n$:
\begin{center}
\begin{minted}{c}
x = ~x;
x >>= i;
x &= 1;
x -= 1ull;
\end{minted}
\end{center}
In this snippet, after the third line, $x$ is equal to 0 if its $i^\text{th}$ bit was initially one, and 1 otherwise. The subtraction by one in line four thus produces the desired result.

\subsection{A faster algorithm exploiting matrix structure}
\label{shuff}

\autoref{alg:broadcast} is already reasonably efficient, and has the
advantage of being completely generic w.r.t. the matrix. Yet, better solutions may exist in more specific cases.
We present here an alternative that can be much faster when the matrix possesses a particular structure.

The idea behind this second algorithm is to take advantage of the fact that many coefficients of a matrix may be identical, thence many of the finite-field multiplications performed in
a matrix-vector product consist in multiplications by the same constants. Thus, we may be tempted
to take advantage of this fact by performing those in parallel.
A motivation of this approach is that the multiplications being by constants
(the known coefficients of the matrix), they can be
performed by \pshufb-implemented table lookups, which are more efficient than
a broadcast-based algorithm.

\subsubsection{Description of \autoref{alg:shuffle}}
We first give a high-level idea of the algorithm.
Let us first consider a small example, and compute $\mat\veci$ defined as:
\begin{equation}
\label{matey}
	\begin{pmatrix}
	    1 & 0 & 2 & 2 \\
		3 & 1 & 2 & 3 \\
		2 & 3 & 3 & 2 \\
		0 & 2 & 3 & 1 \\
	\end{pmatrix}\cdot
	\begin{pmatrix}
	x_0 \\
	x_1 \\
	x_2 \\
	x_3 \\
	\end{pmatrix}.
\end{equation}
It is obvious that this is equal to:
\[
	\begin{pmatrix}
	x_0 \\
	x_1 \\
	0 \\
	x_3 \\
	\end{pmatrix}
	+~
	2\cdot
	\begin{pmatrix}
	x_2 \\
	x_2 \\
	x_0 \\
	x_1 \\
	\end{pmatrix}
	+~
	2\cdot
	\begin{pmatrix}
	x_3 \\
	0 \\
	x_3 \\
	0 \\
	\end{pmatrix}
	+~
	3\cdot
	\begin{pmatrix}
	0 \\
	x_0 \\
	x_1 \\
	x_2 \\
	\end{pmatrix}
	+
	3\cdot
	\begin{pmatrix}
	0 \\
	x_3 \\
	x_2 \\
	0 \\
	\end{pmatrix},
\]
where both the constant multiplications of the vector $(x_0, x_1, x_2, x_3)^t$ and the shuffles of its coefficients can be computed
with a single \pshufb{} instruction each, while none of these operations directly depends on the value of the vector.
This type of decomposition can be done for any matrix, but the number of operations depends on the value of its coefficients.

\medskip

We now sketch one way of obtaining an optimal decomposition as above.
We consider a matrix product $\mat\veci$ with $\mat$ constant and $\veci$ unknown,
where $\veci$ is seen as the formal arrangement of variables $\veci_i$.
Let us define $\shuff(\mat, \lambda)$ as one of the minimal sets of shuffles of coefficients of $\veci$, such that
there exists a unique vector $\vect \in \shuff(\mat, \lambda)$ with $\vect_i = \veci_j$ iff $\mat_{i,j} = \lambda$.
For instance, in the above example, we have $\shuff(\mat,2) = \{(x_2,x_2,x_0,x_1)^t, (x_3,0,x_3,0)^t\}$. Equivalently, we could have taken
$\shuff(\mat, 2) = \{(x_3,x_2,x_3,0)^t, (x_2,0,x_0,x_1)^t\}$.
These sets are straightforward to compute from this particular matrix, and so are they in the general case.

From the definition of $\shuff$, it is clear that we have $\mat\veci = \sum_{\lambda \in \Fst^*}\sum_{\vect \in \shuff(\mat,\lambda)} \lambda\cdot \vect$.
Once the values of the sets $\shuff$ have been determined, it is clear that we only need to compute this sum to get our result, and 
this is precisely what this second algorithm does.

We give a complete description of this shuffle-based process in \autoref{alg:shuffle}.

\begin{algorithm}[!h]
  \caption{Shuffle-based matrix-vector multiplication}
\label{alg:shuffle}
  \LinesNumbered
  \KwIn{   $\veci \in \Fst^{16}$, $\mat \in \mappi_{16}(\Fst)$}
  \KwOut{  $\veco = \mat\cdot\veci$}
  Offline phase\\
  \For{$i \in \{1\ldots15\}$}
  {
  $s_i[\,] \defas \shuff(M, i)$ $\triangleright$ \emph{Initialize the array $s_i$ with one of the possible sets of shuffles.}
  }
  Online phase\\
  $\veco \defas \mathbf{0}_{64}$\\
  \For{$i \in \{1\ldots15\}$}
  {
  \For{$j < \#s_i$}
  {
  $\veco \defas \veco + i\cdot s_i[j]$
  }
  }
  \Return{$\veco$}
\end{algorithm}

\subsubsection{Cost of \autoref{alg:shuffle}.} The cost of computing a matrix-vector product with \autoref{alg:shuffle} depends
on the coefficients of the matrix, since the size of the sets $\shuff(\mat, \lambda)$ depends both on the density
of the matrix and of how its coefficients are arranged.

If we suppose that a vector implementation of this algorithm is used, and if the dimension and the field of the matrix are well chosen,
we can assume that both the scalar multiplication of $\veci$ by a constant and a shuffle can be computed with a single
\pshufb{} each, and a few ancillary instructions. We take the number of calls to \pshufb as a basis to define a cost function for matrices with respect to their implementations with \autoref{alg:shuffle}.
It is defined as:
%\[
%\cost(M) \defas \big(\sum_{\lambda \in \Fst^*} \indic(\shuff(\mat, \lambda)) + \#\shuff(\mat, \lambda)\big) - \indic(\shuff(\mat, 1)),
%\]
\[
\cost(M) \defas \sum_{\lambda \in \Fst^*}\#\shuff(\mat, \lambda) + \sum_{\lambda \in \Fst^*-\{1\}}\indic(\shuff(\mat, \lambda)).
\]
The first term corresponds to the number of shuffles that need to be performed for each constant value, and the second to the number
of constant multiplications to do.
We may notice that $\#\shuff(\mat, \lambda)$ is equal to the maximum number of occurrence of $\lambda$ in a single row of $\mat$, and the $\cost$ function
is therefore easy to compute. As an example the cost of the matrix $\mat$ from \autoref{matey} is 7. In the special case of a matrix that has a diagonal
with identical non-zero coefficients, the actual cost is one less than computed with $\cost$.

\medskip

A matrix with minimal cost w.r.t. $\cost$ is one that
minimizes the sum of the maximum number of occurrence of $\lambda$ in any row, for every $\lambda \in \Fst^*$.
A simple observation is that for matrices with the same number of non-zero coefficients, this amount is the smallest when every row can be deduced by permutation of a single one;
this is for instance the case for circulant matrices, by definition.
More generally, we can heuristically hope that the cost of a matrix will be low if all of its rows can be obtained from the permutation of, say, only two rows.

We can try to estimate the minimum cost of an arbitrary dense circulant matrix of dimension sixteen over $\Fst$.
If the coefficients of the (unique, up to permutation) row are chosen at random, we should expect them to be made of about $2/3$ of the elements of $\Fst$.
Additionally, fifteen permutations of the input will need to be computed (all rows need to be different for the matrix to be of full rank ---which is something we will
desire in our applications--- but the coefficients on the diagonal are constant).
Thus we can estimate the cost of such a random matrix to be about 25.

\medskip

Before concluding this section, let us notice that special cases of this algorithm have already been used for circulant matrices, namely in the case of the \AES{} \mc{} matrix~\cite{vpaes}.

\subsubsection{Implementation of \autoref{alg:shuffle} with SSSE3 instructions.}

The implementation of \autoref{alg:shuffle} is quite straightforward. We give nonetheless a small code snippet in \autoref{app:spec}.

\subsection{Performance}

In \autoref{tbl:perf_sam} and \autoref{tbl:perf_eric} of \autoref{sec:appli}, we give a few performance figures for ciphers with a \shark{} structure using assembly implementations of \autoref{alg:broadcast} and \autoref{alg:shuffle}  for their linear mapping.
From there it can be seen without surprise that \autoref{alg:shuffle} is more efficient if the matrix is well chosen. However, \autoref{alg:broadcast}
still performs reasonably well, without imposing any condition on the matrix.
