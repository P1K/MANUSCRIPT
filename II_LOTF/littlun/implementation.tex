\section{Implementation of \littlunOne}
\label{sec:implem}

\subsection{Hardware implementation}
We give a circuit (using OR, AND and XOR gates) implementing \littlunS in \autoref{sb4_circ} in \autoref{sec:circ}.
A hardware implementation of the entire S-box can easily be deduced by plugging this circuit into the one of \autoref{littlunS}.
As previously mentioned, \littlunOne can be implemented with 12 non-linear (OR and AND) gates and 24 XOR gates. With a typical cell library
such as the Virtual Silicon standard cell library, OR and AND gates cost 1.33 gate equivalent (GE), and XOR gates 2.67 GE. Thus synthesising
the S-box with this library would cost 80 GE. 

\subsection{Bitsliced software implementation}
One of our main objective w.r.t. implementation was to obtain an S-box with an efficient \emph{bitsliced} implementation in
software. This is closely related to the simplicity of the circuit of the S-box, though not exactly equivalent.
We purposefully chose a 4-bit S-box from the \emph{class 13} of Ullrich \etal \cite{skew} because of its very
efficient bitsliced implementation that requires only 9 instructions
on a wide variety of platforms.
%(and offers good instruction-level parallelism).
Such an implementation is given in \autoref{bitsliced4}.
From this, it is easy to obtain an efficient bitsliced implementation for the whole S-box, as shown in \autoref{bitsliced8}.
This implementation typically requires 43 instructions and 13 registers.

\begin{figure}[ht]
\begin{minted}[breaklines]{c}
t  = b;    b |= a;    b ^= c; // (B): c ^ (a | b)
c &= t;    c ^= d;            // (C): d ^ (c & b)
d &= b;    d ^= a;            // (D): a ^ (d & B)
a |= c;    a ^= t;            // (A): b ^ (a | C)
\end{minted}
\caption[Snippet for a bitsliced \C implementation of \littlunS.]{Snippet for a bitsliced \C implementation of \littlunS \label{bitsliced4} with input and output in registers $a,b,c,d$ (the word holding the most significant bit is taken to be $a$), using one extra register $t$.}
\end{figure}

\begin{figure}[ht]
\begin{minted}[breaklines]{c}
t = a ^ e;
u = b ^ f;
v = c ^ g;
w = d ^ h;
S4(t,u,v,w); // uses one more extra register x
a ^= t;    e ^= t;
b ^= u;    f ^= u;
c ^= v;    g ^= v;
d ^= w;    h ^= w;
S4(a,b,c,d); // reuses t as extra
S4(e,f,g,h); // reuses u as extra
\end{minted}
\caption[Snippet for a bitsliced \C implementation of \littlunOne.]{Snippet for a bitsliced \C implementation of \littlunOne, using the code of \autoref{bitsliced4} as subroutine\label{bitsliced8}. The input and output registers
are $a,b,c,d,e,f,g,h$ (with the most significant bit in word $a$), the five extra registers are $t,u,v,w,x$.}
\end{figure}


\subsection{Masking}
The low number of non-linear gates needed to implement \littlunOne makes it a suitable choice for applications where counter-measures against side-channel attacks are required.
Indeed, it directly implies a lower cost when using Boolean masking schemes (both hardware and software), which represent the primitive to be masked as a circuit \cite{isw,DBLP:conf/fse/CarletGPQR12}.
In particular, \littlunOne is competitive with the S-boxes proposed by Grosso \etal \cite{lsdesigns}: it has the same gate count as the S-box used for \robin and only
one more non-linear (and one less XOR) gate than the one used for \fantomas. All three S-boxes are comparable in terms of cryptographic properties.

One could alternatively consider that the chief non-linear component to take into account in that context is actually \littlunS, the 4-bit S-box underlying \littlunOne, rather than the full S-box.
Indeed, any cryptosystem using \littlunOne (in combination with an arbitrary linear layer)
can be re-written as using only \littlunS for its non-linear part. In that respect, the number of non-linear gates to consider for masking would only be 4\footnote{One could object that
additional factors need to be taken into account, such as for instance the total number of application of the S-box in an execution of the cipher. Yet if we jump a little ahead and
consider the block cipher \fly of \autoref{sec:fly}, we can see that in terms of the 4-bit S-box, \fly needs $20\times8\times3 = 480$ calls to \littlunS, which is
comparable to the $31\times16 = 496$ of \present (discounting the key-schedule).}.

The ability to express \littlunOne only in terms of a 4-bit S-box is also convenient when considering threshold implementations (although these chiefly apply to hardware implementations,
which are not the focus of this chapter).
For instance, it allows one to benefit from the recent progresses in such protected implementations of small S-boxes~\cite{ti-4bit}.

We further discuss the cost of masking a concrete block cipher instance using \littlunOne in \autoref{sec:flyimplem}.

\subsection{Inverse S-box}
The inverse $\littlunOne^{-1}$ of \littlunOne is slightly costlier to implement, because of a more expensive inverse for \littlunS. As a circuit, the latter requires 5 XOR gates, 4 non-linear (OR and AND) gates
and one NOT gate (costing 0.67 GE). The total hardware cost of $\littlunOne^{-1}$ is thus 90 GE.

Software bitsliced implementations are also more expensive. We give a snippet for the inverse of \littlunS in
\autoref{bitslicedI4} that requires 11 instructions and 5 registers. The complete inverse can be implemented with 49 instructions and 13 registers in a straightforward adaptation of \autoref{bitsliced8}\footnote{Because
the output registers form a non-trivial permutation of the input ones, additional instructions may also be needed in the cases where this cannot be dealt with implicitly.}.

\begin{figure}[ht]
\begin{minted}[breaklines]{c}
t  = c;    c &= b;    c ^= d; // (A): d ^ (b & c)
d |= t;    d ^= a;            // (B): a ^ (c | d) 
a &= c;    a ^= b;    a ^= d; // (C): b ^ B ^ (a & A) 
b = ~b;    b &= d;    b ^= t; // (D): c ^ (~b & B)
\end{minted}
\caption[Snippet for a bitsliced \C implementation of the inverse of \littlunS.]{Snippet for a bitsliced \C implementation of the inverse of \littlunS \label{bitslicedI4} with inputs in registers $a,b,c,d$ (the word holding the most significant bit is taken to be $a$), using one extra register $t$.
The output is in $c,d,a,b$.}
\end{figure}
