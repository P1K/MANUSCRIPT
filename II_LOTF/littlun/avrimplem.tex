\section{AVR implementation of the \fly round function}
\label{sec:mega}

We give pseudo AVR assembly code for the S-box layer, the permutation
and on-the-fly computation of the key-schedule of \fly.
All the state, key and temporary variables fit in the 32 registers of
an ATtiny or ATmega.

\begin{minted}[breaklines]{scheme}
; input/output in s0,...,s7 (with MSBs in s0)
; temporary values held in t0,...,t4
; top XOR
movw t0, s0 ; moves t1:s0 <- s1:s0
movw t2, s2
eor t0, s4
eor t1, s5
eor t2, s6
eor t3, s7
; middle S-box
mov t4, t1
or  t1, t0
eor t1, t2
and t2, t4
eor t2, t3
and t3, t1
eor t3, t0
or  t0, t2
eor t0, t4
; bottom XOR
eor s0, t0
eor s1, t1
eor s2, t2
eor s3, t3
eor s4, t0
eor s5, t1
eor s6, t2
eor s7, t3
; bottom S-boxes
mov t0, s1
or  s1, s0
eor s1, s2
and s2, t0
eor s2, s3
and s3, s1
eor s3, s0
or  s0, s2
eor s0, t0

mov t0, s5
or  s5, s4
eor s5, s6
and s6, t0
eor s6, s7
and s7, s5
eor s7, s4
or  s4, s6
eor s4, t0
\end{minted}
\begin{figure}[ht]
\caption{The \littlunOne S-box on ATmega, using 41 instructions.\label{fig:avrS}}
\end{figure}

\begin{figure}[ht]
\begin{minted}[breaklines]{scheme}
; input/output in s0,...,s7
rol s1
rol s2
rol s2
swap s3
ror s3
swap s4
swap s5
rol s5
ror s6
ror s6
ror s7
\end{minted}
\caption{The $\rot$ permutation on ATmega/ATtiny, using 11 instructions.\label{fig:avrP}}
\end{figure}

\begin{figure}[ht]
\begin{minted}[breaklines]{scheme}
; key input in k0,...,k15
; cipher state in s0,...,s7
; round constant in c0
; temporary register in t0

; add the current round key & round constant to the state
eor s0, k0
eor s1, k1
eor s2, k2
eor s3, k3
eor s4, k4
eor s5, k5
eor s6, k6
eor s7, k7

eor s0, c0
eor s1, 255

; update k0,...,k7 to the next round key
eor k0, k8
eor k1, k9
eor k2, k10
eor k3, k11
eor k4, k12
eor k5, k13
eor k6, k14
eor k7, k15

; update c0 to the next round constant
mov  t0, c0
andi t0, 1
dec  t0
andi t0, 177
lsr  c0
eor  c0, t0
\end{minted}
\caption{Key addition, and the $\ksOne$ key-schedule on ATmega/ATtiny, using 24 instructions.\label{fig:avrK}}
\end{figure}

\FloatBarrier

Taken together, \autoref{fig:avrS}, \autoref{fig:avrP} and \autoref{fig:avrK} entirely
define the round function of \fly. They thus demonstrate that true to its lightweight nature,
\fly can be implemented in a very compact way. This implementation is also easy to obtain from
the description of the cipher, and no particular optimisation work is necessary to derive it.
