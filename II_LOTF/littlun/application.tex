\section{An application: the \littlunpride block cipher}
\label{sec:fly}

In this section, we present the \fly block cipher as an application of the \littlunOne S-box. It is a 64-bit
block cipher with 128-bit keys.
Thanks to the branch
number of the S-box, it is easy to design a round function with good resistance to statistical attacks by combining
its bitsliced application with a simple bit permutation. This results in a cipher with a structure similar to \present \cite{present}
with a tradeoff: the S-box is bigger (and thus more expensive to implement, in particular in hardware) but the
permutation is simpler (and thus cheaper to implement in software).
This cipher was designed to be used in the same cases as \pride, and its chief implementation target is 8-bit microcontrollers.

\subsection{Specifications}

We first give the specification of the round function $\rfly$ of \fly. It takes a 64-bit block and 64-bit round key as input.
Let $x \defas (x_0||x_1||x_2||x_3||x_4||x_5||x_6||x_7)$, $rk \defas (rk_0||rk_1||rk_2||rk_3||rk_4||rk_5||rk_6|| rk_7)$ be such
an input, with $x_i$, $rk_i$ 8-bit words\footnote{The big endian convention is used to convert from $x$ and $rk$ to the $x_i$s and $rk_i$s.}.

Let us first define $f_i(t) \defas \iota_i(t_0)||\kappa(t_1)||t_2||t_3||t_4||t_5||t_6||t_7$, with $\kappa(x) \defas x \oplus \texttt{0xFF}$;
$\iota_i(x) \defas x \oplus \mathcal{L}(i)$,
$\mathcal{L}(i)$ being a round constant produced as the $i^\text{th}$ iteration of the LFSR shown in
\autoref{rcon_lfsr}, initialized with zero. Algebraically speaking, this LFSR implements the mapping $x \mapsto \alpha(x + \alpha^7)$
in $\mathbf{F}_2[\alpha]/\alpha^8+\alpha^7+\alpha^3+\alpha^2+1$. Note however that the mapping of elements of this field to the state of the
LFSR (or equivalently 8-bit machine words) uses the inverse ordering of the usual convention, \ie{} the highest coefficient is
stored in the LSB. This (as well as the addition of $\alpha^7$ prior to the multiplication by $\alpha$) is done to ease software implementations
of this round constant generation. An example of such an implementation is given in 
\autoref{fig:avrK}.
\begin{figure}[ht]
\begin{minted}[breaklines]{c}
flip = r[0] ^ 1;    r[n] = r[n + 1], n = 6...0;    r[7] = 0;
r[0] = r[0] ^ flip; r[4] = r[4] ^ flip;
r[5] = r[5] ^ flip; r[7] = r[7] ^ flip;
\end{minted}
% // 7
% // 3
% // 2
% // 0
\caption[The round-constant-generating LFSR.]{The LFSR which $i^\text{th}$ iteration (starting from a zero state) defines the $i^\text{th}$ round constant. This pseudo-code
assumes an 8-bit state $r$, which entry $r[0]$ maps to the LSB in a machine representation\label{rcon_lfsr}.}
\end{figure}

We write $\ark_i$ the addition of the $i^\text{th}$ round key: $\ark_i(rk_i, x) \defas f_i(x \oplus rk_i)$;
$\blit(x)$ a bitsliced application of
\littlunOne (such as \eg the one shown in \autoref{bitsliced8}), with $x_0$ holding the most significant bits of the input to the S-boxes;
and $\rot$ the ``\shiftrow'' word-wise rotation (with $\rott$ denoting bitwise rotation to the left)
\[
\rot(x) \defas (x_0||x_1 \rott 1||x_2 \rott 2||x_3 \rott 3||x_4 \rott 4||x_5 \rott 5||x_6 \rott 6||x_7 \rott 7)
\]
which can alternatively be defined at the bit level as the permutation $\perm(i) \defas (i + 8(i \mod 8)) \mod 64$
applied to a suitable binary representation of $x = b_0\ldots b_{63}$. Then we simply have %(omitting the addition of the round constant)
$\rfly(\cdot,\cdot) \defas \rot \circ \blit \circ \ark$.
We give a (slightly simplified) graphical representation of the SPN structure of this round function at the bit level in \autoref{fig:fly_spn}.

\begin{figure}[ht]
\centering
\begin{tikzpicture}[scale=0.7]

	%% Subkey XORs
    \foreach \z in {0,...,63} {
        \node[XOR, scale=0.5] (xor\z) at ($\z*(0.75em, 0)$) {};
        \node (xorr\z) at ($\z*(0.75em, 0)+(0,-16em)$) {};
    }

    %% Nodes positions
    \foreach \z in {0,...,63} {
        \node (i\z) [above = 0.75em of xor\z] {};
        \node (o\z) [below = 2.5em of xor\z] {};
        \node (ii\z) [above = 0.25em of xorr\z] {};
        \draw[thick] (i\z) -- (xor\z);
	}

	%% Permutation layer
	\foreach \z [evaluate=\z as \zz using {int(mod(mod(\z,8)*8+\z,64))}] in {0,...,63} {
		\draw[thick] (xor\z)  -- (o\z.center)  -- (ii\zz.center) -- (xorr\zz);
	}

	%% SBoxes
    \foreach \z in {0,...,7} {
    		\node[draw,thick,minimum width=4.2em,minimum height=2em,fill=white] (p4) at ($\z*(6em,0) + (2.6em,-2em)$) {$\mathcal{S}$};
	}

	\node[left = 0em of xor0] {$rk_i$};
	
\end{tikzpicture}

\caption[The round function of \fly.]{The round function of \fly.\label{fig:fly_spn} Bits are numbered left to right from 0 to 63 (w.r.t. the bit permutation). The addition of the round constant is
omitted.}
\end{figure}

\medskip

%the following functions (borrowed from \pride):
%\begin{align*}
%\alpha_i(x) \defas x + 193\times i \mod 256 && \beta_i(x) \defas x + 165\times i \mod 256\\
%\gamma_i(x) \defas x + \phantom{1}81\times i \mod 256 && \delta_i(x) \defas x + 197\times i \mod 256
%\end{align*}
%to break the simple arithmetic relation between the subkeys that otherwise could lead to potential related-key-like multitarget slide attacks.

We propose two key-schedules, $\ksOne$ and $\ksTwo$, depending on whether resistance to related-key attacks is required (in the case of $\ksTwo$) or not.
In order to distinguish between the two block ciphers, we write \fly for the (default) case where $\ksOne$ is used and $\flyrk$ when $\ksTwo$ is used.
We describe $\ksOne$ first, which in fact performs only an elementary scheduling:
let $k := k_0||k_1$ be the 128-bit master key, and $k_0$ (resp. $k_1$) its first (resp. second) half; then 
the sequence $(rk)$ of round keys of $\ksOne$ is the simple alternation of $k_0$ and $k_0 \oplus k_1$ defined as $rk_i \defas k_0 \oplus i\times k_1$\footnote{By writing $i\times k_1$,
we mean the all-zero key for even values of $i$ and $k_1$ otherwise.}. A round constant is also added in $\ark$ through the function $f_i$ to prevent self-similarity attacks.

\medskip

For \fly to be resistant to related-key attacks, we use the same approach as \noekeon \cite{noekeon} to define $\ksTwo$ as follows. Let us denote by $\fly(0, \cdot)/12$ twelve applications of the round function of \fly
with the all-zero 128-bit key and define $k' \defas k_0'||k_1' = \fly(0, k_0)/12||\linebreak \fly(0, k_1)/12$. Then $\ksTwo$ is defined through $\flyrk$ as $\flyrk(k, \cdot) \defas \fly(k', \cdot)$. 

\medskip

The round function of \fly is applied 20 times, the same as \pride. The entire cipher can thus finally be defined
as $\fly(k,\cdot) \defas \ark(rk_{20},\cdot) \circ \rfly(rk_{19}, \cdot) \circ \ldots \circ \rfly(rk_1, \cdot) \circ \rfly(rk_0,\cdot)$.


\subsubsection*{Design rationale.}
The core of \fly is the \littlunOne S-box, which was designed to have a branch number of three. This allows to achieve a good diffusion when combining the S-box
application with a simple bit permutation. The latter was chosen so that all eight bits at the output of an S-box go to one different S-box each (similarly, all
input bits come from a different S-box). Unlike in \present, this permutation also has cycles of different lengths (discounting fixed points),
namely 2 (on 8 values), 4 (on 16) and 8 (on 32). This might reduce the impact of (linear and differential) trail clustering.
The round constants break the self-similarity and self-symmetry of the round function (through $\iota$)
and the self-symmetry of the S-box (through $\kappa$)\footnote{This symmetry is actually already broken by $\iota$ and (most of the times) by the round keys, but using an
extra constant allows for a simple and clean argument at a negligible cost.}.

The two components of the round function can be efficiently implemented on an 8-bit architecture through a bitsliced
application of the S-box and word rotations respectively (cf. \autoref{sec:flyimplem}).

The two key-schedules were designed according to different possible scenarios. Most applications do not require resistance to related-key attacks and a simple
alternating key-schedule is enough in that case. We chose not to use an FX construction as in \pride as we did not consider the slight gain in efficiency
it offers to be worth the generic security loss it implies. In the spirit of \noekeon, we propose a second key-schedule to offer resistance to related-key attacks
that consists in ``scrambling'' the master key with a permutation of good differential uniformity before it is used as in the first key-schedule.

\subsection{Preliminary cryptanalysis}

We now analyse the security of \fly against various types of attacks. Considering the similarity of the design with \present and the published analysis on this cipher, the
most efficient attacks on \fly are likely to be (variants of) classical statistical (differential and linear) attacks, which we analyse first (in the single-key setting).
We then give an overview of the resistance against other attack techniques. 

\subsubsection{Statistical attacks.}
\label{sec:stats}
We can use the branch number of \littlunOne together with the properties of the bit permutation $\rot$ to easily derive a lower bound on the number of
(differentially and linearly) active S-boxes. Indeed, as the branch number is 3, we are guaranteed to have at least 6 active S-boxes
every four rounds of any non-trivial differential or linear characteristic. This is a consequence of the following proposition:
%\begin{enumerate}
%%\item The first two rounds may have one active S-box each.
%%\item Except for the first and last rounds, a round with one active S-box either follows or is followed by a round with more than one active S-box;
%%\ie no transitions of active S-boxes on three rounds of the form $* \rightarrow 1 \rightarrow 1 \rightarrow 1 \rightarrow *$ are possible.
%\item No three consecutive rounds may have a single active S-box each,
%\ie no transitions of active S-boxes on three rounds of the form $1 \rightarrow 1 \rightarrow 1$ are possible.
%\item No transitions of active S-boxes on three rounds of the form $1 \rightarrow n \rightarrow 1$, $n > 1$ are possible\footnote{This is a consequence
%of the fact that in a round following one with a single active S-boxes, all $n$ active S-boxes are active in a single bit of their input, and consequently
%each of their outputs activates at least 2 S-boxes.}.
%%\item Any four rounds (excluding the first and last rounds) have at least 6 active S-boxes.
%\end{enumerate}
\begin{prop}
%No transitions of active S-boxes on three rounds of \fly of the form $1 \rightarrow n \rightarrow 1$ are possible.
There is no 3-round trail on \fly activating 1, then $n$, then 1 S-box, for any value of $n$. 
\end{prop}
\begin{proof}
In a round following one round with a single active S-box, all $n$ active S-boxes are active in a single bit of their input, and consequently
each of their outputs activates at least 2 S-boxes.
\end{proof}

The block size of \fly being 64 bits, we want any differential characteristic to have a probability $p \approx 2^{-64}$ when averaged over the key and
message space. Similarly, we want any linear characteristic to have an average bias $b \approx 2^{-32}$. From
the \autoref{statprop} of \littlunOne, by multiplying the differential probabilities and applying the piling-up lemma respectively, this means that we want a differential
(respectively linear) characteristic to have \emph{at least} 16 active S-boxes. This happens at the latest after 12 rounds, for which at least 18 S-boxes are guaranteed to be active.
Even by discounting the additional 2 S-boxes and assuming that a distinguisher can be found for this amount of rounds, this gives a very comfortable margin of 8 rounds,
which we estimate to be much beyond the ability of an attacker to convert the distinguisher into, say, key-recovery (in particular, this is twice the number of rounds needed
for full diffusion).
This also leaves some margin to ensure that even in the case where \fly would exhibit a strong differential or linear hull it would be unlikely for an attacker to be able
to mount a meaningful attack. For instance, after 16 rounds, an attacker would need about $2^{32}$ ``optimal'' contributing characteristics to obtain a distinguisher with non-trivial
probability, and would still be facing 4 rounds to mount an attack.

Thus, we conjecture that \fly with 20 rounds offers good resistance to statistical attacks.

\paragraph{Brief comparison with \present.}
The best attacks to date on \present are based on multidimensional linear cryptanalysis~\cite{DBLP:conf/ctrsa/Cho10,DBLP:journals/iacr/BogdanovTV16}. These attacks exploit the presence of linear trails
that constantly activate only one S-box per round (\ie ``single-bit trails''). As there are no such trails in \fly, we believe that these attacks would be less effective
on the latter. Similarly, some other good attacks on \present exploit the fact that half of the bits of some groups of S-boxes remain in the same group~\cite{DBLP:conf/ctrsa/CollardS09},
and there is no such property for \fly.

\subsubsection{Other attacks}

\paragraph{Algebraic attacks.}
We would like to estimate how many rounds of \fly are necessary for the degree of the cipher to reach the maximum of 63, as a lower degree could be exploited in
``algebraic'' attacks. Computing the exact degree of an iterated function is a difficult problem in general, but we should at least compute the upper bound of Boura, Canteaut and De Cannière
to estimate how quickly the degree increases \cite{permdegree}. In our case, this bound states that $\deg(G \circ F) \leq n - (n - \deg(G))/(n_0 - 2)$, where $n$ is the block size and $n_0$ the
size of the S-box. Combining this bound with the fact that the degree of the S-box is 5 (and thus that $\deg(\blit \circ F) \leq 5\cdot\deg(F)$), we can see that 5 rounds of
\fly are necessary to reach a full degree. If we assume this bound to be an equality, any (algebraic) distinguisher on more than about twice this number of rounds is unlikely to exist. 
Even when relaxing this latter assumption, 20 rounds seem to be well enough to make \fly resistant to algebraic attacks.

\paragraph{Meet-in-the-middle attacks.}
We analysed how many rounds are necessary to ensure that every bit of the (intermediate) ciphertext depends on every bit of the key, as a basic way to estimate the resistance
of \fly to meet-in-the-middle (MitM) attacks, which typically exploit the opposite effect. We did this by performing random trials with $2^{20}$ pairs of random
keys and random plaintexts and found that this happens after at most 5 rounds. Any MitM attack on more than about twice this number of rounds is unlikely to exist, and
we therefore conjecture that \fly is resistant to such attacks\footnote{Which key-schedule ($\ksOne$ or $\ksTwo$) is used is irrelevant, as $\ksTwo$ is equivalent to
using $\ksOne$ with a different ``effective'' master key, which a MitM attacker can recover in the exact same way as the ``true'' master key produced by $\ksOne$.}.

\paragraph{Invariant subspace attacks.}
Invariant subspace attacks exploit the propensity of a round function to map inputs from a certain (non-trivial) affine coset to another, when in addition a trivial key-schedule and sparse round constants are used \cite{briceinv}.
As the two latter points appear in \fly, we analysed its round function in order to see if it could also meet the first, critical condition.
%In that respect, we believe that the structure of the linear layer should
%prevent such a property, as it rotates bits by a different amount (and is thus distinct) on each row.
We ran the automated search tool provided by the authors of \cite{briceinv}\footnote{Available at \url{http://invariant-space.gforge.inria.fr/}.}
for about $2^{36}$ iterations without finding any invariant subspace; this shows that with good probability, no such subspace of dimension greater than $64 - 36 = 28$ exists.

\paragraph{Integral attacks, impossible differentials, zero correlation.}
We did not analyse in detail the security of \fly against integral attacks, nor against impossible differentials and zero correlation
attacks. Indeed, none of these techniques seem to be able to attack a significant number of rounds of bit-oriented ciphers such as \present
(see \eg \cite{bitintegral,presentzero,newdivision}, where the obtained distinguishers reach significantly less rounds than the best statistical ones)
or \fly and we do not consider them to be a threat
for our cipher.

\paragraph{Related-key attacks.}
We now study the resistance of \fly against (XOR-induced, differential) related-key attacks.
\fly equipped with the simple key-alternating key-schedule $\ksOne$ offers (nearly) no resistance to related-key attacks. With $\ksTwo$, however, an attacker is unable to control the
differences between two different effective master keys $k_0'||k_1'$ and $\widetilde{k_0'||k_1'}$ with a probability much better than $2^{-2\cdot64}$, as each difference pair $(k_0,\widetilde{k_0})$
and $(k_1,\widetilde{k_1})$ goes through a permutation with maximum expected differential probability not significantly above $2^{-64}$. Furthermore, unlike single-key differential attacks, which introduce differences
on the plaintext, we do not expect an attacker to easily be able to force a change of (related) keys
if their effective master keys fail to verify a difference relation. Thus, even
if a differential on $\ksTwo$ with probability $p$ higher than $2^{-128}$ were found, it would only lead to a related-key attack on a weak-key class 
	(of size $\approx p/2^{-128}$) or to an attack requiring a huge amount of keys.
Putting everything together, we believe $\flyrk$ to be resistant to XOR-induced related-key attacks.

\paragraph{Known- and chosen-key distinguishers, compression function mode.}
We do not claim any resistance of \fly against known-key and chosen-key distinguishers. We do not make any claim about its suitability
to build a cryptographically strong compression function.

\subsection{Implementation}
\label{sec:flyimplem}

%We conclude this discussion on \fly by showing how it can be implemented efficiently on 8-bit AVR microcontrollers, and also discuss the additional cost of masking.

\subsubsection{Microcontrollers implementations.}
The S-box application can take advantage of the bitsliced expression of \littlunOne from \autoref{sec:implem}, which can easily be implemented with instructions available on the cheapest ATtiny chips \cite{ATtiny}. It is even possible to save 2 instructions from
the 43 quoted in \autoref{sec:implem} on higher-end architectures such as the ATmega family \cite{ATmega} by using word-wise 16-bit \texttt{movw} instructions,
resulting in the implementation given in \autoref{fig:avrS} of \autoref{sec:mega}. A straightforward implementation of the inverse S-box application requires 59 instructions ---a significant overhead of 44\,\%.
However, as a lightweight cipher is precisely used in cases where the available resources are limited, we would mostly expect it to be used in a mode of operation that only uses encryption,
such as \eg \textsf{CTR} (for encryption only) or \textsf{CLOC} \cite{cloc} (for authenticated-encryption). Hence we do not believe that a slower inverse is a significant drawback.

Even though the AVR instruction set does not include rotations by an arbitrary constant, the permutation $\rot$ can still
be compactly implemented with only 11 instructions, as shown in \autoref{fig:avrP} of \autoref{sec:mega}.

The entire substitution and permutation layers of \fly can therefore be implemented with only 52 instructions on ATmega (54 on ATtiny), which is 4 less than the 56 of \pride \cite{pride},
while at the same time having at least 1.5 times more \emph{equivalent} active S-boxes every four rounds\footnote{There are at least four active 4-bit S-boxes of maximum differential probability $2^{-2}$ and
best linear correlation $2^{-1}$ every two rounds of \pride and there are at least 3 active 8-bit S-boxes of maximum differential probability $2^{-4}$ and best linear correlation $2^{-2}$ every
two rounds of \fly.}.

On-the-fly computation of one round-key of the key-schedule $\ksOne$ can be done in 8 instructions. The complete key expansion and round constant addition can be done in 24 instructions
as shown in \autoref{fig:avrK}.

The total round function of \fly including the key-schedule can thus be implemented in 76 instructions, which is eight more than \pride. Note however that the conjectured security
margin of \fly is much bigger, and unlike \pride, its resistance to generic attacks does not decrease with the amount of data available to the adversary\footnote{\pride uses an FX construction,
where one half of its 128-bit key is only used for pre- and post-whitening. This
leads to a simple way to merge on-the-fly round-key generation with the round function, but significantly degrades the security of the cipher to generic attacks
from 128 bits to $128 - \log(D)$, with $D$ the amount of data available to an attacker~\cite{desx}, while also leading to more efficient time-memory-data trade-offs \cite{itaitmd}.}. Furthermore, as the
key schedules of \fly and \pride are rather similar, one could consider using the round function of \fly with the key schedule of \pride. This would result in a cipher slightly more efficient
than both, with the same profile as the latter; we do not expect such a swap to introduce any specific weakness.

\subsubsection{Masked implementations.}
We have just considered the good performance of \fly on AVR processors. 
However, on such platforms, discounting the overhead of protection 
against side-channel attacks may be misleading. 
Indeed, these devices are rather prone to leakage, and it might not be 
entirely reasonable to deploy unprotected cryptosystems on them \cite{avrleak}.
Consequently, we consider here the cost of masked implementations of \fly, 
and compare it to \pride and the ``NSA ciphers'' \simon and \speck~\cite{NSAciph} in the same setting.
All the masked implementations have been generated automatically by 
using the compiler of Barthe \etal{} \cite{maskingcomp}.
This compiler takes a \C implementation of a cipher as an input and generates \C code for a corresponding masked implementation. This code is then instrumented to count the number of basic instructions
(\eg{} logical and arithmetic) executed in the encryption of one block.
 
\begin{table}
\begin{center}
\begin{tabular}{l r r r r r}
\toprule
Cipher     & Unmasked & Order 2 & Order 4 & Order 7 & Order 11 \\
\midrule
\fly(8)    & 1909~~      & 10741~~     & 27253~~     &  66421~~    &  145525~~      \\
\midrule
\simonC (8)  & 3400~~      & 14056~~     & 30344~~     &  65336~~    &  131704~~      \\
\midrule
\simonC (32) & 1012~~      & 3926~~     & 8240~~     &  17336~~    &  34364~~      \\
\midrule
\pride (8)  & 1374~~      & 22922~~     & 60550~~     &  150592~~    &  333368~~      \\
\midrule
\speckC (32) & 486~~      & 48198~~     & 132652~~     & 337843~~    & 757983~~      \\
\bottomrule
\end{tabular}
\end{center}
\caption{Count of operations needed to encrypt one block with each cipher, masked at various orders.}
\label{table:maskingbench}
\end{table}

We report the cost (in terms of number of instruction) for the studied ciphers and configurations in \autoref{table:maskingbench}.
The first column gives the name of the cipher, followed by the
basic word-size used in the implementation (for 8-bit microcontrollers, the most relevant value for this number is understandingly 8).
The next columns give the number of instructions (over the same word-size as the cipher) taken to encrypt one block with an unmasked
implementation, and then masked implementations at various orders
(an order-$t$ implementation ensures security in the $t$-probing model \cite{isw}).

From these results, a first important remark we can make is that neither \pride nor \speck seem to be well suited to masked implementations.
This is due to their conjoined use of bitwise operations and integer modular addition
(80 8-bit additions for \pride (in its key-schedule) and 54 32-bit additions for \speckC).
Masking bitwise operations can be done relatively efficiently by using a Boolean sharing scheme
but it is costly to do so with an additive scheme, while the converse holds for modular additions.
In practice, the compiler of Barthe \etal{} uses the algorithm of Coron \etal{} from CHES~2014 to mask
modular additions with a Boolean scheme \cite{DBLP:conf/ches/CoronGV14}\footnote{If we were to restrict ourselves to first-order masking,
it would be possible to use the more efficient algorithm of Coron \etal{} from FSE~2015 \cite{DBLP:conf/fse/CoronGTV15}.}. 

On the contrary, both \fly and \simon are quite efficient to mask with a Boolean scheme.
Note that we implemented \simonC in two ways: one using 8-bit words, suitable for 8-bit microcontrollers
(and thus directly comparable with the intended use of \fly), and one using 32-bit words (which is more straightforward in a way as all instances of \simon on $2n$-bit blocks
can be expressed naturally with $n$-bit arithmetic).

On 8-bit platforms, our unmasked implementation of \fly is more efficient than \simonC.
This advantage is maintained up to a small number of shares, but starting from 8 (\ie{} an implementation secure at order 7),
\simon becomes more efficient.
This behaviour can be explained by the breakdown of the cost for the two ciphers:
although an unmasked \simonC is costlier than \fly overall,
our masked implementation of the former uses only 176 \emph{refresh} and \emph{and} operations, while \fly needs 240.
As the cost of these operations is quadratic in
the number of shares, implementing \simon at high order becomes cheaper than implementing \fly, but the latter starts with
a significant initial advantage.

In conclusion both \fly and \simon are well suited to masked implementations on 8-bit microcontrollers. While the latter is the most efficient of the
two for $7+$-order implementations, \fly is cheaper in the (in our opinion more relevant) case of low-order ones.

We give the code of a masked version of \fly in \autoref{sec:mask}.
