\section{The \littlun S-box construction}
\label{sec:litt}

%We present the design rationale for our S-box and an instantiation as the ``\littlunOne'' S-box. 

\subsection{The Lai-Massey structure}
Our S-box uses the \emph{Lai-Massey structure}, which was proposed in 1991 for the design
of the block cipher \idea \cite{idea}. The structure is similar in its objective to a Feistel or Misty structure,
see \eg{}~\cite{sac15} for definitions,
as it allows to construct $n$-bit functions out of smaller components. It is in particular well-suited
to build efficient 8-bit S-boxes from 4-bit S-boxes all the while amplifying the good cryptographic
properties of the 4-bit S-boxes.
It was already used as such for the design of the second S-box of the \whirlpool hash function
\cite{whirlpool} (an early version of \whirlpool used a randomly-generated S-box) using five
4-bit S-boxes, see \autoref{whirlpoolS},
and for the design of the S-box of the \fox block cipher \cite{fox} which uses a three-round iterated structure.
In our construction, we use only one round of the more classical variant of the structure, with only three S-boxes,
see \autoref{littlunS}, which allows nonetheless to square the differential probability
and the linearity of the underlying 4-bit S-box\footnote{We do not require the 4-bit S-boxes to
be orthomorphisms of any group; hence we in fact only partially adhere to the Lai-Massey structure as later defined
by Vaudenay~\cite{vaudmassey}.}.

The choice of the Lai-Massey structure was mainly motivated by our objective of building
an S-box with a branch number of three, both differential and linear; this will in turn be useful to design a good lightweight round function.
Indeed, it is easy to see that the S-box will
have this property for the differential branch number by construction as soon as the 4-bit S-boxes have differential branch number three,
and such S-boxes are well-known, see \eg \serpent \cite{serpent}. So much cannot be said however for
the linear branch number, as no 4-bit S-box with optimal resistance to differential and linear cryptanalysis exists with this property, as
demonstrated by an exhaustive search we performed on the optimal classes described \eg{} in~\cite{class4bit}. In fact, we are not aware
of previous examples of 8-bit S-boxes with this feature either.

Other good properties of the structure are that it yields S-boxes with a
circuit depth of two S-boxes and it allows for efficient vector implementations using SIMD instructions,
see \autoref{sec:simdimplem}.
On the downside, it requires the 4-bit S-boxes to be permutations if we want the
8-bit S-box to be one. Canteaut, Duval and Leurent recently showed how the absence of such a restriction
for Feistel structures could be used to build compact S-boxes with particularly low differential probability
\cite{sac15}. We should note however that for the applications we have in mind, see \autoref{sec:fly},
the linearity of the
S-box is as important as the differential probability, and the linearity of the S-box of Canteaut \etal is
average, and in particular not better than ours.
Finally, we should mention that the good and bad points of the Lai-Massey structure cited so far
are shared with the Misty structure. Choosing Lai-Massey in our case was mainly due to a matter of
taste, though it is noteworthy that Misty yields S-boxes with a rather sparse algebraic expression,
meaning that the polynomials in the ANF of such S-boxes tend to have many zero coefficients.

\subsection{An instantiation: \littlunOne}

We now define \littlunOne, a concrete instantiation of the Lai-Massey structure which
achieves a differential and linear branch number of three.
Although we have seen that we could guarantee this in the differential case by using a 4-bit
S-box of differential branch number three, this is actually not necessary and we use instead a very compact
member of the \emph{class 13} of Ullrich \etal \cite{skew}. This S-box uses only 4 non-linear
and 4 XOR gates, which is minimal for an optimal S-box of this size. This leads to an
8-bit S-box using 12 non-linear and 24 XOR gates.
We give the table of the 4-bit S-box ``\littlunS'' in \autoref{tab4} and of the complete 8-bit
S-box in \autoref{tab8}, in \autoref{sec:sbox_tables}, and conclude this section by a summary of the cryptographic properties
of \littlunOne.

\begin{figure}[ht]
\centering
\begin{subfigure}[b]{0.45\textwidth}
\centering
\begin{tikzpicture}[scale=1,transform shape,thick]

\hspace{.5mm}
  \tikzstyle{every node}=[transform shape];
  \tikzstyle{every node}=[node distance=1cm];

	% thingies
    \node (hin) {\textsf{Hi}};
	\node (lon) [right of=hin,node distance=3cm]{\textsf{Lo}};
	\node (f-4) [below of=hin,draw,rectangle,very thick,minimum width=1cm,minimum height=0.5cm] {$\sbo_4$};
	\node (f-5) [below of=lon,draw,rectangle,very thick,minimum width=1cm,minimum height=0.5cm] {$\sbo^{-1}_4$};
	\coordinate[right of=f-4,node distance=1.5cm] (mid);
	\node (XOR-1) [XOR,scale=0.8,thick,below of=mid] {};
	\coordinate[left of=XOR-1,node distance=1.5cm] (midleft);
	\coordinate[right of=XOR-1,node distance=1.5cm] (midright);
	\node (f-1) [below of=XOR-1,draw,rectangle,very thick,minimum width=1cm,minimum height=0.5cm] {$\sbo'_4$};
	\node (XOR-2) [XOR,scale=0.8,thick,below of=hin,node distance=4.5cm] {};
	\coordinate[right of=XOR-2,node distance=1.5cm] (midbot);
	\node (XOR-3) [XOR,scale=0.8,thick,below of=lon,node distance=4.5cm] {};
	\node (f-2) [below of=XOR-2,draw,rectangle,very thick,minimum width=1cm,minimum height=0.5cm] {$\sbo_4$};
	\node (f-3) [below of=XOR-3,draw,rectangle,very thick,minimum width=1cm,minimum height=0.5cm] {$\sbo^{-1}_4$};

	% data path
	\draw (hin) -- (f-4);
	\draw (lon) -- (f-5);
	%
	\draw (f-4) -- (XOR-2);
	\draw (midleft) -- (XOR-1);
	\draw (f-5) -- (XOR-3);
	\draw (midright) -- (XOR-1);
	%
	\draw (XOR-1) -- (f-1);
	\draw (f-1) -- (midbot);
	\draw (midbot) -- (XOR-2);
	\draw (midbot) -- (XOR-3);
	%
	\draw (XOR-2) -- (f-2);
	\draw (XOR-3) -- (f-3);

%	\draw (XOR-1) edge (f-1);
%
%    \node (XOR-2)[right of=f-1,XOR,scale=0.8,thick] {};
%	\draw (f-1) edge (XOR-2);
%	\node (f-2) [right of=XOR-2] {};
%	\path[line] (XOR-2) edge (f-2) ;
%	\node [right of=XOR-2,node distance=1.2cm] {$y$};
%
%	%% Subkeys
%	\node (k-0) [above=0.4cm of XOR-1]{\small $k_{0}$};
%	\draw (k-0) edge (XOR-1);
%
%	\node (k-1) [above=0.4cm of XOR-2]{\small $k_{1}$};
%	\draw (k-1) edge (XOR-2);
	
\end{tikzpicture}

\caption{Lai-Massey as in the \whirlpool second S-box\label{whirlpoolS}}
\end{subfigure}
\begin{subfigure}[b]{0.45\textwidth}
\centering
\begin{tikzpicture}[scale=1,transform shape,thick]

\hspace{.5mm}
  \tikzstyle{every node}=[transform shape];
  \tikzstyle{every node}=[node distance=1cm];

	% thingies
    \node (hin) {\textsf{Hi}};
	\node (lon) [right of=hin,node distance=3cm]{\textsf{Lo}};
	\coordinate[right of=hin,node distance=1.5cm] (mid);
	\node (XOR-1) [XOR,scale=0.8,thick,below of=mid] {};
	\coordinate[left of=XOR-1,node distance=1.5cm] (midleft);
	\coordinate[right of=XOR-1,node distance=1.5cm] (midright);
	\node (f-1) [below of=XOR-1,draw,rectangle,very thick,minimum width=1cm,minimum height=0.5cm] {$\sbo_4$};
	\node (XOR-2) [XOR,scale=0.8,thick,below of=hin,node distance=4.5cm] {};
	\coordinate[right of=XOR-2,node distance=1.5cm] (midbot);
	\node (XOR-3) [XOR,scale=0.8,thick,below of=lon,node distance=4.5cm] {};
	\node (f-2) [below of=XOR-2,draw,rectangle,very thick,minimum width=1cm,minimum height=0.5cm] {$\sbo_4$};
	\node (f-3) [below of=XOR-3,draw,rectangle,very thick,minimum width=1cm,minimum height=0.5cm] {$\sbo_4$};

	% data path
	\draw (hin) -- (XOR-2);
	\draw (midleft) -- (XOR-1);
	\draw (lon) -- (XOR-3);
	\draw (midright) -- (XOR-1);
	%
	\draw (XOR-1) -- (f-1);
	\draw (f-1) -- (midbot);
	\draw (midbot) -- (XOR-2);
	\draw (midbot) -- (XOR-3);
	%
	\draw (XOR-2) -- (f-2);
	\draw (XOR-3) -- (f-3);

%	\draw (XOR-1) edge (f-1);
%
%    \node (XOR-2)[right of=f-1,XOR,scale=0.8,thick] {};
%	\draw (f-1) edge (XOR-2);
%	\node (f-2) [right of=XOR-2] {};
%	\path[line] (XOR-2) edge (f-2) ;
%	\node [right of=XOR-2,node distance=1.2cm] {$y$};
%
%	%% Subkeys
%	\node (k-0) [above=0.4cm of XOR-1]{\small $k_{0}$};
%	\draw (k-0) edge (XOR-1);
%
%	\node (k-1) [above=0.4cm of XOR-2]{\small $k_{1}$};
%	\draw (k-1) edge (XOR-2);
	
\end{tikzpicture}

\caption{Lai-Massey as in the \littlun construction\label{littlunS}}
\end{subfigure}
\caption{The Lai-Massey structure}
\end{figure}

\begin{prop}[Statistical properties]
\label{statprop}
The differential uniformity of \littlunOne and of its inverse is 16 and its linearity is 64, as proven by a direct computation.
Its DDT and LAT are shown in \autoref{littDDT} and \autoref{littLAT} respectively.
\end{prop}

In essence, \autoref{statprop} means that the probability (taken over all the inputs) of any non-trivial differential relation going through
the S-box is upper-bounded by $2^{-4}$ and the bias of any non-trivial linear approximation is upper-bounded by $2^{-3}$.

\begin{prop}[Diffusion properties]
The differential and linear branch number of \littlunOne and of its inverse is 3.
\end{prop}

As we already mentioned, several 4-bit S-boxes from the literature have a differential branch number of 3, and it is not hard to construct
8-bit S-boxes with this property from them. This is not the case for the linear branch number, and we find the fact that \littlunOne
has such a property to be quite more remarkable.

\begin{prop}[Algebraic properties]
The maximal degree of the ANF of \littlunOne is 5 in four of the eight output bits,
4 in two other and 3 in the remaining two. The maximal degree of its inverse is 5 in six of the eight output bits
and 4 in the other two.
%The complete ANF of \littlunOne and its inverse are given in Table~\ref{littANF} and
%Table~\ref{littInvANF} respectively.
\end{prop}

%\begin{prop}[Fixed points]
%\littlunOne has two fixed points: 0 and 187.
%\end{prop}


%%%%% POTENTIAL TBR %%%%
%It can be seen from the DDT and LAT of \littlunOne that it is quite structured in a way; it would be very unlikely to obtain
%an S-box with such tables if it were drawn at random uniformly among permutations of $\{0,\ldots,255\}$. This is actually to be
%expected as it would be quite improbable that a random S-box would exhibit so strong a distinguishing property as having a branch number of three.
%However, we do not believe that it is possible to exploit this structure in an attack.
%%%%%
