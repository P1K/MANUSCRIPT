\section{Preliminaries}
\label{sec:elprelim}

We start by giving an overview of the main notions that will be used in evaluating the cryptographic properties of our construction.
A more detailed and rigorous treatment of the following can for instance be found in~\cite{phdRoue}.

\bigskip

Although we will mostly consider S-boxes as defined over binary strings we may see an $n$-bit S-box as a mapping
$\ftwo^n \rightarrow \ftwo^n$ whenever convenient, in particular so that the addition of a difference to an S-box
input is a well-defined operation.

\begin{defi}[Differential uniformity of an S-box]
\label{diffS}
Let $\sbo$ be an $n$-bit S-box. We define its
\emph{difference distribution table} (or DDT) as the function $\Ldiff_{\sbo}$ defined extensively by:
\[
\Ldiff_{\sbo}(a,b) \defas \#\{x \in \ftwo^n | \sbo(x) + \sbo(x + a) = b\}.
\]
The \emph{differential uniformity} $\LdiffU$ of $\sbo$ is defined as:
\[
\max_{(a,b) \neq (0,0)} \Ldiff_{\sbo}(a,b).
\]
\end{defi}
Put another way, an $n$-bit S-box with differential uniformity $\LdiffU$ has a maximal differential probability
of $\LdiffU/2^n$ over its inputs.


\begin{defi}[Linearity of an S-box]
\label{linS}
Let $\sbo$ be an $n$-bit S-box. We define its
\emph{linear approximation table} (or LAT) as the function $\lin_{\sbo}$ defined extensively by:
\[
\lin_{\sbo}(a,b) \defas \sum_{x \in \ftwo^n} (-1)^{\langle b,\sbo(x)\rangle + \langle a, x\rangle}.
\]
The \emph{linearity} $\linU$ of $\sbo$ is defined as:
\[
\max_{(a,b) \neq (0,0)} |\lin_{\sbo}(a,b)|.
\]
\end{defi}
Roughly speaking, the linearity measures the maximum absolute difference between how many times a non-trivial linear approximation
takes the value 1 and how many times it takes the value 0. For an $n$-bit S-box, it is therefore twice the difference between $2^{n-1}$
and how many times either value is taken.
In particular, if we define the \emph{bias} $b$ of a probability $p$ as $p - 1/2$, it means that
the absolute bias of any linear approximation of an $n$-bit S-box of linearity $\linU$ is upper-bounded by $(\linU/2)/2^n$.
We will in fact mostly use the alternative notion of squared \emph{correlation} of linear approximations, 
where the correlation of a given approximation is defined as $\Cor_{\sbo}(a,b) \defas \lin_{\sbo}(a,b)/2^n$, \ie twice the bias.

\begin{defi}[Differential branch number of an S-box]
The \emph{differential branch number} of an S-box $\sbo$ is:
\[
\min_{\{(a,b) \neq (0,0) \,|\, \Ldiff_{\sbo}(a,b) \neq 0\}} \wt(a) + \wt(b),
\]
where $\wt(x)$ is the Hamming weight of $x$.
\end{defi}

\begin{defi}[Linear branch number of an S-box]
The \emph{linear branch number} of an S-box $\sbo$ is:
\[
\min_{\{a\neq 0,b\neq 0\, |\, \lin_{\sbo}(a,b) \neq 0\}} \wt(a) + \wt(b).
\]

\end{defi}

\begin{defi}[Algebraic normal form]
Let $f : \ftwo^n \rightarrow \ftwo$ be an $n$-bit Boolean function, its \emph{algebraic normal form}
(or ANF) is defined as the unique polynomial $g \in \ftwo[x_0,x_1,\ldots x_{n-1}]/\langle\,x_i^2-x_i\,\rangle_{i<n}$
such that for all $x \in \ftwo^n$, $f(x) = g(x[0],\ldots,\allowbreak x[n-1])$.
Similarly, the ANF of an $n$-bit S-box $\sbo$
is the sequence of the ANFs of its $n$ constituent Boolean
functions $\langle\sbo(\cdot),e_i\rangle$, with $(e_i)$ the canonical basis of $\ftwo^n$.
\end{defi}

The previous definitions focused on properties of individual S-boxes; in particular they involved
no key or other kind of parameterisation. As we are eventually interested in the properties of block ciphers,
\ie families of permutations indexed by a key, we may also require definitions that average
some of the above properties in a meaningful way.
We then get the two following definitions, where we naturally
extend the notion of differential uniformity and linearity to any function rather than just
S-boxes.

\begin{defi}[Maximum expected differential probability of a family of functions]
\label{diffC}
The \emph{Maximum expected differential probability} (MEDP) of
$\Fun : \keyspace \times \{0,1\}^n \rightarrow \{0,1\}^n$ is defined as:
\[
\max_{(a,b) \neq (0,0)} ~ \frac{1}{\#\keyspace}~\cdot~ \sum_{k \in \keyspace} \frac{\Ldiff_{\Fun(k,\cdot)}(a,b)}{2^n}.
\]
\end{defi}

\begin{defi}[Maximum expected linear potential of a family of functions]
\label{linC}
The \emph{Maximum expected linear potential} (MELP) of
$\Fun : \keyspace \times \{0,1\}^n \rightarrow \{0,1\}^n$ is defined as:
\[
\max_{(a,b) \neq (0,0)} ~ \frac{1}{\#\keyspace}~\cdot~ \sum_{k \in \keyspace} \Bigg(\frac{\lin_{\Fun(k,\cdot)}(a,b)}{2^n}\Bigg)^2.
\]
\end{defi}

The MEDP and MELP of (reduced versions of) a block cipher are good indicators of its resistance against differential and linear cryptanalysis.
Unfortunately, it is usually a hard problem to compute the MEDP and MELP for a non-trivial number of rounds of a block cipher.
However, in the case of SPN ciphers,
it is generally easier to derive upper bounds on the probability and potential of given differential and linear
\emph{trails} (or \emph{characteristics}), for which the intermediate differences and linear masks are specified at every round. For some
SPN ciphers, it may even be possible to provide bounds for \emph{any} trail on a given number of rounds. These notions are then often
used instead of the MEDP and MELP to argue about the security of the cipher.

\medskip

In the following, we consider an iterative $n$-bit block cipher $\E$ with round function $\R$. If we write $\R_k$ the
round function with pre-addition of a subkey $k$, \ie $\R_k(x) = \R(x \oplus k)$,
then the $r$-round $\E$ is defined as $\E(K,x) = (\R_{k_{r-1}} \circ \cdots \circ \R_{k_0}(x)) \oplus k_r$, where the tuple
$(k_0,\ldots,k_r)$ is the image of $K$ by some key expansion mapping. We then have:

\begin{defi}[Differential trail]
An $r$-round \emph{differential trail} for a block cipher $\E$ is an $(r+1)$-tuple $(a_0,\ldots,a_r)$ of $n$-bit \emph{differences}.
An input $x$ to $\E$ is said to \emph{follow} the trail for the subkeys $(k_0,\ldots,k_r)$
if $\forall\, i < r,\,\R_{k_i}(x) \oplus \R_{k_i}(x \oplus a_i) = a_{i+1}$.

\noindent
When the key (equivalently the tuple of subkeys) of $\E$ is fixed, we call \emph{differential trail probability} (DTP) of a trail the probability
that it is followed by an input to $\E$.
\end{defi}

\begin{defi}[Linear trail]
\label{linT}
An $r$-round \emph{linear trail} for $\E$ is an $(r+1)$-tuple $(a_0,\ldots,a_r)$ of $n$-bit \emph{linear selection masks}.

\noindent
For a fixed key of $\E$, the \emph{linear trail correlation} (LTC) of a trail is given by:
\[
\prod_{i=0}^{r-1} \Cor_{\R_{k_i}}(a_i, a_{i+1}).
\]
\end{defi}

In the same spirit as \autoref{diffC} and \autoref{linC}, we will usually be interested in the expected values of the DTP
and the squared LTC over the choice of the key.
%We then call EDTP and ELTP the \emph{expected differential trail probability}
%and \emph{expected linear trail potential} respectively.

\medskip

The above definitions are in fact not completely specific to SPN block ciphers. However, for such ciphers, one can additionally
define the notion of \emph{active} S-box in a trail, which we define informally as follows:

\begin{defi}[Active S-box in a trail]
An $i^\text{th}$-round S-box is \emph{active} in an $r$-round differential (resp. linear) trail $(a_0,\ldots,a_r)$
if it has a non-zero input difference as per $a_i$ (resp. if some of its input bits are selected
by the mask $a_i$).
\end{defi}

\noindent
For example, taking the block and S-box size to be respectively 32 and 4 and using hexadecimal notation, all S-boxes are active in the
$2$-round trails $(\texttt{0xFFFFFFFF}, \texttt{0xFFFF}\linebreak{}\texttt{FFFF}, \texttt{0xFFFFFFFF})$ and
$(\texttt{0x12345678}, \texttt{0x9ABCDEFE}, \texttt{0xDCBA9876})$, and only one S-box is active at every round of the trail $(\texttt{0x10000000},
\texttt{0x30000000}, \texttt{0x00000004})$.

We can define the \emph{weight} of a trail $(a_n)$ as the number of S-boxes active for $(a_n)$. Note that whether
an S-box is active or not in a trail solely depends on the definition of the latter and not, say, on subkey values.
Consequently, even though the probability of the differential trails of a block cipher (say) may in fact depend on the key, the possible weights of
trails of non-zero probability are only a consequence of the round function.

If we allow ourselves to make some independence hypotheses, assuming that consecutive round functions
behave independently of each other, then the number of active S-boxes in a differential trail can be used to upper-bound its expected probability.
This is expressed informally as the following:

\begin{assu}
\label{assu:mark}
Let $(a_n)$ be a differential trail of weight $w$ for the SPN block cipher $\E$ which uses a unique $b$-bit S-box $\sbo$ of differential uniformity
$\LdiffU$. Call $p_{\sbo} \defas \LdiffU/2^b$ the maximum differential probability of $\sbo$.
Then the expected probability over the choice of $K$ that an input to $\E(K,\cdot)$ follows the trail $(a_n)$ is upper-bounded by $(p_{\sbo})^w$.
\end{assu}

In other words, \autoref{assu:mark} is a consequence of the assumption that the expected probability of an $r$-round
differential trail is equal to the
product of the expected probabilities of the one-round trails it is made of, \ie the so-called Markov assumption~\cite{idea}.

The case of linear trails is treated somewhat similarly. In particular, from \autoref{linT}, the correlation of
a trail for a fixed key is easy to compute from the correlation of the one-round trails it is made of. Hence, we get:

\begin{prop}
\label{prop:lina}
Let $(a_n)$ be a linear trail of weight $w$ for the SPN block cipher $\E$ which uses a unique $b$-bit S-box $\sbo$ of linearity
$\linU$. Call $\Cor_{\sbo} \defas \linU/2^b$ the maximal absolute correlation of a linear approximation for $\sbo$.
Then, fixing the key $K$, the absolute correlation of the trail $(a_n)$ for $\E(K,\cdot)$ is upper-bounded by $(\Cor_{\sbo})^w$
and thus the squared correlation by $(\Cor_{\sbo}^2)^w$.
\end{prop}

Depending on the properties of the S-box and on the block size of a cipher, one can use the upper bounds
from \autoref{assu:mark} and \autoref{prop:lina} to set an objective for the minimum number of active S-boxes in any trail.

For resistance against differential cryptanalysis, one typically requires trails to have at least $\mathit{wd}$ differentially active
S-boxes so that $(p_{\sbo})^\mathit{wd} < D2^{-n}$, with $n$ the block size of the cipher under consideration and
$D$ a number. This comes from
the fact that differentials of probability $2^{-n+1}$ are bound to exist, and one wishes both that no single trail has a
higher expected probability and more importantly that no set of more than $D$ trails with a minimum number of active
S-boxes and equal starting and ending differences can be found.

In the case of linear cryptanalysis, one similarly typically requires at least $\mathit{wl}$ linearly active
S-boxes in any trail so that $(\Cor_{\sbo}^2)^\mathit{wl} < L2^{-n}$. This comes from the fact that the data complexity required to exploit
an approximation of correlation $C$ grows with the square of $C$, and that we do not want the net number of trails with
equal starting and ending masks, activating
the minimum number of S-boxes, and interfering constructively,
to be likely more than $L$ for some fixed keys.

The entire approach based on counting the number of active S-boxes becomes particularly useful
if one is able to use the structure of the round function of a block cipher to lower-bound the
number of differentially and linearly active S-boxes for any trail of a given number of round. If such a bound can be computed, the objective on the number
of S-boxes then becomes one on the number of rounds.
