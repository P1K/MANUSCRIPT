\section{C masked implementation of \fly}
\label{sec:mask}

We give here the code of a masked version of the \littlunOne S-box and of the \fly block cipher in \autoref{ms} and \autoref{mf} respectively.
These implementations use masked versions of basic logical operations defined in \autoref{mp}.
This further shows that \fly can easily be implemented and that its structure does not need to be changed if masking is added.

\begin{minted}[breaklines]{c}
typedef uint8_t bint8_t[NUM_SHARES];

#define NOT(x)      (~x)
#define XOR(x,y)    (x ^ y)
#define AND(x,y)    (x & y)

/* Masked logical NOT */
void bint8_not(bint8_t r, bint8_t a) {
  r[0] = NOT(a[0]);
  // This can be removed if we know that a and r are equal
  memcpy(r+1, a+1, NUM_SHARES - 1);
  return;
}

/* Masked exclusive OR */
void bint8_xor(bint8_t r, bint8_t a, bint8_t b) {
  int i;
  for (i = 0; i < NUM_SHARES; i++) r[i] = XOR(a[i], b[i]);
  return;
}

/* Masked exclusive OR with a public value */
void bint8_xor_pub(bint8_t r, bint8_t a, uint8_t b) {
  r[0] = XOR(a[0], b);
  memcpy(r+1, a+1, NUM_SHARES - 1);
  return;
}

/* Masked logical AND */
void bint8_and(bint8_t rc, bint8_t a,bint8_t b) {
  int i, j;
  uint8_t zij, zji, tmp;
  bint8_t r;

  // Diagonal
  for (i = 0; i < NUM_SHARES; i++) {
    r[i] = AND(a[i],b[i]);
  }
  // Triangles + Row-Wise Sums
  for (i = 0; i < NUM_SHARES; i++) {
    for (j = i + 1; j < NUM_SHARES; j++) {
      zij  = uint8_rand();
      zji  = AND(a[i],b[j]);
      zji  = XOR(zij,zji);
      tmp  = AND(a[j],b[i]);
      zji  = XOR(zji,tmp);
      r[i] = XOR(r[i],zij);
      r[j] = XOR(r[j],zji);
    }
  }

  memcpy(rc,r,NUM_SHARES);

  return;
}

/* Masked rotation */
void bint8_rotl(bint8_t r, bint8_t a, uint8_t b) {
  int i;
  for (i = 0; i < NUM_SHARES; i++) r[i] = ROTL8(a[i], b);
  return;
}

/* Masked logical OR as a series of NOTs and ANDs */
void bint8_or (bint8_t r, bint8_t x, bint8_t y) {
  bint8_t aux, x0, y0;

  bint8_not(x0, x);
  bint8_not(y0, y);
  bint8_and(aux, x0, y0);
  bint8_not(r, aux);
  return;
}

/* The mask-refreshing function  */
void bint8_refresh(bint8_t r, bint8_t a) {
  int i, j;
  uint8_t aux;

  memcpy(r,a,NUM_SHARES);

  for (i = 0; i < NUM_SHARES; i++) {
    for (j = i + 1; j < NUM_SHARES; j++) {
      aux = uint8_rand();
      r[i] = XOR(r[i], aux);
      r[j] = XOR(r[j], aux);
    }
  }

  return;
}
\end{minted}
\begin{figure}[!hbt]
\caption{Masked implementations of logical operations\label{mp}.} 
\end{figure}

The masked implementations of logical operations defined in the above code can be used
to implement masked version of any logical circuit. It can be directly observed
from these functions that the cost of a masked XOR is linear in the number of
shares \mintinline{c}{NUM_SHARES}, as can be seen in \mintinline{c}{bint8_xor},
while it is quadratic for the non-linear AND function \mintinline{c}{bint8_and}.

The masked implementation of the \littlun S-box shown in the next \autoref{ms} follows in a
straightforward way the standard bitsliced implementation of \autoref{bitsliced8}. The main
difference is that the masked versions of \autoref{mp} are used to implement the logical
operations and that regular calls to \mintinline{c}{bint8_refresh} need to be
made to refresh the random masks. 

\begin{minted}[breaklines]{c}
void s8 (bint8_t x[8]) {
  bint8_t t, a, b, c, d, aux;

  bint8_xor(a, x[0], x[4]);
  bint8_xor(b, x[1], x[5]);
  bint8_xor(c, x[2], x[6]);
  bint8_xor(d, x[3], x[7]);
  bint8_copy(t, b);
  bint8_refresh(aux, a);
  bint8_or(b, b, aux);
  bint8_xor(b, b, c);
  bint8_refresh(aux, t);
  bint8_and(c, c, aux);
  bint8_xor(c, c, d);
  bint8_refresh(aux, b);
  bint8_and(d, d, aux);
  bint8_xor(d, d, a);
  bint8_refresh(aux, c);
  bint8_or(a, a, aux);
  bint8_xor(a, a, t);
  bint8_xor(x[0], x[0], a);
  bint8_xor(x[1], x[1], b);
  bint8_xor(x[2], x[2], c);
  bint8_xor(x[3], x[3], d);
  bint8_xor(x[4], x[4], a);
  bint8_xor(x[5], x[5], b);
  bint8_xor(x[6], x[6], c);
  bint8_xor(x[7], x[7], d);
  bint8_copy(t, x[1]);
  bint8_refresh(aux, x[0]);
  bint8_or(x[1], x[1], aux);
  bint8_xor(x[1], x[1], x[2]);
  bint8_refresh(aux, t);
  bint8_and(x[2], x[2], aux);
  bint8_xor(x[2], x[2], x[3]);
  bint8_refresh(aux, x[1]);
  bint8_and(x[3], x[3], aux);
  bint8_xor(x[3], x[3], x[0]);
  bint8_refresh(aux, x[2]);
  bint8_or(x[0], x[0], aux);
  bint8_xor(x[0], x[0], t);
  bint8_copy(t, x[5]);
  bint8_refresh(aux, x[4]);
  bint8_or(x[5], x[5], aux);
  bint8_xor(x[5], x[5], x[6]);
  bint8_refresh(aux, t);
  bint8_and(x[6], x[6], aux);
  bint8_xor(x[6], x[6], x[7]);
  bint8_refresh(aux, x[5]);
  bint8_and(x[7], x[7], aux);
  bint8_xor(x[7], x[7], x[4]);
  bint8_refresh(aux, x[6]);
  bint8_or(x[4], x[4], aux);
  bint8_xor(x[4], x[4], t);
  return;
}
\end{minted}

\vspace{-1mm}
\begin{figure}[ht]
\caption{\C masked implementation of the \littlunOne S-box\label{ms}.} 
\end{figure}

\newpage

The masked implementation of \fly given in \autoref{mf} is a simple implementation of the
cipher whose only notable fact is that it uses a masked implementation for the S-box.

\begin{minted}[breaklines]{c}
void fly (bint8_t x[8], bint8_t k[16]) {
  uint8_t rcon, t, i, j, l;

  rcon = 0;
  for(i = 0; i < 20; i++) {
    for(j = 0; j < 8; j++) {
      bint8_xor(x[j], x[j], k[j]);
    }
    bint8_xor_pub(x[0], x[0], rcon);
    bint8_xor_pub(x[1], x[1], 0xFF);
    s8(x);
    for(l = 1; l < 8; l++) {
      bint8_rotl(x[l], x[l], l);
    }
    t = rcon & 1;
    t = t - 1;
    t = t & 177;
    rcon = rcon >> 1;
    rcon = rcon ^ t;
    for(j = 0; j < 8; j++) {
      bint8_xor(k[j], k[j], k[8 + j]);
    }
  }
  for(j = 0; j < 8; j++) {
    bint8_xor(x[j], x[j], k[j]);
  }
  bint8_xor_pub(x[0], x[0], rcon);
  bint8_xor_pub(x[1], x[1], 0xFF);
  return;
}
\end{minted}
\begin{figure}[ht]
\caption{\C masked implementation of \fly.\label{mf}}
\end{figure}
