\section{Introduction}
Since the late 1990's and the end of the \aes competition, the academic community and the industry have been provided with excellent block ciphers.
In most cases where a cipher is needed, \aes \cite{rijndael} can readily be used and there is currently little need for a replacement.
Consequently, the symmetric cryptographic community shifted focus to \eg the wider picture of \emph{authenticated encryption} through
the \textsc{Caesar} competition, or to more specific applications of block ciphers. In the latter case, an important topic is
the design of ``lightweight'' block ciphers intended to be implemented on low-cost, resource-constraint devices. An early
successful example following this trend is the block cipher \present \cite{present}, which can be implemented in small hardware circuits.
Most lightweight algorithms similarly target a few platforms on which they are expected to perform particularly well;
good performance in other cases are however not usually expected and lightweight ciphers are in general not very versatile.
Typical platforms of interest include hardware circuits and 8-bit to 32-bit microcontrollers.
Recently, NIST released a draft report on lightweight cryptography with the objective of creating a portfolio of lightweight primitives
through an open process~\cite{NistLightDraft}. The topic of lightweight cryptography is thus ever more timely. 

In this work, we design a conceptually simple block cipher targeting efficient \emph{light} implementations on 8-bit microcontrollers, where
by \emph{light} implementations,
we mean in particular that the size of the code is small, typically of the order of 200 bytes. If more resources are available, the best current block cipher is
probably the \aes, see \eg \cite{DBLP:journals/iacr/BeaulieuSSTWW15}.
%, which is also expected to achieve reasonable performance in hardware.
The chief academic proposal to date for this scenario is the \pride block cipher,
that was presented at CRYPTO~2014~\cite{pride};
notable
``non-academic'' ciphers for the same scenario are the ``NSA ciphers'' \simon and \speck \cite{NSAciph}.
Our block cipher is built around \littlunOne, a compact 8-bit S-box with branch number 3. This allows
to define a round function similar to a scaled-up variant of \present, composing the S-box application with a simple bit permutation. As a bit permutation
obviously does not increase the number of active bits, an important part of the diffusion in such a cipher is played by the S-box. The typical measure of the quality
of the diffusion of an S-box is its ``branch number'' which plays a role similar to the minimum distance of the linear codes used in \aes-like designs.
We thus get a trade-off between hardware and light software implementations: \littlunOne is more expensive in hardware than two applications
of the S-box of \present, but the bit permutation is simple to implement with 8-bit rotations. Owing to Golding, we name our
block cipher ``\fly''.

Excluding on-the-fly key expansion, the round function of \fly costs four instructions less to implement than \pride's on AVR. Using the
good branch number of \littlunOne, we can show that with a similar number of rounds, \fly is more resistant than \pride
to statistical attacks. This is all the more relevant as the security margin of \pride seems to be quite
thin~\cite{prideattack}. Taking the key-schedule into account, one round of \fly costs eight more instructions
than one round of \pride. However, unlike \pride, we do not use an FX construction for the key-schedule and thus
the generic security of \fly does not decrease with the amount of data available
to the adversary. Dinur also showed how the FX construction can lead to more efficient time-memory-data trade-offs~\cite{itaitmd}.

As implementations on resource-constraint devices are more likely to be vulnerable to side-channel attacks, one should also consider
the additional cost of protection against, say, differential power analysis when evaluating schemes that target such platforms.
In that respect, the small number of gates
necessary to implement the \littlunOne S-box, as well as its simple expression in terms of light 4-bit S-boxes allows one to produce
masked implementations of \fly with limited overhead.

\paragraph{Related work.} The block cipher literature is so numerous that most new proposal will bear some similarity with past
designs. In that respect, apart from \present, \fly is quite similar to \rectangle \cite{rectangle}, which also combines a \serpent-like
bitsliced application of an S-box \cite{serpent} with a rotation-implemented bit permutation. However, the S-box in \rectangle is on 4 bits, it does
not have a branch number of 3 and the rotations are on 16-bit words. The construction of the \littlun S-box uses
the Lai-Massey structure from the \idea block cipher~\cite{idea}; this structure was already used to build the second S-box
of the \whirlpool hash function \cite{whirlpool} and the S-box of the block cipher \fox \cite{fox}.
