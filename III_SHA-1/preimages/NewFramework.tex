\section{Higher-Order Differential Meet-in-The-Middle}
\label{NewFramework}

We now describe how to modify the framework of \autoref{KKFramework} to use higher-order differentials.
Let us denote by 
$(\{\alpha_1,\alpha_2\}, \{\beta_1,\beta_2\}) \overset{\Fs}{\underset{p}{\longrightarrow}} \gamma$
the fact that
  $\underset{(x,y)}{\Pr}\big[\Fs(x \oplus \alpha_1 \oplus \alpha_2, y \oplus \beta_1 \oplus \beta_2) \oplus
  \Fs(x \oplus \alpha_1, y \oplus \beta_1) \oplus  \Fs(x \oplus \alpha_2 , y \oplus \beta_2) =
  \Fs(x,y )\oplus \gamma \big] = p$
, meaning
that $(\{\alpha_1,\alpha_2\}, \{\beta_1,\beta_2\})$ is a related-key order-2 differential for $\Fs$ that holds with probability $p$.

Similarly as in \autoref{KKFramework}, the goal of the attacker is to find four linear subspaces
$D_{1,1}, D_{1,2}, D_{2,1}, D_{2,2}$ of $\{0,1\}^m$ in direct sum such that:
\begin{equation}
\label{eq:Dir_sum}
  D_{1,1} \bigoplus D_{1,2} \bigoplus D_{2,1} \bigoplus D_{2,2} 
\end{equation}
\vspace{-4mm}
\begin{equation}
\forall \delta_{1,1} , \delta_{1,2} \in D_{1,1} \times D_{1,2}~\exists~ \Delta_1 \in \{0,1\}^v \text{ s.t. }
(\{\delta_{1,1}, \delta_{1,2}\}, \{0,0\})
\overset{\Fuu}{\underset{1}{\longrightarrow}} \Delta_1
\label{Differential_def_1}
\end{equation}
\vspace{-4mm}
\begin{equation}
\forall \delta_{2,1} , \delta_{2,2} \in D_{2,1} \times D_{2,2}~\exists~ \Delta_2 \in \{0,1\}^v \text{ s.t. }
(\{\delta_{2,1}, \delta_{2,2}\}, \{0,0\})
\overset{\Fd^{-1}}{\underset{1}{\longrightarrow}} \Delta_2.
\label{Differential_def_2}
\end{equation}
Then $M \oplus \delta_{1,1} \oplus \delta_{1,2} \oplus \delta_{2,1} \oplus \delta_{2,2}$ 
is a \emph{preimage} of $c+p$ if and only if $
    \Fuu(\mu \oplus \delta_{1,1} \oplus \delta_{1,2} \oplus 
    \delta_{2,1} \oplus \delta_{2,2}, c) = 
    \Fd^{-1}(\mu \oplus \delta_{1,1} \oplus \delta_{1,2} \oplus 
    \delta_{2,1} \oplus \delta_{2,2}, p) $
which is equivalent by the \autoref{Differential_def_1} 
and \autoref{Differential_def_2} to the equality: 
\begin{equation}
  \label{eq:Hmitm_eq}
  \begin{split}
  \Fuu(\mu \oplus \delta_{1,1} \oplus \delta_{2,1} \oplus \delta_{2,2}, p) & 
    \oplus \quad \qquad \quad 
    \Fd^{-1}(\mu \oplus \delta_{2,1},\oplus \delta_{1,1} \oplus \delta_{1,2}, c) \oplus \\
    \Fuu(\mu \oplus \delta_{1,2} \oplus \delta_{2,1} \oplus \delta_{2,2}, p) & 
    \oplus\quad  = \quad \quad   
    \Fd^{-1}(\mu \oplus \delta_{2,2}, \oplus \delta_{1,1} \oplus \delta_{1,2}, c) \oplus \\ 
    \Fuu(\mu \oplus \delta_{2,1} \oplus \delta_{2,2}, p) \oplus \Delta_1 & 
    \qquad \quad \quad \quad   
    \Fd^{-1}(\mu \oplus \delta_{1,1}  \oplus \delta_{1,2}, c) \oplus \Delta_2.
    \end{split}
\end{equation}

We denote by $d_{i,j}$ the dimension of the 
sub-space $D_{i,j}$ for $i,j=1,2$. Then for a set $M$ of messages $\mu \in \{0,1\}^m)$ one can define $\# M$ affine sub-sets 
$\mu_i \oplus D_{1,1} \oplus D_{1,2} \oplus D_{2,1} \oplus D_{2,2}$ of dimension $d_{1,1} + d_{1,2} + d_{2,1} +d_{2,2}$
(since the sub-spaces $D_{i,j}$ are in direct sum by hypothesis), which can be tested for a preimage using \autoref{eq:Hmitm_eq}.
This can be done efficiently by a modification of \autoref{affine_testing} into the following \autoref{affine_testing_II}.

\begin{algorithm}[ht]
\LinesNumbered 
\KwIn{
	$D_{1,1}, D_{1,2},D_{2,1},D_{2,2} \subset \{0,1\}^m$, $\mu \in \{0,1\}^m$, $p$, $c$
}
\KwOut{
  A preimage of $c + p$ if there is one in $\mu \oplus D_{1,1} \oplus D_{1,2}, \oplus D_{2,1} \oplus D_{2,2} $, $\bot$ otherwise
}
\KwData{ Six lists:\\
$L_{1,1}$ indexed by $\delta_{1,2} , \delta_{2,1} , \delta_{2,2}$\\
$L_{1,2}$ indexed by $\delta_{1,1},  \delta_{2,1},  \delta_{2,2}$\\
$L_{2,1}$ indexed by $\delta_{1,1},  \delta_{1,2},  \delta_{2,2}$\\
$L_{2,2}$ indexed by $\delta_{1,1} , \delta_{1,2},  \delta_{2,1}$\\
$L_{1}$ indexed by $\delta_{2,2},  \delta_{2,1}$\\
$L_{2}$ indexed by $\delta_{1,1} , \delta_{1,2}$
}
        \ForAll{
          $ \delta_{1,2}, \delta_{2,1}, \delta_{2,2} \in
            D_{1,2} \times D_{2,1} \times D_{2,2} $
        }
        {
          $ L_{1,1}[\delta_{1,2}, \delta_{2,1}, \delta_{2,2} ] \leftarrow 
          \Fd^{-1}(\mu \oplus \delta_{1,2} \oplus \delta_{2,1} \oplus \delta_{2,2}, c)$ \;
        }

        \ForAll{
          $ \delta_{1,1}, \delta_{2,1}, \delta_{2,2}\in 
          D_{1,1} \times D_{2,1} \times D_{2,2}$
        }
        {
          $ L_{1,2}[\delta_{1,1}, \delta_{2,1}, \delta_{2,2}] \leftarrow 
          \Fd^{-1}(\mu \oplus \delta_{1,1} \oplus \delta_{2,1} \oplus \delta_{2,2}, c)$ \;
        }

        \ForAll{
          $ \delta_{1,1}, \delta_{1,2}, \delta_{2,2}\in 
          D_{1,1} \times D_{1,2} \times D_{2,2} \times$
        }
        {
          $ L_{2,1}[\delta_{1,1}, \delta_{1,2}, \delta_{2,2}] \leftarrow 
          \Fuu(\mu \oplus \delta_{1,1} \oplus \delta_{1,2} \oplus \delta_{2,2},p)$ \;
        }

        \ForAll{
          $ \delta_{1,1}, \delta_{1,2}, \delta_{2,1} \in 
          D_{1,1} \times D_{1,2} \times D_{2,1}$
        }
        {
          $ L_{2,2}[\delta_{1,1}, \delta_{1,2}, \delta_{2,1}] \leftarrow
         \Fuu(\mu \oplus \delta_{1,1} \oplus \delta_{1,2} \oplus \delta_{2,1}, p)$ \;
        }

        \ForAll{
          $ \delta_{1,1}, \delta_{1,2} \in D_{1,1} \times D_{1,2}$
        }
        {
          $ L_{2}[\delta_{1,1}, \delta_{1,2}] \leftarrow
          \Fd^{-1}(\mu \oplus \delta_{1,1} \oplus \delta_{1,2}, c) \oplus \Delta_1$ \;
        }

        \ForAll{
          $ \delta_{2,1}, \delta_{2,2}\in D_{2,1} \times D_{2,2} $
        }
        {
          $ L_{1}[\delta_{2,1}, \delta_{2,2}] \leftarrow
          \Fuu(\mu \oplus \delta_{2,1} \oplus \delta_{2,2}, 
          p  ) \oplus \Delta_2$ \;
        }

        \ForAll{
          $ \delta_{1,1},\delta_{1,2}, \delta_{2,1}, \delta_{2,2} \in 
            D_{1,1} \times D_{1,2} \times D_{2,1} \times D_{2,2} $
        }
        { 
          \If {$L_{1,1}[\delta_{1,2}, \delta_{2,1}, \delta_{2,2} ] \oplus 
            L_{1,2}[\delta_{1,1}, \delta_{2,1}, \delta_{2,2} ] \oplus 
            L_1[\delta_{2,1}, \delta_{2,2}] = 
            L_{2,1}[\delta_{1,1}, \delta_{1,2}, \delta_{2,2} ] \oplus 
            L_{2,2}[\delta_{1,1}, \delta_{1,2}, \delta_{2,1} ] \oplus 
          L_2[\delta_{1,1}, \delta_{1,2}] $ 
        }
          {
            \Return $\mu  \oplus \delta_{1,1} \oplus \delta_{1,2} 
                        \oplus \delta_{2,1} \oplus \delta_{2,2}$ 
          }
        }
        \Return {$\bot$}
        \caption{\label{affine_testing_II} Testing $\mu \oplus D_{1,1} \oplus D_{1,2} \oplus D_{2,1} \oplus D_{2,2}$ for a preimage}

      \end{algorithm}

    \subsection{Analysis of \autoref{affine_testing_II}.}
      If we denote by $\Gamma_1$ and $\Gamma_2$ the cost of evaluating
      of $\Fuu$ and $\Fd^{-1}$ and $\Gamma_{match}$ the cost of the test on line~14,
      then the algorithm allows to test
      $2^{d_{1,1}+d_{1,2}+d_{2,1}+d_{2,2}}$ messages with a complexity of 
          $2^{d_{1,2}+d_{2,1}+d_{2,2}} \Gamma_2 + 2^{d_{1,1}+d_{2,1}+d_{2,2}} \Gamma_2 +
          2^{d_{1,1}+d_{1,2}+d_{2,1}} \Gamma_1 +2^{d_{1,1}+d_{1,2}+d_{2,1}} \Gamma_1 +
          2^{d_{1,1}+d_{1,2}} \Gamma_2 + 2^{d_{2,1}+d_{2,2}} \Gamma_1 + \Gamma_{\text{match}}$. 
      The algorithm must then be run $2^{n - (d_{1,1}+d_{1,2}+d_{2,1}+d_{2,2})}$ times in order to test
      $2^n$ messages. 
      In the special case where all the linear spaces have the same 
      dimension $d$ and if we consider that $\Gamma_{\text{match}}$ is negligible
      with respect to the total complexity, the total complexity of an attack is
      then of : 
      $2^{n-4d}\cdot(2^{3d}\cdot(2\Gamma_1+2\Gamma_2)+2^{2d}\cdot(\Gamma_1+\Gamma_2)) = 
      2^{n-d+1} \Gamma + 2^{n-2d} \Gamma = \bigo(2^{n-d})$ where $\Gamma$ is the cost of the 
      evaluation of the total compression function $\Fs$. We think that the assumption on
      the cost of $\Gamma_\text{match}$ is
      reasonable given the small size of $d$ in actual attacks and the fact that performing
      a single match is much faster than computing $\Fs$.

      The factor that is gained from a brute-force search of complexity $\bigo(2^n)$ is hence of
      $2^d$, which is the same as for \autoref{affine_testing}. However, one now needs
      four spaces of differences of size $2^d$ instead of only two,
      which might look like a setback. Indeed the real
      interest of this method does not lie in a simpler attack but in the fact that
      using higher-order differentials may now allow to attack functions for which no
      good-enough order-1 differentials are available.


    \subsection{Using probabilistic truncated differentials.}
    Similarly as in  \autoref{KKFramework}, \autoref{affine_testing_II} can be modified in order
    to use probabilistic truncated differentials instead of probability-1 differentials
    on the full state. The changes to the algorithm and the complexity evaluation are
    identical to the ones described in \autoref{probtrunc}.
