\section{Applications to \shaone}
\label{sec:sha1}

%\subsection{Description of SHA-1}
%
%SHA-1 is an NSA-designed hash function standardized by the NIST~\cite{sha1}. It combines a compression
%function which is a block cipher with 512-bit keys and 160-bit messages used in Davies-Meyer mode
%with a Merkle-Damg\aa rd mode of operation~\cite[Chap. 9]{HAC96}. Thus, the initial vector (IV)
%as well as the final hash
%are 160-bit values, and messages are processed in 512-bit blocks.
%The underlying block cipher of the compression function can be described as follows:
%let us denote by $m_0, \ldots m_{15}$ the 512-bit key as 16 32-bit words. The \emph{expanded} key
%$w_0, \ldots w_{79}$ is defined as:
%\[
%  w_i = \left\{\begin{array}{ll}
%      m_i & \text{if } i < 16\\
%      (w_{i - 3} \oplus w_{i - 8} \oplus w_{i - 14} \oplus w_{i - 16}) \lll 1 & \text{otherwise}.
%    \end{array}
%    \right.
%  \]
%  Then, if we denote by $a, b, c, d, e$ a 160-bit state made of 5 32-bit words and initialized
%  with the plaintext, the ciphertext is the value held in this state
%  after iterating the following procedure (parametered by the round number $i$) 80 times:
%  \[
%    \begin{array}{l}
%      t \leftarrow (a \lll 5) + \bool_{i \div 20}(b, c, d) + e + k_{i \div 20} + w_i  \\
%      e \leftarrow d\\
%      d \leftarrow c\\
%      c \leftarrow b \lll 30\\
%      b \leftarrow a\\
%      a \leftarrow t,\\
%    \end{array}\quad
%  \]
%  where `$\div$' denotes the integer division, $\Phi_{0\ldots3}$ are four bitwise Boolean functions,
%  and $k_{0\ldots3}$ are four constants (we refer to~\cite{sha1} for their definition).
%
%  Importantly, before being hashed, a message is \emph{always} padded with
%  at least 65 bits, made of a `1' bit, a (possibly zero) number of `0' bits,
%  and the length of the message in bits as a 64-bit integer. This padding places
%  an additional constraint on the attacker as it means that even a preimage for the compression function with
%  a valid IV is not necessarily a preimage for the hash function.


\subsection{One-block preimages without padding}
\label{sec:one_wo_pad}
  We apply the framework of \autoref{NewFramework} to mount attacks on \shaone
  for \emph{one-block preimages without padding}. These are rather direct applications
  of the framework, the only difference being the fact that we use \emph{sets} of
  differentials instead of \emph{linear spaces}. This has no impact
  on \autoref{affine_testing_II}, but makes the description of the attack parameters
  less compact.

  As was noted in~\cite{DBLP:conf/crypto/KnellwolfK12}, the message expansion of \shaone being linear, it is possible to attack 15 steps both in the forward and backward 
  direction, for a total of 30, without advanced matching techniques: it is sufficient to use a message difference
  in the kernel of the 15 first steps of
  the message expansion. 
  When applying our framework to attack more steps, say 55 to 62, we have observed experimentally
  that splitting the forward and backward parts around steps 22 to 27 seems to give the best results.
  A similar behaviour was observed by Knellwolf and Khovratovich in their attacks, and this can be explained by the fact
  that the \shaone step function has a somewhat weaker diffusion when computed backward compared to forward.

  We use \autoref{difference-getter} to
  construct a suitable set of differences in the preparation of an attack.
  This algorithm was run on input differences of low Hamming weight; these
  are kept only when they result in output differences
  with truncation masks that are long enough and with good overall
  probabilities. The sampling parameter $Q$ that we used was $2^{15}$;
  the threshold value $t$ was subjected to a tradeoff: the larger it is,
  the less bits are chosen in the truncation mask, but the better the probability of
  the resulting differential. In practice, we used values between 2 and 5, 
  depending on the differential considered. 

  \begin{algorithm}[htb]
        \LinesNumbered
    \KwIn{
      A chunk $\Fs_i$ of the compression function, $\delta_{i,1}, \delta_{i,2} \in \{0,1\}^m$, a threshold value $t$,
      a sample size $Q$, an internal state $c$.
    }
    \KwOut{
      An output difference $\mathcal{S}$ , and a mask $T_\mathcal{S}$ for the differential 
      $((\delta_{i,1}, \delta_{i,2}), 0) \overset{\Fs_i}{{\rightsquigarrow}}~\mathcal{S}$
    }
    \KwData{ 
      An array $d$ of $n$ counters initially set to $0$.
    }
    \For{$ q = 0$ to $Q$}
    {
      Choose $\mu \in \{0,1\}^m$ at random\;
      $\Delta \leftarrow \Fs_i (\mu \oplus \delta_{i,1} \oplus \delta_{i,2} , c) \oplus \Fs_i (\mu \oplus \delta_{i,1} , c) \oplus
      \Fs_i(\mu \oplus \delta_{i,2},c) \oplus \Fs_i (\mu,c)$\;
      \For{$i = 0$ to $ n-1$ }
      {
        \If{ the i$^\text{th}$ bit of $\Delta$ is 1}
        {
          $d[i] \leftarrow d[i] + 1$\;
        }
      }
    }
    \For{$i = 0$ to $ n-1$ }
    {
      \If{ $d[i] \geq  t $}
      {
        Set the $i$-th bit of the output difference $\mathcal{S}$ to $1$\;
      }
    }
    \caption{\label{difference-getter}Computing a suitable output difference for a given input difference}
  \end{algorithm}

  Once input and output differences have been chosen, we use an adapted version of \cite[Algorithm~2]{DBLP:conf/crypto/KnellwolfK12} given in
  \autoref{mask-getter} to compute suitable truncation masks. 

  \begin{algorithm}[ht]
        \LinesNumbered
    \KwIn{
      $D_{1,1}, D_{1,2}, D_{2,1}, D_{2,2} \subset \{0,1\}^m$, a sample size $Q$, a mask size $d$.
    }
    \KwOut{
      A truncation mask $T \in \{0,1\}^n$ of Hamming weight $d$.
    }
    \KwData{ 
      An array $k$ of $n$ counters initially set to $0$.
    }

    \For{$ q = 0$ to $Q$}
    {
      Choose $\mu \in \{0,1\}^m$ at random \;
      $c \leftarrow \Fs(\mu, \iv)$\;
      Choose $(\delta_{1,1},\delta_{1,2},\delta_{2,1},\delta_{2,2}) \in D_{1,1} \times D_{1,2} \times D_{2,1} \times D_{2,2} $ at random\; 
      $\Delta \leftarrow 
      \Fuu (\mu \oplus \delta_{1,1} \oplus \delta_{1,2} , c) \oplus 
      \Fuu (\mu \oplus \delta_{1,1} , c) \oplus 
      \Fuu(\mu \oplus \delta_{1,2},c)$\;
      $\Delta \leftarrow \Delta \oplus
      \Fd^{-1} (\mu \oplus \delta_{2,1} \oplus \delta_{2,2}, c) \oplus 
      \Fd^{-1}(\mu \oplus \delta_{2,2} , c) \oplus 
      \Fd^{-1} (\mu \oplus \delta_{2,2} , c) $\;
      \For{$i = 0$ to $ n-1$ }
      {
        \If{ the i$^\text{th}$ bit of $\Delta$ is 1}
        {
          $k[i] \leftarrow k[i] + 1$\;
        }
      }
    }
    Set the d bits of lowest counter value in $k$ to 1 in T.
    \caption{\label{mask-getter}Find truncation mask T for matching}
  \end{algorithm}

  The choice of the size of the truncation mask $d$ in this algorithm
  offers a tradeoff between the probability one can hope to achieve for the resulting truncated differential
  and how efficient a filtering of ``\,bad\,'' messages it will offer.
  In our applications to \shaone, we chose masks of size about
  $\min(\log_2(|D_{1,1}|), \log_2(|D_{1,2}|)$, $\log_2(|D_{2,1}|), \log_2(|D_{2,2}|))$,
  which is consistent with taking masks of size the dimension of the affine
  spaces as is done in~\cite{DBLP:conf/crypto/KnellwolfK12}.

  \medskip

  We similarly adapt \cite[Algorithm 3]{DBLP:conf/crypto/KnellwolfK12} as \autoref{typeI-getter} in order to estimate the average
  false negative probability associated with the final truncated differential.

  \begin{algorithm}[ht]
        \LinesNumbered
    \KwIn{
      $D_{1,1}, D_{1,2}, D_{2,1}, D_{2,2} \subset \{0,1\}^m, T \in \{0,1\}^n$, a sample size $Q$
    }
    \KwOut{
      Average false negative probability $\alpha$.
    }
    \KwData{ 
      A counter $k$ initially set to $0$.
    }

    \For{$q = 0$ to $Q$}
    {
      Choose $\mu \in \{0,1\}^m$ at random \;
      $c \leftarrow \Fs(\mu, \iv)$\;
      Choose $(\delta_{1,1},\delta_{1,2},\delta_{2,1},\delta_{2,2}) \in D_{1,1} \times D_{1,2} \times D_{2,1} \times D_{2,2} $ at random\; 
      $\Delta \leftarrow 
      \Fuu (\mu \oplus \delta_{1,1} \oplus \delta_{1,2} , c) \oplus 
      \Fuu (\mu \oplus \delta_{1,1} , c) \oplus 
      \Fuu(\mu \oplus \delta_{1,2},c)$\;
      $\Delta \leftarrow \Delta \oplus
      \Fd^{-1} (\mu \oplus \delta_{2,1} \oplus \delta_{2,2}, c) \oplus 
      \Fd^{-1}(\mu \oplus \delta_{2,2} , c) \oplus 
      \Fd^{-1} (\mu \oplus \delta_{2,2} , c) $\;
      \For{$i = 0$ to $ n-1$ }
      {
        \If{$\Delta \not\equiv_T 0^n$}
        {
          $k \leftarrow k+1$\;
        }
      }
    }
    \Return {$k/Q$}
    \caption{\label{typeI-getter}Estimate the average false negative probability}
  \end{algorithm}

\bigskip

  We do not explicitly list the difference sets used in our attack, but conclude this section by giving the statistics for the best attacks that we found for
  various reduced versions of \shaone \emph{without padding} in \autoref{SHA11}, the highest number of
  attacked rounds being 62.
%  Because the difference spaces
%  are no longer affine, they do not lend themselves to a compact description and their size
%  is not necessarily a power of 2 anymore. The ones we use do not have many elements, however,
%  which makes them easy to enumerate.

  \begin{table}[htb]
    \caption[One block preimage without padding.]{One block preimage without padding. $N$ is the 
              number of attacked steps, \emph{Split} is the separation step between the
              forward and the backward chunk, $d_{i,j}$ is the $\log_2$ of the cardinal
              of $D_{i,j}$ and $\alpha$ is the estimate for the false negative probability. The complexity
              is computed as described in  \autoref{NewFramework}.\label{SHA11}}
    \begin{center}
      \begin{tabular}{l c c c c c c r @{}} \toprule
        $N\qquad$ &  \emph{Split} & $d_{1,1}$ &  $d_{1,2}$ & $d_{2,1}$ & $d_{2,2}$ & $\alpha $ & Complexity \\\midrule
        58    & 25  & 7.6  & 9.0 & 9.2 & 9.0 & 0.73  & 157.4\\ 
        59    & 25  & 7.6  & 9.0 & 6.7 & 6.7  & 0.69  & 157.7\\ 
        60    & 26  & 6.5 & 6.0 & 6.7 & 6.0  & 0.60  & 158.0\\ 
        61    & 27  & 4.7 & 4.8 & 5.7  & 5.8  & 0.51  & 158.5\\ 
        62    & 27  & 4.7 & 4.8 & 4.3  & 4.6  & 0.63 & 159.0 \\ 
        \bottomrule
        \hline
      \end{tabular}
    \end{center}
  \end{table}

  \subsection{One-block preimages with padding}
\label{sec:one_wi_pad}


  If we want the message to be properly padded, 65 out of the 512 bits of the last message
  blocks need to be fixed according to the padding rules, and this naturally restricts the positions
  of where one can now use message differences. This has in particular an adverse effect on
  the differences in the backward step, which Hamming weight increases because of some
  features of \shaone's message expansion algorithm. The overall process of finding attack
  parameters is otherwise unchanged from the non-padded case. We give statistics for
  the best attacks that we found in \autoref{SHA12}. One will note that the highest number of attacked
  rounds dropped from 62 to 56 when compared to \autoref{SHA11}.
 
  \begin{table}[htb]
    \caption[One block preimage with padding.]{One block preimage with padding. $N$ is the 
              number of attacked steps, \emph{Split} is the separation step between the
              forward and the backward chunk, $d_{i,j}$ is the $\log_2$ of the cardinal
              of $D_{i,j}$ and $\alpha$ is the estimation for false negative probability. The complexity
              is computed as described in \autoref{NewFramework}.\label{SHA12}}
    \begin{center}
      \begin{tabular}{l c c c c c c r @{}} \toprule
        $N\qquad$ &  \emph{Split} & $d_{1,1}$ &  $d_{1,2}$ & $d_{2,1}$ & $d_{2,2}$ & $\alpha $ & Complexity \\\midrule
      51    & 23  & 8.7  & 8.7 & 8.7 & 8.7  & 0.72  & 155.6\\ 
      52    & 23  & 9.1  & 9.1 & 8.2 & 8.2  & 0.61  & 156.7\\ 
      53    & 23  & 9.1  & 9.1 & 3.5 & 3.5  & 0.61  & 157.7\\ 
      55    & 25  & 6.5  & 6.5 & 5.9 & 5.7  & 0.52  & 158.2\\ 
      56    & 25  & 6  & 6.2 & 7.2 & 7.2  & 0.6  & 159.4\\ 
        \bottomrule
        \hline
      \end{tabular}
    \end{center}
  \end{table}



  \subsection{Two-block preimages with padding}
\label{sec:one_two}


  We can increase the number of rounds for which we can find a preimage with a properly padded
  message at the cost of using a slightly longer message of two blocks: if we are able to find
  \emph{one-block pseudo preimages with padding} on enough rounds, we can then use the
  \emph{one-block preimage without padding} to bridge the former to the IV. Indeed,
  in a pseudo-preimage
  setting, the additional freedom degrees gained from removing any constraint on the IV
  more than compensate for the ones added by the padding. We first describe
  how to compute such pseudo-preimages.

  \subsubsection{One-block pseudo-preimages.}

  If we relax the conditions on the IV and do not impose anymore that it is fixed
  to the one of the specifications, it becomes possible to use a splice-and-cut
  decomposition of the function, as well as short properly padded bicliques built in the
  same way as the ones used by Knellwolf and Khovratovich~\cite{DBLP:conf/crypto/KnellwolfK12}.

  The reduced compression function of \shaone $\Fs$ is now decomposed into three smaller
  functions as $\Fs = \Fd^t \circ \Fuu \circ \Ft \circ \Fd^b$, $\Ft$ being the rounds covered by the
  biclique. The function $\Fuu$ covers the steps $s_1$ to $e$, $\Fd = \Fd^t \circ \Fd^b$ covers
  $s_2$ to $e + 1$ through the feedforward, and $\Ft$ covers $s_2 + 1$ to $s_1 - 1$, as shown in
  \autoref{biclique}.

    \begin{figure}
	\center
%      \includegraphics[scale=0.6]{../III_SHA-1/preimages/SHA.pdf}
      \includegraphics[scale=1.4]{figures/splice-and-cut.pdf}
      \caption{A splice-and-cut decomposition using a biclique.\label{biclique}}
    \end{figure}

    Finding the parameters is done in the exact same way as for the one-block preimage attacks.
    Since we only use bicliques covering $7$ steps, one can generate many of them
    from a single one by modifying some of the remaining message words outside of the biclique proper.
    Therefore, the amortized cost of their construction is small and considered negligible w.r.t. the
    rest of the attack. The statistics of the resulting attacks are shown in \autoref{SHA13}.

  \begin{table}[htb]
    \caption[One block pseudo-preimage with padding.]{One block pseudo-preimage with padding. $N$ is the 
              number of attacked steps, $d_{i,j}$ is the $\log_2$ of the cardinal
              of the set $D_{1,2}$ and $\alpha$ is the estimate for the false negative probability.
              The various splits are labeled as in \autoref{biclique}.
            The complexity is computed as described in \autoref{NewFramework}.\label{SHA13}}
    \begin{center}
      \begin{tabular}{l c c c c c  c c c r @{}} \toprule
        $N\qquad$ & $s_1$ & $e$ & $s_2$ & $d_{1,1}$ &  $d_{1,2}$ & $d_{2,1}$ & $d_{2,2}$ & $\alpha $ & Complexity \\\midrule
        61    & 27 & 49 & 20   & 7.0  & 7.0 & 7.5 & 7.5  & 0.56  & 156.4\\ 
        62    & 27 & 50 & 20   & 5.8  & 5.7 & 7.2 & 7.2  & 0.57  & 157.0\\ 
        63    & 27 & 50 & 20   & 6.7  & 6.7 & 7.7 & 7.7  & 0.57  & 157.6\\ 
        64    & 27 & 50 & 20   & 3 & 3 & 4.5 & 4.7  & 0.69  & 158.7\\ 
        \bottomrule
        \hline
      \end{tabular}
    \end{center}
  \end{table}

  \subsubsection{Complexity of the two-block attacks.}
  Using two one-block attacks, it is simple to mount a two-block attack at the combined cost of each of them.
  For a given target $c$, the process is the following:
    \begin{enumerate}
      \item The attacker uses a properly-padded pseudo-preimage attack, yielding the second message block $\mu_2$ and an IV $\iv'$;
      \item The attacker then uses a non-padded preimage attack with target $\iv'$, yielding a first message block $\mu_2$.
    \end{enumerate}
From the Merkle-Damg\aa rd structure of \shaone, it follows that the two-block message $(\mu_1, \mu_2)$ 
    is a preimage of $c$.

    For attacks up to 60 rounds, we can use the pseudo-preimage attacks of~\cite{DBLP:conf/crypto/KnellwolfK12}; for 61 and 62 rounds,
    we use the ones of this section. This results in attacks of complexities as shown in \autoref{SHA14}.

  \begin{table}[htb]
    \caption[Two-block preimage.]{Two-block preimage attacks on \shaone reduced to $N$ steps. The pseudo-preimage attacks followed by
    ``$\star$'' come from \cite{DBLP:conf/crypto/KnellwolfK12}.\label{SHA14}}
    \begin{center}
      \begin{tabular}{l c c c } \toprule
        $N\qquad$ & Second block complexity\phantom{bla} & First block complexity\phantom{bla}  & Total Complexity \\\midrule
        58    & $156.3^\star$ & 157.4& 157.9\\ 
        59    & $156.7^\star$ & 157.7& 158.3\\ 
        60    & $157.5^\star$ & 158.0& 158.7\\ 
        61    & 156.4    & 158.5& 158.8\\ 
        62    & 157.0    & 159.0& 159.3\\ 
        \bottomrule
        \hline
      \end{tabular}
    \end{center}
  \end{table}

\section{Conclusion}
This chapter showed how to extend the differential variant of the \mitm framework for hash function preimage attacks with higher-order differentials,
and how this could be applied to find better attacks on \shaone.
The source of the improvements come from the fact that higher-order differentials lead to better message partitions, which
then mechanically allow to mount attacks up to a larger number of steps than the previous best
results.

This new technique is however not without downsides of its own, as it makes the attack procedure more complex. A consequence is
that it does not seem to improve the best attacks on very weak functions such as \mdfour. Yet, it still produces the best known
attacks on the more complex \shaone, when the objective is to find a preimage for the highest possible number of rounds.
It should also be noted that due to the strong similarity with the original \mitm framework, this technique may be
much less successful when applied to functions with a heavier message expansion, such as \shatwo, or in general to
function with better diffusion. A striking example in the latter case is that by applying the current framework
to the \blake-512 hash function~\cite{blake}, we were only able to attack 2.75 out of 16 with a complexity close to a generic attack. 

