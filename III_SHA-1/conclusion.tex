\section{Conclusion}
\label{sec:conclusion}

The work described in this article culminated in the computation of an explicit freestart collision for the full \shaone, but a collision for the entire hash algorithm is still unknown.
There is no known generic and efficient algorithm that can turn a freestart collision into a plain collision for the hash function.
However, the advances we have made do allow us to precisely estimate and update the computational and financial cost to generate such a collision, using the latest cryptanalysis advances \cite{DBLP:conf/eurocrypt/Stevens13}
(the computational cost required to generate such a collision is actually a recurrent debate in the academic community since the first theoretical attack from Wang~\etal~\cite{DBLP:conf/crypto/WangYY05a}).

Schneier's projections~\cite{schneierSHA1} on the cost of \shaone collisions, made in 2012 (on Amazon EC2: $\approx$700K US\$ by 2015, $\approx$173K US\$ by 2018 and $\approx$43K US\$ by 2021) were based on
(an early announcement of) \cite{DBLP:conf/eurocrypt/Stevens13}. These projections have been used to establish the timeline of migrating away from \shaone-based signatures for secure Internet websites,
resulting in a migration by January 2017 ---one year before Schneier estimated that a \shaone collision would be within the resources of criminal syndicates. 

This work demonstrated that GPUs are much faster for this type of attacks (compared to CPUs)
and we now precisely estimate that a full \shaone collision should not cost more than between 75K and 120K US\$ by renting Amazon EC2 cloud over a few months at the time when this research was done, in early autumn 2015.
Our new GPU-based projections are now more accurate and they are significantly below Schneier's estimations. More worrying, they are theoretically already within Schneier's estimated resources of criminal syndicates as of the end of 2015,
almost two years earlier than previously expected, and one year before \shaone being marked as unsafe in modern Internet browsers.
This lead us to recommend that migration from \shaone to the secure \shatwo or \shathree hash algorithms should be done sooner than previously planned.

Note that it has previously been shown that a more advanced so-called chosen-prefix collision attack on \mdfive allowed the creation of a rogue Certification Authority undermining the security of all secure websites \cite{DBLP:conf/crypto/StevensSALMOW09}. 
Collisions on \shaone can result in \eg{} signature forgeries, but do not directly undermine the security of the Internet at large. More advanced so-called chosen-prefix collisions \cite{DBLP:conf/crypto/StevensSALMOW09}
are significantly more threatening, but currently much costlier to mount. Yet, given the lessons learned with the \mdfive full collision break, it is not advisable to wait until these become practically possible.

\medskip

At the time of the find of the 80-step freestart collision, in October 2015, we learned that in an ironic turn of events the CA/Browser Forum\footnote{The CA/Browser Forum is the main association of industries regulating the use of digital certificates on the Internet.}
was planning to hold a ballot to decide whether to extend issuance of \shaone certificates through the year 2016 \cite{cabforum}.
With our new cost projections in mind, we strongly recommended against this extension and the ballot was subsequently withdrawn \cite{cabforum2}.
Further action was subsequently considered by major browser providers such as Microsoft \cite{MS_sha} and Mozilla \cite{Moz_sha} to speed up the deprectation of \shaone certificates.
