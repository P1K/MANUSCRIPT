\section{A framework for freestart collisions for SHA-1}
\label{sec:framework}
% TODO Explain message bit relations somewhere (not necessarily this section)
% + homogeneize notations (bit / path /state conditions)

This section presents in detail our framework for freestart collision attacks on \shaone. We start by presenting our general approach to such attacks
in \autoref{sec:why_free}, and the addition and improvements to existing techniques that were developped to make the attacks possible in \autoref{sec:attack_techs}.

\subsection{Faster collisions by exploiting more freedom}
\label{sec:why_free}

The main interest of freestart collisions is that they consist in an attack against a meaningful (though weakened) security notion, while granting more freedom to
the attacker by providing him with an additional input. It is thus expected that such attacks should be more efficient than ones targeting stronger notions (\eg hash function
collisions), not unlike attacking a reduced version of a function, using fewer steps.

However, it is not necessarily clear in general how to efficiently exploit the additional input in a freestart attack. We describe below our strategy in that respect
in the case of \shaone.

\medskip

The basic idea underlying the structure of our freestart collisions is to start the initialisation of the hash function from a ``middle'' state rather than from the \iv,
and to similarly initialize the message with an offset. As both of the step function and the message expansion of \shaone can easily be inverted, any computation started
from the middle can be equivalently defined as starting from the beginning. We call \emph{main block offset} the offset in the initialization of the message, giving us
the technical definition:

\begin{defi}[Main block offset]
A freestart collision attack on \shaone uses a \emph{main block offset} of $i \in \{0, \ldots, 64\}$ if the candidate (expanded) messages $\expmess$ tested for a collision are defined
through the (possibly backward) message expansion of the words $\expmess_i, \ldots, \expmess_{i + 15}$.
\end{defi}

The desired effect of using a (non-zero) main block offset is to move down the part where freedom is available, enabling to satisfy conditions (either deterministically or with accelerating techniques)
up to a later point in the attack. Being able to do so would mechanically make the attack more efficient, at the condition that this gain is not entirely offset by the cost
of satisfying potential conditions when computing backward to the corresponding \iv (this computation being eventually necessary to fully determine the colliding inputs).

In the case of a freestart attack, the entire freedom of the \iv implies that no \emph{a priori} constraints are set on the values for the middle initializations. This is in contrast to a hash function attack, where
the necessity for the backward-computed \iv to be of a specific value makes the approach less powerful, altough potentially still useful.

We briefly review below some cases of past attacks on hash functions (not necessarily freestart) using this sort of approach.

\medskip

A first example is due to Dobbertin, who used such a technique in a collision attack on \mdfour~\cite{DBLP:conf/fse/Dobbertin96}.
This method has also been used to improve collision attacks on functions based on parallel branches, such as \ripemdote~\cite{DBLP:conf/eurocrypt/LandelleP13}.
A reduced version of \groestl (which is much more recent than the \mdsha family and structurally quite different) was also attacked by using such a shifted initialization~\cite{DBLP:conf/fse/MendelRS14}

Starting from the middle may also be useful in weaker attack models, such as ``distinguishers''. The approach was for instance used in
Saarinen's slide distinguisher on the \shaone compression function~\cite{DBLP:conf/fse/Saarinen03} and in
rebound attacks in general~\cite{DBLP:conf/fse/MendelRST09}, which can be thought of as a start-from-the-middle strategy adapted to the specific case of \aes-like primitives.

We conclude this discussion by considering more closely how we applied this technique to \shaone.

\subsubsection{Different choices for the disturbance vectors.}
We already mentioned at length the importance of the disturbance vectors in attacks on \shaone.
When shifting the window where actual freedom is used, it seems natural to also shift the disturbance vector.
This is in part a consequence of the fact that good \dvs tend to naturally include a series of steps for which they impose a lot of conditions. These can easily be resolved provided that
enough freedom is available, but they would heavily impact the complexity of an attack were they to be satisfied probabilistically. It is thus important that this part remains
in the window of available freedom.

As the equivalence classes of Manuel already include vertical shifts in their definitions, the two known good classes are already suitable for the search for good \dvs for freestart collisions.
We simply expect to use \dvs with different shift values than the ones usually considered.

\medskip

The structure of the attack also imposes additional conditions on the \dv, compared to a usual two-block hash function attack. Both conditions come from the aforementioned fact that bit conditions
(ensuring that the message pair follows a proper differential path)
may need to be satisfied (probabilistically) when computing the \iv backward from the values of the middle initialization. To make this phase efficient, we would like to have as few such
bit conditions as possible. This translates to the two high-level conditions:
\begin{enumerate}
\item The \dv should not imply too many differences in the last five steps of the attack. This is because we aim for a one-block attack, thus corresponding differences will need to be inserted in the
backward-computed \iv for the pair to define a collision.
\item The \dv itself should not include too many differences in its early words (corresponding to the ones used in the backward computation), as we observed that being dense on these positions makes
it harder to connect to the \iv with few conditions. Additionally, it may decrease the overall probability of the computation.
\end{enumerate}

\subsubsection{New constraints for the accelerating techniques.}
The necessity of satisfying bit conditions in a backward computation also imposes some constraints on the accelerating techniques that can be used. These all have in common that they work by
selectively changing the value of some bits of the message. As the structure of our attacks also includes shifting down the free window of the message, changing a bit in this window may also impact
the value of the few first message words that are computed backward. This may potentially result in unwanted interactions with the ability to satisfy the backward bit conditions.

In short, the selection of accelerating techniques (\eg neutral bits) must take into account their ``back effects'', which may disqualify some otherwise good candidates.

\subsection{New techniques for freestart collisions}
\label{sec:attack_techs}

There is a certain number of points that need to be specified to make the description of our framework complete. In particular, we will discuss here:
\begin{enumerate}
\item The construction of non-linear paths adapted to the shifted initialization and to the requirements of high-probability backward computations, and improvements to existing
methods (\autoref{sec:free_nl}).
\item The accelerating techniques used in our attacks (\autoref{sec:free_nb}).
\item The modified attack process adapted to the shifted initialization (\autoref{sec:free_att}).
\end{enumerate}

\subsubsection{Differential paths for freestart attacks}
\label{sec:free_nl}

%TODO actually, we rather want lo back interaction proba
We have the requirements in the freestart attacks that the computation from the initialized state back to the \iv be of high probability; ideally, we would like to be able to satisfy deterministically
all the conditions associated to this computation (while still performing a shifted initialization). A direct consequence of this is that the differential path should be sparse (\ie include few
conditions, and especially few differences) in the steps involved in the backward computation.

The known methods for the construction of non-linear differential paths for \sha do not provide good guarantees that the path will be sparse in \emph{a priori} specified state words. However, it is
not hard to search independently for a sparse prefix for the differential path and then only to search for paths that are extensions of this prefix.
We implemented this strategy by first greedily searching for prefixes of various lengths and with few differences, and then by searching for good paths extending these prefixes.
We show in \autoref{fig:lo_start} the starting point that was eventually used in the search for the path of the 76-step attack (using a representation suitable for a guess-and-determine
path search)\footnote{Note that in order to ensure a collision through the Davies-Meyer feed-forward, a sign has been imposed in the differences in the \iv.}.

\begin{figure}[htb]\begin{center}
\begin{tabular}{lcc}
$i$ & $\state_i$ & $\mess_i$\\
-4 & \nodiff \nodiff \nodiff \nodiff \nodiff \nodiff \nodiff \nodiff \nodiff \nodiff \nodiff \nodiff \nodiff \nodiff
\nodiff \nodiff \nodiff \nodiff \nodiff \nodiff \nodiff \nodiff \nodiff \nodiff \nodiff \nodiff \nodiff \nodiff \nodiff \nodiff \nodiff \nodiff&\\ 
-3 & \nodiff \nodiff \nodiff \nodiff \nodiff \nodiff \nodiff \nodiff \nodiff \nodiff \nodiff \nodiff \nodiff \nodiff
\nodiff \nodiff \nodiff \nodiff \nodiff \nodiff \nodiff \nodiff \nodiff \nodiff \nodiff \nodiff \nodiff \nodiff \nodiff \nodiff \nodiff \nodiff& \\
-2 & \nodiff \nodiff \nodiff \nodiff \nodiff \nodiff \nodiff \nodiff \nodiff \nodiff \nodiff \nodiff \nodiff \nodiff
\nodiff \nodiff \nodiff \nodiff \nodiff \nodiff \nodiff \nodiff \nodiff \nodiff \nodiff \nodiff \nodiff \nodiff \nodiff \nodiff $\monediffu$\nodiff& \\
-1 & \nodiff \nodiff \nodiff \nodiff \nodiff \nodiff \nodiff \nodiff \nodiff \nodiff \nodiff \nodiff \nodiff \nodiff
\nodiff \nodiff \nodiff \nodiff \nodiff \nodiff \nodiff \nodiff \nodiff \nodiff \nodiff \nodiff \nodiff \nodiff \nodiff \nodiff \nodiff \onediffu& \\
0 &\nodiff \nodiff \nodiff \nodiff \nodiff \nodiff \nodiff \nodiff \nodiff \nodiff \nodiff \nodiff \nodiff \nodiff \nodiff \nodiff \nodiff \nodiff \nodiff \nodiff \nodiff \nodiff \nodiff \nodiff \nodiff \nodiff \nodiff \nodiff \nodiff \nodiff \nodiff \nodiff &    \nodiff \nodiff \nodiff \nodiff \nodiff \nodiff \nodiff \nodiff \nodiff \nodiff \nodiff \nodiff \nodiff \nodiff \nodiff \nodiff \nodiff \nodiff \nodiff \nodiff \nodiff \nodiff \nodiff \nodiff \nodiff \nodiff \nodiff \onediff \nodiff \nodiff \nodiff \nodiff \\
1 &\nodiff \nodiff \nodiff \nodiff \nodiff \nodiff \nodiff \nodiff \nodiff \nodiff \nodiff \nodiff \nodiff \nodiff \nodiff \nodiff \nodiff \nodiff \nodiff \nodiff \nodiff \nodiff \nodiff \nodiff \nodiff \nodiff \nodiff \onediff \nodiff \nodiff \nodiff \nodiff &   \nodiff \nodiff \nodiff \nodiff \nodiff \onediff \nodiff \nodiff \nodiff \nodiff \nodiff \nodiff \nodiff \nodiff \nodiff \nodiff \nodiff \nodiff \nodiff \nodiff \nodiff \nodiff \nodiff \nodiff \nodiff \nodiff \nodiff \onediff \onediff \onediff \nodiff \nodiff \\
2 &\nodiff \nodiff \nodiff \nodiff \nodiff \onediff \nodiff \nodiff \nodiff \nodiff \nodiff \nodiff \nodiff \nodiff \nodiff \nodiff \nodiff \nodiff \nodiff \nodiff \nodiff \nodiff \onediff \nodiff \nodiff \nodiff \nodiff \nodiff \nodiff \onediff \nodiff \nodiff &   \onediff \onediff \nodiff \nodiff \onediff \onediff \nodiff \nodiff \nodiff \nodiff \nodiff \nodiff \nodiff \nodiff \nodiff \nodiff \nodiff \nodiff \nodiff \nodiff \nodiff \nodiff \nodiff \nodiff \nodiff \nodiff \nodiff \onediff \nodiff \onediff \nodiff \nodiff \\
3 & \nodiff \nodiff \nodiff \nodiff \nodiff \onediff \nodiff \nodiff \nodiff \nodiff \nodiff \nodiff \nodiff \nodiff \nodiff \nodiff \nodiff \onediff \nodiff \nodiff \nodiff \nodiff \nodiff \nodiff \onediff \nodiff \nodiff \nodiff \nodiff \onediff \nodiff \nodiff &   \nodiff \nodiff \nodiff \nodiff \onediff \onediff \nodiff \nodiff \nodiff \nodiff \nodiff \nodiff \nodiff \nodiff \nodiff \nodiff \nodiff \nodiff \nodiff \nodiff \nodiff \nodiff \nodiff \nodiff \nodiff \nodiff \nodiff \nodiff \nodiff \nodiff \onediff \nodiff \\
4 & \nodiff \onediff \nodiff \nodiff \nodiff \nodiff \nodiff \nodiff \nodiff \nodiff \nodiff \nodiff \onediff \nodiff \nodiff \nodiff \nodiff \nodiff \nodiff \onediff \nodiff \nodiff \nodiff \nodiff \onediff \nodiff \nodiff \nodiff \nodiff \nodiff \onediff \nodiff &   \onediff \onediff \nodiff \nodiff \nodiff \nodiff \nodiff \nodiff \nodiff \nodiff \nodiff \nodiff \nodiff \nodiff \nodiff \nodiff \nodiff \nodiff \nodiff \nodiff \nodiff \nodiff \nodiff \nodiff \nodiff \nodiff \nodiff \onediff \nodiff \nodiff \nodiff \nodiff \\
5 &\nodiff \onediff \nodiff \nodiff \nodiff \nodiff \nodiff \nodiff \nodiff \nodiff \nodiff \nodiff \nodiff \nodiff \onediff \nodiff \nodiff \nodiff \nodiff \onediff \nodiff \nodiff \nodiff \nodiff \nodiff \nodiff \onediff \nodiff \onediff \nodiff \nodiff \nodiff  &  \onediff \nodiff \onediff \onediff \nodiff \onediff \nodiff \nodiff \nodiff \nodiff \nodiff \nodiff \nodiff \nodiff \nodiff \nodiff \nodiff \nodiff \nodiff \nodiff \nodiff \nodiff \nodiff \nodiff \nodiff \nodiff \nodiff \onediff \onediff \onediff \nodiff \nodiff\\ 
6 & \dunnodiff \dunnodiff \dunnodiff \dunnodiff \dunnodiff \dunnodiff \dunnodiff \dunnodiff \dunnodiff \dunnodiff \dunnodiff \dunnodiff \dunnodiff \dunnodiff \dunnodiff \dunnodiff \dunnodiff \dunnodiff \dunnodiff \dunnodiff \dunnodiff \dunnodiff \dunnodiff \dunnodiff \dunnodiff \dunnodiff \dunnodiff \dunnodiff \dunnodiff \dunnodiff \dunnodiff \dunnodiff &   \nodiff \nodiff \onediff \onediff \onediff \onediff \nodiff \nodiff \nodiff \nodiff \nodiff \nodiff \nodiff \nodiff \nodiff \nodiff \nodiff \nodiff \nodiff \nodiff \nodiff \nodiff \nodiff \nodiff \nodiff \nodiff \nodiff \nodiff \nodiff \onediff \nodiff \nodiff \\
7 & \dunnodiff \dunnodiff \dunnodiff \dunnodiff \dunnodiff \dunnodiff \dunnodiff \dunnodiff \dunnodiff \dunnodiff \dunnodiff \dunnodiff \dunnodiff \dunnodiff \dunnodiff \dunnodiff \dunnodiff \dunnodiff \dunnodiff \dunnodiff \dunnodiff \dunnodiff \dunnodiff \dunnodiff \dunnodiff \dunnodiff \dunnodiff \dunnodiff \dunnodiff \dunnodiff \dunnodiff \dunnodiff &  \onediff \nodiff \onediff \onediff \onediff \onediff \nodiff \nodiff \nodiff \nodiff \nodiff \nodiff \nodiff \nodiff \nodiff \nodiff \nodiff \nodiff \nodiff \nodiff \nodiff \nodiff \nodiff \nodiff \nodiff \nodiff \nodiff \onediff \onediff \nodiff \onediff \nodiff \\
8 & \dunnodiff \dunnodiff \dunnodiff \dunnodiff \dunnodiff \dunnodiff \dunnodiff \dunnodiff \dunnodiff \dunnodiff \dunnodiff \dunnodiff \dunnodiff \dunnodiff \dunnodiff \dunnodiff \dunnodiff \dunnodiff \dunnodiff \dunnodiff \dunnodiff \dunnodiff \dunnodiff \dunnodiff \dunnodiff \dunnodiff \dunnodiff \dunnodiff \dunnodiff \dunnodiff \dunnodiff \dunnodiff &  \nodiff \nodiff \onediff \nodiff \nodiff \nodiff \nodiff \nodiff \nodiff \nodiff \nodiff \nodiff \nodiff \nodiff \nodiff \nodiff \nodiff \nodiff \nodiff \nodiff \nodiff \nodiff \nodiff \nodiff \nodiff \nodiff \nodiff \onediff \nodiff \nodiff \nodiff \nodiff \\
9 & \dunnodiff \dunnodiff \dunnodiff \dunnodiff \dunnodiff \dunnodiff \dunnodiff \dunnodiff \dunnodiff \dunnodiff \dunnodiff \dunnodiff \dunnodiff \dunnodiff \dunnodiff \dunnodiff \dunnodiff \dunnodiff \dunnodiff \dunnodiff \dunnodiff \dunnodiff \dunnodiff \dunnodiff \dunnodiff \dunnodiff \dunnodiff \dunnodiff \dunnodiff \dunnodiff \dunnodiff \dunnodiff &  \nodiff \nodiff \onediff \nodiff \nodiff \onediff \nodiff \nodiff \nodiff \nodiff \nodiff \nodiff \nodiff \nodiff \nodiff \nodiff \nodiff \nodiff \nodiff \nodiff \nodiff \nodiff \nodiff \nodiff \nodiff \nodiff \nodiff \onediff \onediff \onediff \nodiff \nodiff \\
10 & \dunnodiff \dunnodiff \dunnodiff \dunnodiff \dunnodiff \dunnodiff \dunnodiff \dunnodiff \dunnodiff \dunnodiff \dunnodiff \dunnodiff \dunnodiff \dunnodiff \dunnodiff \dunnodiff \dunnodiff \dunnodiff \dunnodiff \dunnodiff \dunnodiff \dunnodiff \dunnodiff \dunnodiff \dunnodiff \dunnodiff \dunnodiff \dunnodiff \dunnodiff \dunnodiff \dunnodiff \dunnodiff & \\ 
\end{tabular}
\end{center}
\caption{A sparse prefix for the non-linear differential path of a freestart attack. The symbol ``\dunnodiff'' denotes the absence of conditions.\label{fig:lo_start}}
\end{figure}

Through the course of the preparation of the attacks, we tried both methods for constructing the extension of the non-linear path, \ie a guess-and-determine and a meet-in-the-middle approach.
We have found that the meet-in-the-middle approach allowed to find paths with fewer conditions, and that it allowed a better control of the position where some of these conditions were located.
A slight downside of our meet-in-the-middle implementation is that it does not give strict guarantees that the paths it produces are completely valid, as it may introduce contradictory conditions,
for instance in the high-density part of the path where many differences are located. However, this issue can usually be resolved by considering slight variations of the impossible path, which
we were able to do efficiently by using our implementation of the guess-and-determine method.
% TODO doesn't completely reflect the 80-step. Is that a real problem tho?

\medskip

The evaluation and eventual selection of the \dvs used in our attacks was done using joint local collision analysis. We modified the original implementation of Stevens~\cite{DBLP:conf/eurocrypt/Stevens13}
to cover \emph{all} steps (including the non-linear part) and to produce the entire set of sufficient state conditions and message bit-relations as used by collision attacks.
In particular, we improved JLCA in the following ways:
\begin{enumerate}
\item \textbf{Include the non-linear part.} Recall that originally JLCA considered the entire set of differential paths that conform to the disturbance vector only
over the linear part. 
This was done by considering sets $\mathcal{Q}_i$ of allowed state differences for each state word $\state_i$ given the disturbance vector (including carries).
We extended this by defining sets $\mathcal{Q}_i$ for the non-linear part as the state difference given by the previously constructed differential path of the non-linear part.
Here one actually has a few options: only consider exact state difference of the non-linear path or also consider changes in carries and/or signs, as well as include state differences conforming to the disturbance vector.
We found that allowing changes in carries and/or signs for the state differences given by the non-linear path made JLCA impractical,
yet including state differences conforming to the disturbance vector was practical and had a positive effect on the overall probability of the full differential path.

\item \textbf{Do not consider auxiliary carries not used in the attack.} Originally JLCA considers local collisions with carries as this improves the overall probability, the probability of variants of paths adding up. 
However, collision attacks employ sufficient conditions for, say, the first 25 steps, where such auxiliary carries are not used. For these steps JLCA would thus optimize for the wrong model with auxiliary carries. 
We propose to improve this by not adding up the probability of paths over the first 25 steps, but only to take the maximum probability. 
We propose to do this by redefining the cumulative probabilities $p_{(\mathcal{P},w)}$ from~\cite[Sec 4.6]{DBLP:conf/eurocrypt/Stevens13} to:
\[ p_{(\mathcal{P},w)} = \max_{\widehat{\mathcal{P}}_{[0,25]}\in\mathcal{D}_{[0,25]}} 
  \sum_{\substack{
	    \mathcal{P}'\in\mathcal{D}_[0,t_e]\\
			\mathcal{P}'|_{[0,25]}=\widehat{\mathcal{P}}_{[0,25]}\\
			\mathcal{P}=\mathrm{Reduce}(\mathcal{P}'), w=w(\mathcal{P}')
			}} 
	 \Pr[\mathcal{P}'-\mathcal{P}]. 
\]
In our JLCA implementation this can be simply implemented by replacing the addition of two probabilities by taking their maximum conditional on the current \shaone step.

\item \textbf{Determine sufficient conditions.} Originally JLCA only gave as output starting differences, ending differences, message bit-relations and the optimal success probability. 
We propose to extend JLCA to reconstruct the set of differential paths over steps, say, $[0,25]$, and to determine minimal sets of sufficient conditions and message bit-relations.
This can be made possible by keeping the intermediate sets of reduced differential paths $\mathcal{R}_{[t_b,t_e]}$ which were constructed backwards starting at a zero-difference intermediate state of \shaone.
Then one can iteratively construct sets $\mathcal{O}_{[0,i)}$ of optimal differential paths over steps $0,\ldots,i-1$, \ie, differential paths compatible with some combination of the optimal starting differences, ending differences and message bit-relations such that the optimal success probability can be achieved.
One starts with the set $\mathcal{O}_{[0,0)}$ determined by the optimal starting differences.
Given $\mathcal{O}_{[0,i)}$ one can compute $\mathcal{O}_{[0,i+1)}$ by considering all possible extensions of every differential path in $\mathcal{O}_{[0,i)}$ with step $i$ (under the predefined constraints, \ie $\Delta Q_j \in \mathcal{Q}_j$, $\delta W_i \in \mathcal{W}-i$, see~\cite{DBLP:conf/eurocrypt/Stevens13}). 
From all those paths, one only stores in $\mathcal{O}_{[0,i+1)}$ those that can be complemented by a reduced differential path over steps $i+1,\ldots,t_e$ from $\mathcal{R}_{[i+1,t_e]}$ such that the optimal success probability is achieved over steps $0,\ldots,t_e$.

Now given, say, $\mathcal{O}_{[0,26)}$, we can select any path and determine its necessary and sufficient conditions for steps $0,\ldots,25$ and the optimal set of message bit-relations that goes with it.
Although we use only one path, having the entire set $\mathcal{O}_{[0,26)}$ opens even more avenues for future work.
For instance, one might consider an entire subclass of $2^k$ differential paths from $\mathcal{O}_{[0,26)}$ that can be described by state conditions linear in message bits and a set of (linear) message bit-relations.
This would provide $k$ bits more in degrees of freedom that can be exploited by speed up techniques.
\end{enumerate}
In short, we proposed several extensions to JLCA that allow us to determine sufficient state conditions and message bit-relations optimized for collision attacks,
\ie minimal set of conditions attaining the highest success probability paths (where auxiliary carries are only allowed after a certain step).

\subsubsection{Instantiating accelerating techniques}
\label{sec:free_nb}

Generally speaking, the accelerating technique used in our attacks were neutral bits. Only ``single-bit'' neutral bits were used in the attack on 76 steps, but
the 80-step attack also used ``boomerang'' neutral bits where three (and in one occasion, four) carefully chosen bits are flipped at once to compute the related
pair of messages. Additionally, in both attacks, some additional changes in the message may be necessary (depending on which neutral bits are activated)
to ensure that no message bit-relation becomes violated as a result. In the following, we use the term \emph{neutral bit} to mean either of the ``single-bit''
or ``boomerang'' neutral bits.

The main difference in the selection of neutral bits compared to a usual (non-freestart) attack is the presence of the main block offset, which
determines the offset of the message freedom window used during the attack.
We have selected a main block offset of 6 (resp. 5) in the 76-step (resp. 80-step) case, as this led to the best distribution of usable neutral bits and boomerangs.
This means that all the neutral bits (including potential boomerangs) directly lead to changes in the state from steps 6 (resp. 5) up to 21 (resp. 20) and that these changes propagate to steps 5 (resp. 4) down to 0 backwards and steps
22 (resp. 21) up to 79 forwards.

We describe below the selection process more specifically in the case of the 80-step attack; the approach was similar in the 76-step case, with the omission of the boomerangs neutral bits which were not used.

\medskip

The search for both single and boomerang neutral bits requires to evaluate the probability that a candidate does not interact badly with path conditions up to a certain point.
This probability is estimated experimentally by observing the effect of activating potential neutral bits on many partial solutions for the differential path.
Because the dense area of the attack conditions may implicitly force certain other bits to specific values (resulting in hidden conditions),
we used more than 4000 partial solutions (over steps 1 up to 16) in the analysis.
The 16 steps fully determine the message block, and also verify the sufficient conditions in the \iv and in the dense non-linear differential path of the first round.
It should be noted that for this step it is important to generate every sample independently.
Indeed using \eg{} message modification techniques to generate many samples from a single one would result in a biased distribution where many samples would only differ in the last few steps.

\paragraph{Searching for Boomerangs.}
We analyze potential boomerangs from the framework of Joux and Peyrin \cite{DBLP:conf/crypto/JouxP07}, which initially flip a single state bit together with 3 or more message bits.
Each boomerang should be orthogonal to the attack conditions, \ie, the state bit where a difference is introduced
should be free of sufficient conditions, while flipping the message bits should not break any of the message bit relations
(either directly or through the message expansion).
Let $t\in[6,16], b\in[0,31]$ be such that the state bit $\state_t[b]$ has no sufficient condition.

First, we determine the best usable boomerang on $\state_t[b]$ as follows.
For every sampled solution, we flip that state bit and compute the signed bit differences between the resulting and the unaltered message words $\expmess_5,\ldots,\expmess_{20}$.
We verify that the boomerang is usable by checking that flipping its constituting bits breaks none of the message bit relations.
We normalize these signed bit differences by negating them all when the state bit is flipped from 1 to 0.
In this manner we obtain a set of usable boomerangs for $\state_t[b]$.
We determine the auxiliary conditions on message bits and state bits and only keep the best usable boomerang that has the fewest auxiliary conditions.

Secondly, we analyze the behaviour of the boomerang over the backwards steps.
For every sampled solution, we simulate the application of the boomerang by flipping the bits of the boomerang.
We then recompute steps 4 to 0 backwards and verify if any sufficient condition on these steps is broken.
Any boomerang that breaks any sufficient conditions on the early steps with probability higher than 0.1 is dismissed.

Thirdly, we analyze the behaviour of the boomerang over the forward steps.
For every sampled solution, we simulate the application of the boomerang by flipping its constituting bits.
We then recompute steps 21 up to 79 forwards and keep track of any sufficient condition for the differential path that becomes violated.
A boomerang will be used at step $i$ in our attack if it does not break any sufficient condition up to step $i-1$ with probability more than 0.1.

\paragraph{Searching for single neutral bits.}
The neutral bit analysis uses the same overall approach as the boomerangs, with the following changes.
After boomerangs are determined, their conditions are added to the previous attack conditions and used to generate a new set of solution samples.
Usable neutral bits consist of a set of one or more message bits that are flipped simultaneously.
However, unlike for boomerangs, the reason for flipping more than one bit is to preserve message bit relations, and not to control the
propagation of a state difference.
Let $t\in[5,20], b\in[0,31]$ be a candidate neutral bit; flipping $\expmess_t[b]$ may possibly break certain message bit relations.
We express each message bit relation over $\expmess_5,\ldots,\expmess_{20}$ using linear algebra, 
and use Gaussian elimination to ensure that each of them has a unique last message bit $\expmess_i[j]$ (\ie where $i*32+j$ is maximal).
For each relation involving $\expmess_t[b]$, let $\expmess_i[j]$ be its last message bit.
If $(i,j)$ equals $(t,b)$ then this neutral bit is not usable (indeed, it would mean that its value is fully determined by earlier message bits).
Otherwise we add bit $\expmess_i[j]$ to be flipped together with $\expmess_t[b]$ as part of the neutral bit.
Similarly to boomerangs, we dismiss any neutral bit that breaks sufficient conditions backwards with probability higher than 0.1.
The step~$i$ in which the neutral bit is used is determined in the same way as for the boomerangs.

%The search for single neutral bits followed a straightforward application of the technique: many partial solutions for the differential path were generated, which
%allowed to test the effect changing the value of individual bits had on the satisfiability of the path (including the backward conditions). We experimentally
%set a threshold of $p \approx 0.9$ for a bit to qualify at a given step; \ie a bit is chosen to be a neutral bit at the latest step where it does not interact with
%path conditions with probability more than $p$ (although this choice was slightly moderated by some implementation considerations).
%
%The selection of the boomerang neutral bits was done by following the approach of Joux and Peyrin~\cite{DBLP:conf/crypto/JouxP07}, where sufficient conditions are
%added in the differential path to ensure that activating the boomerang does not introduce differences before a certain (late) step. From the potential boomerang
%candidates, we then only kept the ones that did not interact badly with the backward conditions and that were compatible with the message bit-relations.
%
%\medskip
%
%The process mentioned above only departs from the classical approach by its additional requirements on backward interactions. While this can be dealt with in
%a straightforward way,
%the main task with respect to the neutral bits when developing the attacks was to choose which offset to use for the window of freedom in the message. This is
%the main value that determines how much will be gained compared to the usual setting, but there is a tradeoff to be found between allowing freedom up to a late step
%(\ie choosing a large offset) and ensuring that the backward propagation can be satisfied with high probability (which is easier to achieve when choosing a small
%offset).
%
%The choice of this offset was done experimentally in our attacks, and resulted in choosing an offset of 6 (resp. 5) in the 76-step (resp. 80-step) case.

\subsubsection{The shifted initialization and the attack process}
\label{sec:free_att}

In the implementations of our attacks, we only apply neutral bits on fully determined message blocks. In other words, neutral bits are used to try to extend to more steps
partial solutions for the differential path which cannot be extended by using more freedom in the message.
We call \emph{base solutions} such partial solutions; each of them consists of an (expanded) message and an \iv
such that 21 consecutive state words (one for each free message word plus the \iv) follow the differential path:

\begin{defi}[Base solution]
A \emph{base solution} for a differential path $\mathcal{P}$ of offset $i \in \{0,\ldots,64\}$ is a pair $(\expmess,  \wp_5(\state) = (\state_{j},\ldots,\state_{j+4}; j \in \{i,\ldots i + 16\})$
such that the state words $\state_i, \ldots, \state_{i+20}$ entailed by $\expmess$ and $\wp_5(\state)$ satisfy all necessary conditions imposed on them by $\mathcal{P}$ (\ie is a partial solution for $\mathcal{P}$).
\end{defi}

Although we may define base solutions with an offset as above, one should note that this offset is not directly related to the main block offset; in particular, both offsets do not need to be equal, and they
are indeed different in our attacks. The reason of that difference is that the two offsets represent different things: the base solution offset defines which state words have their conditions pre-satisfied
before any neutral bits are applied, while the main block offset defines which state words are modified through the action of the neutral bits. Thus we can see that we typically want the base solution
to have an offset only if there are no path conditions on the lower state words that become ignored by the initialization as a result. If this were not the case, the conditions would need to be
satisfied probabilistically, likely with no accelerating technique to make that process efficient. On the other hand, the main block offset is chosen to be the one yielding the best neutral bits, and
it is chiefly limited by the magnitude of the backward interactions it entails.
Let us make this more concrete by discussing the specific cases of our attacks.

\medskip

The first state condition in both attacks appears on $\state_{-3}$, \ie the second (original) \iv word. Consequently, we chose a base solution offset of one in both cases. 
Thus, a base solution consists of state words $\state_{-3},\ldots,\state_{17}$ specified through a five-word \iv and an expanded message such that all path conditions are verified by
this partially computed state.

The search for a base solution consists in specifying a suitable ``\iv'' $\wp_5(\state)$ and then finding a message producing a valid partial solution. For instance, in the case of the 76-step
attack, this process is instantiated by first choosing an initial solution over $\state_8,\ldots,\state_{12}$ and then extending it backward using message words $\expmess_{11},\ldots,\expmess_1$
and forward using words $\expmess_{12},\ldots,\expmess_{16}$.

For both attacks, the implementation of this search directly uses the path conditions, such as they are computed at the end of the generation of the differential path.
%TODO expand?

%TODO restructure ? Attack process ?
\paragraph{Attack process.}
We already mentioned that the main block offset is 6 (resp. 5) for the 76-step (resp. 80-step) attack. Let us detail here the effect of this offset on the base solution and on the attack process
in a bit more details, considering the 76-step case (without losing much generality).

In the course of the attack, the window of freedom for the message is made of $\expmess_6,\ldots,\expmess_{21}$, and all state conditions are known to be satisfied for steps $-4$ to $17$.
The value of the message words cannot be changed entirely, because they must conform to the base solution. It is thus not possible to \eg fix the value of $\expmess_{17}$ so that all conditions
for $\state_{18}$ become satisfied. However, a certain number of neutral bits have been determined for this message window. The attack then consists in searching for good patterns of active and
inactive neutral bits that allow to satisfy path conditions in state words past $\state_{17}$. These bits were chosen so that they do not interact badly with the already fulfilled conditions,
first of the base solution, and then of the further state words that are iteratively added to the partial solution, while allowing to try many different messages. The best neutral bit does not
interact w.h.p. with any condition before $\state_{26}$ excluded; in general, for each previous step starting from $\state_{18}$, some fresh neutral bits are available. Past $\state_{26}$,
the attack switches to a purely probabilistic phase.

We show in \autoref{fig:attack_layout} the starting point for the attack process in the 76-step case. This figure shows the state words for which the conditions were satisfied in the base solution
(\,\begin{tikzpicture}[scale=0.25]\draw[thick,pattern=horizontal lines] (0,0) rectangle (1,1);\end{tikzpicture}\,\,\begin{tikzpicture}[scale=0.25]\draw[thick,pattern=vertical lines] (0,0) rectangle (1,1);\end{tikzpicture}\,),
the message words where neutral bits are located (\,\begin{tikzpicture}[scale=0.25]\draw[thick,pattern=crosshatch] (0,0) rectangle (1,1);\end{tikzpicture}\,) and the remaining (untouched) free message
window (\,\begin{tikzpicture}[scale=0.25]\draw[thick,pattern=north west lines] (0,0) rectangle (1,1);\end{tikzpicture}\,),
and the state words where the cost of satisfying the path conditions is amortized through the action of the neutral bits (\,\begin{tikzpicture}[scale=0.25]\draw[thick,pattern=dots] (0,0) rectangle (1,1);\end{tikzpicture}\,).

\begin{figure}[htb]
  \begin{center}
  \begin{tikzpicture}[scale=0.22,>=latex']
    % teh boxez
    \draw[thick] (0,0) -- (20,0) -- (20,-34);
    \draw[thick,dashed] (20,-34) -- (15,-33) -- (10,-34) -- (5,-33) -- (0,-34);
    \draw[thick] (0,-34) -- (0,0);
    \draw[thick] (25,-4) -- (45,-4) -- (45,-34);
    \draw[thick,dashed] (45,-34) -- (40,-33) -- (35,-34) -- (30,-33) -- (25,-34);
    \draw[thick] (25,-34) -- (25,-4);
    % teh row indices for the state
    \node at (-2,0) {-4};
    \node at (-2,-12) {8};
    \node at (-2,-16) {12};
    \node at (-2,-18) {14};
    \node at (-2,-21) {17};
    \node at (-2,-25) {21};
    \node at (-2,-30) {26};
    % teh row indices for the message
    \node at (23,-4) {0};
    \node at (23,-9) {5};
    \node at (23,-14) {10};
    \node at (23,-19) {15};
    \node at (23,-24) {20};
    \node at (23,-29) {25};
    \node at (23,-34) {30};
    % the fillings
    \draw[pattern=vertical lines] (0,-1) rectangle (20,-12); % base solution
    \draw[pattern=horizontal lines] (0,-12) rectangle (20,-16); % IV
    \draw[pattern=vertical lines] (0,-16) rectangle (20,-21); % base solution
    \draw[pattern=dots] (0,-21) rectangle (20,-30); % where neutral bits act
    \draw[pattern=north west lines] (25,-10) rectangle (45,-25); % free message words
    \draw[pattern=crosshatch] (25,-18) rectangle (45,-25); % message words with neutral bits

    % legends
    \draw[thick] (50,-14) rectangle (51.5,-15.5);
    \node[anchor=text] at (52.5,-15.5) {Starting point};
    \draw[pattern=horizontal lines] (50,-14) rectangle (51.5,-15.5);
    %
    \draw[thick] (50,-18) rectangle (51.5,-19.5);
    \node[anchor=text] at (52.5,-19.5) {Base solution};
    \draw[pattern=vertical lines] (50,-18) rectangle (51.5,-19.5);
    %
    \draw[thick] (50,-22) rectangle (51.5,-23.5);
    \node[anchor=text] at (55,-23.5) {Free};
    \node[anchor=text] at (50,-25) {message window};
    \draw[pattern=north west lines] (50,-22) rectangle (51.5,-23.5);
    \draw[thick] (52.5,-22) rectangle (54,-23.5);
    \draw[pattern=crosshatch] (52.5,-22) rectangle (54,-23.5);
    %
    \draw[thick] (50,-26) rectangle (51.5,-27.5);
    \node[anchor=text] at (52.5,-27.5) {Message with};
    \node[anchor=text] at (52.5,-29) {neutral bits};
    \draw[pattern=crosshatch] (50,-26) rectangle (51.5,-27.5);
    %
    \draw[thick] (50,-30) rectangle (51.5,-31.5);
    \node[anchor=text] at (52.5,-31.5) {Affected by};
    \node[anchor=text] at (52.5,-33) {neutral bits};
    \draw[pattern=dots] (50,-30) rectangle (51.5,-31.5);

    %title
    \node[anchor=text] at (0,3) {\shaone State words ($\mathcal{A}_i$)};
    \node[anchor=text] at (25,3) {\shaone Message words ($\mathcal{W}_i$)};

\end{tikzpicture}

  \end{center}
  \caption{The freestart attack layout made of a base solution and a message with offset.}
  \label{fig:attack_layout}
\end{figure}
