\section{Hash functions}
\label{sec:intro}

A (binary) hash function is a mapping from bit strings of arbitrary length (the messages) to strings of a fixed predetermined length (the digests, or hashes):
$\hash : \{0,1\}^* \rightarrow \{0,1\}^n$ for some integer $n$.
Many hash functions do not strictly adhere to this definition, as they upper-bound the length of their inputs by a (usually) large integer such as $2^{64}$. This currently has
no impact in practice, as all potential inputs to a hash function are much shorter than these upper limits; it is of course debatable whether this will always remain the case. 
The typical output length of hash functions is of a few hundred bits, often a multiple of thirty-two; popular hash functions of the present and the past may be found with
$n \in \{128, 160, 224, 256, 384, 512\}$.

A \emph{cryptographic} hash function\footnote{We will obviously only consider such functions from now on.} is a hash function $\hash$ that verifies a certain number of \emph{security properties}, which express the difficulty of computing inputs
to $\hash$ that verify some conditions. There are three ``classical'' security properties that must be met (at least to some extent)
by any secure hash function; in an informal way, they can be expressed as:
\begin{defi}[Preimage resistance] Given a target $\targ$, an \emph{adversary} must not be able to find a message $\messhash$ such that $\hash(\messhash) = \targ$ with non-negligible probability using
less computational resources than what is required to compute $\hash$ $\bigo(2^n)$ times.
\label{def:pre}
\end{defi}
\begin{defi}[Second preimage resistance] Given a message $\messhash$, an adversary must not be able to find a different message $\messhash'$ such that
$\hash(\messhash) = \hash(\messhash')$  with non-negligible probability using less computational resources than what is required to compute $\hash$ $\bigo(2^n)$ times.
\label{def:2pre}
\end{defi}
\begin{defi}[Collision resistance] One must not be able to find two distinct ``fresh'' messages $\messhash$ and $\messhash'$ such that $\hash(\messhash) =
\hash(\messhash')$ with non-negligible probability using less computational resources than what is required to compute $\hash$ $\bigo(2^{n/2})$ times.
\label{def:coll}
\end{defi}

The complexities associated with these definitions correspond to the ones of the best known \emph{generic} algorithms that can find messages with the desired
properties with high probability for any function. An \emph{attack} on a specific hash function is an algorithm that can achieve one of the above tasks (significantly)
more efficiently than the best generic algorithm. By definition, this violates the associated security property.
Above, we define the complexity of these generic algorithms in terms of calls to $\hash$. In practice, it may be necessary to adjust
this to a finer granularity; for instance, for the \merkdam hash functions that we describe in \autoref{sec:mdhf}, it makes more sense to evaluate the complexity
in terms of calls to what will be defined as a \emph{compression function}.

As a side remark, the lower complexity target of $\bigo(2^{n/2})$ associated with collision resistance is a direct consequence of the so-called \emph{birthday paradox} (which
is encountered frequently in cryptography): informally,
$N$ elements define $\bigo(N^2)$ pairs, thus if these elements are drawn at random from a set of size $S$, we expect two of them to be the same (\ie to collide)
with high probability after drawing $\bigo(\sqrt{S})$ of them.
One can also notice that unlike (second) preimage resistance, the definition of collision resistance does not include an input target or message. As such, it does not in itself
exclude (generic) algorithms that simply return two constant colliding messages, in constant time. Such algorithms trivially violate the security property, but
they do not say anything about the security of the functions (not the least because of their generic nature). Consequently, they are usually considered irrelevant in a cryptographic context, and they do not make
the definition devoid of sense.

\subsection{Applications of hash functions}

The nature of the properties used to define security for cryptographic hash functions is based on requirements from the various cases where they may be used in concrete cryptosystems or protocols.
Depending on the situation, resistance w.r.t. all three properties from above may not be necessary; for instance, it may be the case that collision resistance is not required, or that, say, only second preimage resistance
is relevant. However, a hash function is usually expected to be used in a certain number of settings, and it is thus understandingly expected to be secure against all attacks.

We only briefly sketch some possible uses of hash functions here, as an illustration.

\paragraph{Hash and sign signatures} are one of the main settings where hash functions may be employed; the objective is, given a digital signature
algorithm $\sign$, to efficiently and securely sign \emph{long} messages. Directly doing so using $\sign$ is usually not an option, because signature algorithms are rather slow (especially compared
to hash functions, which are typically at least $500 \sim 1000$ times faster) and may not behave well on long messages. A useful alternative is instead to first compute the digest of the
message that needs to be signed (say ``$\messhash$'') and to sign the digest instead of the whole message. One then gives an output of the form $(\messhash, \sign(\key,\hash(\messhash)))$, with $\key$ a private key for $\sign$. 

It is obvious that for such a scheme to work, the function $\hash$ needs to be at least resistant to second preimages. If this were not the case, an adversary could intercept a message $\messhash$
signed by $A$, replace $\messhash$ by $\messhash'$ such that the two messages collide through $\hash$, and claim that $A$ signed $\messhash'$.
Collision resistance is also important for similar reasons.

\paragraph{Password hashing} is another common setting where hash functions may be useful. In this case, we assume that a certain entity wishes to allow users to authenticate themselves
through passwords. Consequently, it must remember the valid password of each user. As there would be obvious security issues with the entity storing the passwords themselves, an idea
is to instead store their images through a hash function $\hash$. Thus, if an adversary finds the database of users and their associated (hashed) passwords, he would be unable to find
the passwords (or some equivalent input), provided that $\hash$ is preimage-resistant.

In this setting, second preimage and collision resistance are not strictly needed. However, it should be noted that even when built from a secure hash function,
the scheme just described above has severe issues, which descriptions are beyond the scope of this article.

\paragraph{Hash-based signatures.} A less common use of hash functions is to utilise them to directly define signature schemes (see \eg \cite{DBLP:conf/crypto/Merkle87}). We will not describe such schemes in detail
here, but it is interesting to mention a few of their specificities. Similarly as other systems using hash functions, only preimage resistance is needed. The main
idea of the schemes is that the signing party makes some digests public while keeping their inputs secret; signing a message then consists in selectively revealing some of
these inputs. Thus, being able to compute preimages for the hash function breaks the scheme, but collisions are not a threat.
A distinguishing feature of hash-based signatures is that the hash function may only need to be used on inputs of short, possibly fixed size. 

\paragraph{Message authentication codes.} The last possible usage of hash functions that we describe here is the building of \emph{message authentication codes}, or MACs, which
can somehow be seen as the keyed variants of hash functions. A MAC takes as input a key $\key$ and a message $\messhash$ and outputs a tag $\tagg = \mac(\key, \messhash)$.
If an adversary does not know the key, it should be hard for him to find a valid (message, tag) pair for $\mac(\key, \cdot)$, with the message either being of his choosing
or imposed by a challenger. These notions correspond to \emph{existential} (for the former) and \emph{universal} (for the latter) \emph{forgery}. Of course, it is also
highly necessary that no tag collisions occur, whether or not these happen on specific messages requested by an adversary.

Hash functions seem to be good candidates to build MACs, and indeed generic hash function based constructions such as HMAC~\cite{DBLP:conf/crypto/BellareCK96} are popular. However, the exact security of
these constructions is not always easy to establish, and there are usually faster alternatives such as MACs based on universal/polynomial hash functions (see \eg~\cite{DBLP:conf/crypto/BlackHKKR99}).

\section{\merkdam hash functions}
\label{sec:mdhf}

One of the first frameworks for hash function design to have been developed is the so-called \merkdam construction, that was independently developed by
Merkle and Damg\aa rd in 1989~\cite{DBLP:conf/crypto/Merkle89a,DBLP:conf/crypto/Damgard89a}. The idea of this construction is to make the arbitrary-length
inputs to a hash function manageable by defining the latter as the iteration of a \emph{compression function} with a small fixed-size (co-)domain.
This makes the overall design much easier, but without \emph{a priori} ensuring that the resulting function will be secure. The main contribution of
Merkle and Damg\aa rd in that respect is to give a construction such that the security of the function can be (partially) reduced to the one of
the compression function: they show that for \eg collision attacks, an attack on the hash function can be exploited to build an attack on the compression function.
Taking the contrapositive, as long as there are no collision attacks on the compression function, we can be confident that the hash function is secure against this kind of attacks.
This is quite similar in a way to building symmetric cryptosystems by combining a secure block cipher and a secure mode of operation.

The construction works as following. A compression function $\compress : \{0,1\}^n \times \{0,1\}^b \rightarrow \{0,1\}^n$
takes two inputs: a \emph{chaining value} $\chain$ and a \emph{message block} $\messblock$, and produces another chaining value as output.
The hash function $\hash$ associated with $\compress$ is built by extending the domain of the latter to $\{0,1\}^*$ (or rather $\{0,1\}^N$ for a large $N$, most of the time).
This is done by specifying an \emph{initial value} \iv for the first chaining value $\chain_0$, which is a constant for the hash function, and by defining the image
of a message $\mess$ through $\hash$ by the following process:
\begin{enumerate}
\item $\mess$ is padded to a size multiple of the message block size $b$. Various padding rules may be employed, but it is obviously important that they should not
introduce trivial collisions. Most of the time, the size (usually in bits) of the non-padded message $\mess$ is included in the padding in one way or the other.
This is usually called \merkdam-strengthening, and it is essential to make many of the common instantiations of the framework secure\footnote{One issue that
could arise in the absence of such strengthening is that fixed-points for the compression function could be used to produce collisions; such fixed-points
are easy to build for some popular compression function constructions (including the \emph{Davies-Meyer} construction
of the \mdsha family described in \autoref{sec:mdsha}).}.
\item The padded message is then iteratively fed to the compression function: call $\messblock_0||\messblock_1||\linebreak\ldots||\messblock_r$ this message (assuming
it is on $r+1$ blocks, w.l.o.g.), then define $\chain_{i+1} \defas \compress(\chain_i, \messblock_i)$. The digest $\hash(\mess)$ is equal to the last
chaining value $\chain_{r+1}$.
\end{enumerate}
We give an illustration of this process in \autoref{fig:merkdam}.

\begin{figure}[!htb]
\begin{center}
%\documentclass{standalone}
%\usepackage{rotating}	%% introduce sideways env
%\usepackage{tikz}
%
%%% Public TikZ libraries
%\usetikzlibrary{positioning}
%\usetikzlibrary{shapes}
%
%%% Custom TikZ addons
%\usetikzlibrary{crypto.symbols}
%\tikzset{shadows=no}        % Option: add shadows to XOR, ADD, etc.
%
%%% Document
%\begin{document}
\begin{tikzpicture}[scale=0.4]
	\path[anchor=east] (-1,0.5) node {$pad(m)=$};
	\draw[fill=Fuchsia!20,thick,inner sep=2ex] (0,0) rectangle (16,1);

	%% Separations in the message
	\draw[thick] ++( 4,0) -- ++(0,1); \path (   2,0.5) node {$\messblock_{0}$};
	\draw[thick] ++( 8,0) -- ++(0,1); \path ( 4+2,0.5) node {$\messblock_{1}$};
	\draw[thick] ++(12,0) -- ++(0,1); \path ( 8+2,0.5) node {$\messblock_{2}$};
	\draw[thick] ++(16,0) -- ++(0,1); \path (12+2,0.5) node {$\messblock_{3}$};

	%% Compressions functions 
	\begin{scope}[shift={(0.5,-4)}]
		\node [draw,trapezium,trapezium left angle=70,trapezium right angle=70,minimum height=0.7cm,thick,fill=YellowOrange!30,shift={(1.15,0.4)},rotate=-90] 
		{\begin{sideways}$\compress$\end{sideways}};
		\draw[->,thick] ++(1.5,+4) -- ++(0,-2.5) -- ++(0.5,0);
		\draw[->,thick] ++(0,0.5) node[left] {$\chain_{0}=\iv$}-- ++(2,0);
	\end{scope}

	\begin{scope}[shift={(4.5,-4)}]
		\node [draw,trapezium,trapezium left angle=70,trapezium right angle=70,minimum height=0.7cm,thick,fill=YellowOrange!30,shift={(1.15,0.4)},rotate=-90] 
		{\begin{sideways}$\compress$\end{sideways}};
		\draw[->,thick] ++(1.5,+4) -- ++(0,-2.5) -- ++(0.5,0);
		\draw[->,thick] ++(-0.2,0.5) -- node[below] {$\chain_{1}$} ++(2.2,0);
	\end{scope}

	\begin{scope}[shift={(8.5,-4)}]
		\node [draw,trapezium,trapezium left angle=70,trapezium right angle=70,minimum height=0.7cm,thick,fill=YellowOrange!30,shift={(1.15,0.4)},rotate=-90] 
		{\begin{sideways}$\compress$\end{sideways}};
		\draw[->,thick] ++(1.5,+4) -- ++(0,-2.5) -- ++(0.5,0);
		\draw[->,thick] ++(-0.2,0.5) -- node[below] {$\chain_{2}$} ++(2.2,0);
	\end{scope}

	\begin{scope}[shift={(12.5,-4)}]
		\node [draw,trapezium,trapezium left angle=70,trapezium right angle=70,minimum height=0.7cm,thick,fill=YellowOrange!30,shift={(1.15,0.4)},rotate=-90] 
		{\begin{sideways}$\compress$\end{sideways}};
		\draw[->,thick] ++(1.5,+4) -- ++(0,-2.5) -- ++(0.5,0);
		\draw[->,thick] ++(-0.2,0.5) -- node[below] {$\chain_{3}$} ++(2.2,0);
	\end{scope}

	\begin{scope}[shift={(16.5,-4)}]
		\draw[->,thick] ++(-0.2,0.5) -- ++(0.75,0) node[right] {$\chain_4 = \hash(m)$} ;
	\end{scope}
	
\end{tikzpicture}
%\end{document}

\caption{A \merkdam hash function processing a four-block input. Figure adapted from \cite{TiKZ:Cryptographers}.\label{fig:merkdam}}
\end{center}
\end{figure} 

We conclude this presentation by sketching the reduction proofs of the construction for collision and first preimage attacks. There can be no such proof
for second preimages, as there exists some generic attacks on \merkdam functions independently of the security of the compression function. We will
briefly discuss these in \autoref{sec:refining_md}.
It is also important to notice that these reductions are one-way. There is no similarly generic way to convert, say a collision for a compression function $\compress$,
to a collision for a \merkdam function $\hash$ based on it. Still, such a collision would violate the security reduction, and thus no formal guarantee could be given
anymore on the collision resistance of $\hash$. We will discuss such issues in slightly more details in \autoref{sec:refining_md}.

\begin{prop}
A collision on a \merkdam hash function $\hash$ implies a collision on its compression function $\compress$\footnote{Without losing much generality, we assume in this proof
that \merkdam strengthening is used, that the message length is appended at the end of the padding, and that it fits in a single message block.}.
\end{prop}
\begin{proof}
Assume we have $\mess$, $\mess' \neq \mess$ s.t. $\hash(\mess) = \hash(\mess')$.

Case 1: $\mess$ and $\mess'$ have a different length.
The last message blocks $\messblock_r$ and $\messblock'_{r'}$ both include the length
of the message, which is different. Thus $(\chain_r,\messblock_r)$ and $(\chain'_{r'}, \messblock'_{r'})$ are distinct and collide through $\compress$.

Case 2: $\mess$ and $\mess'$ are of the same length. Assume w.l.o.g. that the messages fit on $r + 1$ blocks after padding.
Call $i$ the highest block number such that $\messblock_{i} \neq \messblock'_{i}$. If $i = r$, then $(\chain_r,\messblock_r)$ and $(\chain'_{r}, \messblock'_{r})$
are distinct and collide through $\compress$. If $i < r$, either $\chain_{i+1} = \chain'_{i+1}$ (thus $(\chain_i,\messblock_i)$ and $(\chain'_{i}, \messblock'_{i})$
form a valid collision pair for $\compress$), or, we have a non-empty sequence of input pairs $(\chain_j, \messblock_j)$ and $(\chain'_j, \messblock'_j)$,
$j = i+1\ldots r$ such that $\messblock_j = \messblock'_j$ for all $j$. As $\chain_{r+1} = \chain'_{r+1}$, at least one element of the sequence
collides through $\compress$. The two input pairs in the first one to do so are different, and thus form a valid collision for $\compress$.
\end{proof}

\begin{prop}
A preimage on a \merkdam hash function $\hash$ implies a preimage on its compression function $\compress$.
\end{prop}
\begin{proof}
Let $\mess$ be a message fitting on $r+1$ blocks after padding s.t. $\hash(\mess) = \targ$, with $\targ$ the preimage target.
Then $\compress(\chain_r, \messblock_r) = \targ$, and $(\chain_r, \messblock_r)$ is thus a valid preimage input for $\compress$.
\end{proof}

\section{Refining the security of hash functions}
\label{sec:refining_md}

The three security notions of collision and preimage resistance can be refined and completed with additional ones. These are not always directly relevant to actual uses of hash functions, but they may nonetheless be useful to evaluate the security
of a function in a finer way. We may roughly distinguish between two kinds of additional properties by what they characterize: 1) Non-ideal behaviours of certain frameworks for hash function constructions; 2) Non-ideal
behaviours of specific instances of hash functions or of their building blocks.
To allow for more explicit definitions of these additional properties, it is useful to define a stronger, idealized view of hash functions:

\begin{defi}[Random oracle]
An $n$-bit \emph{random oracle} $\ro$ is a mapping $\{0,1\}^* \rightarrow \{0,1\}^n$ such that for every input $x$, its image $\ro(x)$ is drawn uniformly at random over $\{0,1\}^n$.
\end{defi}

According to the way we defined attacks, a random oracle is not vulnerable to them, as only generic algorithms may be used against it. It thus completely captures what
can be realized with hash functions: if a high-level construction is not secure when it is idealized as using a random oracle (when analysed in the \emph{random oracle model}),
one can never hope to make it so when instantiating the oracle by a concrete hash function. However, even if a construction is secure in such a model, it is not necessarily true that
this will still be the case once instantiated.

The \merkdam functions from \autoref{sec:mdhf} exhibit some of this \emph{non-ideal behaviours},
in the sense that there are some properties that can be realised through algorithms that are more efficient for \merkdam functions than for random oracles.

A good example of such a property is the concept of \emph{multicollision}, where one is required to find $r > 2$ messages $\messhash_{0},\ldots,\messhash_{r-1}$ which images through $\hash$
are all equal. The generic complexity (\eg for a random oracle) of this problem is $\bigo(2^{n\times (r-1)/r})$ calls to $\hash$ for an $n$-bit function, but Joux showed how to
find $2^r$-multicollisions in time $\bigo(r\times 2^{n/2})$ for \merkdam hash functions~\cite{DBLP:conf/crypto/Joux04}.
The basic idea used in this attack is that collisions for a \merkdam function $\hash$ can be chained together to lead to an exponentially growing number of distinct messages hashing
to the same value. We can easily see this with a small example: assume that an attacker found two distinct colliding messages $\mess_0$ and $\mess'_0$ of the same length;
this can be done generically at a cost of $2^{n/2}$. One then looks for a second collision for the function $\widetilde \hash$ obtained by replacing the initial chaining
value $\chain_0$ by $\hash(\mess_0) = \hash(\mess'_0)$; this can again be obtained for a cost of $2^{n/2}$, resulting in $\mess_1$ and $\mess'_1$. Then, thanks to
the chaining property of the \merkdam construction, we have found four colliding messages $\mess_0||\mess_1$, $\mess'_0||\mess_1$, $\mess_0||\mess'_1$, $\mess'_0||\mess'_1$\footnote{We ignore
padding details here, but these can be easily worked out.}. It is easy to see how this generalizes to longer messages, leading to the quoted complexity of $\bigo(r \times 2^{n/2})$.
We can observe that one of the weaknesses of the construction that is exploited here is that a hash collision is also
a collision for the \emph{internal state} of the hash function. We will briefly see in \autoref{sec:betterhash} that increasing the size of the state of a function is indeed a way to make it resistant
to such attacks.

Another good example for \merkdam, which this time directly violates the security property from \autoref{def:2pre}, consists in second preimage attacks for long messages.
Dean~\cite{dean}, and later Kelsey and Schneier~\cite{DBLP:conf/eurocrypt/KelseyS05} showed how one can again exploit the structure of the function and internal collisions to compute a second preimage
of a long message more efficiently than with a generic algorithm. The complexity of Kelsey and Schneier's attack to find a second preimage for a message
of $2^k$ \emph{blocks} is $\approx \bigo(2^{n-k+1})$ calls to $\hash$. This means that \merkdam hash functions are actually inherently insecure, if we adhere
strictly to what we stated as security objectives. However, although significant, these attacks remain expensive, especially for messages of usual sizes. As such, they are not usually considered as threatening
the practical use of \merkdam functions, and indeed functions following this framework such as \shatwo~\cite{Nist-SHA} are still widely used.

\bigskip

We already mentioned in \autoref{sec:mdhf} that an attack on a compression function would void the security reduction of a \merkdam function employing it. Thus it seems natural to also
analyse the security of the compression functions themselves. An attack would in this case demonstrate a non-ideal behaviour of the second kind we mentioned above, not targeting
a hash function in itself but one of its building blocks. There is a natural way to generalise the security properties expressed in \autoref{def:pre} $\sim$ \autoref{def:coll} to this context,
leading to the notion of \emph{(semi-)freestart} attacks:

\begin{defi}[Freestart preimage]
A \emph{freestart preimage} for a Merkle-Damg\aa rd hash function $\hash$ is a pair $(\freeiv,\messblock)$
of an \iv and a message such that $\hash_\freeiv(\messblock) = \targ$, with $\hash_\freeiv(\cdot)$ denoting
the hash function $\hash$ with its original \iv replaced by $\freeiv$.
\label{def:free_pre}
\end{defi}

\begin{defi}[Semi-freestart collision]
A \emph{semi-freestart collision} for a Merkle-Damg\aa rd hash function $\hash$ is a pair $((\freeiv,\messblock), (\freeiv,\messblock'))$
of two \iv and message pairs such that $\hash_\freeiv(\messblock) = \hash_{\freeiv}(\messblock')$.
\label{def:semi-free_coll}
\end{defi}

\begin{defi}[Freestart collision]
A \emph{freestart collision} for a Merkle-Damg\aa rd hash function $\hash$ is a pair $((\freeiv,\messblock), (\freeiv',\messblock'))$
of two \iv and message pairs such that $\hash_\freeiv(\messblock) = \hash_{\freeiv'}(\messblock')$.
\label{def:free_coll}
\end{defi}

It can be noted that if the two messages of, say, a freestart collision pair are one-block long, the definition above becomes equivalent to the one of a collision
on the compression function $\compress$ used to build $\hash$.
There is little difference between the two in general, the feature of freestart attacks precisely being that compared to hash function attacks, they may (partially) exploit the additional available input offered by the chaining value
of the compression function.


\bigskip

Lastly, a somehow loosely defined addition to the security notions presented so far is the concept of \emph{distinguishers}, which denote non-ideal behaviours
of a hash function that are not otherwise captured by the previous definitions. We will not give a precise definition here, as these are made tricky by
the unkeyed nature of hash functions. Instead, we just briefly mention an example of such a distinguisher for the compression function of \shaone.

Jumping ahead, we will see in \autoref{sec:description} that the \iv and the message block of \shaone's compression function are made of five and sixteen
32-bit words respectively. In 2003, Saarinen showed that \emph{slid pairs} could be found for this compression function for a cost equivalent to $2^{32}$
function calls~\cite{DBLP:conf/fse/Saarinen03}. Such pairs are made of two \ivs $\state_{0,\ldots4}$, $\state'_{0,\ldots,4}$ and messages $\mess_{0,\ldots,15}$,
$\mess'_{0,\ldots,15}$ with $\state'_i = \state_{i-1}$ and $\mess'_i = \mess_{i-1}$\footnote{We skip here the details of how to determine the starting values $\state'_0$ and $\mess'_0$.}, such that the
pair of outputs of the function called on these inputs also has this property. Although it is not expected of a random function to exhibit such a property, there is no clear way to use it to mount
an attack against the main and secondary security properties defined above.


\section{Modern hash function frameworks}
\label{sec:betterhash}

We have mentioned some generic weaknesses of the \merkdam framework in \autoref{sec:refining_md}. As a consequence of these, modern hash function designs are usually based on alternative,
more secure constructions. We briefly review two of them: the \emph{wide-pipe} variation of \merkdam, or \emph{chop-MD}, and the \emph{sponge} construction.

\paragraph{Wide-pipe \merkdam.} The wide-pipe construction was introduced in 2005 by Lucks~\cite{DBLP:conf/asiacrypt/Lucks05} and Coron \etal under the name chop-MD~\cite{DBLP:conf/crypto/CoronDMP05}.
It is conceptually simple, and consists in using the \merkdam construction with a compression function of output size larger than the one of the chaining value. If we write $\lfloor\cdot\rfloor_n$
an arbitrary truncation function from $m > n$ to $n$ bits, we may define the $n$-bit chop-MD construction based on a compression function $\compress : \{0,1\}^m \times \{0,1\}^b \rightarrow \{0,1\}^m$
as $\lfloor\hash(\cdot)\rfloor_n$, with $\hash$ a standard \merkdam function built from $\compress$. Some variations are possible, for instance by considering other mappings from $m$ to $n$
bits instead of just a truncation.

One can easily see that a hash collision for such a function does not anymore imply a collision for its internal state. By choosing $m$ to be sufficiently large (for instance taking $m = 2n$),
one can achieve generic resistance to, say, multicollisions. In fact, Coron \etal proved that this construction is a secure \emph{domain extender for random oracles}, in the sense that
if the compression function is a fixed-size random oracle, using it in a chop-MD mode yields a function that is $\varepsilon$-indifferentiable from a random oracle (in the sense of Maurer \etal~\cite{DBLP:conf/tcc/MaurerRH04})
with $\varepsilon \approx 2^{-t}q^2$, $t = m - n$ being the number of truncated bits and $q$ the number of queries to $\compress$. In the light of \autoref{sec:refining_md}, this is a very useful result, as it
says that unlike plain \merkdam, no non-ideal behaviour is introduced by the domain-extending construction. Such hash functions are thus expected to behave closely to the random oracles they may be expected to instantiate, as long as
their compression functions are ``ideal'' and that they are not queried too much w.r.t. to the size of their parameters.

\paragraph{Sponge construction.} The sponge construction was introduced in 2007 by Bertoni \etal \cite{SpongeFunctions}. It is quite distinct from the \merkdam framework, notably because it does not use a compression
function as building block, but a function of equally-sized domain and co-domain. Usually, this function is even taken to be bijective.

The construction in itself is simple. Assume we want to build an $n$-bit function based on a $b$-bit permutation $\perm$. We define the \emph{rate} $r$ and the \emph{capacity} $c$ as two integers such that
$b = r + c$. Then, hashing the message $\mess$ consists in padding it to a length multiple of $r$ and to process it iteratively
in two phases. The \emph{absorbing} phase computes an internal state value $\freeiv \defas \perm(\perm(\ldots\perm(\messblock_0||0^c)\oplus\messblock_1||0^c)\ldots)$. The squeezing phase then produces the
$n$-bit output as $\hash(\mess) \defas \lfloor\freeiv\rfloor_r||\lfloor\perm(\freeiv)\rfloor_r||\ldots||\lfloor\perm^{n \div r}(\freeiv)\rfloor_{n \mod r}$.

A distinguishing feature of the sponge construction (apart from the fact that it does not use a compression function) is that the output length of an instance is entirely decorrelated from the size of its building
block. Thus, it allows to swimmingly build variable-length hash functions. A given permutation can also be used in different instantiations offering a tradeoff between speed (a larger rate giving
faster functions) and security (a larger capacity giving more secure functions).
In particular, Bertoni \etal showed in 2008 that similarly as chop-MD, the sponge construction instantiated with a random function or permutation is $\varepsilon$-indifferentiable from a random oracle
of the same output size, with $\varepsilon \approx 2^{-c}q^2$~\cite{DBLP:conf/eurocrypt/BertoniDPA08}. To achieve the classical security requirements of a hash function, it is thus optimal to take $c = 2n$.

One of the best examples of a sponge function is \keccak~\cite{KeccakReference}, which became the \shathree standard in 2015~\cite{Nist-SHA3}. 

\section{The \mdsha family}
\label{sec:mdsha}
We now briefly present the ``\mdsha'' hash function family, both because it is of somewhat historical importance and because, the function studied in this article, \shaone, is one of its members.

The family originated in 1990 with the \mdfour function, introduced by Rivest~\cite{Rivest-md4}. An attack on a reduced version was quickly found by den Boer and Bosselaers~\cite{DBLP:conf/crypto/BoerB91},
and Rivest proposed \mdfive as a strengthened version of \mdfour~\cite{Rivest-md5}. Bosselaers proposed \ripemd in 1992 as another attempt to strengthen \mdfour~\cite[Chap. 3]{DBLP:books/sp/BosselaersP95},
and the NSA did the same the following year by introducing the first generation of the SHS/SHA algorithms~\cite{Nist-SHA0}. Both algorithms were quickly modified in 1996 and 1995 respectively~\cite{DBLP:conf/fse/DobbertinBP96,Nist-SHA1}.
Some later algorithms such as \shatwo~\cite{Nist-SHA}, introduced in 2002, also trace their roots back to \mdfour.

Their are some variations inside members of the family; notably, \ripemd uses a parallel structure for its compression function. We specifically list below features that are shared by \mdfour, \mdfive and \sha, but they are also
true for other \mdsha functions to a large extent.
\begin{itemize}
\item The \merkdam construction is used as a domain extender.
\item The compression function is built from an \emph{ad-hoc} block cipher used in a \emph{Davies-Meyer} mode: let $\blockE(x,y)$ be the encryption of the plaintext $y$ with the key $x$ by the cipher $\blockE$, then
the compression function is defined by
$\chain_{i+1} \defas \blockE(\messblock_{i+1},\chain_i) \boxplus \chain_i$.
\item The block cipher inside the compression function is an unbalanced Feistel network that uses modular additions, XORs, bit rotations, and bitwise Boolean functions as constitutive elements. Its key schedule is linear, and (very) fast to compute. 
\end{itemize}

With the advent of the NIST \shathree competition, which ran from 2007 to 2012, much more diversity was introduced to hash function design. However, the \mdsha family still had an influence on many competition candidates, and it remains
influential as of today.

%\subsection{Contributions of this thesis}
%
%In this chapter, we present a joint work with Thomas Peyrin and Marc Stevens on freestart collisions for the \shaone hash function, \ie attacks breaking the security of the function according to \autoref{def:free_coll}. Hash function
%collisions (attacks w.r.t. \autoref{def:coll}) had already been presented by Wang \etal in 2005~\cite{DBLP:conf/crypto/WangYY05a}, but their high complexity made the explicit computation of a collision for the full function unfeasible
%at the time (and no such collision was publicly known at the time where this thesis was written). The novelty of our work comes from the fact that we were able to explicitly compute collisions for the entire compression function
%of \shaone. Furthermore, most of the methods developed during this research can be smoothly translated to the hash function case, which lead to more accurate, cheaper estimates for the cost of a hash function collision.
%
%This work was published in two papers: a first attack on \shaone reduced to 76 steps was published at CRYPTO~2015~\cite{DBLP:conf/crypto/KarpmanPS15},
%and an improved attack on the full 80-step compression function was published at EUROCRYPT~2016~\cite{DBLP:conf/eurocrypt/StevensKP16}. 
%
%\medskip
%
%In a joint work with Thomas Espitau and Pierre-Alain Fouque, we also improved the preimage attacks on \shaone. This led to another paper published at CRYPTO~2015 \cite{DBLP:conf/crypto/EspitauFK15}. However, we do not detail this contribution
%in this manuscript.
