\section{Notation}
\label{sec:not}

We use some notation in this chapter, which also carries over to the next one.
\autoref{table:symbs} gives the meaning of various standard symbols and \autoref{table:appbitconditions} shows the signification of the symbols used to denote bit differences between two states.
Additionally, we use the following conventions: \shaone states, messages, and expanded messages are respectively denoted by $\state$, $\mess$, $\expmess$; for a variable $x$, the corresponding variable related by a specific
difference is noted $\widetilde{x}$, the difference itself is denoted by $\diff(x,\widetilde{x})$ or $\diff x$; two different variables of the same type are noted $x$, $x'$ when their difference is not specific;
a variable that can be seen as an array can have its words accessed through a subscript, indices starting from zero, \eg $x_2$ is the third word of $x$; a variable that can be seen as a fixed-size binary word
can have its bits accessed through a bracket notation, indices starting from zero, \eg $x[31]$ is the thirty-second bit of $x$; 
numbers written in hexadecimal use a fixed-space font and the \emph{0x} prefix, \eg \texttt{0x1337}; we sometimes also write numbers in base two, in which case they are written with a subscript $2$, \eg $1010_2$. Finally, `$\defas$' is used to denote equality by definition.
Various additional shorthands are introduced throughout the text.

\begin{table}[!htb]
\caption{Meaning of standard symbols.}\label{table:symbs}
\begin{center}
\begin{tabular}{c c}
\toprule
Symbol & Meaning\\
\midrule
$\oplus$ & Bitwise exclusive or\\
$+$ & Modular addition\\
$\boxplus$ & Modular addition, word-wise modular addition\\
$-$ & Modular subtraction\\
$\vee$ & Bitwise logical or\\
$\wedge$ & Bitwise logical and\\
$\neg$ & Bitwise complementation\\
$\circlearrowleft r$ & Bit rotation by $r$ to the left\\
$\circlearrowright r$ & Bit rotation by $r$ to the right\\
\bottomrule
\end{tabular}
\end{center}
\end{table}

\begin{table}[!htb]
{\centering\small%
\caption{Meaning of the bit difference symbols, for a symbol located on $\state_t[i]$. The same symbols are also used for $\expmess$.}\label{table:appbitconditions}
\begin{tabular}{c c}
  \toprule
  Symbol & Condition on ($\state$,$\dstate$) \\
	\midrule
  \nodiff & $\state_t[i] = \dstate_t[i]$  \\
  \onediff & $\state_t[i] \neq \dstate_t[i]$  \\
  \onediffu & $\state_t[i] = 0$, \quad $\dstate_t[i] = 1$  \\
  \onediffd & $\state_t[i] = 1$, \quad $\dstate_t[i] = 0$  \\
  $\mnodiffz$ & $\state_t[i] = \dstate_t[i] = 0$\\
  $\mnodiffo$ & $\state_t[i] = \dstate_t[i] = 1$ \\
  \midrule
  \equaup & $\state_t[i] = \dstate_t[i] = A_{t-1}[i]$ \\
  \diffup & $\state_t[i] = \dstate_t[i] \neq A_{t-1}[i]$ \\
	\equarightup & $\state_t[i] = \dstate_t[i] = (A_{t-1} \circlearrowright 2)[i]$  \\
	\diffrightup & $\state_t[i] = \dstate_t[i] \neq (A_{t-1} \circlearrowright 2)[i]$  \\
	\equarightupup & $\state_t[i] = \dstate_t[i] = (A_{t-2}  \circlearrowright 2)[i]$  \\
	\diffrightupup & $\state_t[i] = \dstate_t[i] \neq (A_{t-2} \circlearrowright 2)[i]$  \\
	\midrule
	\dunnodiff & No condition on $\state_t[i]$, $\dstate_t[i]$\\
  \bottomrule
\end{tabular}\\}
\end{table}
