\section{The SHA-1 hash function}
\label{sec:description}

This section gives a brief description of the \shaone hash function, going again over some features already presented in \autoref{chap:hashfun}. We refer to the most recent NIST standard document~\cite{Nist-SHA} for a more thorough
presentation.

\shaone is a hash function from the \mdsha family which produces digests of $160$ bits.
Its high-level structure follows the Merkle-Damg{\aa}rd framework~\cite{DBLP:conf/crypto/Merkle89a,DBLP:conf/crypto/Damgard89a}: the input message to the function
is first padded to a length multiple of the block size, which is 512 bits, defining $k$ similarly-sized blocks.
Each block $\messblock_i$ is then fed to a compression function $\compress$ which is used to update a 160-bit chaining value $\chain_i$:
$\chain_{i+1} \defas \compress(\chain_i,\messblock_{i})$.
The first chaining value $\chain_0$ is a predefined constant set to an initial value \iv given in the specifications of the function, and the last value $\chain_k$ is the final output of the hash function.
The padding rule of \shaone is a straightforward application of \merkdam strengthening; it is at least 65-bit long and is made of one `1' bit followed by a (possibly zero)
number of `0' bits, and the length of the message to be hashed (without the padding) in bits as a 64-bit integer. The value of the \iv is given in \autoref{tab:sha_iv} as five 32-bit words
(each of these words initializes one of the five internal registers of the compression function, described below).
Note that neither the padding nor the \iv actually play any role in our attacks.

\begin{table}[ht]
\caption{\label{tab:sha_iv}The initial value (\iv) of \shaone.}
\begin{center}
\begin{tabular}{c c c c c} \toprule
$\mathcal{A}_0$:\texttt{0x67452301} & $\mathcal{B}_0$:\texttt{0xefcfab89} & $\mathcal{C}_0$:\texttt{0x98badcfe} & $\mathcal{D}_0$:\texttt{0x10325476} & $\mathcal{E}_0$:\texttt{0xc3d2e1f0} \\ 
\bottomrule
\end{tabular}
\end{center}
\end{table}

Similarly to other members of the \mdsha family, the compression function $\compress$ is built around an \emph{ad hoc} block cipher $\blockE$ used in a Davies-Meyer construction.
%$\chain_{i+1} \defas \blockE(\messblock_{i+1},\chain_i) \boxplus \chain_i$, where $\blockE(x,y)$ is the encryption of the plaintext $y$ with the key $x$ and ``$\boxplus$'' denotes
%wordiwse addition of the five state registers (cf. below) in $\ztt$.
%and ``$\boxplus$'' denotes word-wise addition in $\mathbf{Z}/2^{32}\,\mathbf{Z}$. 
The block cipher itself is a five-branch generalized Feistel network using an Add-Rotate-XOR ``ARX'' step function with the addition of two non-linear Boolean functions, see \autoref{tab:sha_spec}.
In its full version, the step function is iterated 80 times, divided in four rounds of 20 steps.

The internal state of $\blockE$ consists of five 32-bit registers $(\mathcal{A}_i, \mathcal{B}_i, \mathcal{C}_i, \mathcal{D}_i, \mathcal{E}_i)$; at each step, a 32-bit \emph{expanded} message word $\expmess_i$ derived from
a message block $\messblock$
is used to update the five registers:

\[
\left\{
\begin{array}{lcl}
\mathcal{A}_{i+1} & = & (\mathcal{A}_i \circlearrowleft 5) + \boolF_i(\mathcal{B}_i,\mathcal{C}_i,\mathcal{D}_i) + \mathcal{E}_i + \mathcal{K}_i + \expmess_i\\
\mathcal{B}_{i+1} & = & \mathcal{A}_i\\
\mathcal{C}_{i+1} & = & \mathcal{B}_i \circlearrowright 2 \\
\mathcal{D}_{i+1} & = & \mathcal{C}_i \\
\mathcal{E}_{i+1} & = & \mathcal{D}_i \\
\end{array},
\right.	
\]

\noindent with $\mathcal{K}_i$ a constant and $\boolF_i$ one of three possible bitwise Boolean functions (see \autoref{tab:sha_spec} for their specifications).
We give a graphical representation of this step function in \autoref{fig:sha1_step}.
From this figure and the definition of the function, one can notice that all updated registers except $\mathcal{A}_{i+1}$ are just rotated copies of another;
thus it is possible to equivalently express \shaone's step function in a recursive way, using only (past values of) the register $\mathcal{A}$. The definition then becomes:
\begin{equation}
\label{eq:rec_step}
\mathcal{A}_{i+1} = (\mathcal{A}_i \circlearrowleft 5) + \boolF_i(\mathcal{A}_{i-1},\mathcal{A}_{i-2}\circlearrowright 2,\mathcal{A}_{i-3}\circlearrowright 2) + (\mathcal{A}_{i-4}\circlearrowright 2) + \mathcal{K}_i + \expmess_i.
\end{equation}

\noindent
With this notation, the output of one application of the compression function is made of the (possibly rotated) last five state words, $\mathcal{A}_{76}\ldots \mathcal{A}_{80}$.

\begin{figure}[ht]
\begin{center}
%% Public TikZ libraries
\usetikzlibrary{positioning}

%% Custom TikZ addons
\usetikzlibrary{crypto.symbols}
\tikzset{shadows=no}        % Option: add shadows to XOR, ADD, etc.

%% Document
\begin{tikzpicture}[
	thick,
	node distance=2.1cm and 0cm,
	]

    \tikzstyle{word} = [
		draw,
		rectangle,
		fill=Fuchsia!50,
		text centered,
		text width=3em,
		minimum width=2cm,	
		%blur shadow={shadow blur steps=5},	
	]

    \tikzstyle{dot} = [
		fill,
		shape=circle,
		minimum size=5pt,
		inner sep=0pt,
	]	

    \tikzstyle{F} = [
		draw,
		rectangle,
		fill=LimeGreen!80,
		text centered,
		text width=2em,
		minimum width=1em,
		minimum height=2em,	
		%blur shadow={shadow blur steps=5},	
	]

    \tikzstyle{rotatebox} = [
		draw,
		rectangle,
		fill=YellowOrange!70,
		text centered,
		text width=1cm,
		minimum width=1cm,	
		%blur shadow={shadow blur steps=5},		
	]
	
	\node[word] (a0) {$\mathcal{A}_{i}$};
	\node[word,right of = a0] (b0) {$\mathcal{B}_{i}$};
	\node[word,right of = b0] (c0) {$\mathcal{C}_{i}$};
	\node[word,right of = c0] (d0) {$\mathcal{D}_{i}$};
	\node[word,right of = d0] (e0) {$\mathcal{E}_{i}$};
    
	\node[MODADD,scale=1.5,below = 2em of e0] (add1) {};
	\node[MODADD,scale=1.5,below = 2em of add1] (add2) {};
	\node[MODADD,scale=1.5,below = 2em of add2] (add3) {};
	\node[MODADD,scale=1.5,below = 2em of add3] (add4) {};

	\path let \p1=(a0.east),\p2=(add2) in node[rotatebox] (rot1) at (\x1+0.5em,\y2) {$\circlearrowleft 5$};	
	\path let \p1=(b0),\p2=(add3) in node[rotatebox] (rot2) at (\x1,\y2) {$\circlearrowright 2$};

	\node[F,left = 1em of add1] (F) {$\boolF_{i}$};

	\node[word,below of = add4] (e1) {$\mathcal{E}_{i+1}$};
	\node[word,left of = e1] (d1) {$\mathcal{D}_{i+1}$};
	\node[word,left of = d1] (c1) {$\mathcal{C}_{i+1}$};
	\node[word,left of = c1] (b1) {$\mathcal{B}_{i+1}$};
	\node[word,left of = b1] (a1) {$\mathcal{A}_{i+1}$};

	\draw[line] (e0) -- (add1);
	\draw[line] (add1) -- (add2);
	\draw[line] (add2) -- (add3);
	\draw[line] (add3) -- (add4);
	\draw[line] (b0) -- (rot2);
	\draw[line] (rot1) -- (add2);

	\draw[line] let \p1=(add4.south),\p2=(a1.north),\p3=(add4.south) in 
		(\x1,\y1) 
		-- (\x1,\y3-0.5em)
		-- (\x2, \y2+1em)
		-- (\x2, \y2);

	\draw[line] let \p1=(c0.south),\p2=(d1.north),\p3=(add4.south) in 
		(\x1,\y1) 
		-- (\x1,\y3-0.5em)
		-- (\x2, \y2+1em)
		-- (\x2, \y2);

	\draw[line] let \p1=(rot2.south),\p2=(c1.north),\p3=(add4.south) in 
		(\x1,\y1) 
		-- (\x1,\y3-0.5em)
		-- (\x2, \y2+1em)
		-- (\x2, \y2);
		
	\draw[line] let \p1=(d0.south),\p2=(e1.north),\p3=(add4.south) in 
		(\x1,\y1) 
		-- (\x1,\y3-0.5em)
		-- (\x2, \y2+1em)
		-- (\x2, \y2);
				
	\draw[line] let \p1=(a0.south),\p2=(b1.north),\p3=(add4.south) in 
		(\x1,\y1) 
		-- (\x1,\y3-0.5em)
		-- (\x2, \y2+1em)
		-- (\x2, \y2);

	\draw[line] (F) -- (add1);

	\node[right = 2em of add3] (m) {$\mathcal{W}_{i}$};
	\draw[line] (m) -- (add3);

	\node[right = 2em of add4] (k) {$\mathcal{K}_{i}$};
	\draw[line] (k) -- (add4);

	\path let \p1=(rot1.west),\p2=(a0.south) in node[dot] (dotA) at (\x2,\y1) {};
	
	\path let \p1=(F.west),\p2=(d0.south) in node[dot] (dotD) at (\x2,\y1-0.6em) {};
	\path let \p1=(F.west),\p2=(c0.south) in node[dot] (dotC) at (\x2,\y1) {};
	\path let \p1=(F.west),\p2=(b0.south) in node[dot] (dotB) at (\x2,\y1+0.6em) {};
	
	\draw[line] (dotA) -- (rot1);
	\draw[line] let \p1=(dotD),\p2=(F.west) in (dotD) -- (\x2,\y1);
	\draw[line] let \p1=(dotC),\p2=(F.west) in (dotC) -- (\x2,\y1);
	\draw[line] let \p1=(dotB),\p2=(F.west) in (dotB) -- (\x2,\y1);

\end{tikzpicture}

\caption[One step of \shaone.]{One step of \shaone. Figure adapted from \cite{TiKZ:Cryptographers}.\label{fig:sha1_step}}
\end{center}
\end{figure}


\renewcommand{\arraystretch}{1.2}
\begin{table}[ht]
\caption{\label{tab:sha_spec}Boolean functions and constants of \shaone.}
\begin{center}
% why was it like this again?? \begin{tabular}{c c c c r @{}} \toprule
\begin{tabular}{c c c c} \toprule
$\;\; \textnormal{round} \;\;$ & step $i$ & $\boolF_i(x,y,z)$ &  $\mathcal{K}_i$ \\ \midrule
1 & $\;\;  \:\:0 \leq i <  20 \;\;$  & $\boolF_{\text{IF}} = (x \wedge y) \vee (\neg x \wedge z)$ & $\;\; \texttt{0x5a827999} \;\;$ \\
2 & $20 \leq i <  40$ & $\boolF_{\text{XOR}} = x \oplus y \oplus z$ & \texttt{0x6ed6eba1} \\
3 & $40 \leq i <  60$ & $\;\;  \boolF_{\text{MAJ}} = (x \wedge y) \vee (x \wedge z) \vee (y \wedge z) \;\;$  & \texttt{0x8fabbcdc} \\
4 & $60 \leq i <  80$ & $\boolF_{\text{XOR}} = x \oplus y \oplus z $ & \texttt{0xca62c1d6} \\
\bottomrule
\end{tabular}
\end{center}
\end{table}

\noindent
Finally, the expanded message words $\expmess_i$ are computed from the 512-bit message block $\messblock$. This message is first expressed as
sixteen 32-bit words $\mess_0,\ldots, \mess_{15}$, which are then used to recursively define the eighty 32-bit words $\expmess_i$:
\begin{equation}
\label{eq:exp_mess}
\expmess_i=
\left\{
\begin{array}{ll}
\mess_i, & \textnormal{ for } 0\leq i\leq 15 \\
(\expmess_{i-3} \oplus \expmess_{i-8} \oplus \expmess_{i-14} \oplus \expmess_{i-16}) \circlearrowleft 1, & \textnormal{ for } 16\leq i\leq 79
\end{array}.
\right.
\end{equation}

\noindent
The step function and the message expansion can both easily be inverted as follows:
\begin{equation}
\label{eq:rec_step_inv}
\mathcal{A}_{i}\hspace{-0.7mm}=\hspace{-0.8mm}(\mathcal{A}_{i+5} - \expmess_{i+4} - \mathcal{K}_{i+4} - \boolF_{i+4}(\mathcal{A}_{i+3},\mathcal{A}_{i+2}\circlearrowright 2,\mathcal{A}_{i+1}\circlearrowright 2) -
(\mathcal{A}_{i+4} \circlearrowleft 5))\hspace{-0.8mm}\circlearrowleft 2,
\end{equation}
\begin{equation}
\label{eq:exp_mess_inv}
\expmess_{i} = (\expmess_{i+16} \circlearrowright 1) \oplus \expmess_{i+13} \oplus \expmess_{i+8} \oplus \expmess_{i+2}.
\end{equation}
