%\begin{abstract}
%We show that a distinguishing attack in the related key model on an Even-Mansour block cipher can readily be converted into an extremely efficient key recovery attack.
%
%Concerned ciphers include in particular all iterated Even-Mansour schemes with independent keys.
%We apply this observation to the \caesar candidate \proestotr and are able to recover
%the whole key with a number of requests linear in its size. This improves on recent
%forgery attacks in a similar setting.
%\keywords{Even-Mansour, related-key attacks, \proestotr.}
%\end{abstract}


% TODO RK model ``first''
% Colourful figures in appendix (sorry for colorz)

\label{chap:emrka}


\section{Introduction}


The Even-Mansour scheme, or simply Even-Mansour, or \EM, is arguably the simplest way to construct a block cipher from publicly available
components. It defines the encryption $\E((k_1, k_0), p)$ of the plaintext $p$ under the possibly equal keys $k_0$
and $k_1$ as $\Perm(p \oplus k_0) \oplus k_1$, where $\Perm$ is a public permutation. Even and Mansour proved
in~1991 that for a permutation over $n$-bit values, the probability of recovering the keys
is upper-bounded by $\bigo(qt \cdot 2^{-n})$ when the attacker considers the permutation as a black box,
where $q$ is the data complexity (the number of accesses to the encryption oracle) and $t$ is the time
complexity (the number of accesses to the permutation) of the attack~\cite{EM}. Although
of considerable interest, this bound also shows at the same time that the construction is not ideal,
as one gets security only up to $\bigo(2^\frac{n}{2})$ queries, which
is less than the $\bigo(2^n)$ one would expect for an $n$-bit block cipher. For this
reason, much later work investigated the security of variants of the original construction. A simple
one is the iterated Even-Mansour scheme ($\IEM$). When using independent keys and independent permutations,
its $r$-round version is defined as
$\IEM^r((k_r, k_{r-1}, \ldots, k_0), p) \defas \Perm_{r-1}(\Perm_{r-2}(\ldots\Perm_0\linebreak[1](p \oplus k_0) \oplus k_1)\ldots) \oplus k_r$,
and it has been established by Chen and Steinberger that this construction is secure up to $\bigo(2^\frac{rn}{r+1})$ queries~\cite{CS14}.
On the other hand, in a related-key model, the same construction lends itself to trivial
distinguishing attacks, and one must consider alternatives if security in this model is necessary.
Yet until the recent work of Cogliati and Seurin~\cite{CS15} and Farshim and Procter~\cite{FP14},
no variant of Even-Mansour was proved to be secure in the related-key model. This is not the case
anymore and it has now been proven that one can reach a non-trivial level of related-key security for $\IEM^r$
starting from $r = 3$ when using keys linearly derived from a single master key instead
of using independent keys, or even for the original non-iterated \EM scheme if one uses
a non-linear key derivation meeting some conditions.
While related-key analysis obviously gives much more power to the attacker than the
single-key setting, it is a widely accepted
model that may provide useful results on primitives studied in a general context.
%especially as related keys may naturally arise in some protocols.

%\subsubsection{Our contribution.}
%We show that the distinguishing attacks on Even-Mansour ciphers in a related-key
%model can be extended to much more powerful key-recovery attacks by considering modular
%additive differences instead of XOR differences. This applies both to the trivial
%distinguishers on iterated Even-Mansour with independent keys and to the more complex distinguisher
%of Cogliati and Seurin for 2-round Even-Mansour with a linear key-schedule. While these observations
%are somewhat elementary, they eventually lead to a key-recovery attack
%on the authenticated-encryption scheme and \caesar candidate \proestotr in a related-key model. This improves on the recent work
%from FSE~2015 of Dobraunig, Eichlseder and Mendel who use similar methods but only produce forgeries~\cite{DEM15}.

\section{Notation}

We use $||$ to denote string concatenation, $a^i$ with $i$ an integer
to denote the string
made of the concatenation of $i$ copies of the character $a$, and $a^*$
to denote any string of the set $\{a^i, i \in \mathbf{N}\}$, $a^0$
denoting the empty string $\varepsilon$. For any string $s$, we use
$s[i]$ to denote its $i^\text{th}$ element, starting from zero.

We also use $\Delta_i^n$ to denote the string
$0^{n-i-1} || 1 || 0^{i - 1}$. The superscript $n$ will always
be clear from the context and therefore omitted.

Finally, we identify strings of length $n$ over the binary alphabet $\{0,1\}$ with elements of
the vector-space $\mathbf{F}_2^n$ and with the binary representation of elements of
the group $\mathbf{Z}/2^n\mathbf{Z}$. The addition operations on these structures
are respectively denoted by $\oplus$, the bitwise exclusive or (XOR) and $+$, the modular addition.

\section{Generic related-key key-recovery attacks on $\IEM$}
\label{sec:gen}


Since the work of Bellare and Kohno~\cite{BK03}, it is well-known that no block cipher can resist
related-key attacks (RKA) when an attacker may request encryptions under related keys
using two relation classes. A simple example showing why this cannot be the case
is to consider the classes $\xrka(k)$ and $\arka(k)$ of keys
related to $k$ by the XOR and the modular addition of any constant chosen by the attacker respectively.
If we have access to related-key encryption oracles
$\E(k,\cdot)$, $\E(\xrka(k),\cdot)$ and $\E(\arka(k),\cdot)$ for the block cipher $\E$ with $\kappa$-bit keys,
we can easily learn the value of the bit $k[i]$ of $k$ by comparing the results of the queries
$\E(k + \Delta_i, p)$
and $\E(k \oplus \Delta_i, p)$. For $i < \kappa - 1$, the plaintext
$p$ is encrypted under the same key if
$k[i] = 0$, then resulting in the same ciphertext, and is
encrypted under different keys if $k[i] = 1$, then resulting in different ciphertexts
with overwhelming probability.
Doing this test for every bit of $k$ thus allows to recover the whole key
with a complexity linear in $\kappa$, except its most significant bit, as
the carry of a modular addition on the MSB never propagates.% and thus never there will never be a difference between the related keys.
This latter bit can of course easily be recovered once all the others have been determined.

In the same paper, Bellare and Kohno also show that no such trivial generic attack exists when
the attacker is restricted to using
only one of the two classes $\xrka$ or $\arka$, and they prove that an ideal
cipher is in this case resistant to RKA. Taken together, these results mean in essence
that a related-key attack on a block cipher $\E$ using both classes $\xrka(k)$ and $\arka(k)$ does not say
much on $\E$, as nearly all ciphers fall to an attack
in the same model. On the other hand, an attack using either of $\xrka$ or $\arka$ \emph{is} meaningful,
because an ideal cipher is secure in that case.
Nonetheless, one should note that when
queries with many different related keys are allowed, this security remains lower than in a single-key model,
because of the ability to create collisions. One can
for instance query $\E(k\oplus\delta_i,p)$ with $2^{k/2}$ different values for $\delta_i$ and a constant $p$
and then compute $\E(\gamma_i,p)$ with $2^{k/2}$ guesses $\gamma_i$ for $k$. With high probability, one
of the $\gamma_i$s will be equal to one of the $k\oplus\Delta_i$s, which allows to recover $k$.

\subsection{Key-recovery attacks on $\IEM^r$ with independent keys}

Going back to $\IEM$, we explicit the
trivial related-key distinguishers mentioned in the introduction. These distinguishers exist
for $r$-round iterated Even-Mansour schemes with independent keys, for any value of $r$.
As they only use keys related with, say, the $\xrka$ class,
they are therefore meaningful when considering the related-key security of $\IEM$.

From the very definition of $\IEM^r$, it
is obvious that the two values $\E((k_{r-1}, k_{r-2},\linebreak[1] \ldots, k_0), p)$
and $\E((k_{r-1}, k_{r-2}, \ldots, k_0 \oplus \delta), p \oplus \delta)$ are equal for any
difference $\delta$ when
$\E = \IEM^r$ and that this equality does not hold in general, thence allowing to distinguish
$\IEM^r$ from an ideal cipher. This is illustrated for the original \EM in \autoref{fig:em3}.

\begin{figure}[!htb]
\begin{center}
\begin{tikzpicture}[scale=1,transform shape, thick]

\hspace{.5mm}
  \tikzstyle{every node}=[transform shape];
  \tikzstyle{every node}=[node distance=1.2cm];

	% data path
    \node (XOR-1)[XOR,scale=0.8,thick] {};
    \node [left of=XOR-1] (p) {$p$};
	\node (f-1) [right of=XOR-1,draw,rectangle,very thick,minimum width=1cm,minimum height=1cm] {$\Perm$};
	\draw (XOR-1) edge (f-1);

    \node (XOR-2)[right of=f-1,XOR,scale=0.8,thick] {};
	\draw (f-1) edge (XOR-2);
	\node (f-2) [right of=XOR-2] {};
	\path[line] (XOR-2) edge (f-2) ;
	\node [right of=XOR-2,node distance=1.2cm] {$c$};

	%% Subkeys
	\node (k-0) [above=1.2cm of XOR-1]{\small $k_{0}$} ;

	\node (k-1) [above=1.2cm of XOR-2]{\small $k_{1}$};
	\draw (k-1) edge (XOR-2);

	% Diff
	\coordinate [above=7.17mm of p] (d);
	\node (delta) [right=3.21mm of d] {$\delta$};
	\node (XOR-3)[below=2.2mm of k-0,XOR,scale=0.8] {};
    \node (XOR-4)[right=1.6mm of p,XOR,scale=0.8] {};
	
	% moar datapath
	\draw (p) edge (XOR-4);
	\draw[dashed] (XOR-4) edge (XOR-1);
	%
	\draw (k-0) edge (XOR-3);
	\draw[dashed] (XOR-3) edge (XOR-1);
	%
	\coordinate[right=7.5mm of d] (dp) ;
	\draw[dashed] (dp) edge (XOR-3);
	\draw[dashed] (delta) edge (XOR-4);
	
\end{tikzpicture}

\caption[A related-key distinguisher on \EM.]{A related-key distinguisher on \EM. Dashed lines represent datapaths where a difference is injected and not canceled.\label{fig:em3}}
\end{center}
\end{figure}

\medskip

We now show how these distinguishers can be combined with the two-class attack of Bellare and Kohno
in order to extend it to a very efficient key-recovery attack. We give a description in the case of one-round
Even-Mansour, but it can easily be extended to an arbitrary $r$. The attack is very simple and works
as follows: consider again $\E((k_1, k_0), p) = \Perm(p \oplus k_0) \oplus k_1$; one can learn
the value of the bit $k_0[i]$ by querying $\E((k_1, k_0), p)$ and
$\E((k_1, k_0 + \Delta_i), p \oplus \Delta_i)$
and by comparing their values. These  differ with overwhelming probability
if $k_0[i] = 1$ and are equal otherwise.
This is illustrated in \autoref{fig:em4.1.tex} and \autoref{fig:em4.2.tex}

\begin{figure}[!htb]
\begin{center}
\begin{tikzpicture}[scale=1,transform shape, thick]

\hspace{.5mm}
  \tikzstyle{every node}=[transform shape];
  \tikzstyle{every node}=[node distance=1.2cm];

	% data path
    \node (XOR-1)[XOR,scale=0.8,thick] {};
    \node [left of=XOR-1] (p) {$p$};
	\node (f-1) [right of=XOR-1,draw,rectangle,very thick,minimum width=1cm,minimum height=1cm] {$\Perm$};
	\draw (XOR-1) edge (f-1);

    \node (XOR-2)[right of=f-1,XOR,scale=0.8,thick] {};
	\draw (f-1) edge (XOR-2);
	\node (f-2) [right of=XOR-2] {};
	\path[line] (XOR-2) edge (f-2) ;
	\node [right of=XOR-2,node distance=1.2cm] {$c$};

	%% Subkeys
	\node (k-0) [above=1.2cm of XOR-1]{\small $k_{0}$} ;

	\node (k-1) [above=1.2cm of XOR-2]{\small $k_{1}$};
	\draw (k-1) edge (XOR-2);

	% Diff
	\coordinate [above=7.17mm of p] (d);
	\node (delta) [right=2.68mm of d] {$\Delta_i$};
	\node (XOR-3)[below=2.65mm of k-0,ADD,scale=0.8] {};
    \node (XOR-4)[right=2.25mm of p,XOR,scale=0.8] {};
	
	% moar datapath
	\draw (p) edge (XOR-4);
	\draw[dashed] (XOR-4) edge (XOR-1);
	%
	\draw (k-0) edge (XOR-3);
	\draw[dashed] (XOR-3) edge (XOR-1);
	%
	\coordinate[right=9.5mm of d] (dp) ;
	\draw[dashed] (dp) edge (XOR-3);
	\draw[dashed] (delta) edge (XOR-4);

	% cap
	\node [right=.25cm of k-0,scale=0.7] {\emph{If $k_0[i] = 0$}};
	
\end{tikzpicture}

\caption{Related-key queries to $\IEM^1$ (\ie \EM) with no output difference.\label{fig:em4.1.tex}}
\end{center}
\end{figure}

\begin{figure}[!htb]
\begin{center}
\begin{tikzpicture}[scale=2,transform shape, thick]

\hspace{.5mm}
  \tikzstyle{every node}=[transform shape];
  \tikzstyle{every node}=[node distance=1.2cm];

	% data path
    \node (XOR-1)[XOR,scale=0.8,thick] {};
    \node [left of=XOR-1] (p) {$p$};
	\node (f-1) [right of=XOR-1,draw,rectangle,very thick,minimum width=1cm,minimum height=1cm] {$\Perm$};
	\draw[dashed] (XOR-1) edge (f-1);

    \node (XOR-2)[right of=f-1,XOR,scale=0.8,thick] {};
	\draw[dashed] (f-1) edge (XOR-2);
	\node (f-2) [right of=XOR-2] {};
	\path[line,dashed] (XOR-2) edge (f-2) ;
	\node [right of=XOR-2,node distance=1.2cm] {$c$};

	%% Subkeys
	\node (k-0) [above=1.2cm of XOR-1]{\small $k_{0}$} ;

	\node (k-1) [above=1.2cm of XOR-2]{\small $k_{1}$};
	\draw (k-1) edge (XOR-2);

	% Diff
	\coordinate [above=7.17mm of p] (d);
	\node (delta) [right=2.68mm of d] {$\Delta_i$};
	\node (XOR-3)[below=2.65mm of k-0,ADD,scale=0.8] {};
    \node (XOR-4)[right=2.25mm of p,XOR,scale=0.8] {};
	
	% moar datapath
	\draw (p) edge (XOR-4);
	\draw[dashed] (XOR-4) edge (XOR-1);
	%
	\draw (k-0) edge (XOR-3);
	\draw[dashed] (XOR-3) edge (XOR-1);
	%
	\coordinate[right=9.5mm of d] (dp) ;
	\draw[dashed] (dp) edge (XOR-3);
	\draw[dashed] (delta) edge (XOR-4);

	% cap
	\node [right=.25cm of k-0,scale=0.7] {\emph{If $k_0[i] = 1$}};
	
\end{tikzpicture}

\caption{Related-key queries to $\IEM^1$ (\ie \EM) with an output difference.\label{fig:em4.2.tex}}
\end{center}
\end{figure}


A similar attack works on the variant of the (iterated) \EM
that uses modular addition instead of XOR for the combination of the key with the plaintext.
This variant was first analyzed by Dunkelman, Keller and Shamir~\cite{DKS12} and offers the same security bounds as the
original Even-Mansour. An attack in that case works similarly
by querying \eg $\E((k_1, k_0), \Delta_i)$ and
$\E((k_1, k_0 \oplus \Delta_i), 0^\kappa)$.

Both attacks use a single difference class for the related keys, either $\xrka$ or $\arka$,
and they are therefore meaningful as related-key attacks. They simply emulate the attack that uses both
classes simultaneously by taking advantage of the fact that the usage of key material is very simple in
Even-Mansour.
Finally, we can see that in the particular case of a one-round construction, the attack still works without
adaptation if one
chooses the keys $k_1$ and $k_0$ to be equal.

\subsection{Extension to $\IEM^2$ with a linear key schedule}

Cogliati and Seurin~\cite{CS15} showed that it is also possible to very efficiently distinguish
$\IEM^2$ with related keys, even when the keys are equal or derived
from a master key by a linear key schedule. Similarly as for independent-key $\IEM$, we can adapt
the distinguisher and transform it into a key-recovery attack. The idea remains the same: one
replaces the $\xrka$ class of the original distinguisher with $\arka$, which makes its success
conditioned on the value of a few key bits, hence allowing their recovery. We give
the description of our modified distinguisher for $\E(k, p) := \Perm(\Perm(k \oplus p) \oplus k) \oplus k$,
where we let the $\delta_i$s denote arbitrary differences:

\begin{enumerate}[leftmargin=4em]
	\item Query $y_1 := \E(k + \delta_1, x_1)$.
	\item Set $x_2$ to $x_1 \oplus \delta_1 \oplus \delta_2$ and query $y_2 := \E(k + \delta_2, x_2)$.
	\item Set $y_3$ to $y_1 \oplus \delta_1 \oplus \delta_3$ and query $x_3 := \E^{-1}(k + \delta_3, y_3)$.
	\item Set $y_4$  to $y_2 \oplus \delta_1 \oplus \delta_3$ and query $x_4 := \E^{-1}(k + (\delta_1 \oplus \delta_2 \oplus \delta_3), y_4)$.
	\item Test if $x_4 = x_3 \oplus \delta_1 \oplus \delta_2$.
\end{enumerate}

If the test is successful, it means that with overwhelming probability
the key bits at the positions of the differences $\delta_1$, $\delta_2$,
$\delta_3$ are all zero, as in that case  $k + \delta_i = k \oplus \delta_i$ and
the distinguisher works ``as intended'', and as otherwise at least one uncontrolled difference goes through $\Perm$ or $\Perm^{-1}$.
It is possible to restrict oneself to using differences in only two bits for the $\delta_i$s, and as soon as two zero key bits
have been found (which happens after an expected four trials for
random keys), the rest of the key can be determined one bit at a time.

We conclude this short section by showing why the test of line~5 is successful when $k + \delta_i = k \oplus \delta_i$,
but refer to Cogliati and Seurin for a complete description of their distinguisher, including the general case of distinct
permutations and keys linearly derived from a master key.

For the sake of clarity, we write $k \oplus \delta_i$ for $k + \delta_i$, as these values are equal by hypothesis.
By definition,
$y_1 = \Perm(\Perm(x_1 \oplus k \oplus \delta_1) \oplus k \oplus \delta_1) \oplus k \oplus \delta_1$ and
$y_3 = \Perm(\Perm(x_1 \oplus k \oplus \delta_1) \oplus k \oplus \delta_1) \oplus k \oplus \underline{\color{Green}\delta_1} \oplus {\color{NavyBlue}\overline{\delta_1}} \oplus \delta_3$
which simplifies to
$\Perm(\Perm(x_1 \oplus k \oplus \delta_1) \oplus k \oplus \delta_1) \oplus k \oplus \delta_3$. This yields the following
expression for $x_3$:
\begin{align*}
x_3 &= \Perm^{-1}(\Perm^{-1}(\Perm(\Perm(x_1 \oplus k \oplus \delta_1) \oplus k \oplus \delta_1) \oplus \underline{\color{Green}k \oplus \delta_3} \oplus {\color{NavyBlue}\overline{k \oplus \delta_3}})
\oplus k \oplus \delta_3) \oplus k \oplus \delta_3 \\
&= \Perm^{-1}(\underline{\color{Green}\Perm^{-1}(}{\color{NavyBlue}\overline{\Perm(}}\Perm(x_1 \oplus k \oplus \delta_1) \oplus k \oplus \delta_1{\color{NavyBlue}\overline{)}}\underline{\color{Green})}
\oplus k \oplus \delta_3) \oplus k \oplus \delta_3 \\
&= \Perm^{-1}(\Perm(x_1 \oplus k \oplus \delta_1) \oplus \underline{\color{Green}k} \oplus \delta_1 \oplus {\color{NavyBlue}\overline{k}} \oplus \delta_3) \oplus k \oplus \delta_3 \\
&= \Perm^{-1}(\Perm(x_1 \oplus k \oplus \delta_1) \oplus \delta_1 \oplus \delta_3) \oplus k \oplus \delta_3
\end{align*}
Similarly,
$y_2 = \Perm(\Perm(x_1 \oplus k \oplus \delta_1) \oplus k \oplus \delta_2) \oplus k \oplus \delta_2$ and
$y_4 = \Perm(\Perm(x_1 \oplus k \oplus \delta_1) \oplus k \oplus \delta_2) \oplus k \oplus \delta_2 \oplus \delta_1 \oplus \delta_3$, which
yields the following expression for $x_4$:
\begin{align*}
x_4 &=  \Perm^{-1}(\Perm^{-1}(\Perm(\Perm(x_1 \oplus k \oplus \delta_1) \oplus k \oplus \delta_2) \oplus \underline{\color{Green}k \oplus \delta_2 \oplus \delta_1
\oplus \delta_3} \oplus {\color{NavyBlue}\overline{k \oplus \delta_1 \oplus \delta_2 \oplus \delta_3}}) \\
&\oplus k \oplus \delta_1 \oplus \delta_2 \oplus \delta_3) \oplus k \oplus \delta_1 \oplus \delta_2 \oplus \delta_3\\
&= \Perm^{-1}(\underline{\color{Green}\Perm^{-1}(}{\color{NavyBlue}\overline{\Perm(}}\Perm(x_1 \oplus k \oplus \delta_1) \oplus k \oplus \delta_2{\color{NavyBlue}\overline{)}}\underline{\color{Green})}
\oplus k \oplus \delta_1 \oplus \delta_2
\oplus \delta_3) \oplus k \oplus \delta_1 \oplus \delta_2 \oplus \delta_3 \\
&= \Perm^{-1}(\Perm(x_1 \oplus k \oplus \delta_1) \oplus \underline{\color{Green}k \oplus \delta_2} \oplus {\color{NavyBlue}\overline{k}} \oplus \delta_1 \oplus {\color{NavyBlue}\overline{\delta_2}} \oplus
\delta_3) \oplus k \oplus \delta_1 \oplus \delta_2 \oplus \delta_3\\
&= \Perm^{-1}(\Perm(x_1 \oplus k \oplus \delta_1) \oplus \delta_1 \oplus \delta_3) \oplus k \oplus \delta_1 \oplus \delta_2 \oplus \delta_3
\end{align*}
From the final expressions of $x_3$ and $x_4$, we see that their XOR difference is indeed $\delta_1 \oplus \delta_2$.

\section{Application to \proestotr}
\label{sec:appli}

We now apply the simple generic key-recovery attack of \autoref{sec:gen} to the \caesar candidate \proestotr, which is an authenticated-encryption
scheme member of the \proest family~\cite{proest}. This family is based on the \proest permutation and defines
three schemes instantiating as many modes of operation, namely COPA, OTR and APE. Only
the latter can readily be instantiated with a permutation, and both COPA and OTR rely on a keyed primitive. In order
to be instantiated with \proest, the permutation is expanded to a block cipher
defined as a one-round Even-Mansour scheme with identical keys
$\E(k,p) := \Perm(p \oplus k) \oplus k$,
with the \proest permutation as $\Perm$. We will denote this cipher as \proestem.

Although the attack of \autoref{sec:gen} could readily be applied to \proestem, this cipher is only
meant to be embedded into a specific instantiation of a mode such as OTR, and attacking it out of context may not
be relevant to its intended use.
Hence we must
be able to mount an attack on \proestcopa or \proestotr as a whole for it to be really significant,
which is precisely what we describe now for the latter.

Because our attack solely relies on the Even-Mansour structure of the cipher, we refer the interested reader to the
submission document of \proest for the definition of its permutation.
The same goes for the OTR mode~\cite{M14}, as we only need to focus on a small part to describe the attack.
Consequently, we just describe how the encryption of the first block of plaintext is performed in \proestotr. Note that this is not necessarily the
same as encrypting the first block in every instantiations of OTR, as there is some flexibility in the definition of the mode.

\medskip

The mode of operation OTR is nonce-based; it takes as input a key $k$, a nonce $n$, a message $m$,
possibly empty associated data $a$, and produces a ciphertext $c$ corresponding
to the encryption of the message with $k$, and a tag $t$ authenticating $m$ and
$a$ together with the key $k$. It is important for the security of the mode to ensure
that one cannot encrypt twice using the same nonce. However, there are no
specific additional restrictions on the nonces, and we consider that an attacker
has full control over their non-repeating values.

The encryption of the first block of ciphertext $c_1$ by \proestotr is defined as a function
$\fun(k, n, m_1, m_2)$
of $k$, $n$, and the first two blocks of plaintext $m_1$ and $m_2$:
let $\ell := \E(k, n || 10^*)$ be the encryption of the padded nonce and
$\ell' := \permi(\ell)$, with $\permi$ a linear permutation, in this case the multiplication by $4$ in some finite field,
then $c_1$ is simply equal to $\E(k, \ell' \oplus m_1) \oplus m_2$. We show this schematically along with the encryption
of the second block in \autoref{fig:otr}.
Let us now apply the attack from \autoref{sec:gen}.

\begin{figure}[ht]
\begin{center}
\begin{tikzpicture}[auto,scale=1,transform shape]
\tikzstyle{block} = [rectangle, draw, thick,
    text width=3em, text centered, minimum height=3em]
\tikzstyle{line} = [draw, -latex']

\node [block] (Et) {$\E_K$};
\coordinate [above of=Et, node distance=1cm] (top);
\coordinate [left of=Et, node distance=2cm] (midleft);
\node [left of=top, node distance=2cm] (m1) {$m_1$};
\node [left of=Et, node distance=1.2cm, thick, scale=1.2] (xor1) {$\mathbf{\oplus}$};
\node [above of=xor1, node distance=0.7cm] (4L1) {$\ell'$};
\coordinate [left of=xor1, node distance=1mm] (xor11);
\path [line] (m1) -- (midleft) -- (xor11);
\coordinate [above of=xor1, node distance=1mm] (xor12);
\path [line] (4L1) -- (xor12);
\coordinate [right of=xor1, node distance=1mm] (xor13);
\path [line] (xor13) -- (Et);
%
\node [right of=top, node distance=2cm] (m2) {$m_2$};
\node [right of=Et, node distance=2cm, thick, scale=1.2] (xor2) {$\mathbf{\oplus}$};
\coordinate [left of=xor2, node distance=1mm] (xor21);
\path [line] (Et) -- (xor21);
\coordinate [above of=xor2, node distance=1mm] (xor22);
\path [line] (m2) -- (xor22);
%
\coordinate [below of=midleft, node distance=.75cm] (botleft);
\coordinate [below of=botleft, node distance=.75cm] (bumleft);
\coordinate [right of=botleft, node distance=4cm] (botright);
\coordinate [right of=bumleft, node distance=4cm] (bumright);
\coordinate [below of=xor2, node distance=1mm] (xor23);
\node [block, below of=Et, node distance=2.5cm] (Eb) {$\E_K$};
%
\coordinate [left of=Eb, node distance=2cm] (bimleft);
\node [left of=Eb, node distance=1.2cm, thick, scale=1.2] (xor3) {$\mathbf{\oplus}$};
\coordinate [left  of=xor3, node distance=1mm] (xor31);
\coordinate [above of=xor3, node distance=1mm] (xor32);
\coordinate [right of=xor3, node distance=1mm] (xor33);
\coordinate [below of=xor3, node distance=1mm] (xor34);
\node [above of=xor3, node distance=0.7cm] (4L2) {$\ell'$};
\node [below of=xor3, node distance=0.7cm] (L) {$\ell$};
%
\path [line] (xor23) -- (botright) -- (bumleft) -- (bimleft) -- (xor31);
\path [line] (4L2) -- (xor32);
\path [line] (L) -- (xor34);
\path [line] (xor33) -- (Eb);
%
\node [right of=Eb, node distance=2cm, thick, scale=1.2] (xor4) {$\mathbf{\oplus}$};
\coordinate [below of=xor4, node distance=1mm] (xor41);
\coordinate [left of=xor4, node distance=1mm] (xor42);
\coordinate [above of=xor4, node distance=1mm] (xor43);
\path [line] (midleft) -- (botleft) -- (bumright) -- (xor43);
\path [line] (Eb) -- (xor42);
%
\node [below of=m1, node distance=4.5cm] (c1) {$c_1$};
\node [right of=c1, node distance=4cm] (c2) {$c_2$};
\path [line] (bimleft) -- (c1);
\path [line] (xor41) -- (c2);

\end{tikzpicture}
\end{center}
\caption{\label{fig:otr}The encryption of the first two blocks of message in \proestotr.}
\end{figure}

\subsection{Step 1: Recovering the most significant half of the key}

It is straightforward to see that one can recover the value
of the bit $k[i]$ by performing only two queries with related keys
and different nonces and messages. One just has to compare
$c_1 = \fun(k, n, m_1, m_2)$ and
$\hat{c}_1 = \fun(k + \Delta_i,n \oplus \Delta_i,
m_1 \oplus \Delta_i \oplus \permi(\Delta_i), m_2)$. Indeed, if $k[i] = 0$,
then the value $\hat{\ell}$ obtained in the computation of $\hat{c}_1$ is equal to
$\ell \oplus \Delta_i$ and $\hat{\ell'} = \ell' \oplus \pi(\Delta_i)$, hence
$\hat{c}_1 = c_1 \oplus \Delta_i$. If $k[i] = 1$, the latter equality does not
hold with overwhelming probability.
We illustrate the propagation of differences in the computation of $\ell'$ and $\hat{\ell'}$ in
\autoref{fig:otr21} and \autoref{fig:otr22}.


\begin{figure}[!htb]
\begin{center}
\begin{tikzpicture}[scale=1,transform shape, thick]

\hspace{-2.5mm}
  \tikzstyle{every node}=[transform shape];
  \tikzstyle{every node}=[node distance=1.2cm];

	% data path
    \node (XOR-1)[XOR,scale=0.8,thick] {};
	\node (f-1) [right=4mm of XOR-1,draw,rectangle,very thick,minimum width=1cm,minimum height=1cm] {$\Perm$};
    \node (XOR-2)[right=4mm of f-1,XOR,scale=0.8,thick] {};
	\node [left of=XOR-1] (p) {$n$};
	\draw (XOR-1) edge (f-1);

	\draw (f-1) edge (XOR-2);
	\node (f-2) [right of=XOR-2,node distance=.5cm] {$\hat{\ell}$};
	\node (XOR-6) [right of=f-2,node distance=.5cm,MUL,scale=0.8] {};
	\node (f-3) [right of=XOR-6,node distance=.7cm] {$\hat{\ell'}$};
	\draw[dashed] (XOR-2) edge (f-2);
	\draw[dashed] (f-2) edge (XOR-6);
	\draw[line,dashed] (XOR-6) edge (f-3);

	%% Subkeys
	\node (k-0) [above=1.2cm of XOR-1]{\small $k$} ;
	\node (k-1) [above=1.2cm of XOR-2]{\small $k$};

	% cst
	\node (four) [above=1.2cm of XOR-6]{\small 4};
	\draw (four) -- (XOR-6);

	% Diff
	\coordinate [above=7.51mm of p] (d);
	\node (delta) [right=2.68mm of d] {$\Delta_i$};
	\node (XOR-3)[below=2.65mm of k-0,ADD,scale=0.8] {};
    \node (XOR-4)[right=2.25mm of p,XOR,scale=0.8] {};
	\node (XOR-5)[below=2.65mm of k-1,ADD,scale=0.8] {};
	
	% moar datapath
	\draw (p) edge (XOR-4);
	\draw[dashed] (XOR-4) edge (XOR-1);
	%
	\draw (k-0) edge (XOR-3);
	\draw[dashed] (XOR-3) edge (XOR-1);
	%
	\coordinate[right=9.5mm of d] (dp) ;
	\draw[dashed] (dp) edge (XOR-3);
	\draw[dashed] (delta) edge (XOR-4);
	%
	\draw[dashed] (XOR-3) -- (XOR-5);
	%
	\draw (k-1) -- (XOR-5);
	\draw[dashed] (XOR-5) -- (XOR-2);

	% cap
	\node [right=.25cm of k-0,scale=0.7] {\emph{If $k[i] = 0$}};
	\node [below=.2cm of XOR-6,scale=0.7] {$\hat{\ell'} = \ell'\oplus4\otimes\Delta_i$};

	% white square
	\node [below=.3mm of f-1] {};
	
\end{tikzpicture}

\caption{Related-key queries to \proestotr with predictable output difference.\label{fig:otr21}}
\end{center}
\end{figure}

\begin{figure}[!htb]
\begin{center}
\begin{tikzpicture}[scale=2,transform shape, thick]

\hspace{-2.5mm}
  \tikzstyle{every node}=[transform shape];
  \tikzstyle{every node}=[node distance=1.2cm];

	% data path
    \node (XOR-1)[XOR,scale=0.8,thick] {};
	\node (f-1) [right=4mm of XOR-1,draw,rectangle,very thick,minimum width=1cm,minimum height=1cm] {$\Perm$};
    \node (XOR-2)[right=4mm of f-1,XOR,scale=0.8,thick] {};
    \node [left of=XOR-1] (p) {$n$};
	\draw[dashed] (XOR-1) edge (f-1);

	\draw[dashed] (f-1) edge (XOR-2);
	\node (f-2) [right of=XOR-2,node distance=.5cm] {$\hat{\ell}$};
	\node (XOR-6) [right of=f-2,node distance=.5cm,MUL,scale=0.8] {};
	\node (f-3) [right of=XOR-6,node distance=.7cm] {$\hat{\ell'}$};
	\draw[dashed] (XOR-2) edge (f-2);
	\draw[dashed] (f-2) edge (XOR-6);
	\draw[line,dashed] (XOR-6) edge (f-3);

	%% Subkeys
	\node (k-0) [above=1.2cm of XOR-1]{\small $k$} ;
	\node (k-1) [above=1.2cm of XOR-2]{\small $k$};

	% cst
	\node (four) [above=1.2cm of XOR-6]{\small 4};
	\draw (four) -- (XOR-6);

	% Diff
	\coordinate [above=7.51mm of p] (d);
	\node (delta) [right=2.68mm of d] {$\Delta_i$};
	\node (XOR-3)[below=2.65mm of k-0,ADD,scale=0.8] {};
    \node (XOR-4)[right=2.25mm of p,XOR,scale=0.8] {};
	\node (XOR-5)[below=2.65mm of k-1,ADD,scale=0.8] {};
	
	% moar datapath
	\draw (p) edge (XOR-4);
	\draw[dashed] (XOR-4) edge (XOR-1);
	%
	\draw (k-0) edge (XOR-3);
	\draw[dashed] (XOR-3) edge (XOR-1);
	%
	\coordinate[right=9.5mm of d] (dp) ;
	\draw[dashed] (dp) edge (XOR-3);
	\draw[dashed] (delta) edge (XOR-4);
	%
	\draw[dashed] (XOR-3) -- (XOR-5);
	%
	\draw (k-1) -- (XOR-5);
	\draw[dashed] (XOR-5) -- (XOR-2);

	% cap
	\node [right=.25cm of k-0,scale=0.7] {\emph{If $k[i] = 1$}};
	\node (cap2) [below=.2cm of XOR-6,scale=0.7] {$\hat{\ell'} = \ell'\oplus \$$};

	% white square
	\node [below=.3mm of f-1] {};
	
\end{tikzpicture}

\caption{Related-key queries to \proestotr with unpredictable output difference.\label{fig:otr22}}
\end{center}
\end{figure}

Yet this does not allow to recover the whole key because the nonce in \proestotr is restricted to a length
half of the width of the block cipher $\E$, or equivalently of the underlying \proest permutation,
\ie $\frac{\kappa}{2}$. It is then possible to recover only
half of the bits of $k$ using this procedure, as one cannot introduce appropriate differences in
the computation of $\ell$ for the other half. The targeted security of the
whole primitive being precisely $\frac{\kappa}{2}$ because of the generic single key
attacks on Even-Mansour,
one does not make a significant gain by recovering only half of the key.
Even though, it should still be noted that this yields an
attack with very little data requirements and with the same time complexity as the best
point on the tradeoff curve of generic attacks, which in that case has a much higher data complexity of
$2^\frac{\kappa}{2}$.
%Yet this does not allow to recover the whole key because the nonce in \proestotr is restricted to a length
%half of the width of the block cipher $\E$, or equivalently of the underlying \proest permutation,
%\ie $\frac{\kappa}{2}$. It is then possible to recover only
%the upper half of the bits of $k$, written $k^\MSB$,
%when using this procedure,
%as one cannot introduce appropriate differences in
%the computation of $\ell$ for the lower half. The targeted security of the
%whole primitive being precisely $\frac{\kappa}{2}$ because of the generic single key
%attacks on Even-Mansour,
%one does not make a significant gain by recovering only half of the key.
%While one could still complete this attack in time $\times$ data complexity
%$\bigo(2^{\kappa/4})$ by applying the standard attack on Even-Mansour and looking for collisions in the output of queries of the form $(k^\MSB||x)$ and $(0||x)$ to
%the encryption oracle and the public permutation respectively, it is possible to do better as we describe next.


\subsection{Step 2: Recovering the least significant half of the key}
\label{sec:reco_lsb}

Even though the generic attack in its most simple form does not allow to recover the full key
of \proestotr,
we can use the fact that the padding of the nonce is done on the least significant
bits to our advantage, and by slightly adapting the procedure of the first step, we can iteratively recover the value
of the least significant half of the key with no more effort than for the most significant half.

Let us first show how we can recover the most significant bit of the least significant half of
the key $k[\msbtwo]$, \ie the first bit for which we cannot use the previous method,
after a single encryption by $\E$.
%For the sake of simplicity, we assume for now that
%unlike in \proestem the two keys of $\E$ are independent, and we omit to write the second (outer) key in the queries to $\E$.

We note $k^\MSB$ the known most significant half of the key $k$.
To mount the attack, one queries $\E(k - k^\MSB + \Delta_\msbtwo, p \oplus \Delta_{\msbtwoone})$ and
$\E(k - k^\MSB - \Delta_\msbtwo, p)$. We can see that the inputs to $\Perm$ in these two cases are equal
iff $k[\msbtwo] = 1$. Indeed, in that case,
the carry in the addition $(k - k^\MSB) + \Delta_\msbtwo$ propagates by exactly one position and is ``cancelled'' by
the difference in $p$, and there is no carry propagation in $(k - k^\MSB) - \Delta_\msbtwo$.
Set $C \defas \Perm(p \oplus (k - k^\MSB - \Delta_\msbtwo))$, then
the result of the two queries are equal to
$C \oplus (k - k^\MSB + \Delta_\msbtwo) = C \oplus (k \oplus k^\MSB \oplus \Delta_\msbtwo \oplus \Delta_\msbtwoone)$ and
$C \oplus (k - k^\MSB - \Delta_\msbtwo) = C \oplus (k \oplus k^\MSB \oplus \Delta_\msbtwo)$
Consequently, the XOR difference between the two results is known and equal to $\Delta_\msbtwoone$.
If on the other hand
$k[\msbtwo] = 0$, the carry in $((k - k^\MSB) - \Delta_\msbtwo)$ propagates all the way to the most significant bit of $k$, whereas only
two differences are introduced in the input to $\Perm$ in the first query. This allows to distinguish between the two cases and thus to recover the value of this key bit.

Once the value of $k[\msbtwo]$ has been learned, one can iterate the process to recover the remaining bits of $k$.
The only subtlety is that we want to ensure that if there is a carry propagation in
$(k - k^\MSB) + \Delta_{\msbtwo - i}$ (resp. $(k - k^\MSB) - \Delta_{\msbtwo - i}$),
it should propagate up to $k_{\msbtwoone}$, the position where we cancel it with an XOR difference
(resp. up to the most significant bit); this can easily be achieved by adding two terms to both keys.
Let us define $\gamma_i$ as the value of the key $k$ only on positions $\msbtwo\ldots\msbtwoone - i$, completed with zeros left and right;
that is $\gamma_i[j] = k[j]$ if $\msbtwo \geq j \geq \msbtwoone - i$, and $\gamma_i[j] = 0$ otherwise.
Let us also define $\widetilde{\gamma_i}$ as the binary complement of $\gamma_i$ on its non-zero support,
that is $\widetilde{\gamma_i}[j] = \widetilde{k[j]}$ if $\msbtwo \geq j \geq \msbtwoone - i$, and $\widetilde{\gamma_i}[j] = 0$ otherwise.
The modified queries then become $\E(k - k^\MSB + \Delta_{\msbtwo - i} + \widetilde{\gamma_i}, p \oplus \Delta_{\msbtwoone})$ and
$\E(k - k^\MSB - \Delta_{\msbtwo - i} - \gamma_i, p)$, for which the propagation of the carries is ensured. Note that
the difference between the results of these two queries when $k[\msbtwo - i] = 1$ is independent of $i$ and always equal
to $\Delta_\msbtwoone$.


We conclude by showing how to apply this procedure to \proestotr. For the sake of readability, let us denote
by $\Delta_i^+$ and $\Delta_i^-$ the complete modular differences used to recover one less significant bit $k[i]$.
We then simply perform the two queries $\fun(k + \Delta_i^+, n \oplus \Delta_{\msbtwoone}, m_1 \oplus \Delta_{\msbtwoone}, m_2)$
and $\fun(k + \Delta_i^-, n, m_1 \oplus \permi(\Delta_{\msbtwoone}), m_2)$, which differ by $\Delta_{\msbtwoone}$ iff $k_i$ is one,
with overwhelming probability.

All in all, one can retrieve the whole key of size
$\kappa$ using only $2\kappa$ chosen-nonce related-key encryption requests with $\kappa + \kappa/2 + 1$ different keys, ignoring
everything in the output apart from the value of the first block
of ciphertext. We give the full attack in \autoref{alg:kr}. Note that it makes use
of a procedure \textsc{Refresh} which picks fresh values for two message words and most importantly
for the nonce. Because
the attack is very fast, it can easily be tested. We give an implementation
for a 64-bit toy cipher in \autoref{sec:toy}.

\begin{algorithm}[h]
\LinesNumbered
\KwIn{ Oracle access to $\fun(k, \cdot, \cdot, \cdot)$ and $\fun(\arka(k), \cdot, \cdot, \cdot)$
for a fixed unknown key $k$ of even length $\kappa$}
\KwOut{ Two candidates for the key $k$ }

$k'$ := $0^\kappa$\\
\For{ $i$ := $\kappa - 2$ to $\kappa/2$}
{
	\textsc{Refresh}($n$, $m_1$, $m_2$)\\
	$x$ := $\fun(k, n, m_1, m_2)$\\
	$y$ := $\fun(k + \Delta_i, n \oplus \Delta_i,
                 m_1 \oplus \Delta_i \oplus \permi(\Delta_i), m_2)$\\
	\If{$x = y \oplus \Delta_i$}
	{
		$k'[i]$ := 0	
	}
	\Else
	{
		$k'[i]$ := 1
	}
}
\For{ $i$ := $\kappa/2 - 1$ to $0$}
{
	\textsc{Refresh}($n$, $m_1$, $m_2$)\\
	$x$ := $\fun(k + \Delta_i^+, n \oplus \Delta_{\msbtwoone}, m_1 \oplus \Delta_{\msbtwoone}, m_2)$\\
	$y$ := $\fun(k + \Delta_i^-, n, m_1 \oplus \permi(\Delta_{\msbtwoone}), m_2)$\\
	\tcc{The values of $\Delta_i^+$ and $\Delta_i^-$ are computed as in \autoref{sec:reco_lsb}}
	\If{$x = y \oplus \Delta_\msbtwoone$}
	{
		$k'[i]$ := 1	
	}
	\Else
	{
		$k'[i]$ := 0
	}
}
$k''$ := $k'$\\
$k''[\kappa - 1]$ := 1\\
\Return{$(k',k'')$}
\caption{Related-key key recovery for \proestotr\label{alg:kr}}
\end{algorithm}


\paragraph{\emph{\textsc{Remark.}}} If the padding of the nonce in \proestotr were done on the most significant bits, no attack similar
to Step~2 could recover
the corresponding key bits: the modular addition is a triangular function, meaning that the result of $a + b$ on a bit $i$ only
depends on the value of bits of position less than $i$ in $a$ and $b$, and therefore no XOR in the nonce in the less
significant bits could control modular differences introduced in the padding in the more significant bits. An attack in that case would thus most likely
be applicable to general ciphers when using only the $\arka$ class, and it is proven that no such attack is efficient.
However, one could always imagine using a related-key class using an addition operation reading the bits in reverse. While
admittedly unorthodox, this would not result in a stronger model than using $\arka$, strictly speaking.

\paragraph{Discussion.}

In a recent independent work, Dobraunig, Eichlseder and Mendel use similar methods to produce forgeries for \proestotr by considering
related keys with XOR differences~\cite{DEM15}. On the one hand, one could argue that the class $\xrka$ seems more natural than $\arka$
and could be more likely to arise in actual situations, which would
make their attack more applicable than ours. On the other hand, an ideal cipher is expected to give a similar security against
RKA using either class, which means that our model is not theoretically stronger than the one of Dobraunig~\etal, while resulting in a
much more powerful key recovery attack.

\section{Conclusion}

We made a simple observation that allows to convert related-key distinguishing attacks on some Even-Mansour schemes into much more
powerful key-recovery attacks, and we used this observation to derive an extremely efficient key-recovery attack on the
\proestotr \caesar candidate, in the related-key model.

Primitives based on \EM are quite common, and it is natural to wonder if we could mount similar
attacks on other ciphers. A natural first target would be \proestcopa which is also based on the \proestem cipher.
However, in this mode,
encryption and tag generation depend on the encryption of a fixed plaintext $\ell \defas \E(k, 0)$ which is different
for different keys with overwhelming probability and makes our attack fail. The forgery attacks of
Dobraunig~\etal seem to fail in that case for the same reason.
Keeping with \caesar candidates, another good target would be \minalpher~\cite{minalpher}, which also
uses a one-round Even-Mansour block cipher as one of its components. The attack also fails in this
case, though, because the masking key used in the Even-Mansour cipher is derived from the master key
in a highly non-linear way. In fact, Mennink recently proved that both ciphers are resistant
to related-key attacks~\cite{DBLP:conf/crypto/Mennink16}.
Finally, leaving aside authentication and
going back to traditional block ciphers, we could consider designs such as LED \cite{LED}. The attack
also fails in that case, however, because the cipher uses an iterated construction with at least 8 rounds and
only one or two keys.

This lack of other results is not very surprising, as we only improve existing distinguishing
attacks, and this improvement cannot be used without a distinguisher as its basis.
Therefore, any primitive for which resistance to related-key attacks is important should already be resistant
to the distinguishing attacks and thus to ours. Yet it would be reasonable to allow the presence
of a simple related-key distinguisher when designing a primitive, as this is a very weak type of
attack; in fact, this is for instance the approach taken by PRINCE, among others~\cite{PRINCE}, which admits
a trivial distinguisher due to its \textsf{FX} construction.
What we have shown is that one must be
careful when contemplating such a decision for \EM, and \textsf{FX} constructions, to some extent, as in that case it is actually
equivalent to allowing key recovery, the most powerful of all attacks.

\setcounter{section}{0}
\renewcommand\thesection{\Alph{section}}

\section{Proof-of-concept implementation for a 64-bit permutation}
\label{sec:toy}

We give the source of a C program that recovers a 64-bit key from a design similar to \proestotr where
the permutation has been replaced by a small ARX, for compactness. This allows to verify the correctness
of the attack, all the while showing that it is indeed very simple to implement. For the sake of simplicity,
we do not ensure that the nonce does not repeat in the queries.

\begin{minted}[breaklines,tabsize=2]{c}
#include <stdio.h>
#include <stdint.h>
#include <stdlib.h>

#define ROL32(x,r) (((x) << (r)) ^ ((x) >> (32 - (r))))
#define MIX(hi,lo,r) { (hi) += (lo); (lo) = ROL32((lo),(r)) ; (lo) ^= (hi); }

#define TIMES2(x) ((x & 0x8000000000000000ULL) ? ((x) << 1ULL) ^  0x000000000000001BULL : (x << 1ULL))
#define TIMES4(x) TIMES2(TIMES2((x)))

#define DELTA(x) (1ULL << (x))
#define MSB(x) ((x) & 0xFFFFFFFF00000000ULL)
#define LSB(x) ((x) & 0x00000000FFFFFFFFULL)

/* Replace arc4random() by your favourite PRNG */

/* 64-bit permutation using Skein's MIX */
uint64_t p64(uint64_t x)
{
	uint32_t hi = x >> 32;
	uint32_t lo = LSB(x);
	unsigned rcon[8] = {1, 29, 4, 8, 17, 12, 3, 14};

	for (int i = 0; i < 32; i++)
	{
	   MIX(hi, lo, rcon[i % 8]);
	   lo += i;
	}

	return ((((uint64_t)hi) << 32) ^ lo);
}

uint64_t em64(uint64_t k, uint64_t p)
{
	return p64(k ^ p) ^ k;
}

uint64_t potr_1(uint64_t k, uint64_t n, uint64_t m1, uint64_t m2)
{
	uint64_t l, c;

	l = TIMES4(em64(k, n));
	c = em64(k, l ^ m1) ^ m2;

	return c;
}

uint64_t recover_hi(uint64_t secret_key)
{
	uint64_t kk = 0;

	for (int i = 62; i >= 32; i--)
	{
		uint64_t m1, m2, c11, c12, n;

		m1 = (((uint64_t)arc4random()) << 32) ^ arc4random();
		m2 = (((uint64_t)arc4random()) << 32) ^ arc4random();
		n  = (((uint64_t)arc4random()) << 32) ^ 0x80000000ULL;
		c11 = potr_1(secret_key, n, m1, m2);
		c12 = potr_1(secret_key + DELTA(i), n ^ DELTA(i), m1 ^ DELTA(i) ^ TIMES4(DELTA(i)), m2);

		if (c11 != (c12 ^ DELTA(i)))
			kk |= DELTA(i);
	}

	return kk;
}

uint64_t recover_lo(uint64_t secret_key, uint64_t hi_key)
{
	uint64_t kk = hi_key;

	for (int i = 31; i >= 0; i--)
	{
		uint64_t m1, m2, c11, c12, n;
		uint64_t delta_p, delta_m;

		m1 = (((uint64_t)arc4random()) << 32) ^ arc4random();
		m2 = (((uint64_t)arc4random()) << 32) ^ arc4random();
		n  = (((uint64_t)arc4random()) << 32) ^ 0x80000000ULL;

		delta_p = DELTA(i) - MSB(kk) + (((LSB(~kk)) >> (i + 1)) << (i + 1));
		delta_m = DELTA(i) + MSB(kk) + LSB(kk);
		c11 = potr_1(secret_key + delta_p, n ^ DELTA(32), m1 ^ DELTA(32), m2);
		c12 = potr_1(secret_key - delta_m, n, m1 ^ TIMES4(DELTA(32)), m2);

		if (c11 == (c12 ^ DELTA(32)))
			kk |= DELTA(i);
	}

	return kk;
}

int main()
{
	uint64_t secret_key = (((uint64_t)arc4random()) << 32) ^ arc4random();
	uint64_t kk1 = recover_lo(secret_key, recover_hi(secret_key));
	uint64_t kk2 = kk1 ^ 0x8000000000000000ULL;

	printf("The real key is %016llx, the key candidates are %016llx, %016llx    ", secret_key, kk1, kk2);
	if ((kk1 == secret_key) || (kk2 == secret_key))
		printf("SUCCESS!\n");
	else
		printf("FAILURE!\n");

	return 0;
}
\end{minted}

\renewcommand\thesection{\arabic{section}}
