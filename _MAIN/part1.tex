\part[Prolégomènes]
	{Prolégomènes}
\label{part:prolego}

\section*{Overview}

The first part of this thesis serves as an introduction to the other two and to some of the
main topics of symmetric-key cryptography. The first two chapters respectively introduce block ciphers
and hash functions, which are the main objects studied in our work. The third and last chapter
describes a simple attack on \proestotr, an authenticated-encryption scheme. The prerequisites necessary
to understand this attack are minimal, and we believe it to be a fit introductory example to cryptanalysis.  



\cleardoublepage
\chapter*{Contents}
\parttoc

%%%%%%%%%%%%%%%%%%%%%%%%%%%%%%%%%%%%%%%%%%%%%%%%%%%%%%%%%%%%%%%%%%%%%%%%
%%%%%%%%%%%%%%%%%%%%%%%%%%%%%%%%%%%%%%%%%%%%%%%%%%%%%%%%%%%%%%%%%%%%%%%%

\chapter[Introduction aux chiffres par bloc]{Block ciphers basics}
\label{cha:block_intro}

\section{Block ciphers}

A \emph{block cipher} is a family of injective mappings over finite domains and co-domains, indexed by a finite set of \emph{keys}. This very broad definition will
in fact always be specialized, taking domains and co-domains of identical sizes, and all parameters living in the binary world. Hence, a block cipher
is a mapping $\E : \{0,1\}^\kappa \times \{0,1\}^n \rightarrow \{0,1\}^n$ such that for all $k \in \{0,1\}^\kappa$, $\E(k,\cdot)$ is a permutation.
We call $\kappa$ the \emph{key size} and $n$ the \emph{block size} of $\E$. Typical parameter sizes are $\kappa \in \{64, 80, 128, 192, 256\}$ (though
64 and 80-bit keys are now considered to be too short to provide adequate security) and $n \in \{64, 128, 256\}$.
We usually require $\E$ and its inverse $\E^{-1}$ to be efficiently computable (depending on the intended application, it may be enough for only
one of these to be efficient).

The most immediate purpose of block ciphers is to provide confidentiality of communications. Two parties $A$ and $B$ who share a key $k$\footnote{We
completely ignore the problem of obtaining such a shared key.} for the same
block cipher are able to send encrypted messages $c \defas \E(k,p)$, $c' \defas \E(k,p')$, etc. The non-key input to $\E$ is generally called
the \emph{plaintext}, and the output of $\E$ is called the \emph{ciphertext}.

If $\E$ is such that the permutations $\E(k,\cdot)$ are hard to invert when $k$ is unknown, $A$ and $B$ may suppose that a secure channel of communication
between them consists in injecting their messages to strings $m_0||m_1||\ldots||m_\ell$ of sizes multiple of $n$ and sending encrypted messages
$\E(k,m_0)||\E(k,m_1)||\ldots||\E(k,m_\ell)$. There are two major problems with this scheme, however, regardless of the security of the block
cipher: 1) The scheme is not \emph{randomised}, \ie encrypting the same plaintext twice always results in the same ciphertext. An eavesdropper
(a ``passive adversary'') on the channel between $A$ and $B$ can thus detect when identical message blocks have been sent. 2) The
communication is not authenticated. An active adversary on the channel may delete or modify some of the blocks of a message, append to a message
some blocks from a previous message, or add randomly generated blocks. All of this can be done without $A$ and $B$ noticing that someone
is maliciously tampering with the channel.

Problems such as the ones above are solved by designing secure \emph{modes of operation}. We do not study this topic in this thesis, but
we mention some elements related to modes in \autoref{sec:bc_modes}. But first, we make the intuition behind the evaluation of the security
of block ciphers themselves more explicit in \autoref{sec:bc_sec}.

\section{Security of block ciphers}
\label{sec:bc_sec}

We keep this section relatively informal. Our goal is to be able to specify what it means for $\E$ to be a good block cipher from
a practical point of view. Yet, we start by defining the useful notion of \emph{ideal block cipher}.

\begin{defi}[Ideal block cipher]
An \emph{ideal block cipher} $\E$ is a mapping $\{0,1\}^\kappa \times \{0,1\}^n \rightarrow \{0,1\}^n$ s.t. all the permutations
$\E(k,\cdot)$ are drawn independently and uniformly at random among the permutations of $\{0,1\}^n$.
\end{defi}

This definition intuitively corresponds to the best we can achieve from the definition of a block cipher. For small values of $n$
(\eg up to $20 \sim 32$ depending on the desired performance), one can implement ideal block ciphers by using an appropriate
shuffling algorithm (such as the one variously attributed to Fisher, Yates, Knuth, etc.~\cite{uniform_shuffle}, which we will
call ``FYK''). As this method
requires $\bigo(2^n)$ setup time and memory per key, it is obviously impractical for cryptographically common block sizes of $n \geq 64$.
Even for small values of $n$, running the FYK shuffle requires a considerable amount of randomness parameterized by the keys, which
is not something trivial to fulfill. All of this leads to the fact that we are forced most of the time to use ``approximations'' of
ideal block ciphers. A useful (mostly theoretical) way of quantifying the security of a specific block cipher is to measure ``how far'' it
is from being ideal. Informally, this is done by upper-bounding the \emph{advantage} (over a random answer) that any adversary
(with some bounded resources)
has of distinguishing whether he is given black-box access to a randomly-drawn permutation or to an instance of the block cipher
with a randomly chosen (unknown) key. This statement can be made more precise in the form of the following definition
(similar to the one that can be found \eg in \cite{DBLP:journals/jcss/BellareKR00}):

\begin{defi}[Pseudo-random permutations (PRP)]
We consider a block cipher $\E$ of key size $\kappa$ and block size $n$.
We write $\Pi_{2^n}$ for the set of permutations on binary strings of length $n$; $x \overset{\$}{\leftarrow} \mathcal{S}$
the action of drawing $x$ uniformly at random among elements of the set $\mathcal{S}$; $\mathcal{A}^{f}$ an algorithm with
oracle (black-box) access to the function $f$ and which outputs a single bit.
Then we define the \emph{PRP advantage} of $\mathcal{A}$ over $\E$, written $\Adv^{\text{PRP}}_{\E}(\mathcal{A})$ as:
\[
\Adv^{\text{PRP}}_{\E}(\mathcal{A}) = |\Pr[\mathcal{A}^f = 1~|~f \overset{\$}{\leftarrow} \Pi_{2^n}] - \Pr[\mathcal{A}^f = 1~|~f \defas \E(k,\cdot), k \overset{\$}{\leftarrow} \{0,1\}^\kappa]|.
\]
The \emph{PRP security} of $\E$ w.r.t. the \emph{data complexity} $q$ and \emph{time complexity} $t$ is:
\[
\Adv^{\text{PRP}}_{\E}(q,t) \defas \max_{\mathcal{A}\,\in\,\text{Alg}^{f\backslash q, \E\backslash t}} \{\Adv^{\text{PRP}}_{\E}(\mathcal{A})\}.
\]
Here, $\text{Alg}^{f\backslash q, \E\backslash t}$ is the set of all algorithms $\mathcal{A}$ with oracle access to $f$ that perform at most $q$ oracle accesses
and which run in time $\bigo(t)$, with the time unit being the time necessary to compute $\E$ once.
\label{def:prp}
\end{defi}
There exists a related notion of \emph{strong pseudo-random permutation} (SPRP) where one considers algorithms given oracle access both to $f$ and its inverse.

\medskip

\autoref{def:prp} is quite useful in some contexts, for instance to prove that a construction using a block cipher is not significantly less secure than the latter. This is
typically done by defining an advantage function similar to PRP security for the higher-level construction (this being for instance CBC-MAC in the case of \cite{DBLP:journals/jcss/BellareKR00}) and by showing that
it is not more than the PRP security of the block cipher plus some (reasonably small) extra terms.

However, this definition is not constructive, in the sense that it does not provide any (efficient)
way of computing the PRP security of a block cipher in general (some results do exist for specific block cipher constructions (usually modulo access to a lower-level primitive such as a
``random permutation'') such as the one due to Even and Mansour~\cite{EM}, which we will see again in \autoref{chap:emrka}).
A major topic in symmetric cryptography is to analyse explicit block ciphers in order to assess their concrete security against attacks. In the language of \autoref{def:prp}, this
consists in finding algorithms for which $q$, $t$ and the PRP advantage is known. Any such attack on a block cipher $\E$ allows to lower-bound its PRP-security at a given point.
In reality, though, the world of block cipher cryptanalysis is more nuanced than what \autoref{def:prp} may lead us to believe; practically important characteristics of an attack
are also its memory complexity, distinguishing between its online and offline time complexity, whether it applies equally well to all keys or if it is only successful
for some ``weak'' subset thereof, whether it also recovers $k$ when $f$ was instantiated from $\E$, or an algorithm equivalent to $\E(k,\cdot)$, etc. We devote the remainder of this
section to sketching some typical elements of attacks on block ciphers.

\subsection{Distinguishers and attacks}
The core of many concrete attacks on block ciphers is made of \emph{distinguishers}, which can be defined as algorithms using reasonable resources which have a non-negligible advantage according to \autoref{def:prp}.
There is no easy answer as to what ``reasonable'' and ``non-negligible'' should mean in the context of actual cryptanalysis, as the key and block size of a specific cipher are fixed values. While some ciphers or potential distinguishers
may be parameterized in a way that helps to make the definition meaningful, this does not have to be the case. Sometimes, one is easily convinced by the performance of an algorithm so that there is
consensus that it can be called a distinguisher (\eg distinguishing $\E$ of key and block size $2^{128}$ with $q = 2$, $t = 2^{20}$, probability $\approx 1$), while some other times the picture is much less clear
(\eg $q = t =  2^{120}$ and probability $\approx 1$). We will ignore this issue altogether and assume that all the attacks of this chapter are consensual.

\subsubsection{Classes of distinguishers for block ciphers}

We now briefly describe two examples of types of distinguishers, which exploit ``non-ideal'' behaviours of different nature.

\bigskip

We start with \emph{differential distinguishers}, which are part of the broader class of \emph{statistical} distinguishers.
The basic idea of the latter is to define an event which has a different probability distribution for the target (the block cipher $\E$)
than for a random permutation drawn from $\Pi_{2^n}$. Running the distinguisher then consists in collecting a certain number of samples (obtained through
the oracle) and deciding from which distribution
%(the one entailed by $\E$ or the one entailed by a random permutation)
those are the
most likely to have been drawn.
A differential distinguisher instantiates this idea by considering a certain type of statistical events. Another major class of
statistical distinguishers is the one of \emph{linear distinguishers}.

Consider a block cipher $\E$; a \emph{differential} for $\E$ is a pair $(\Delta,\delta)$ of input and output \emph{differences},
according to some group law $+$\footnote{We implicitly only consider non-trivial differences where $\Delta \neq 0$.}.
In the huge majority of cases, $+$ is the addition in $\mathbf{F}_2^n$,
\ie the bitwise exclusive OR (XOR); in this case we usually use the alternative notation $\oplus$. Sometimes, $+$
is taken to be the addition in $\mathbf{Z}/2^n\mathbf{Z}$, and some other times differences according to the two laws may be jointly used.
A \emph{differential pair} for the difference $(\Delta,\delta)$ is an ordered pair of plaintexts and their corresponding ciphertexts (for some
key $k$)
$p$, $c \defas \E(k,p)$, $p'$, $c' \defas \E(k,p')$ such that $p - p' = \Delta$, $c - c' = \delta$. When differences are over $\mathbf{F}_2^n$,
subtraction coincides with addition and the pair can be unordered. We consider this to be the case in the remainder of this description.

We call \emph{differential probability} of a differential w.r.t. a permutation $\Perm$ the probability of obtaining a differential pair
for $\Perm$:
$\DP^{\Perm}(\Delta,\delta) \defas \Pr_{p\,\in\,\{0,1\}^n}[\Perm(k,p) \oplus \Perm(k,p \oplus \Delta) = \delta]$.
The most important characteristic of a differential for a block cipher is its \emph{expected differential probability}, which
is simply its differential probability for $\E(k,\cdot)$ averaged over $k$:
$\EDP^{\E}(\Delta,\delta) \defas 2^{-\kappa}\sum_{k\,\in\,\{0,1\}^\kappa} \DP^{\E(k,\cdot)}(\Delta,\delta)$.
A common assumption is that for most keys and differentials, the fixed-key DP is close to the average EDP.
The DP of a random differential w.r.t. a random permutation can be approximated by
a Poisson distribution: the (approximate) number of differential pairs is $\sim \poi(2^{-1})$, of mean and variance $2^{-1}$
(see~\cite{DBLP:journals/jmc/DaemenR07}, using an earlier result~\cite{DBLP:journals/joc/OConnor95}).
As there are $2^{n-1}$ possible pairs, the expected DP is thus $2^{-n}$\footnote{Note however that the DP is in fact restricted to values multiple of $2^{-n+1}$.}.
For a distinguisher on $\E$ to be of any use, we need its EDP not to be equal to $2^{-n}$. If it is far enough
from that (\eg $2^{-3n/4}$), we usually make the simplifying working hypothesis that all DPs are equal to their expected value (or rather the nearest
possible values).
In such a case, using the distinguisher consists in collecting $\propto 1/\EDP^{\E}(\Delta,\delta)$ plaintext pairs verifying
the input difference and counting how many of them verify the output difference. We decide that we are interacting with $\E$ if and only if this is one or more.

\bigskip

Another kind of distinguishers is based on \emph{algebraic} representations of block ciphers. One can always redefine a block cipher
$\E : \{0,1\}^\kappa \times \{0,1\}^n \rightarrow \{0,1\}^n$ as an ordered set of functions $\F_i : \{0,1\}^{\kappa+n} \rightarrow \{0,1\}$ that project
$\E$ on its $i^\text{th}$ output bit: $\E \equiv \langle \F_0, \ldots, \F_{n-1} \rangle$. The $\F_i$s can be understood as Boolean functions
$\mathbf{F}_{2^{\kappa + n}} \rightarrow \Ftwo$ which are themselves in bijection with elements of $\Ftwo[x_0,x_1,\ldots x_{\kappa + n-1}]/<x_i^2-x_i>_{i<\kappa + n}$,
\ie multivariate polynomials in $\kappa + n$ variables over $\Ftwo$. The polynomial to which a Boolean function is mapped is called its \emph{algebraic normal form} (ANF);
the ANF of $\E$ is the ordered set of ANFs of its projections.

An important characteristic of an ANF is its (maximal) degree, which can be used to define simple yet efficient distinguishers. The degree
of (the ANF of) an $n$-bit permutation is at most $n - 1$, and it is expected of a random permutation to be of maximal degree. If a block cipher
has degree $d < n - 1$, it can be distinguished by differentiating it on enough values. This simply requires to evaluate the oracle on $2^d$ properly chosen
values (essentially a cube of dimension $d$) and to sum them together. If the result is the all-zero ciphertext, the oracle is likely to be of degree less than
$d$ and is hence assumed to be $\E$; if this is not the case, it is necessarily of degree strictly more than $d$ and hence assumed to be a random permutation.

\subsubsection{Extending distinguishers to key-recovery attacks}

In the definition of PRP security, we were content with the notion of distinguisher. In actual attacks on block ciphers, however, the end objective
would ideally be to recover the unknown key used by the oracle. The context of a concrete attack is also different from a PRP security game as one usually knows that he is interacting with a specific
cipher $\E$ and not a random permutation, and there is seemingly no point in running a distinguisher at all.
Despite these observations, distinguishers are in fact useful in many cases, and are often at the basis of key-recovery attacks\footnote{Some attacks
are not distinguisher-based. Though many of them are quite interesting, we do not describe them here.}. We briefly explain
the basic idea of this conversion; to do this, we need to assume that $\E$ possesses a certain structure (this is completely without loss of generality).

An \emph{iterative block cipher} is a cipher $\E$ that can be described as the multiple composition of a \emph{round function} $\R$ (possibly with additional
composition of an initialization or finalization function that we ignore here) : $\E \equiv \R \circ \cdots \circ \R$. Let us assume that a ``full''
application of $\E$ is made of $r$ rounds. A distinguisher-based key-recovery attack first consists in finding a distinguisher on
a \emph{reduced-round} version of $\E$ made of the composition of $d < r$ round functions. The next step simply consists in querying the oracle on inputs verifying the distinguisher condition (for instance
plaintexts with difference $\Delta$, in a differential case); as one obtains encryption with the full block cipher, one is not expected to be able to directly run the distinguisher
on these values. The main idea comes from the third step, where one guesses values for part of the unknown key $k$ of $\E$ which allow him to partially decrypt the ciphertexts by $r-d$ rounds. Then, if
the guess was correct, he obtains ciphertexts for the cipher reduced to $d$ rounds, on which the distinguisher is expected to be successful. On the other hand, if the guess was incorrect, one obtains
ciphertexts somehow equivalent to the ones of a $(2r-d)$-round cipher and the distinguisher should fail. Thus, this overall approach gives us a method to verify a guess for part of the unknown key.

The procedure as described above calls for several comments. 1) The cost of guessing part of the key obviously adds to the complexity of the distinguisher, so that the overall complexity of the attack
is higher than the latter. Thus, only distinguishers that leave a sufficient ``margin'' may be converted to key-recovery attack. 2) There are various reasons why a distinguisher may be able to work
even though only part of the key was guessed, for instance because the entire key is spread over many rounds, or because the distinguisher may be run on only ``part of the state'' of $\E$, which computation
does not require the entire ``round key''. 3) The part of the key that was not recovered thanks to the distinguisher can be obtained by different means. For instance, another distinguisher may be
used which leads to another part of the key, or it can simply be guessed exhaustively.


\subsubsection{Attack models}

So far we have discussed how to express the security of block ciphers and how to attack them in a rather simple case when one is given access to a single ``secret'' oracle. This setting may be generalised
in some ways, for instance by providing more than one oracle. One such common generalisation is to attack a cipher in the \emph{related-key model}, where one is given oracle access to
$\E(k,\cdot)$, $\E(\rka(k),\cdot)$, with $\rka(\cdot)$ one or more mappings on the key space. A crucial observation in this case is that $\rka$ cannot be arbitrary, as some mappings may
be so powerful that they allow to attack (almost) every cipher; speaking of the security of $\E$ in such a model is then meaningless. We will mention this matter again in \autoref{chap:emrka}.

The potential problems arising from ill-defined related-key models are a useful reminder that attacks should be specified in a well-founded way. While some of them may be
considered too unrealistic to be of practical significance (this is in fact a rather common reproach to related-key attacks at large), this question is only secondary to their not allowing to trivially
attack any cipher.

\section{Using block ciphers}
\label{sec:bc_modes}

We mentioned in the beginning of this chapter that block ciphers do not provide adequate security if they are used directly and not as part of a wider construction. One calls \emph{mode
of operation} such a construction that results in a (hopefully) functional cryptosystem. We do not describe modes in this section, but reiterate from the introduction the essential conditions that they
must meet.

A foremost requirement is that a mode be randomised, in the sense that encrypting the same message with the same key twice should not result in the same ciphertext. This can be
enforced through the notion of \emph{indistinguishability} in a \emph{chosen-plaintext attack} scenario (\textsf{IND-CPA}) and its close relatives. Roughly, this is
defined thanks to the following process: an adversary is given a black-box access to the encryption procedure of a certain cryptosystem, then prepares two messages $m_0$ and $m_1$ and sends them to an oracle. This oracle randomly selects one of the two messages and
returns its associated ciphertext. Finally, the adversary is again given access to the cryptosystem and then tries to guess which message was encrypted. The cryptosystem is \textsf{IND-CPA}
if no adversary (with appropriately bounded resources) is successful in his guess with a non-marginal advantage. It is clear in particular that a deterministic cryptosystem cannot be secure according
to this definition.

We also already mentioned that a cryptosystem should provide authentication of the communicating parties. This is either done directly by the mode of operation (which is then
called an \emph{authenticated encryption} mode, or AE) or by combining an encryption-only mode with a \emph{message authentication code} (MAC) in an appropriate way. The current
trend is to favour the former approach, as it tends to lead to more efficient schemes.

We conclude this chapter by highlighting the prevalence of the \emph{birthday bound} (coming from the so-called \emph{birthday paradox}, which we will
only define in \autoref{chap:hashfun}) in cryptography by stating the following fact: when using a block cipher of block size $n$, most modes of operation are only
secure up to the encryption of $\approx 2^{n/2}$ blocks, even if the key size of the block cipher can be much bigger than that.


% TODO REFS.... Where???


\chapter[Introduction aux fonctions de hachage]{Hash functions basics}
\label{chap:hashfun}
\section{Introduction}
\label{sec:intro}

A (binary) hash function is a mapping from bit strings of arbitrary length (the messages) to strings of a fixed predetermined length (the digests, or hashes):
$\hash : \{0,1\}^* \rightarrow \{0,1\}^n$ for some integer $n$.
Many hash functions do not strictly adhere to this definition, as they upper-bound the length of their inputs by a (usually) large integer such as $2^{64}$. This currently has
no impact in practice, as all potential inputs to a hash function are much shorter than these upper limits; it is of course debatable whether this will always remain the case. 
The typical output length of hash functions is of a few hundred bits, often a multiple of thirty-two; popular hash functions of the present and the past may be found with
$n \in \{128, 160, 224, 256, 384, 512\}$.

\subsection{Cryptographic hash functions}

A \emph{cryptographic} hash function\footnote{We will obviously only consider such functions from now on.} is a hash function $\hash$ that verifies a certain number of \emph{security properties}, which express the difficulty of computing inputs
to $\hash$ that verify some conditions. There are three ``classical'' security properties that must be met (at least to some extent)
by any secure hash function; in an informal way, they can be expressed as:
\begin{defi}[Preimage resistance] Given a target $\targ$, an \emph{adversary} must not be able to find a message $\messhash$ such that $\hash(\messhash) = \targ$ with non-negligible probability using
less computational ressources than what is required to compute $\hash$ $\bigo(2^n)$ times.
\label{def:pre}
\end{defi}
\begin{defi}[Second preimage resistance] Given a message $\messhash$, an adversary must not be able to find a different message $\messhash'$ such that
$\hash(\messhash) = \hash(\messhash')$  with non-negligible probability using less computational ressources than what is required to compute $\hash$ $\bigo(2^n)$ times.
\label{def:2pre}
\end{defi}
\begin{defi}[Collision resistance] One must not be able to find two distinct ``fresh'' messages $\messhash$ and $\messhash'$ such that $\hash(\messhash) =
\hash(\messhash')$ with non-negligible probability using less computational ressources than what is required to compute $\hash$ $\bigo(2^{n/2})$ times.
\label{def:coll}
\end{defi}

The complexities associated with these definitions correspond to the ones of the best known \emph{generic} algorithms that can find messages with the desired
properties with high probability for any function. An \emph{attack} on a specific hash function is an algorithm that can achieve one of the above tasks (significantly)
more efficiently than the best generic algorithm. By definition, this violates the associated security property.
Above, we define the complexity of these generic algorithms in terms of calls to $\hash$. In practice, it may be necessary to adjust
this to a finer granularity; for instance, for the \merkdam hash functions that we describe in \autoref{sec:mdhf}, it makes more sense to evaluate the complexity
in terms of calls to what will be defined as a \emph{compression function}.

As a side remark, the lower complexity target of $\bigo(2^{n/2})$ associated with collision resistance is a direct consequence of the so-called \emph{birthday paradox} (which
is encountered frequently in cryptography): informally,
$N$ elements define $\bigo(N^2)$ pairs, thus if these elements are drawn at random from a set of size $S$, we expect two of them to be the same (\ie to collide)
with high probability after drawing $\bigo(\sqrt{S})$ of them.
One can also notice that unlike (second) preimage resistance, the definition of collision resistance does not include an input target or message. As such, it does not in itself
exclude (generic) algorithms that simply return two constant colliding messages, in constant time. Such algorithms trivially violate the security property, but
they do not say anything about the security of the functions (not the least because of their generic nature). Consequently, they are usually considered irrelevant in a cryptographic context, and they do not make
the definition devoid of sense.

\bigskip

The nature of the notions used to define security for cryptographic hash functions is based on requirements from the various cases where they may be used in concrete cryptosystems or protocols.
Depending on situation, not all three above resistance may be necessary; for instance, it may be the case that collision resistance is not required, or that, say, only second preimage resistance
is relevant. However, a hash function is usually expected to be used in a certain number of settings, and it is thus understandingly expected to be secure against all attacks.

We only briefly sketch some possible uses of hash functions here, as an illustration.

\paragraph{Hash and sign signatures} are one of the main settings where hash functions may be employed; the objective is, given a digital signature
algorithm $\sign$, to efficiently and securely sign \emph{long} messages. Directly doing so using $\sign$ is usually not an option, because signature algorithms are rather slow (especially compared
to hash functions, which are typically at least $500 \sim 1000$ times faster) and may not behave well on long messages. A useful alternative is instead to first compute the digest of the
message that needs to be signed (say ``$\messhash$'') and to sign the digest instead of the whole message. One then gives an output of the form $(\messhash, \sign(\key,\hash(\messhash)))$. 

It is obvious that for such a scheme to work, the function $\hash$ needs to be at least resistant to second preimages. If this were not the case, an adversary could intercept a message $\messhash$
signed by $A$, replace $\messhash$ by $\messhash'$ such that the two messages collide through $\hash$, and claim that $A$ signed $\messhash'$.
Collision resistance is also important for similar reasons.

\paragraph{Password hashing} is another common setting where hash functions may be useful. In this case, we assume that a certain entity wishes to allow users to authenticate themselves
through passwords; consequently, it must remember the valid password of each user. As there would be obvious security issues with the entity storing the passwords themselves, an idea
is to instead store their images through a hash function $\hash$. Thus, if an adversary finds the database of users and their associated (hashed) passwords, he would be unable to find
the passwords (or some equivalent input), provided that $\hash$ is preimage-resistant.

In this setting, second preimage and collision resistance are not strictly needed. However, it should be noted that even when built with a secure hash function,
the scheme just described above has severe issues, which descriptions are beyond the scope of this article.

\paragraph{Hash-based signatures.} A less common use of hash functions is to utilise them to directly define signature schemes (see \eg \cite{DBLP:conf/crypto/Merkle87}). We will not describe such schemes in detail
here, but it is interesting to mention a few of their specificities. Similarly as other schemes using hash functions, only preimage resistance is needed. The main
idea of the schemes is that the signing party makes some digests public while keeping their inputs secret; signing a message then consists in selectively revealing some of
these inputs. Thus, being able to compute preimages for the hash function breaks the scheme, but collisions are not a threat.
A distinguishing feature of hash-based signatures is that the hash function may only need to be used on inputs of short, possibly fixed size. 

\paragraph{Message authentication codes.} The last possible usage of hash functions that we detail here is the building of \emph{message authentication codes}, or MACs, which
can somehow be seen as the keyed variants of hash functions. A MAC takes as input a key $\key$ and a message $\messhash$ and outputs a tag $\tagg = \mac(\key, \messhash)$.
If an adversary does not know the key, it should be hard for him to find a valid (message, tag) pair for $\mac(\key, \cdot)$, with the message either being of his choosing
or imposed by a challenger. These notions correspond to \emph{existential} (for the former) and \emph{universal} (for the latter) \emph{forgery}. Of course, it is also
higly necessary that no tag collisions occur, whether or not these happen on specific messages requested by an adversary.

Hash functions seem to be good candidates to build MACs, and indeed generic hash function based constructiosn such as HMAC~\cite{DBLP:conf/crypto/BellareCK96} are popular. However, the exact security of
these constructions is not always easy to establish, and there are usually faster alternatives such as MACs based on universal/polynomial hash functions (see \eg~\cite{DBLP:conf/crypto/BlackHKKR99}).

\subsubsection{\merkdam hash functions}
\label{sec:mdhf}

One of the first frameworks for hash function design to have been developed is the so-called \merkdam construction, that was independently developed by
Merkle and Damg\aa rd in 1989~\cite{DBLP:conf/crypto/Merkle89a,DBLP:conf/crypto/Damgard89a}. The idea of this construction is to make the arbitrary-length
inputs to a hash function manageable by defining it as the iteration of \emph{compression function} with a small fixed-size (co-)domain.
This makes the overall design much easier, but without \emph{a priori} ensuring that the resulting function will be secure. The main contribution of
Merkle and Damg\aa rd in that respect is to give a construction such that the security of the function can be (partially) reduced to the one of
the compression function: they show that for \eg collision attacks, an attack on the hash function can be exploited to build an attack on the compression function.
Taking the contrapositive, as long as there are no collision attacks on the compression function, we can be confident that the hash function is secure in that respect.
This is quite similar in a way to building symmetric cryptosystems by combining a secure block cipher and a secure mode of operation.

The construction works as following. A compression function $\compress : \{0,1\}^n \times \{0,1\}^b \rightarrow \{0,1\}^n$
takes two inputs: a \emph{chaining value} $\chain$ and a \emph{message block} $\messblock$, and produces another chaining value as output.
The hash function $\hash$ associated with $\compress$ is built by extending the domain of the latter to $\{0,1\}^*$ (or rather $\{0,1\}^N$ for a large $N$, most of the time).
This is done by specifying an \emph{initial value} \iv for the first chaining value $\chain_0$, which is a constant for the hash function, and by defining the image
of a message $\mess$ by $\hash$ through the following process:
\begin{enumerate}
\item $\mess$ is padded to a size multiple of the message block size $b$. Various padding rules may be employed, but it is obviously important that they should not
introduce trivial collisions. Most of the time, the size (usually in bits) of the non-padded message $\mess$ is included in the padding in one way or the other.
This is usually called \merkdam-strengthening, and it is essential to make many of the common instantiations of the framework secure\footnote{One issue that
could arise in the absence of such strengthening is that fixed-points of the compression function could be used to produce collisions; such fixed-points
are easy to build for some popular compression function constructions (including the \emph{Davies-Meyer} construction
of the \mdsha family described in \autoref{sec:mdsha}).}.
\item The padded message is then iteratively fed to the compression function: call $\messblock_0||\messblock_1||\linebreak\ldots||\messblock_r$ this message (assuming
it is on $r+1$ blocks, w.l.o.g.), then define $\chain_{i+1} \defas \compress(\chain_i, \messblock_i)$. The digest $\hash(\mess)$ is equal to the last
chaining value $\chain_{r+1}$.
\end{enumerate}
We give an illustration of this process in \autoref{fig:merkdam}.

\begin{figure}[!htb]
\begin{center}
%\documentclass{standalone}
%\usepackage{rotating}	%% introduce sideways env
%\usepackage{tikz}
%
%%% Public TikZ libraries
%\usetikzlibrary{positioning}
%\usetikzlibrary{shapes}
%
%%% Custom TikZ addons
%\usetikzlibrary{crypto.symbols}
%\tikzset{shadows=no}        % Option: add shadows to XOR, ADD, etc.
%
%%% Document
%\begin{document}
\begin{tikzpicture}[scale=0.4]
	\path[anchor=east] (-1,0.5) node {$pad(m)=$};
	\draw[fill=Fuchsia!20,thick,inner sep=2ex] (0,0) rectangle (16,1);

	%% Separations in the message
	\draw[thick] ++( 4,0) -- ++(0,1); \path (   2,0.5) node {$\messblock_{0}$};
	\draw[thick] ++( 8,0) -- ++(0,1); \path ( 4+2,0.5) node {$\messblock_{1}$};
	\draw[thick] ++(12,0) -- ++(0,1); \path ( 8+2,0.5) node {$\messblock_{2}$};
	\draw[thick] ++(16,0) -- ++(0,1); \path (12+2,0.5) node {$\messblock_{3}$};

	%% Compressions functions 
	\begin{scope}[shift={(0.5,-4)}]
		\node [draw,trapezium,trapezium left angle=70,trapezium right angle=70,minimum height=0.7cm,thick,fill=YellowOrange!30,shift={(1.15,0.4)},rotate=-90] 
		{\begin{sideways}$\compress$\end{sideways}};
		\draw[->,thick] ++(1.5,+4) -- ++(0,-2.5) -- ++(0.5,0);
		\draw[->,thick] ++(0,0.5) node[left] {$\chain_{0}=\iv$}-- ++(2,0);
	\end{scope}

	\begin{scope}[shift={(4.5,-4)}]
		\node [draw,trapezium,trapezium left angle=70,trapezium right angle=70,minimum height=0.7cm,thick,fill=YellowOrange!30,shift={(1.15,0.4)},rotate=-90] 
		{\begin{sideways}$\compress$\end{sideways}};
		\draw[->,thick] ++(1.5,+4) -- ++(0,-2.5) -- ++(0.5,0);
		\draw[->,thick] ++(-0.2,0.5) -- node[below] {$\chain_{1}$} ++(2.2,0);
	\end{scope}

	\begin{scope}[shift={(8.5,-4)}]
		\node [draw,trapezium,trapezium left angle=70,trapezium right angle=70,minimum height=0.7cm,thick,fill=YellowOrange!30,shift={(1.15,0.4)},rotate=-90] 
		{\begin{sideways}$\compress$\end{sideways}};
		\draw[->,thick] ++(1.5,+4) -- ++(0,-2.5) -- ++(0.5,0);
		\draw[->,thick] ++(-0.2,0.5) -- node[below] {$\chain_{2}$} ++(2.2,0);
	\end{scope}

	\begin{scope}[shift={(12.5,-4)}]
		\node [draw,trapezium,trapezium left angle=70,trapezium right angle=70,minimum height=0.7cm,thick,fill=YellowOrange!30,shift={(1.15,0.4)},rotate=-90] 
		{\begin{sideways}$\compress$\end{sideways}};
		\draw[->,thick] ++(1.5,+4) -- ++(0,-2.5) -- ++(0.5,0);
		\draw[->,thick] ++(-0.2,0.5) -- node[below] {$\chain_{3}$} ++(2.2,0);
	\end{scope}

	\begin{scope}[shift={(16.5,-4)}]
		\draw[->,thick] ++(-0.2,0.5) -- ++(0.75,0) node[right] {$\chain_4 = \hash(m)$} ;
	\end{scope}
	
\end{tikzpicture}
%\end{document}

\caption{A \merkdam hash function processing a four-block input. Figure adapted from \cite{TiKZ:Cryptographers}.\label{fig:merkdam}}
\end{center}
\end{figure} 

We conclude this presentation by sketching the reduction proofs of the construction for collision and first preimage attacks. There can be no such proof
for second preimages, as there exists some generic attacks on \merkdam functions independently of the security of the compression function. We will
briefly discuss these in \autoref{sec:refining_md}.
It is also important to notice that these reductions are one-way. There is no similarly generic way to convert, say a collision for a compression function $\compress$,
to a collision for a \merkdam function $\hash$ based on it. Still, such a collision would violate the security reduction, and thus no formal guarantee could be given
anymore on the collision resistance of $\hash$. We will discuss such issues in slightly more details in \autoref{sec:refining_md}.

\begin{prop}
A collision on a \merkdam hash function $\hash$ implies a collision on its compression function $\compress$\footnote{Without losing much generality, we assume in this proof
that \merkdam strengthening is used, that the message length is appended at the end of the padding, and that it fits in a single message block.}.
\end{prop}
\begin{proof}
Assume we have $\mess$, $\mess' \neq \mess$ s.t. $\hash(\mess) = \hash(\mess')$.

Case 1: $\mess$ and $\mess'$ have a different length.
The last message blocks $\messblock_r$ and $\messblock'_{r'}$ both include the length
of the message, which is different. Thus $(\chain_r,\messblock_r)$ and $(\chain'_{r'}, \messblock'_{r'})$ are distinct and collide through $\compress$.

Case 2: $\mess$ and $\mess'$ are of the same length. Assume w.l.o.g. that the messages fit on $r + 1$ blocks after padding.
Call $i$ the highest block number such that $\messblock_{i} \neq \messblock'_{i}$. If $i = r$, then $(\chain_r,\messblock_r)$ and $(\chain'_{r}, \messblock'_{r})$
are distinct and collide through $\compress$. If $i < r$, either $\chain_{i+1} = \chain'_{i+1}$, thus $(\chain_i,\messblock_i)$ and $(\chain'_{i}, \messblock'_{i})$
form a valid collision pair for $\compress$, otherwise, we have a non-empty sequence of input pairs $(\chain_j, \messblock_j)$ and $(\chain'_j, \messblock'_j)$,
$j = i+1\ldots r$ such that $\messblock_j = \messblock'_j$ for all $j$. As $\chain_{r+1} = \chain'_{r+1}$, at least one element of the sequence
collides through $\compress$. The two input pairs in the first one to do so are different, and thus form a valid collision for $\compress$.
\end{proof}

\begin{prop}
A preimage on a \merkdam hash function $\hash$ implies a preimage on its compression function $\compress$.
\end{prop}
\begin{proof}
Let $\mess$ be a message fitting on $r+1$ blocks after padding s.t. $\hash(\mess) = \targ$, with $\targ$ the preimage target.
Then $\compress(\chain_r, \messblock_r) = \targ$, and $(\chain_r, \messblock_r)$ is thus a valid preimage input for $\compress$.
\end{proof}

\subsubsection{Refining the security}
\label{sec:refining_md}

The three security notions of collision and preimage resistance can be refined and completed with additional ones. These are not always directly relevant to actual uses of hash functions, but they may nonetheless be useful to evaluate the security
of a function in a finer way. We may roughly distinguish between two kinds of additional properties by what they characterize: 1) Non-ideal behaviours of certain frameworks for hash function constructions; 2) Non-ideal
behaviours of specific instances of hash functions or of their building blocks.
To allow for more explicit definitions of these additional properties, it is useful to define a stronger, idealized view of hash functions:

\begin{defi}[Random oracle]
An $n$-bit \emph{random oracle} $\ro$ is a mapping $\{0,1\}^* \rightarrow \{0,1\}^n$ such that for every input $x$, its image $\ro(x)$ is drawn uniformly at random over $\{0,1\}^n$.
\end{defi}

According to the way we defined attacks, a random oracle is not vulnerable to them, as only generic algorithms may be used against it. It thus completely captures what
can be realized with hash functions: if a high-level construction is not secure when it is idealized as using a random oracle (when analysed in the \emph{random oracle model}),
one can never hope to make it so when instantiating the oracle by a concrete hash function. However, even if a construction is secure in such a model, it is not necessarily true that
this will still be the case once instantiated.

The \merkdam functions from \autoref{sec:mdhf} exhibit som of this \emph{non-ideal behaviours},
in the sense that there are some properties that can be realised through algorithms that are more efficient for \merkdam functions than for random oracles.

A good example of such a property is the concept of \emph{multicollision}, where one is required to find $r > 2$ messages $\messhash_{0},\ldots,\messhash_{r-1}$ which images through $\hash$
are all equal. The generic complexity (\eg for a random oracle) of this problem is $\bigo(2^{n\times (r-1)/r})$ calls to $\hash$ for an $n$-bit function, but Joux showed how to
find $2^r$-multicollisions in time $\bigo(r\times 2^{n/2})$ for \merkdam hash functions~\cite{DBLP:conf/crypto/Joux04}.
The basic idea used in this attack is that collisions for a \merkdam function $\hash$ can be chained together to lead to an exponentially growing number of distinct messages hashing
to the same value. We can easily see this with a simple case: assume that an attacker found two distinct colliding messages $\mess_0$ and $\mess'_0$ of the same length;
this can be done generically at a cost of $2^{n/2}$. One then looks for a second collision for the function $\widetilde \hash$ obtained by replacing the initial chaining
value $\chain_0$ by $\hash(\mess_0) = \hash(\mess'_0)$; this can again be obtained for a cost of $2^{n/2}$, resulting in $\mess_1$ and $\mess'_1$. Then, thanks to
the chaining property of the \merkdam construction, we have found four colliding messages $\mess_0||\mess_1$, $\mess'_0||\mess_1$, $\mess_0||\mess'_1$, $\mess'_0||\mess'_1$\footnote{We ignore
padding details here, but these can be easily worked out.}. It is easy to see how this generalizes to longer messages, leading to the quoted complexity of $\bigo(r \times 2^{n/2})$.
We can observe that one of the weaknesses of the construction that is exploited here is that a hash collision (which can be obtained generically in $2^{n/2}$ time) is also
a collision for the \emph{internal state}. We will briefly see in \autoref{sec:betterhash} that increasing the size of the state of a function is indeed a way to make it resistant
to such attacks.

Another good example, which this time directly violates the security property from \autoref{def:2pre}, are the second preimage attacks for long messages, again on \merkdam functions.
Dean~\cite{dean}, and later Kelsey and Schneier~\cite{DBLP:conf/eurocrypt/KelseyS05} showed how one can again exploit the structure of the function and internal collisions to compute a second preimage
of a long message more efficiently than with a generic algorithm. The complexity of Kelsey and Schneier's attack to find a second preimage for a message
of $2^k$ \emph{blocks} is $\approx \bigo(2^{n-k+1})$ calls to $\hash$. This means that \merkdam hash functions are actually inherently insecure, if we adhere
strictly to what we stated as security objectives. However, although significant, these attacks remain expensive, especially for messages of usual sizes. As such, they are not usually considered as threatening
the practical use of \merkdam functions, and indeed functions following this framework such as \shatwo~\cite{Nist-SHA} are still widely used.

\bigskip

We already hinted in \autoref{sec:mdhf} that an attack on a compression function would void the security reduction of a \merkdam function employing it. Thus it seems natural to also
analyse the security of the compression functions themselves. An attack would in this case demonstrate a non-ideal behaviour of the second kind we mentioned above, not targeting
a hash function in itself but one of its building blocks. There is a natural way to generalise the security properties expressed in \autoref{def:pre} $\sim$ \autoref{def:coll} to this context,
leading to the notion of \emph{(semi-)freestart} attacks:

\begin{defi}[Freestart preimage]
A \emph{freestart preimage} for a Merkle-Damg\aa rd hash function $\hash$ is a pair $(\freeiv,\messblock)$
of an \iv and a message that $\hash_\freeiv(\messblock) = \targ$, with $\hash_\freeiv(\cdot)$ denoting
the hash function $\hash$ with its original \iv replaced by $\freeiv$.
\label{def:free_pre}
\end{defi}

\begin{defi}[Semi-freestart collision]
A \emph{semi-freestart collision} for a Merkle-Damg\aa rd hash function $\hash$ is a pair $((\freeiv,\messblock), (\freeiv,\messblock'))$
of two \iv and message pairs such that $\hash_\freeiv(\messblock) = \hash_{\freeiv}(\messblock')$.
\label{def:semi-free_coll}
\end{defi}

\begin{defi}[Freestart collision]
A \emph{freestart collision} for a Merkle-Damg\aa rd hash function $\hash$ is a pair $((\freeiv,\messblock), (\freeiv',\messblock'))$
of two \iv and message pairs such that $\hash_\freeiv(\messblock) = \hash_{\freeiv'}(\messblock')$.
\label{def:free_coll}
\end{defi}

It can be noted that if the two messages of, say, a freestart collision pair are one-block long, the definition above becomes equivalent to the one of a collision
on the compression function $\compress$ used to build $\hash$.
There is little difference between the two in general, the feature of freestart attacks precisely being that they may (partially) exploit the chaining value
of the compression function as an additional input compared to hash function attacks.


\bigskip

Lastly, a somehow loosely defined addition to the security notions presented so far is the concept of \emph{distinguishers}, which denote non-ideal behaviours
of a hash function that are not otherwise captured by the previous definitions. We will not give a precise definition here, as these are made tricky by
the unkeyed nature of hash functions. Instead, we just briefly mention an example of such a distinguisher for the compression function of the \shaone hash function.

Jumping ahead, we will see in \autoref{sec:description} that the \iv and the message block of \shaone's compression function are made of five and sixteen
32-bit words respectively. In 2003, Saarinen showed that \emph{slid pairs} could be found for this compression function for a cost equivalent to $2^{32}$
function calls~\cite{DBLP:conf/fse/Saarinen03}. Such pairs are made of two \ivs $\state_{0,\ldots4}$, $\state'_{0,\ldots,4}$ and messages $\mess_{0,\ldots,15}$,
$\mess'_{0,\ldots,15}$ with $\state'_i = \state_{i-1}$ and $\mess'_i = \mess_{i-1}$\footnote{We skip here the details of how to determine the starting values $\state'_0$ and $\mess'_0$.}, such that the
pair of outputs of the function called on these inputs also has this property. Although it is not expected of a random function to exhibit such a property, there is no clear way to use it to mount
an attack against the main and secondary security objectives from above.


\subsubsection{Modern hash function frameworks}
\label{sec:betterhash}

We have mentioned some generic weaknesses of the \merkdam framework in \autoref{sec:refining_md}. As a consequence of these, modern hash function designs are usually based on alternative,
more secure constructions. We briefly review two of them: the \emph{wide-pipe} variation of \merkdam, or \emph{chop-MD}, and the \emph{sponge} construction.

\paragraph{Wide-pipe \merkdam.} The wide-pipe construction was introduced in 2005 by Lucks~\cite{DBLP:conf/asiacrypt/Lucks05} and Coron \etal under the name chop-MD~\cite{DBLP:conf/crypto/CoronDMP05}.
It is conceptually simple, and consists in using the \merkdam construction with a compression function of output size larger than the one of the chaining value. If we write $\lfloor\cdot\rfloor_n$
an arbitrary truncation function from $m > n$ to $n$ bits, we may define the $n$-bit chop-MD construction based on a compression function $\compress : \{0,1\}^m \times \{0,1\}^b \rightarrow \{0,1\}^m$
as $\lfloor\hash(\cdot)\rfloor_n$, with $\hash$ a standard \merkdam function built from $\compress$. Some variations are possible, for instance by considering other mappings from $m$ to $n$
bits instead of just a truncation.

One can easily see that a hash collision for such a function does not anymore imply a collision for the internal state. By choosing $m$ to be sufficiently large (for instance taking $m = 2n$),
one can achieve generic resistance to, say, multicollisions. In fact, Coron \etal proved that this construction is a secure \emph{domain extender for random oracles}, in the sense that
if the compression function is a fixed-size random oracle, using it in a chop-MD mode yields a function that is $\varepsilon$-indifferentiable from a random oracle (in the sense of Maurer \etal~\cite{DBLP:conf/tcc/MaurerRH04})
with $\varepsilon \approx 2^{-t}q^2$, $t = m - n$ being the number of truncated bits and $q$ the number of queries to $\compress$. In the light of \autoref{sec:refining_md}, this is a very useful result, as it
says that no non-ideal behaviour is introduced by the domain-extending construction. Such hash functions are thus expected to behave closely to the random oracles they may be expected to instantiate, as long as
their compression functions are ``ideal'' and that they are not queried too much w.r.t. to the size of their parameters.

\paragraph{Sponge construction.} The sponge construction was introduced in 2007 by Bertoni \etal~\cite{SpongeFunctions}. It is quite distinct from the \merkdam framework, notably because it does not use a compression
function as building block, but a function of equally-sized domain and co-domain. In most work, this function is even taken to be bijective.

The construction in itself is simple. Assume we want to build an $n$-bit function based on a $b$-bit permutation $\perm$. We define the \emph{rate} $r$ and the \emph{capacity} $c$ as two integers such that
$b = r + c$. Then, hashing the message $\mess$ consists in padding it to a length multiple of $r$ and to process it iteratively
in two phases. The \emph{absorbing} phase computes an internal state value $\freeiv \defas \perm(\perm(\ldots\perm(\messblock_0||0^c)\oplus\messblock_1||0^c)\ldots)$. The squeezing phase then produces the
$n$-bit output as $\hash(\mess) \defas \lfloor\freeiv\rfloor_r||\lfloor\perm(\freeiv)\rfloor_r||\ldots||\lfloor\perm^{n \div r}(\freeiv)\rfloor_{n \mod r}$.

A distinguishing feature of the sponge construction (apart from the fact that it does not use a compression function) is that the output length of an instance is entirely decorrelated from the size of its building
block. Thus, it allows to swimmingly build variable-length hash functions. A given permutation can also be used in different instantiations offering a tradeoff between speed (a larger rate giving
faster functions) and security (a larger capacity giving more secure functions).
In particular, Bertoni \etal showed in 2008 that similarly as chop-MD, the sponge construction instantiated with a random function or permutation is $\varepsilon$-indifferentiable from a random oracle
of the same output size, with $\varepsilon \approx 2^{-c}q^2$~\cite{DBLP:conf/eurocrypt/BertoniDPA08}. To achieve the classical security requirements of a hash function, it is thus optimal to take $c = 2n$.

One of the best examples of a sponge function is \keccak~\cite{KeccakReference}, which became the \shathree standard in 2015~\cite{Nist-SHA3}. 

\subsection{The \mdsha family}
\label{sec:mdsha}
We briefly present the ``\mdsha'' hash function family, both because it is of somewhat historical importance and because, the function studied in this article, \shaone, is one of its members.

The family originated in 1990 with the \mdfour function, introduced by Rivest~\cite{Rivest-md4}. An attack on a reduced version was quickly found by den Boer and Bosselaers~\cite{DBLP:conf/crypto/BoerB91},
and Rivest proposed \mdfive as a stengthened version of \mdfour~\cite{Rivest-md5}. Bosselaers proposed \ripemd in 1992 as another attempt at strengthening \mdfour~\cite[Chap. 3]{DBLP:books/sp/BosselaersP95},
and the NSA did the same the following year by introducing the first generation of the SHS/SHA algorithms~\cite{Nist-SHA0}. Both algorithms were quickly modified in 1996 and 1995 respectively~\cite{DBLP:conf/fse/DobbertinBP96,Nist-SHA1}.
Some later algorithms such as \shatwo~\cite{Nist-SHA}, introduced in 2002, also trace their roots back to \mdfour.

Their are some variations inside members of the family; notably, \ripemd uses a parallel structure for its compression function. We specifically list below features that are shared by \mdfour, \mdfive and \sha, but they are also
true for other \mdsha functions to a large extent.
\begin{itemize}
\item The \merkdam construction is used as a domain extender.
\item The compression function is built from an \emph{ad-hoc} block cipher used in a \emph{Davies-Meyer} mode. Let $\blockE(x,y)$ be the encryption of the plaintext $y$ with the key $x$ by the cipher $\blockE$, then
the compression function is defined by
$\chain_{i+1} \defas \blockE(\messblock_{i+1},\chain_i) \boxplus \chain_i$.
\item The block cipher inside the compression function is an unbalanced Feistel network that uses modular additions, XORs, bit rotations, and bitwise Boolean functions as constitutive elements. Its key schedule is linear, and (very) fast to compute. 
\end{itemize}

With the advent of the NIST \shathree competition, which ran from 2007 to 2012, much more diversity was introduced to hash function design. However, the \mdsha family was still influential in many competition candidates, and it remains so today.

\subsection{Contributions of this thesis}

In this thesis, we present a joint work with Thomas Peyrin and Marc Stevens on freestart collisions for the \shaone hash function, \ie attacks breaking the security of the function according to \autoref{def:free_coll}. Hash function
collisions (attacks w.r.t. \autoref{def:coll}) had already been presented by Wang \etal in 2005~\cite{DBLP:conf/crypto/WangYY05a}, but their high complexity made the explicit computation of a collision for the full function unfeasible
at the time (and no such collision is publicly known at the time where this thesis was written). The novelty of our work comes from the fact that we were able to explicitly compute collisions for the entire compression function
of \shaone. Furthermore, most of the methods developed during this research can be smoothly translated to the hash function case, which lead to more accurate, cheaper estimates for the cost of a hash function collision.

This work was published in two papers: a first attack on \shaone reduced to 76 steps was published at CRYPTO~2015~\cite{DBLP:conf/crypto/KarpmanPS15},
and an improved attack on the full 80-step compression function was published at EUROCRYPT~2016~\cite{DBLP:conf/eurocrypt/StevensKP16}. 

\medskip

In a joint work with Thomas Espitau and Pierre-Alain Fouque, we also improved the preimage attacks on \shaone. This lead to another paper published at CRYPTO~2015 \cite{DBLP:conf/crypto/EspitauFK15}. However, we do not detail this contribution
in this manuscript.


\chapter[Attaques en clefs liées améliorées sur les schémas Even-Mansour]{Improved related-key attacks on Even-Mansour}
\label{cha:emrka}

%\begin{abstract}
%We show that a distinguishing attack in the related key model on an Even-Mansour block cipher can readily be converted into an extremely efficient key recovery attack.
%
%Concerned ciphers include in particular all iterated Even-Mansour schemes with independent keys.
%We apply this observation to the \caesar candidate \proestotr and are able to recover
%the whole key with a number of requests linear in its size. This improves on recent
%forgery attacks in a similar setting.
%\keywords{Even-Mansour, related-key attacks, \proestotr.}
%\end{abstract}


% TODO RK model ``first''
% Colourful figures in appendix (sorry for colorz)

\label{chap:emrka}


\section{Introduction}


The Even-Mansour scheme (, ``Even-Mansour'', or \EM) is arguably the simplest way to construct a block cipher from publicly available
components. It defines the encryption $\E((k_1, k_0), p)$ of the plaintext $p$ under the (possibly equal) keys $k_0$
and $k_1$ as $\Perm(p \oplus k_0) \oplus k_1$, where $\Perm$ is a public permutation. Even and Mansour proved
in~1991 that for a permutation of size $n$, the probability of recovering the keys
is upper-bounded by $\bigo(qt \cdot 2^{-n})$ when the attacker considers the permutation as a black box,
where $q$ is the data complexity and $t$ is the time
complexity of the attack~\cite{EM}. Although
of considerable interest, this bound also shows at the same time that the construction is not ideal,
as one gets security only up to $\bigo(2^\frac{n}{2})$ queries, which
is less than the $\bigo(2^n)$ one would expect for an $n$-bit block cipher. For this
reason, much later work investigated the security of variants of the original construction. A simple
one is the iterated Even-Mansour scheme ($\IEM$). When using independent keys and independent permutations,
its $r$-round version is defined as
$\IEM^r((k_r, k_{r-1}, \ldots, k_0), p) \defas \Perm_{r-1}(\Perm_{r-2}(\ldots\Perm_0\linebreak[1](p \oplus k_0) \oplus k_1)\ldots) \oplus k_r$,
and it has been established by Chen and Steinberger that this construction is secure up to $\bigo(2^\frac{rn}{r+1})$ queries~\cite{CS14}.
On the other hand, in a related-key model, the same construction lends itself to trivial
distinguishing attacks, and one must consider alternatives if security in this model is necessary.
Yet until the recent work of Cogliati and Seurin~\cite{CS15} and Farshim and Procter~\cite{FP14},
no variant of Even-Mansour was proved to be secure in the related-key model. This is not the case
anymore and it has now been proven that one can reach a non-trivial level of related-key security for $\IEM^r$
starting from $r = 3$ when using keys linearly derived from a single master key (instead
of using independent keys), or even when $r = 1$
when this derivation is non-linear and meets some conditions.
While related-key analysis obviously gives much more power to the attacker than the
single-key setting, it is a widely accepted
model that may provide useful results on primitives studied in a general context.
%especially as related keys may naturally arise in some protocols.

%\subsubsection{Our contribution.}
%We show that the distinguishing attacks on Even-Mansour ciphers in a related-key
%model can be extended to much more powerful key-recovery attacks by considering modular
%additive differences instead of XOR differences. This applies both to the trivial
%distinguishers on iterated Even-Mansour with independent keys and to the more complex distinguisher
%of Cogliati and Seurin for 2-round Even-Mansour with a linear key-schedule. While these observations
%are somewhat elementary, they eventually lead to a key-recovery attack
%on the authenticated-encryption scheme and \caesar candidate \proestotr in a related-key model. This improves on the recent work
%from FSE~2015 of Dobraunig, Eichlseder and Mendel who use similar methods but only produce forgeries~\cite{DEM15}.

\section{Notation}

We use $||$ to denote string concatenation, $\alpha^i$ with $i$ an integer
to denote the string
made of the concatenation of $i$ copies of the character $\alpha$, and $\alpha^*$
to denote any string of the set $\{\alpha^i, i \in \mathbf{N}\}$, $\alpha^0$
denoting the empty string $\varepsilon$. For any string $s$, we use
$s[i]$ to denote its $i^\text{th}$ element (starting from zero).

We also use $\Delta_i^n$ to denote the string
$0^{n-i-1} || 1 || 0^{i - 1}$. The superscript $n$ will always
be clear from the context and therefore omitted.

Finally, we identify strings of length $n$ over the binary alphabet $\{0,1\}$ with elements of
the vector-space $\mathbf{F}_2^n$ and with the binary representation of elements of
the group $\mathbf{Z}/2^n\mathbf{Z}$. The addition operation on these structures
are respectively denoted by $\oplus$ (bitwise exclusive or (XOR)) and $+$ (modular addition).

\section{Generic related-key key-recovery attacks on $\IEM$}
\label{sec:gen}


Since the work of Bellare and Kohno~\cite{BK03}, it is well known that no block cipher can resist
related-key attacks (RKA) when an attacker may request encryptions under related keys
using two relation classes. A simple example showing why this cannot be the case
is to consider the classes $\xrka(k)$ and $\arka(k)$ of keys
related to $k$ by the XOR and the modular addition of any constant chosen by the attacker respectively.
If we have access to (related-key) encryption oracles
$\E(k,\cdot)$, $\E(\xrka(k),\cdot)$ and $\E(\arka(k),\cdot)$ for the block cipher $\E$ with $\kappa$-bit keys,
we can easily learn the value of the bit $k[i]$ of $k$ by comparing the results of the queries
$\E(k + \Delta_i, p)$
and $\E(k \oplus \Delta_i, p)$. For $i < \kappa - 1$, the plaintext
$p$ is encrypted under the same key if
$k[i] = 0$, then resulting in the same ciphertext, and is
encrypted under different keys if $k[i] = 1$, then resulting in different ciphertexts
with overwhelming probability.
Doing this test for every bit of $k$ thus allows to recover the whole key
with a complexity linear in $\kappa$, except its most significant bit, as
the carry of a modular addition on the MSB never propagates.% and thus never there will never be a difference between the related keys.
This key bit can of course easily be recovered once all the others have been determined.

In the same paper, Bellare and Kohno also show that no such trivial generic attack exists when
the attacker is restricted to using
only one of the two classes $\xrka$ or $\arka$, and they prove that an ideal
cipher is in this case resistant to RKA. Taken together, these results mean in essence
that a related-key attack on a block cipher $\E$ using both classes $\xrka(k)$ and $\arka(k)$ does not say
much on $\E$, as nearly all ciphers fall to an attack
in the same model. On the other hand, an attack using either of $\xrka$ or $\arka$ \emph{is} meaningful,
because an ideal cipher is secure in that case\footnote{When
queries with many different related keys are allowed, this security remains lower than in a single-key model,
because of the ability to create collisions. One can
for instance query $\E(k\oplus\Delta_i,p)$ with $2^{k/2}$ different values for $\Delta_i$ and a constant $p$
and then computes $\E(\gamma_i,p)$ with $2^{k/2}$ guesses $\gamma_i$ for $k$. With high probability, one
of the $\gamma_i$s will be equal to one of the $k\oplus\Delta_i$s, which allows to recover $k$.}.

\subsection{Key-recovery attacks on $\IEM^r$ with independent keys}

Going back to $\IEM$, we explicit the
trivial related-key distinguishers mentioned in the introduction. These distinguishers exist
for $r$-round iterated Even-Mansour schemes with independent keys, for any value of $r$.
As they only use keys related with, say, the $\xrka$ class,
they are therefore meaningful when considering the related-key security of $\IEM$.

From the very definition of $\IEM^r$, it
is obvious to see that the two values $\E((k_{r-1}, k_{r-2},\linebreak[1] \ldots, k_0), p)$
and $\E((k_{r-1}, k_{r-2}, \ldots, k_0 \oplus \delta), p \oplus \delta)$ are equal for any
difference $\delta$ when
$\E = \IEM^r$ and that this equality does not hold in general, thence allowing to distinguish
$\IEM^r$ from an ideal cipher. This is illustrated for the original \EM in \autoref{fig:em3}.

\begin{figure}[!htb]
\begin{center}
\begin{tikzpicture}[scale=1,transform shape, thick]

\hspace{.5mm}
  \tikzstyle{every node}=[transform shape];
  \tikzstyle{every node}=[node distance=1.2cm];

	% data path
    \node (XOR-1)[XOR,scale=0.8,thick] {};
    \node [left of=XOR-1] (p) {$p$};
	\node (f-1) [right of=XOR-1,draw,rectangle,very thick,minimum width=1cm,minimum height=1cm] {$\Perm$};
	\draw (XOR-1) edge (f-1);

    \node (XOR-2)[right of=f-1,XOR,scale=0.8,thick] {};
	\draw (f-1) edge (XOR-2);
	\node (f-2) [right of=XOR-2] {};
	\path[line] (XOR-2) edge (f-2) ;
	\node [right of=XOR-2,node distance=1.2cm] {$c$};

	%% Subkeys
	\node (k-0) [above=1.2cm of XOR-1]{\small $k_{0}$} ;

	\node (k-1) [above=1.2cm of XOR-2]{\small $k_{1}$};
	\draw (k-1) edge (XOR-2);

	% Diff
	\coordinate [above=7.17mm of p] (d);
	\node (delta) [right=3.21mm of d] {$\delta$};
	\node (XOR-3)[below=2.2mm of k-0,XOR,scale=0.8] {};
    \node (XOR-4)[right=1.6mm of p,XOR,scale=0.8] {};
	
	% moar datapath
	\draw (p) edge (XOR-4);
	\draw[dashed] (XOR-4) edge (XOR-1);
	%
	\draw (k-0) edge (XOR-3);
	\draw[dashed] (XOR-3) edge (XOR-1);
	%
	\coordinate[right=7.5mm of d] (dp) ;
	\draw[dashed] (dp) edge (XOR-3);
	\draw[dashed] (delta) edge (XOR-4);
	
\end{tikzpicture}

\caption{A related-key distinguisher on \EM. Dashed lines represent datapaths where a difference is injected and not canceled.\label{fig:em3}}
\end{center}
\end{figure}

\medskip

We now show how these distinguishers can be combined with the two-class attack of Bellare and Kohno
in order to extend it to a very efficient key-recovery attack. We give a description in the case of one-round
Even-Mansour, but it can easily be extended to an arbitrary $r$. The attack is very simple and works
as follows: consider again $\E((k_1, k_0), p) = \Perm(p \oplus k_0) \oplus k_1$; one can learn
the value of the bit $k_0[i]$ by querying $\E((k_1, k_0), p)$ and
$\E((k_1, k_0 + \Delta_i), p \oplus \Delta_i)$
and by comparing their values. These  differ with overwhelming probability
if $k_0[i] = 1$ and are equal otherwise.
This is illustrated in \autoref{fig:em4.1.tex} and \autoref{fig:em4.2.tex}

\begin{figure}[!htb]
\begin{center}
\begin{tikzpicture}[scale=1,transform shape, thick]

\hspace{.5mm}
  \tikzstyle{every node}=[transform shape];
  \tikzstyle{every node}=[node distance=1.2cm];

	% data path
    \node (XOR-1)[XOR,scale=0.8,thick] {};
    \node [left of=XOR-1] (p) {$p$};
	\node (f-1) [right of=XOR-1,draw,rectangle,very thick,minimum width=1cm,minimum height=1cm] {$\Perm$};
	\draw (XOR-1) edge (f-1);

    \node (XOR-2)[right of=f-1,XOR,scale=0.8,thick] {};
	\draw (f-1) edge (XOR-2);
	\node (f-2) [right of=XOR-2] {};
	\path[line] (XOR-2) edge (f-2) ;
	\node [right of=XOR-2,node distance=1.2cm] {$c$};

	%% Subkeys
	\node (k-0) [above=1.2cm of XOR-1]{\small $k_{0}$} ;

	\node (k-1) [above=1.2cm of XOR-2]{\small $k_{1}$};
	\draw (k-1) edge (XOR-2);

	% Diff
	\coordinate [above=7.17mm of p] (d);
	\node (delta) [right=2.68mm of d] {$\Delta_i$};
	\node (XOR-3)[below=2.65mm of k-0,ADD,scale=0.8] {};
    \node (XOR-4)[right=2.25mm of p,XOR,scale=0.8] {};
	
	% moar datapath
	\draw (p) edge (XOR-4);
	\draw[dashed] (XOR-4) edge (XOR-1);
	%
	\draw (k-0) edge (XOR-3);
	\draw[dashed] (XOR-3) edge (XOR-1);
	%
	\coordinate[right=9.5mm of d] (dp) ;
	\draw[dashed] (dp) edge (XOR-3);
	\draw[dashed] (delta) edge (XOR-4);

	% cap
	\node [right=.25cm of k-0,scale=0.7] {\emph{If $k_0[i] = 0$}};
	
\end{tikzpicture}

\caption{Related-key queries to $\IEM^1$ with no output difference.\label{fig:em4.1.tex}}
\end{center}
\end{figure}

\begin{figure}[!htb]
\begin{center}
\begin{tikzpicture}[scale=2,transform shape, thick]

\hspace{.5mm}
  \tikzstyle{every node}=[transform shape];
  \tikzstyle{every node}=[node distance=1.2cm];

	% data path
    \node (XOR-1)[XOR,scale=0.8,thick] {};
    \node [left of=XOR-1] (p) {$p$};
	\node (f-1) [right of=XOR-1,draw,rectangle,very thick,minimum width=1cm,minimum height=1cm] {$\Perm$};
	\draw[dashed] (XOR-1) edge (f-1);

    \node (XOR-2)[right of=f-1,XOR,scale=0.8,thick] {};
	\draw[dashed] (f-1) edge (XOR-2);
	\node (f-2) [right of=XOR-2] {};
	\path[line,dashed] (XOR-2) edge (f-2) ;
	\node [right of=XOR-2,node distance=1.2cm] {$c$};

	%% Subkeys
	\node (k-0) [above=1.2cm of XOR-1]{\small $k_{0}$} ;

	\node (k-1) [above=1.2cm of XOR-2]{\small $k_{1}$};
	\draw (k-1) edge (XOR-2);

	% Diff
	\coordinate [above=7.17mm of p] (d);
	\node (delta) [right=2.68mm of d] {$\Delta_i$};
	\node (XOR-3)[below=2.65mm of k-0,ADD,scale=0.8] {};
    \node (XOR-4)[right=2.25mm of p,XOR,scale=0.8] {};
	
	% moar datapath
	\draw (p) edge (XOR-4);
	\draw[dashed] (XOR-4) edge (XOR-1);
	%
	\draw (k-0) edge (XOR-3);
	\draw[dashed] (XOR-3) edge (XOR-1);
	%
	\coordinate[right=9.5mm of d] (dp) ;
	\draw[dashed] (dp) edge (XOR-3);
	\draw[dashed] (delta) edge (XOR-4);

	% cap
	\node [right=.25cm of k-0,scale=0.7] {\emph{If $k_0[i] = 1$}};
	
\end{tikzpicture}

\caption{Related-key queries to $\IEM^1$ with an output difference.\label{fig:em4.2.tex}}
\end{center}
\end{figure}


A similar attack works on the variant of the (iterated) \EM 
that uses modular addition instead of XOR for the combination of the key with the plaintext.
This variant was first analyzed by Dunkelman, Keller and Shamir~\cite{DKS12} and offers the same security bounds as the
original Even-Mansour. An attack in that case works similarly
by querying \eg $\E((k_1, k_0), \Delta_i)$ and
$\E((k_1, k_0 \oplus \Delta_i), 0^\kappa)$.

Both attacks use a single difference class for the related keys (either $\xrka$ or $\arka$),
and they are therefore meaningful as related-key attacks. They simply emulate the attack that uses both
classes simultaneously by taking advantage of the fact that the usage of key material is very simple in
Even-Mansour.
Finally, we can see that in the particular case of a one-round construction, the attack still works without
adaptation if one
chooses the keys $k_1$ and $k_0$ to be equal.

\subsection{Extension to $\IEM^2$ with a linear key schedule}

Cogliati and Seurin~\cite{CS15} showed that it is also possible to very efficiently distinguish
$\IEM^2$ with related keys, even when the keys are equal or derived
from a master key by a linear key schedule. Similarly as for independent-key $\IEM$, we can adapt
the distinguisher and transform it into a key-recovery attack. The idea remains the same: one
replaces the $\xrka$ class of the original distinguisher with $\arka$, which makes its success
conditioned on the value of a few key bits, hence allowing their recovery. We give
the description of our modified distinguisher for $\E(k, p) := \Perm(\Perm(k \oplus p) \oplus k) \oplus k$,
where we let the $\Delta_i$s denote arbitrary differences:

\begin{enumerate}[leftmargin=4em]
	\item Query $y_1 := \E(k + \Delta_1, x_1)$.
	\item Set $x_2$ to $x_1 \oplus \Delta_1 \oplus \Delta_2$ and query $y_2 := \E(k + \Delta_2, x_2)$.
	\item Set $y_3$ to $y_1 \oplus \Delta_1 \oplus \Delta_3$ and query $x_3 := \E^{-1}(k + \Delta_3, y_3)$.
	\item Set $y_4$  to $y_2 \oplus \Delta_1 \oplus \Delta_3$ and query $x_4 := \E^{-1}(k + (\Delta_1 \oplus \Delta_2 \oplus \Delta_3), y_4)$.
	\item Test if $x_4 = x_3 \oplus \Delta_1 \oplus \Delta_2$.
\end{enumerate}

If the test is successful, it means that with overwhelming probability
the key bits at the positions of the differences $\Delta_1$, $\Delta_2$,
$\Delta_3$ are all zero, as in that case  $k + \Delta_i = k \oplus \Delta_i$ and
the distinguisher works ``as intended'', and as otherwise at least one uncontrolled difference goes through $\Perm$ or $\Perm^{-1}$.
It is possible to restrict oneself to using differences in only two bits for the $\Delta_i$s, and as soon as two such zero bits
have been found (which happens after an expected four trials for
random keys), the rest of the key bits can be tested one by one.

We conclude this short section by showing why the test of line~5 is successful when $k + \Delta_i = k \oplus \Delta_i$,
but refer to Cogliati and Seurin for a complete description of their distinguisher, including the general case of distinct
permutations and keys linearly derived from a master key (adapting the general case to our modified setting only requires small modifications which are left as an exercise to the reader).

For the sake of clarity, we write $k \oplus \Delta_i$ for $k + \Delta_i$, as they are equal by hypothesis. 
By definition,
$y_1 = \Perm(\Perm(x_1 \oplus k \oplus \Delta_1) \oplus k \oplus \Delta_1) \oplus k \oplus \Delta_1$ and
$y_3 = \Perm(\Perm(x_1 \oplus k \oplus \Delta_1) \oplus k \oplus \Delta_1) \oplus k \oplus {\color{Green}\Delta_1} \oplus {\color{NavyBlue}\Delta_1} \oplus \Delta_3$
which simplifies to 
$\Perm(\Perm(x_1 \oplus k \oplus \Delta_1) \oplus k \oplus \Delta_1) \oplus k \oplus \Delta_3$. This yields the following
expression for $x_3$:
\begin{align*}
x_3 &= \Perm^{-1}(\Perm^{-1}(\Perm(\Perm(x_1 \oplus k \oplus \Delta_1) \oplus k \oplus \Delta_1) \oplus {\color{Green}k \oplus \Delta_3} \oplus {\color{NavyBlue}k \oplus \Delta_3})
\oplus k \oplus \Delta_3) \oplus k \oplus \Delta_3 \\
&= \Perm^{-1}({\color{Green}\Perm^{-1}(}{\color{NavyBlue}\Perm(}\Perm(x_1 \oplus k \oplus \Delta_1) \oplus k \oplus \Delta_1{\color{NavyBlue})}{\color{Green})}
\oplus k \oplus \Delta_3) \oplus k \oplus \Delta_3 \\
&= \Perm^{-1}(\Perm(x_1 \oplus k \oplus \Delta_1) \oplus {\color{Green}k} \oplus \Delta_1 \oplus {\color{NavyBlue}k} \oplus \Delta_3) \oplus k \oplus \Delta_3 \\
&= \Perm^{-1}(\Perm(x_1 \oplus k \oplus \Delta_1) \oplus \Delta_1 \oplus \Delta_3) \oplus k \oplus \Delta_3
\end{align*}
Similarly,
$y_2 = \Perm(\Perm(x_1 \oplus k \oplus \Delta_1) \oplus k \oplus \Delta_2) \oplus k \oplus \Delta_2$ and
$y_4 = \Perm(\Perm(x_1 \oplus k \oplus \Delta_1) \oplus k \oplus \Delta_2) \oplus k \oplus \Delta_2 \oplus \Delta_1 \oplus \Delta_3$, which
yields the following expression for $x_4$:
\begin{align*}
x_4 &=  \Perm^{-1}(\Perm^{-1}(\Perm(\Perm(x_1 \oplus k \oplus \Delta_1) \oplus k \oplus \Delta_2) \oplus {\color{Green}k \oplus \Delta_2 \oplus \Delta_1
\oplus \Delta_3} \oplus {\color{NavyBlue}k \oplus \Delta_1 \oplus \Delta_2 \oplus \Delta_3}) \\
&\oplus k \oplus \Delta_1 \oplus \Delta_2 \oplus \Delta_3) \oplus k \oplus \Delta_1 \oplus \Delta_2 \oplus \Delta_3\\
&= \Perm^{-1}({\color{Green}\Perm^{-1}(}{\color{NavyBlue}\Perm(}\Perm(x_1 \oplus k \oplus \Delta_1) \oplus k \oplus \Delta_2{\color{NavyBlue})}{\color{Green})}
\oplus k \oplus \Delta_1 \oplus \Delta_2
\oplus \Delta_3) \oplus k \oplus \Delta_1 \oplus \Delta_2 \oplus \Delta_3 \\
&= \Perm^{-1}(\Perm(x_1 \oplus k \oplus \Delta_1) \oplus {\color{Green}k \oplus \Delta_2} \oplus {\color{NavyBlue}k} \oplus \Delta_1 \oplus {\color{NavyBlue}\Delta_2} \oplus
\Delta_3) \oplus k \oplus \Delta_1 \oplus \Delta_2 \oplus \Delta_3\\
&= \Perm^{-1}(\Perm(x_1 \oplus k \oplus \Delta_1) \oplus \Delta_1 \oplus \Delta_3) \oplus k \oplus \Delta_1 \oplus \Delta_2 \oplus \Delta_3
\end{align*}
From the final expressions of $x_3$ and $x_4$, we see that their XOR difference is indeed $\Delta_1 \oplus \Delta_2$.

\section{Application to \proestotr}
\label{sec:appli}

We now apply the simple generic key-recovery attack of \autoref{sec:gen} to the \caesar candidate \proestotr, which is an authenticated-encryption
scheme member of the \proest family~\cite{proest}. This family is based on the \proest permutation and defines
three schemes instantiating as many modes of operation, namely COPA, OTR and APE. Only
the latter can readily be instantiated with a permutation, and both COPA and OTR rely on a keyed primitive. In order
to be instantiated with \proest, the permutation is expanded to a block cipher
defined as a one-round Even-Mansour scheme with identical keys
$\E(k,p) := \Perm(p \oplus k) \oplus k$,
with the \proest permutation as $\Perm$. We will denote this cipher as \proestem.

Although the attack of Section~\ref{sec:gen} could readily be applied to \proestem, this cipher is only
meant to be embedded into a specific instantiation of a mode such as OTR, and attacking it out of context may not
be relevant to its intended use.
Hence we must
be able to mount an attack on \proestcopa or \proestotr as a whole for it to be really significant,
which is precisely what we describe now for the latter.

Because our attack solely relies on the Even-Mansour structure of the cipher, we refer the interested reader to the
submission document of \proest for the definition of its permutation.
The same goes for the OTR mode~\cite{M14}, as we only need to focus on a small part to describe the attack.
Consequently, we just describe how the encryption of the first block of plaintext is performed in \proestotr\footnote{This is not necessarily the
same as encrypting the first block in all instantiations of OTR, as there is some flexibility in the definition of the mode.}.

\medskip

The mode of operation OTR is nonce-based; it takes as input a key $k$, a nonce $n$, a message $m$,
(possibly empty) associated data $a$, and produces a ciphertext $c$ corresponding
to the encryption of the message with $k$, and a tag $t$ authenticating $m$ and
$a$ together with the key $k$. It is important for the security of the mode to ensure
that one cannot encrypt twice using the same nonce. However, there are no
specific restriction as to their value, and we consider that one
can freely choose them.

The encryption of the first block of ciphertext $c_1$ by \proestotr is defined as a function
$\fun(k, n, m_1, m_2)$
of $k$, $n$, and the first two blocks of plaintext $m_1$ and $m_2$:
let $\ell := \E(k, n || 10^*)$ be the encryption of the padded nonce and
$\ell' := \permi(\ell)$, with $\permi$ a linear permutation (the multiplication by $4$ in some finite field),
then $c_1$ is simply equal to $\E(k, \ell' \oplus m_1) \oplus m_2$. We show this schematically along with the encryption
of the second block in Figure~\ref{fig:otr}.
Let us now apply the attack from Section~\ref{sec:gen}.

\begin{figure}[ht]
\begin{center}
\begin{tikzpicture}[auto,scale=1,transform shape]
\tikzstyle{block} = [rectangle, draw, thick,
    text width=3em, text centered, minimum height=3em]
\tikzstyle{line} = [draw, -latex']

\node [block] (Et) {$\E_K$};
\coordinate [above of=Et, node distance=1cm] (top);
\coordinate [left of=Et, node distance=2cm] (midleft);
\node [left of=top, node distance=2cm] (m1) {$m_1$};
\node [left of=Et, node distance=1.2cm, thick, scale=1.2] (xor1) {$\mathbf{\oplus}$};
\node [above of=xor1, node distance=0.7cm] (4L1) {$\ell'$};
\coordinate [left of=xor1, node distance=1mm] (xor11);
\path [line] (m1) -- (midleft) -- (xor11);
\coordinate [above of=xor1, node distance=1mm] (xor12);
\path [line] (4L1) -- (xor12);
\coordinate [right of=xor1, node distance=1mm] (xor13);
\path [line] (xor13) -- (Et);
%
\node [right of=top, node distance=2cm] (m2) {$m_2$};
\node [right of=Et, node distance=2cm, thick, scale=1.2] (xor2) {$\mathbf{\oplus}$};
\coordinate [left of=xor2, node distance=1mm] (xor21);
\path [line] (Et) -- (xor21);
\coordinate [above of=xor2, node distance=1mm] (xor22);
\path [line] (m2) -- (xor22);
%
\coordinate [below of=midleft, node distance=.75cm] (botleft);
\coordinate [below of=botleft, node distance=.75cm] (bumleft);
\coordinate [right of=botleft, node distance=4cm] (botright);
\coordinate [right of=bumleft, node distance=4cm] (bumright);
\coordinate [below of=xor2, node distance=1mm] (xor23);
\node [block, below of=Et, node distance=2.5cm] (Eb) {$\E_K$};
%
\coordinate [left of=Eb, node distance=2cm] (bimleft);
\node [left of=Eb, node distance=1.2cm, thick, scale=1.2] (xor3) {$\mathbf{\oplus}$};
\coordinate [left  of=xor3, node distance=1mm] (xor31);
\coordinate [above of=xor3, node distance=1mm] (xor32);
\coordinate [right of=xor3, node distance=1mm] (xor33);
\coordinate [below of=xor3, node distance=1mm] (xor34);
\node [above of=xor3, node distance=0.7cm] (4L2) {$\ell'$};
\node [below of=xor3, node distance=0.7cm] (L) {$\ell$};
%
\path [line] (xor23) -- (botright) -- (bumleft) -- (bimleft) -- (xor31);
\path [line] (4L2) -- (xor32);
\path [line] (L) -- (xor34);
\path [line] (xor33) -- (Eb);
%
\node [right of=Eb, node distance=2cm, thick, scale=1.2] (xor4) {$\mathbf{\oplus}$};
\coordinate [below of=xor4, node distance=1mm] (xor41);
\coordinate [left of=xor4, node distance=1mm] (xor42);
\coordinate [above of=xor4, node distance=1mm] (xor43);
\path [line] (midleft) -- (botleft) -- (bumright) -- (xor43);
\path [line] (Eb) -- (xor42);
%
\node [below of=m1, node distance=4.5cm] (c1) {$c_1$};
\node [right of=c1, node distance=4cm] (c2) {$c_2$};
\path [line] (bimleft) -- (c1);
\path [line] (xor41) -- (c2);

\end{tikzpicture}
\end{center}
\caption{\label{fig:otr}The encryption of the first two blocks of message in \proestotr.}
\end{figure}

\subsection{Step 1: Recovering the most significant half of the key.}

It is straightforward to see that one can recover the value
of the bit $k[i]$ by performing only two queries with related keys
and different nonces and messages. One just has to compare
$c_1 = \fun(k, n, m_1, m_2)$ and
$\hat{c}_1 = \fun(k + \Delta_i,n \oplus \Delta_i,
m_1 \oplus \Delta_i \oplus \permi(\Delta_i), m_2)$. Indeed, if $k[i] = 0$,
then the value $\hat{\ell}$ obtained in the computation of $\hat{c}_1$ is equal to
$\ell \oplus \Delta_i$ and $\hat{\ell'} = \ell' \oplus \pi(\Delta_i)$, hence
$\hat{c}_1 = c_1 \oplus \Delta_i$. If $k[i] = 1$, the latter equality does not
hold with overwhelming probability.
We illustrate the propagation of differences in the computation of $\ell'$ and $\hat{\ell'}$ in
\autoref{fig:otr21} and \autoref{fig:otr22}.


\begin{figure}[!htb]
\begin{center}
\begin{tikzpicture}[scale=1,transform shape, thick]

\hspace{-2.5mm}
  \tikzstyle{every node}=[transform shape];
  \tikzstyle{every node}=[node distance=1.2cm];

	% data path
    \node (XOR-1)[XOR,scale=0.8,thick] {};
	\node (f-1) [right=4mm of XOR-1,draw,rectangle,very thick,minimum width=1cm,minimum height=1cm] {$\Perm$};
    \node (XOR-2)[right=4mm of f-1,XOR,scale=0.8,thick] {};
	\node [left of=XOR-1] (p) {$n$};
	\draw (XOR-1) edge (f-1);

	\draw (f-1) edge (XOR-2);
	\node (f-2) [right of=XOR-2,node distance=.5cm] {$\hat{\ell}$};
	\node (XOR-6) [right of=f-2,node distance=.5cm,MUL,scale=0.8] {};
	\node (f-3) [right of=XOR-6,node distance=.7cm] {$\hat{\ell'}$};
	\draw[dashed] (XOR-2) edge (f-2);
	\draw[dashed] (f-2) edge (XOR-6);
	\draw[line,dashed] (XOR-6) edge (f-3);

	%% Subkeys
	\node (k-0) [above=1.2cm of XOR-1]{\small $k$} ;
	\node (k-1) [above=1.2cm of XOR-2]{\small $k$};

	% cst
	\node (four) [above=1.2cm of XOR-6]{\small 4};
	\draw (four) -- (XOR-6);

	% Diff
	\coordinate [above=7.51mm of p] (d);
	\node (delta) [right=2.68mm of d] {$\Delta_i$};
	\node (XOR-3)[below=2.65mm of k-0,ADD,scale=0.8] {};
    \node (XOR-4)[right=2.25mm of p,XOR,scale=0.8] {};
	\node (XOR-5)[below=2.65mm of k-1,ADD,scale=0.8] {};
	
	% moar datapath
	\draw (p) edge (XOR-4);
	\draw[dashed] (XOR-4) edge (XOR-1);
	%
	\draw (k-0) edge (XOR-3);
	\draw[dashed] (XOR-3) edge (XOR-1);
	%
	\coordinate[right=9.5mm of d] (dp) ;
	\draw[dashed] (dp) edge (XOR-3);
	\draw[dashed] (delta) edge (XOR-4);
	%
	\draw[dashed] (XOR-3) -- (XOR-5);
	%
	\draw (k-1) -- (XOR-5);
	\draw[dashed] (XOR-5) -- (XOR-2);

	% cap
	\node [right=.25cm of k-0,scale=0.7] {\emph{If $k[i] = 0$}};
	\node [below=.2cm of XOR-6,scale=0.7] {$\hat{\ell'} = \ell'\oplus4\otimes\Delta_i$};

	% white square
	\node [below=.3mm of f-1] {};
	
\end{tikzpicture}

\caption{Related-key queries to \proestotr with predictable output difference.\label{fig:otr21}}
\end{center}
\end{figure}

\begin{figure}[!htb]
\begin{center}
\begin{tikzpicture}[scale=2,transform shape, thick]

\hspace{-2.5mm}
  \tikzstyle{every node}=[transform shape];
  \tikzstyle{every node}=[node distance=1.2cm];

	% data path
    \node (XOR-1)[XOR,scale=0.8,thick] {};
	\node (f-1) [right=4mm of XOR-1,draw,rectangle,very thick,minimum width=1cm,minimum height=1cm] {$\Perm$};
    \node (XOR-2)[right=4mm of f-1,XOR,scale=0.8,thick] {};
    \node [left of=XOR-1] (p) {$n$};
	\draw[dashed] (XOR-1) edge (f-1);

	\draw[dashed] (f-1) edge (XOR-2);
	\node (f-2) [right of=XOR-2,node distance=.5cm] {$\hat{\ell}$};
	\node (XOR-6) [right of=f-2,node distance=.5cm,MUL,scale=0.8] {};
	\node (f-3) [right of=XOR-6,node distance=.7cm] {$\hat{\ell'}$};
	\draw[dashed] (XOR-2) edge (f-2);
	\draw[dashed] (f-2) edge (XOR-6);
	\draw[line,dashed] (XOR-6) edge (f-3);

	%% Subkeys
	\node (k-0) [above=1.2cm of XOR-1]{\small $k$} ;
	\node (k-1) [above=1.2cm of XOR-2]{\small $k$};

	% cst
	\node (four) [above=1.2cm of XOR-6]{\small 4};
	\draw (four) -- (XOR-6);

	% Diff
	\coordinate [above=7.51mm of p] (d);
	\node (delta) [right=2.68mm of d] {$\Delta_i$};
	\node (XOR-3)[below=2.65mm of k-0,ADD,scale=0.8] {};
    \node (XOR-4)[right=2.25mm of p,XOR,scale=0.8] {};
	\node (XOR-5)[below=2.65mm of k-1,ADD,scale=0.8] {};
	
	% moar datapath
	\draw (p) edge (XOR-4);
	\draw[dashed] (XOR-4) edge (XOR-1);
	%
	\draw (k-0) edge (XOR-3);
	\draw[dashed] (XOR-3) edge (XOR-1);
	%
	\coordinate[right=9.5mm of d] (dp) ;
	\draw[dashed] (dp) edge (XOR-3);
	\draw[dashed] (delta) edge (XOR-4);
	%
	\draw[dashed] (XOR-3) -- (XOR-5);
	%
	\draw (k-1) -- (XOR-5);
	\draw[dashed] (XOR-5) -- (XOR-2);

	% cap
	\node [right=.25cm of k-0,scale=0.7] {\emph{If $k[i] = 1$}};
	\node (cap2) [below=.2cm of XOR-6,scale=0.7] {$\hat{\ell'} = \ell'\oplus \$$};

	% white square
	\node [below=.3mm of f-1] {};
	
\end{tikzpicture}

\caption{Related-key queries to \proestotr with unpredictable output difference.\label{fig:otr22}}
\end{center}
\end{figure}

Yet this does not allow to recover the whole key because the nonce in \proestotr is restricted to a length
half of the width of the block cipher $\E$ (or equivalently of the underlying \proest permutation),
\ie $\frac{\kappa}{2}$. It is then possible to recover only
half of the bits of $k$ using this procedure, as one cannot introduce appropriate differences in
the computation of $\ell$ for the other half. The targeted security of the
whole primitive being precisely $\frac{\kappa}{2}$ because of the generic single key
attacks on Even-Mansour,
one does not make a significant gain by recovering only half of the key.
Even though, it should still be noted that this yields an
attack with very little data requirements and with the same time complexity as the best
point on the tradeoff curve of generic attacks, which in that case has a much higher data complexity of
$2^\frac{\kappa}{2}$.


\subsection{Step 2: Recovering the least significant half of the key.}

Even though the generic attack in its most simple form does not allow to recover the full key
of \proestotr,
we can use the fact that the padding of the nonce is done on the least significant
bits to our advantage, and by slightly adapting the procedure of the first step, we can iteratively recover the value
of the least significant half of the key with no more effort than for the most significant half.

Let us first show how we can recover the most significant bit of the least significant half of
the key $k[\msbtwo]$ (\ie the first bit for which we cannot use the previous method)
after a single encryption by $\E$. 
%For the sake of simplicity, we assume for now that
%unlike in \proestem the two keys of $\E$ are independent, and we omit to write the second (outer) key in the queries to $\E$.

We note $k^\MSB$ the (known) most significant half of the key $k$.
To mount the attack, one queries $\E(k - k^\MSB + \Delta_\msbtwo, p \oplus \Delta_{\msbtwoone})$ and
$\E(k - k^\MSB - \Delta_\msbtwo, p)$. We can see that the inputs to $\Perm$ in these two cases are equal
iff $k[\msbtwo] = 1$. Indeed, in that case,
the carry in the addition $(k - k^\MSB) + \Delta_\msbtwo$ propagates by exactly one position and is ``cancelled'' by
the difference in $p$, and there is no carry propagation in $(k - k^\MSB) - \Delta_\msbtwo$.
The result of the two queries are therefore equal to
$\mathfrak{C} \oplus (k - k^\MSB + \Delta_\msbtwo) = \mathfrak{C} \oplus (k \oplus k^\MSB \oplus \Delta_\msbtwo \oplus \Delta_\msbtwoone)$ and
$\mathfrak{C} \oplus (k - k^\MSB - \Delta_\msbtwo) = \mathfrak{C} \oplus (k \oplus k^\MSB \oplus \Delta_\msbtwo)$
with $\mathfrak{C} \defas \Perm(p \oplus (k - k^\MSB - \Delta_\msbtwo))$.
Consequently, the XOR difference between the two results is known and equal to $\Delta_\msbtwoone$.
If on the other hand
$k[\msbtwo] = 0$, the carry in $(k - k^\MSB) - \Delta_\msbtwo)$ propagates all the way to the most significant bit of $k$, whereas only
two differences are introduced in the input to $\Perm$ in the first query. This allows to distinguish between the two cases and thus to recover the value of this key bit.

Once the value of $k[\msbtwo]$ has been learned, one can iterate the process to recover the remaining bits of $k$.
The only subtlety is that we want to ensure that if there is a carry propagation in
$(k - k^\MSB) + \Delta_{\msbtwo - i}$ (resp. $(k - k^\MSB) - \Delta_{\msbtwo - i}$),
it should propagate up to $k_{\msbtwoone}$, the position where we cancel it with an XOR difference
(resp. up to the most significant bit); this can easily be achieved by adding two terms to both keys.
Let us define $\gamma_i$ as the value of the key $k$ only on positions $\msbtwo\ldots\msbtwoone - i$, completed with zeros left and right;
that is $\gamma_i[j] = k[j]$ if $\msbtwo \geq j \geq \msbtwoone - i$, and $\gamma_i[j] = 0$ otherwise.
Let us also define $\widetilde{\gamma_i}$ as the binary complement of $\gamma_i$ on its non-zero support,
that is $\widetilde{\gamma_i}[j] = \widetilde{k[j]}$ if $\msbtwo \geq j \geq \msbtwoone - i$, and $\widetilde{\gamma_i}[j] = 0$ otherwise.
The modified queries then become $\E(k - k^\MSB + \Delta_{\msbtwo - i} + \widetilde{\gamma_i}, p \oplus \Delta_{\msbtwoone})$ and
$\E(k - k^\MSB - \Delta_{\msbtwo - i} - \gamma_i, p)$, for which the propagation of the carries is ensured. Note that
the difference between the results of these two queries when $k[\msbtwo - i] = 1$ is independent of $i$ and always equal
to $\Delta_\msbtwoone$. 


%We now show that the simplifying assumption that the two keys of $\E$ are independent is not necessary and that the attack
%still works when they are equal, as is the case for \proestem.
%In the event of the bit $k[\msbtwo - i]$
%being one, the (identical) results of the queries are equal to encrypting $p$ under $k$ with all its bits of position higher
%than $\msbtwo - i$ set to zero. This shows why even when the outer key is chosen equal to the inner key, we can
%deduce the difference (modular as well as XOR) between the two results, and thence are still able to recover the value of
%this particular key bit; this difference is in fact independent of $i$ and equal to $\Delta_{\msbtwoone}$.

We conclude by showing how to apply this procedure to \proestotr. For the sake of readability, let us denote
by $\Delta_i^+$ and $\Delta_i^-$ the complete modular differences used to recover one less significant bit $k[i]$.
We then simply perform the two queries $\fun(k + \Delta_i^+, n \oplus \Delta_{\msbtwoone}, m_1 \oplus \Delta_{\msbtwoone}, m_2)$
and $\fun(k + \Delta_i^-, n, m_1 \oplus \permi(\Delta_{\msbtwoone}), m_2)$, which differ by $\Delta_{\msbtwoone}$ iff $k_i$ is one,
with overwhelming probability.

All in all, one can retrieve the whole key of size
$\kappa$ using only $2\kappa$ chosen-nonce related-key encryption requests (with $\kappa$ different keys), ignoring
everything in the output (including the tag) apart from the value of the first block
of ciphertext. We give the full attack in \autoref{alg:kr}. Note that it makes use
of a procedure \textsc{Refresh} which picks fresh values for two message words and (most importantly)
for the nonce. Because
the attack is very fast, it can easily be tested. We give an implementation
for a 64-bit toy cipher in \autoref{sec:toy}.

\begin{algorithm}[h]
\LinesNumbered
\KwIn{ Oracle access to $\fun(k, \cdot, \cdot, \cdot)$ and $\fun(\arka(k), \cdot, \cdot, \cdot)$
for a fixed (unknown) key $k$ of even length $\kappa$}
\KwOut{ Two candidates for the key $k$ }

$k'$ := $0^\kappa$\\
\For{ $i$ := $\kappa - 2$ to $\kappa/2$}
{
	\textsc{Refresh}($n$, $m_1$, $m_2$)\\
	$x$ := $\fun(k, n, m_1, m_2)$\\
	$y$ := $\fun(k + \Delta_i, n \oplus \Delta_i,
                 m_1 \oplus \Delta_i \oplus \permi(\Delta_i), m_2)$\\
	\If{$x = y \oplus \Delta_i$}
	{
		$k'[i]$ := 0	
	}
	\Else
	{
		$k'[i]$ := 1
	}
}
\For{ $i$ := $\kappa/2 - 1$ to $0$}
{
	\textsc{Refresh}($n$, $m_1$, $m_2$)\\
	$x$ := $\fun(k + \Delta_i^+, n \oplus \Delta_{\msbtwoone}, m_1 \oplus \Delta_{\msbtwoone}, m_2)$\\
	$y$ := $\fun(k + \Delta_i^-, n, m_1 \oplus \permi(\Delta_{\msbtwoone}), m_2)$\\
	\If{$x = y \oplus \Delta_\msbtwoone$}
	{
		$k'[i]$ := 1	
	}
	\Else
	{
		$k'[i]$ := 0
	}
}
$k''$ := $k'$\\
$k''[\kappa - 1]$ := 1\\
\Return{$(k',k'')$}
\caption{Related-key key recovery for \proestotr\label{alg:kr}}
\end{algorithm}


\paragraph{\emph{\textsc{Remark.}}} If the padding of the nonce in \proestotr were done on the most significant bits, no attack similar
to Step~2 could recover
the corresponding key bits: the modular addition is a triangular function (meaning that the result of $a + b$ on a bit $i$ only
depends on the value of bits of position less than $i$ in $a$ and $b$), and therefore no XOR in the nonce in the less
significant bits could control modular differences introduced in the padding in the more significant bits. An attack in that case would thus most likely
be applicable to general ciphers when using only the $\arka$ class, and it is proven that no such attack is efficient.
However, one could always imagine using a related-key class using an addition operation reading the bits in reverse. While
admittedly unorthodox, this would not result in a stronger model than using $\arka$, strictly speaking.

\paragraph{Discussion.}

In a recent independent work, Dobraunig, Eichlseder and Mendel use similar methods to produce forgeries for \proestotr by considering
related keys with XOR differences~\cite{DEM15}. On the one hand, one could argue that the class $\xrka$ is more natural than $\arka$
and more likely to arise in actual protocols, which would
make their attack more applicable than ours. On the other hand, an ideal cipher is expected to give a similar security against
RKA using either class, which means that our model is not theoretically stronger than the one of Dobraunig~\etal, while resulting in a
much more powerful key recovery attack.

\section{Conclusion}

We made a simple observation that allows to convert related-key distinguishing attacks on some Even-Mansour schemes into much more
powerful key-recovery attacks, and we used this observation to derive an extremely efficient key-recovery attack on the
\proestotr \caesar candidate, in the related-key model.

Primitives based on \EM are quite common, and it is natural to wonder if we could mount similar
attacks on other ciphers. A natural first target would be \proestcopa which is also based on the \proestem cipher.
However, in this mode,
encryption and tag generation depend on the encryption of a fixed plaintext $\ell \defas \E(k, 0)$ which is different
for different keys with overwhelming probability and makes our attack fail. The forgery attacks of
Dobraunig~\etal seem to fail in that case for the same reason.
Keeping with \caesar candidates, another good target would be \minalpher~\cite{minalpher}, which also
uses a one-round Even-Mansour block cipher as one of its components. The attack also fails in this
case, though, because the masking key used in the Even-Mansour cipher is derived from the master key
in a highly non-linear way. In fact, Mennink recently proved that both ciphers are resistant
to related-key attacks~\cite{DBLP:journals/iacr/Mennink15a}.
%the construction used in \minalpher
%may actually benefit from the security proof of Cogliati and Seurin\footnote{Their
%proof requires two distinct non-linear \emph{permutations}, whereas \minalpher uses the same non-linear \emph{function} to
%derive both keys. We did not investigate the impact of these differences, but it is reasonable to expect a good security
%for this construction.}.
Finally, leaving aside authentication and
going back to traditional block ciphers, we could consider designs such as LED \cite{LED}. The attack
also fails in that case, however, because the cipher uses an iterated construction with at least 8 rounds and
only one (or two) keys.

This lack of other results is not very surprising, as we only improve existing distinguishing
attacks, and this improvement cannot be used without a distinguisher as its basis.
Therefore, any primitive for which resistance to related-key attacks is important should already be resistant
to the distinguishing attacks and thus to ours. Yet it would be reasonable to allow the presence
of a simple related-key distinguisher when designing a primitive, as this a very weak type of
attack (in fact, this is for instance the approach taken by PRINCE, among others~\cite{PRINCE}, which admits
a trivial distinguisher due to its \textsf{FX} construction).
What we have shown is that one must be
careful when contemplating such a decision for \EM (and \textsf{FX} constructions, to some extent), as in that case it is actually
equivalent to allowing key recovery, the most powerful of all attacks.

\setcounter{section}{0}
\renewcommand\thesection{\Alph{section}}

\section{Proof-of-concept implementation for a 64-bit permutation}
\label{sec:toy}

We give the source of a C program that recovers a 64-bit key from a design similar to \proestotr where
the permutation has been replaced by a small ARX, for compactness. For the sake of simplicity,
we do not ensure that the nonce does not repeat in the queries.

\begin{minted}[breaklines]{c}
#include <stdio.h>
#include <stdint.h>
#include <stdlib.h>

#define ROL32(x,r) (((x) << (r)) ^ ((x) >> (32 - (r))))
#define MIX(hi,lo,r) { (hi) += (lo); (lo) = ROL32((lo),(r)) ; (lo) ^= (hi); }

#define TIMES2(x) ((x & 0x8000000000000000ULL) ? ((x) << 1ULL) ^                0x000000000000001BULL : (x << 1ULL))
#define TIMES4(x) TIMES2(TIMES2((x)))

#define DELTA(x) (1ULL << (x))
#define MSB(x) ((x) & 0xFFFFFFFF00000000ULL)
#define LSB(x) ((x) & 0x00000000FFFFFFFFULL)

/* Replace arc4random() by your favourite PRNG */

/* 64-bit permutation using Skein's MIX */
uint64_t p64(uint64_t x)
{
	uint32_t hi = x >> 32;
	uint32_t lo = LSB(x);
	unsigned rcon[8] = {1, 29, 4, 8, 17, 12, 3, 14};

	for (int i = 0; i < 32; i++)
	{
	   MIX(hi, lo, rcon[i % 8]);
	   lo += i;
	}

	return ((((uint64_t)hi) << 32) ^ lo);
}

uint64_t em64(uint64_t k, uint64_t p)
{
	return p64(k ^ p) ^ k;
}

uint64_t potr_1(uint64_t k, uint64_t n, uint64_t m1, uint64_t m2)
{
	uint64_t l, c;

	l = TIMES4(em64(k, n));
	c = em64(k, l ^ m1) ^ m2;

	return c;
}

uint64_t recover_hi(uint64_t secret_key)
{
	uint64_t kk = 0;

	for (int i = 62; i >= 32; i--)
	{
		uint64_t m1, m2, c11, c12, n;

		m1 = (((uint64_t)arc4random()) << 32) ^ arc4random();
		m2 = (((uint64_t)arc4random()) << 32) ^ arc4random();
		n  = (((uint64_t)arc4random()) << 32) ^ 0x80000000ULL;
		c11 = potr_1(secret_key, n, m1, m2);
		c12 = potr_1(secret_key + DELTA(i), n ^ DELTA(i), m1 ^ DELTA(i) ^ TIMES4(DELTA(i)), m2);

		if (c11 != (c12 ^ DELTA(i)))
			kk |= DELTA(i);
	}

	return kk;
}

uint64_t recover_lo(uint64_t secret_key, uint64_t hi_key)
{
	uint64_t kk = hi_key;

	for (int i = 31; i >= 0; i--)
	{
		uint64_t m1, m2, c11, c12, n;
		uint64_t delta_p, delta_m;

		m1 = (((uint64_t)arc4random()) << 32) ^ arc4random();
		m2 = (((uint64_t)arc4random()) << 32) ^ arc4random();
		n  = (((uint64_t)arc4random()) << 32) ^ 0x80000000ULL;

		delta_p = DELTA(i) - MSB(kk) + (((LSB(~kk)) >> (i + 1)) << (i + 1));
		delta_m = DELTA(i) + MSB(kk) + LSB(kk);
		c11 = potr_1(secret_key + delta_p, n ^ DELTA(32), m1 ^ DELTA(32), m2);
		c12 = potr_1(secret_key - delta_m, n, m1 ^ TIMES4(DELTA(32)), m2);

		if (c11 == (c12 ^ DELTA(32)))
			kk |= DELTA(i);
	}

	return kk;
}

int main()
{
	uint64_t secret_key = (((uint64_t)arc4random()) << 32) ^ arc4random();
	uint64_t kk1 = recover_lo(secret_key, recover_hi(secret_key));
	uint64_t kk2 = kk1 ^ 0x8000000000000000ULL;

	printf("The real key is %016llx, the key candidates are %016llx, %016llx     ", secret_key, kk1, kk2);
	if ((kk1 == secret_key) || (kk2 == secret_key))
		printf("SUCCESS!\n");
	else
		printf("FAILURE!\n");

	return 0;
}
\end{minted}

\renewcommand\thesection{\arabic{section}}

