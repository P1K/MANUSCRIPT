\part[Nouvelles attaques sur \shaone]
    {New attacks on \shaone} 
\label{part:sha-1}

\section*{Overview}

The \shaone hash function was one of the first widely-deployed hash functions. It was designed by the NSA in 1995 as a modified version of 1993's \shazero, which itself
was more loosely based on 1990's \mdfour.

In 2005, Wang \etal presented the first major attack on \shaone, giving a method to find collisions for the hash function
in time equivalent to $\approx 2^{69}$ evaluations of the compression function, which is significantly faster than the $\approx 2^{80}$ required by a generic process.
However, the high cost of the attack implied that no explicit collision was computed at the time. This remains the case today, even if improvements to the attack
and ever more efficient hardware would make such an endeavour more affordable than it was in 2005.
The considerable amount of work that followed Wang \etal's original contribution then mostly focused on first getting a better understanding of the attack process, and
secondly on bringing innovations allowing to explicitly provide collisions for reduced versions of \shaone with the highest number of steps. 

Considering the efficient collision attacks developped since 2005, it may seem remarkable that \shaone
in fact still offers a comfortable security margin against first preimage attacks. Until our own work,
the attack reaching the highest number of steps was due to Knellwolf and Khovratovich, who showed in 2012 how to attack $57/80$ of the function, with an estimated
complexity of $2^{158.8}$ evaluations of the compression function.


\bigskip

The third and last part of this thesis describes new attacks on \shaone.

We start by presenting explicit collision attacks for the full compression function of \shaone. While this comes short of a practical
attack on the \emph{hash function}, this is the first explicit result on \shaone for a standard security notion. Furthermore, the efficient implementation framework that was developed
for this work could also be reused for hash function attacks, which have a very similar (and in some respect simpler) structure than our ``freestart'' case.

Collision attacks on \shaone \emph{à la} Wang have a rather specific structure, full of small technicalities. Thus, after briefly
introducting \shaone in \autoref{cha:shaone_pres}, we start with a presentation of this topic in \autoref{cha:shaone_hist},
that we hope to be self-contained. Our contributions on freestart collisions are presented next in \autoref{cha:shaone_new}.

We then conclude this thesis by presenting in \autoref{cha:shaone_pre}
the current best preimage attacks on \shaone, in terms of number of attacked steps. By extending the framework developped by Knellwolf and Khovratovich to
higher-order differential attacks, we are able to increase by five the number of steps of \shaone for which a preimage can be computed faster than with
a generic method.



\cleardoublepage
\chapter*{Contents}
\parttoc

%%%%%%%%%%%%%%%%%%%%%%%%%%%%%%%%%%%%%%%%%%%%%%%%%%%%%%%%%%%%%%%%%%%%%%%%
%%%%%%%%%%%%%%%%%%%%%%%%%%%%%%%%%%%%%%%%%%%%%%%%%%%%%%%%%%%%%%%%%%%%%%%%




\chapter[Présentation de \shaone]
        {The \shaone hash function}
\label{cha:shaone_pres}

\section{Notations}
\label{sec:not}

We use a few notations in this article. \autoref{table:symbs} gives the meaning of various standard symbols and \autoref{table:appbitconditions} shows the signification of the symbols used to denote bit differences between two states.
Additionally, we use the following conventions: \shaone states, messages, and expanded messages are respectively denoted by $\state$, $\mess$, $\expmess$; for a variable $x$, the corresponding variable related by a specific
difference is noted $\widetilde{x}$, the difference itself is denoted by $\diff(x,\widetilde{x})$ or $\diff x$; two different variables of the same type are noted $x$, $x'$ when their difference is not specific;
a variable that can be seen as an array can have its words accessed through a subscript, indices starting from zero, \eg $x_2$ is the third word of $x$; a variable that can be seen as a fixed-size binary number
can have its bits accessed through a bracket notation, indices starting from zero, \eg $x[31]$ is the thirty-second bit of $x$; 
numbers written in hexadecimal use a fixed-space font and the \emph{0x} prefix, \eg \texttt{0x1337}; we sometimes also write numbers in base two, in which case they are written with a subscript $b$, \eg $1010_b$. Finally, `$\defas$' is used to denote equality by definition.
Various additional shorthands are introduced throughout the text.

\begin{table}[!htb]
\caption{Meaning of standard symbols.}\label{table:symbs}
\begin{center}
\begin{tabular}{c c}
\toprule
Symbol & Meaning\\
\midrule
$\oplus$ & Bitwise exclusive or\\
$+$ & Modular addition\\
$\boxplus$ & Modular addition, word-wise modular addition\\
$-$ & Modular subtraction\\
$\vee$ & Bitwise logical or\\
$\wedge$ & Bitwise logical and\\
$\neg$ & Bitwise complementation\\
$\circlearrowleft r$ & Bit rotation by $r$ to the left\\
$\circlearrowright r$ & Bit rotation by $r$ to the right\\
\bottomrule
\end{tabular}
\end{center}
\end{table}

\begin{table}[!htb]
{\centering\small%
\caption{Meaning of the bit difference symbols, for a symbol located on $\state_t[i]$. The same symbols are also used for $\expmess$.}\label{table:appbitconditions}
\begin{tabular}{c c}
  \toprule
  Symbol & Condition on ($\state$,$\dstate$) \\
	\midrule
  \nodiff & $\state_t[i] = \dstate_t[i]$  \\
  \onediff & $\state_t[i] \neq \dstate_t[i]$  \\
  \onediffu & $\state_t[i] = 0$, \quad $\dstate_t[i] = 1$  \\
  \onediffd & $\state_t[i] = 1$, \quad $\dstate_t[i] = 0$  \\
  $\mnodiffz$ & $\state_t[i] = \dstate_t[i] = 0$\\
  $\mnodiffo$ & $\state_t[i] = \dstate_t[i] = 1$ \\
  \midrule
  \equaup & $\state_t[i] = \dstate_t[i] = A_{t-1}[i]$ \\
  \diffup & $\state_t[i] = \dstate_t[i] \neq A_{t-1}[i]$ \\
	\equarightup & $\state_t[i] = \dstate_t[i] = (A_{t-1} \circlearrowright 2)[i]$  \\
	\diffrightup & $\state_t[i] = \dstate_t[i] \neq (A_{t-1} \circlearrowright 2)[i]$  \\
	\equarightupup & $\state_t[i] = \dstate_t[i] = (A_{t-2}  \circlearrowright 2)[i]$  \\
	\diffrightupup & $\state_t[i] = \dstate_t[i] \neq (A_{t-2} \circlearrowright 2)[i]$  \\
	\midrule
	\dunnodiff & No condition on $\state_t[i]$, $\dstate_t[i]$\\
  \bottomrule
\end{tabular}\\}
\end{table}


\section{The SHA-1 hash function}
\label{sec:description}

This section gives a brief description of the \shaone hash function, going again over some features already presented in \autoref{chap:hashfun}. We refer to the most recent NIST standard document~\cite{Nist-SHA} for a more thorough
presentation.

\shaone is a hash function from the \mdsha family which produces digests of $160$ bits.
Its high-level structure follows the Merkle-Damg{\aa}rd framework~\cite{DBLP:conf/crypto/Merkle89a,DBLP:conf/crypto/Damgard89a}: the input message to the function
is first padded to a length multiple of the block size, which is 512 bits, defining $k$ similarly-sized blocks.
Each block $\messblock_i$ is then fed to a compression function $\compress$ which is used to update a 160-bit chaining value $\chain_i$:
$\chain_{i+1} \defas \compress(\chain_i,\messblock_{i})$.
The first chaining value $\chain_0$ is a predefined constant set to an initial value \iv given in the specifications of the function, and the last value $\chain_k$ is the final output of the hash function.
The padding rule of \shaone is a straightforward application of \merkdam strengthening; it is at least 65-bit long and is made of one `1' bit followed by a possibly zero
number of `0' bits, and the length of the message to be hashed (without the padding) in bits as a 64-bit integer. The value of the \iv is given in \autoref{tab:sha_iv} as five 32-bit words;
each of these words initializes one of the five internal registers of the compression function, described below.
Note that neither the padding nor the \iv actually play any role in our attacks.

\begin{table}[ht]
\caption{\label{tab:sha_iv}The initial value (\iv) of \shaone.}
\begin{center}
\begin{tabular}{c c c c c} \toprule
$\mathcal{A}_0$:\texttt{0x67452301} & $\mathcal{B}_0$:\texttt{0xefcfab89} & $\mathcal{C}_0$:\texttt{0x98badcfe} & $\mathcal{D}_0$:\texttt{0x10325476} & $\mathcal{E}_0$:\texttt{0xc3d2e1f0} \\ 
\bottomrule
\end{tabular}
\end{center}
\end{table}

Similarly to other members of the \mdsha family, the compression function $\compress$ is built around an \emph{ad hoc} block cipher $\blockE$ used in a Davies-Meyer construction.
The block cipher itself is a five-branch generalized Feistel network using an Add-Rotate-XOR ``ARX'' step function with the addition of two non-linear Boolean functions, see \autoref{tab:sha_spec}.
In its full version, the step function is iterated 80 times, divided in four rounds of 20 steps.

The internal state of $\blockE$ consists of five 32-bit registers $(\mathcal{A}_i, \mathcal{B}_i, \mathcal{C}_i, \mathcal{D}_i, \mathcal{E}_i)$; at each step, a 32-bit \emph{expanded} message word $\expmess_i$ derived from
a message block $\messblock$
is used to update the five registers:

\[
\left\{
\begin{array}{lcl}
\mathcal{A}_{i+1} & = & (\mathcal{A}_i \circlearrowleft 5) + \boolF_i(\mathcal{B}_i,\mathcal{C}_i,\mathcal{D}_i) + \mathcal{E}_i + \mathcal{K}_i + \expmess_i\\
\mathcal{B}_{i+1} & = & \mathcal{A}_i\\
\mathcal{C}_{i+1} & = & \mathcal{B}_i \circlearrowright 2 \\
\mathcal{D}_{i+1} & = & \mathcal{C}_i \\
\mathcal{E}_{i+1} & = & \mathcal{D}_i \\
\end{array},
\right.	
\]

\noindent with $\mathcal{K}_i$ a constant and $\boolF_i$ one of three possible bitwise Boolean functions, see \autoref{tab:sha_spec} for their specifications.
We give a graphical representation of this step function in \autoref{fig:sha1_step}.
From this figure and the definition of the function, one can notice that all updated registers except $\mathcal{A}_{i+1}$ are just rotated copies of another;
thus it is possible to equivalently express \shaone's step function in a recursive way, using only (past values of) the register $\mathcal{A}$. The definition then becomes:
\begin{equation}
\label{eq:rec_step}
\mathcal{A}_{i+1} = (\mathcal{A}_i \circlearrowleft 5) + \boolF_i(\mathcal{A}_{i-1},\mathcal{A}_{i-2}\circlearrowright 2,\mathcal{A}_{i-3}\circlearrowright 2) + (\mathcal{A}_{i-4}\circlearrowright 2) + \mathcal{K}_i + \expmess_i.
\end{equation}

\noindent
With this notation, the output of one application of the compression function is made of the possibly rotated last five state words, $\mathcal{A}_{76}\ldots \mathcal{A}_{80}$.

\begin{figure}[ht]
\begin{center}
%% Public TikZ libraries
\usetikzlibrary{positioning}

%% Custom TikZ addons
\usetikzlibrary{crypto.symbols}
\tikzset{shadows=no}        % Option: add shadows to XOR, ADD, etc.

%% Document
\begin{tikzpicture}[
	thick,
	node distance=2.1cm and 0cm,
	]

    \tikzstyle{word} = [
		draw,
		rectangle,
		fill=Fuchsia!50,
		text centered,
		text width=3em,
		minimum width=2cm,	
		%blur shadow={shadow blur steps=5},	
	]

    \tikzstyle{dot} = [
		fill,
		shape=circle,
		minimum size=5pt,
		inner sep=0pt,
	]	

    \tikzstyle{F} = [
		draw,
		rectangle,
		fill=LimeGreen!80,
		text centered,
		text width=2em,
		minimum width=1em,
		minimum height=2em,	
		%blur shadow={shadow blur steps=5},	
	]

    \tikzstyle{rotatebox} = [
		draw,
		rectangle,
		fill=YellowOrange!70,
		text centered,
		text width=1cm,
		minimum width=1cm,	
		%blur shadow={shadow blur steps=5},		
	]
	
	\node[word] (a0) {$\mathcal{A}_{i}$};
	\node[word,right of = a0] (b0) {$\mathcal{B}_{i}$};
	\node[word,right of = b0] (c0) {$\mathcal{C}_{i}$};
	\node[word,right of = c0] (d0) {$\mathcal{D}_{i}$};
	\node[word,right of = d0] (e0) {$\mathcal{E}_{i}$};
    
	\node[MODADD,scale=1.5,below = 2em of e0] (add1) {};
	\node[MODADD,scale=1.5,below = 2em of add1] (add2) {};
	\node[MODADD,scale=1.5,below = 2em of add2] (add3) {};
	\node[MODADD,scale=1.5,below = 2em of add3] (add4) {};

	\path let \p1=(a0.east),\p2=(add2) in node[rotatebox] (rot1) at (\x1+0.5em,\y2) {$\circlearrowleft 5$};	
	\path let \p1=(b0),\p2=(add3) in node[rotatebox] (rot2) at (\x1,\y2) {$\circlearrowright 2$};

	\node[F,left = 1em of add1] (F) {$\boolF_{i}$};

	\node[word,below of = add4] (e1) {$\mathcal{E}_{i+1}$};
	\node[word,left of = e1] (d1) {$\mathcal{D}_{i+1}$};
	\node[word,left of = d1] (c1) {$\mathcal{C}_{i+1}$};
	\node[word,left of = c1] (b1) {$\mathcal{B}_{i+1}$};
	\node[word,left of = b1] (a1) {$\mathcal{A}_{i+1}$};

	\draw[line] (e0) -- (add1);
	\draw[line] (add1) -- (add2);
	\draw[line] (add2) -- (add3);
	\draw[line] (add3) -- (add4);
	\draw[line] (b0) -- (rot2);
	\draw[line] (rot1) -- (add2);

	\draw[line] let \p1=(add4.south),\p2=(a1.north),\p3=(add4.south) in 
		(\x1,\y1) 
		-- (\x1,\y3-0.5em)
		-- (\x2, \y2+1em)
		-- (\x2, \y2);

	\draw[line] let \p1=(c0.south),\p2=(d1.north),\p3=(add4.south) in 
		(\x1,\y1) 
		-- (\x1,\y3-0.5em)
		-- (\x2, \y2+1em)
		-- (\x2, \y2);

	\draw[line] let \p1=(rot2.south),\p2=(c1.north),\p3=(add4.south) in 
		(\x1,\y1) 
		-- (\x1,\y3-0.5em)
		-- (\x2, \y2+1em)
		-- (\x2, \y2);
		
	\draw[line] let \p1=(d0.south),\p2=(e1.north),\p3=(add4.south) in 
		(\x1,\y1) 
		-- (\x1,\y3-0.5em)
		-- (\x2, \y2+1em)
		-- (\x2, \y2);
				
	\draw[line] let \p1=(a0.south),\p2=(b1.north),\p3=(add4.south) in 
		(\x1,\y1) 
		-- (\x1,\y3-0.5em)
		-- (\x2, \y2+1em)
		-- (\x2, \y2);

	\draw[line] (F) -- (add1);

	\node[right = 2em of add3] (m) {$\mathcal{W}_{i}$};
	\draw[line] (m) -- (add3);

	\node[right = 2em of add4] (k) {$\mathcal{K}_{i}$};
	\draw[line] (k) -- (add4);

	\path let \p1=(rot1.west),\p2=(a0.south) in node[dot] (dotA) at (\x2,\y1) {};
	
	\path let \p1=(F.west),\p2=(d0.south) in node[dot] (dotD) at (\x2,\y1-0.6em) {};
	\path let \p1=(F.west),\p2=(c0.south) in node[dot] (dotC) at (\x2,\y1) {};
	\path let \p1=(F.west),\p2=(b0.south) in node[dot] (dotB) at (\x2,\y1+0.6em) {};
	
	\draw[line] (dotA) -- (rot1);
	\draw[line] let \p1=(dotD),\p2=(F.west) in (dotD) -- (\x2,\y1);
	\draw[line] let \p1=(dotC),\p2=(F.west) in (dotC) -- (\x2,\y1);
	\draw[line] let \p1=(dotB),\p2=(F.west) in (dotB) -- (\x2,\y1);

\end{tikzpicture}

\caption[One step of \shaone.]{One step of \shaone. Figure adapted from \cite{TiKZ:Cryptographers}.\label{fig:sha1_step}}
\end{center}
\end{figure}


\renewcommand{\arraystretch}{1.2}
\begin{table}[ht]
\caption{\label{tab:sha_spec}Boolean functions and constants of \shaone.}
\begin{center}
% why was it like this again?? \begin{tabular}{c c c c r @{}} \toprule
\begin{tabular}{c c c c} \toprule
$\;\; \textnormal{round} \;\;$ & step $i$ & $\boolF_i(x,y,z)$ &  $\mathcal{K}_i$ \\ \midrule
1 & $\;\;  \:\:0 \leq i <  20 \;\;$  & $\boolF_{\text{IF}} = (x \wedge y) \vee (\neg x \wedge z)$ & $\;\; \texttt{0x5a827999} \;\;$ \\
2 & $20 \leq i <  40$ & $\boolF_{\text{XOR}} = x \oplus y \oplus z$ & \texttt{0x6ed6eba1} \\
3 & $40 \leq i <  60$ & $\;\;  \boolF_{\text{MAJ}} = (x \wedge y) \vee (x \wedge z) \vee (y \wedge z) \;\;$  & \texttt{0x8fabbcdc} \\
4 & $60 \leq i <  80$ & $\boolF_{\text{XOR}} = x \oplus y \oplus z $ & \texttt{0xca62c1d6} \\
\bottomrule
\end{tabular}
\end{center}
\end{table}

\noindent
Finally, the expanded message words $\expmess_i$ are computed from the 512-bit message block $\messblock$. This message is first expressed as
sixteen 32-bit words $\mess_0,\ldots, \mess_{15}$, which are then used to recursively define the eighty 32-bit words $\expmess_i$:
\begin{equation}
\label{eq:exp_mess}
\expmess_i=
\left\{
\begin{array}{ll}
\mess_i, & \textnormal{ for } 0\leq i\leq 15 \\
(\expmess_{i-3} \oplus \expmess_{i-8} \oplus \expmess_{i-14} \oplus \expmess_{i-16}) \circlearrowleft 1, & \textnormal{ for } 16\leq i\leq 79
\end{array}.
\right.
\end{equation}

\noindent
The step function and the message expansion can both easily be inverted as follows:
\begin{equation}
\label{eq:rec_step_inv}
\mathcal{A}_{i}\hspace{-0.7mm}=\hspace{-0.8mm}(\mathcal{A}_{i+5} - \expmess_{i+4} - \mathcal{K}_{i+4} - \boolF_{i+4}(\mathcal{A}_{i+3},\mathcal{A}_{i+2}\circlearrowright 2,\mathcal{A}_{i+1}\circlearrowright 2) -
(\mathcal{A}_{i+4} \circlearrowleft 5))\hspace{-0.8mm}\circlearrowleft 2,
\end{equation}
\begin{equation}
\label{eq:exp_mess_inv}
\expmess_{i} = (\expmess_{i+16} \circlearrowright 1) \oplus \expmess_{i+13} \oplus \expmess_{i+8} \oplus \expmess_{i+2}.
\end{equation}



\chapter[Une brève histoire des attaques en collision sur \shaone]
        {A brief history of collision attacks on \shaone}
\label{cha:shaone_hist}

\section{The collision attack on SHA-1 from CRYPTO 2005, its ancestors and its developments}
\label{sec:history}

In this section we give a background on the literature of collision attacks on \shaone, that was initiated by the major work of Wang, Yin and Yu from CRYPTO~2005~\cite{DBLP:conf/crypto/WangYY05a}.
Some of the techniques used to attack \shaone had been previously introduced to attack other hash functions, and in particular \shazero, \shaone's close predecessor. Consequently, we start by
discussing some of these earlier work. We then review some of the more recent developments of the original attacks.

None of the material presented in this section is new, and it may be safely skipped by an experienced reader. We believe however that it may be of some use to a public less familiar with
the matter.

\medskip

All the attacks presented in this section are differential in nature. In its simplest form, the idea is to find a ``good'' \emph{message difference} $\diff(\expmess,\dexpmess)$
(or $\diff\expmess$)\footnote{Strictly speaking, the difference is imposed on the non-expanded message words $\mess_i$. However, the message expansion being linear,
we will usually rather consider the implied difference on the expanded message $\expmess_i$.}
and associated state differential path $\diff(\state,\dstate)$ (or $\diff\state$) such that: (1) A pair of states following the differential path
$\diff\state$ results in a collision; (2) Finding a pair of messages of difference $\diff\expmess$ such that the corresponding pair of states follows $\diff\state$ is
``efficient'' (in particular, this means that searching for a collision in that way is faster than searching for one by brute force).


\subsection{Preliminaries: collision attacks on \shazero}

In the initial SHA standard of 1993, there was no rotation by one to the left in the message expansion of the compression function~\cite{Nist-SHA0}; the corresponding original algorithm was
retrospectively named \shazero, to distinguish it from the updated standard \shaone, first published in 1995~\cite{Nist-SHA1}.
At CRYPTO~1998, Chabaud and Joux presented a theoretical collision attack on \shazero, with an estimated complexity of searching through $2^{61}$ message pairs~\cite{DBLP:conf/crypto/ChabaudJ98}
(equivalent to $2^{58}$ calls to \shazero~\cite{DBLP:journals/joc/BihamCJ15}). However, the modified message
expansion of \shaone prevents a straightforward application of the same approach from leading to an attack better than brute force.

There are four main components in this original attack on \shazero (and its first improvements~\cite{DBLP:conf/crypto/BihamC04,DBLP:conf/eurocrypt/BihamCJCLJ05,DBLP:journals/joc/BihamCJ15}),
all of which found their way to the attacks on \shaone (either \emph{as is} or in a modified form):
\begin{enumerate}
\item \emph{Local collisions} in the step function as a springboard for a collision for the full function (\autoref{sec:local_coll}).
\item Using \emph{signed differences} and a fine analysis of difference conditions in the Boolean functions (\autoref{sec:diffs_ana}).
\item Regrouping local collisions along a \emph{disturbance vector} (\autoref{sec:dv_sha0}).
\item Efficient implementation of the probabilistic search for colliding messages (\autoref{sec:acc_techs_sha0}).
\end{enumerate}

We will only briefly mention the multi-block techniques as used for \shazero~\cite{DBLP:conf/eurocrypt/BihamCJCLJ05,DBLP:journals/joc/BihamCJ15}, as these are not directly relevant to the best attacks on \shaone.
The improvements by Wang \etal used to attack the full \shaone are presented next in \autoref{sec:full_sha1}.

\subsubsection{Local collisions for a few steps of \sha}
\label{sec:local_coll}
An instructive starting point when searching for collisions on \sha is to first consider a \emph{linearized} variant (over $\ftwo$) of the step function, obtained by replacing
the Boolean functions $\boolF_\text{IF}$ and $\boolF_\text{MAJ}$ by $\boolF_\text{XOR}$ and the additions in $\ztt$ by additions in $\ftwo^{32}$ (\ie XORs). Although collisions for this simple
variant (named \shiun by Chabaud and Joux~\cite{DBLP:conf/crypto/ChabaudJ98}) are trivial to find, \shiun is useful as a simple model to build the main structure of the attack
(in particular the differential paths $\diff\expmess$ and $\diff\state$),
one element of which being the concept of \emph{local collisions} for the step function.

It is easy to see (for instance from \autoref{eq:rec_step}) that only five consecutive state words are used in the step function of \sha to determine the value of the next (or previous) one. Consequently, if
we could introduce a difference in the message, such that after some steps the five state words are equal in the two instances, then the final hash values for the
two computations will form a collision
as long as no more differences are present in the remainder of the message. Of course, the latter condition is hard to meet in general, but meeting the former seems to be quite easy as long as some limited
control on the message words is available. It is also a good first objective, as it locally achieves the result that we wish to get at the end of the computation.

\medskip

Let us assume that we have full control over six consecutive expanded message words used in the computation of two related \sha invocations: $\expmess_{i\ldots i + 5}$ and $\dexpmess_{i\ldots i + 5}$, and that
$\expmess_j = \dexpmess_j$ for $j < i$. We note $\state$ and $\dstate$ the respective state words of the two instances.

The first step of a local collision is to introduce a difference (so that the two messages are not consistently equal, as we are not interested in any trivial equality),
for instance in one bit. For the linear variant \shiun, the exact index where this difference is introduced does not matter much;
let us assume w.l.o.g. that $\expmess_i$ and $\dexpmess_i$ are different exactly on bit 8, \ie
\begin{center}
\begin{tabular}{c}
$\diff(\expmess_i, \dexpmess_i) =$ \nodiff \nodiff \nodiff \nodiff \nodiff \nodiff \nodiff \nodiff \nodiff \nodiff \nodiff \nodiff \nodiff \nodiff
\nodiff \nodiff \nodiff \nodiff \nodiff \nodiff \nodiff \nodiff \nodiff \nodiff \onediff \nodiff \nodiff \nodiff \nodiff \nodiff \nodiff \nodiff \\
\end{tabular}.
\end{center}
From \autoref{eq:rec_step}, we see that this introduces a difference between $\state_{i + 1}$ and $\dstate_{i + 1}$ in the same position, \ie on bit 8. Our goal is now to ensure that this
difference does not propagate further, and that there is no difference between $\state_{i + 2\ldots i + 6}$ and $\dstate_{i + 2\ldots i + 6}$:
\begin{itemize}
\item At ($\state_{i + 2},\dstate_{i + 2}$), we must cancel the difference coming from $(\state_{(i + 2) - 1}, \dstate_{(i + 2) - 1}) \circlearrowleft 5 = (\state_{i + 1}, \dstate_{i + 1}) \circlearrowleft 5$;
this is done by inserting a difference in bit 13 of $\expmess_{i + 1}$:
\begin{center}
\begin{tabular}{c}
$\diff(\expmess_{i+1}, \dexpmess_{i+1}) =$ \nodiff \nodiff \nodiff \nodiff \nodiff \nodiff \nodiff \nodiff \nodiff
\nodiff \nodiff \nodiff \nodiff \nodiff \nodiff \nodiff \nodiff \nodiff \nodiff \onediff \nodiff \nodiff \nodiff \nodiff \nodiff \nodiff \nodiff \nodiff \nodiff \nodiff \nodiff \nodiff \\
\end{tabular}.
\end{center}
%
\item At ($\state_{i + 3},\dstate_{i + 3}$), we must cancel the difference coming from $(\state_{(i + 3) - 2}, \dstate_{(i + 3) - 2}) = (\state_{i + 1}, \dstate_{i + 1})$; this is done by inserting a difference in bit 8 of $\expmess_{i + 2}$:
\begin{center}
\begin{tabular}{c}
$\diff(\expmess_{i+2}, \dexpmess_{i+2}) =$ \nodiff \nodiff \nodiff \nodiff \nodiff \nodiff \nodiff \nodiff \nodiff \nodiff \nodiff \nodiff \nodiff \nodiff
\nodiff \nodiff \nodiff \nodiff \nodiff \nodiff \nodiff \nodiff \nodiff \nodiff \onediff \nodiff \nodiff \nodiff \nodiff \nodiff \nodiff \nodiff \\
\end{tabular}.
\end{center}
%
\item At ($\state_{i + 4},\dstate_{i + 4}$), we must cancel the difference coming from $(\state_{(i + 4) - 3}, \dstate_{(i + 4) - 3}) \circlearrowright 2 = (\state_{i + 1}, \dstate_{i + 1}) \circlearrowright 2$;
this is done by inserting a difference in bit 6 of $\expmess_{i + 3}$:
\begin{center}
\begin{tabular}{c}
$\diff(\expmess_{i+3}, \dexpmess_{i+3}) =$  \nodiff \nodiff \nodiff \nodiff \nodiff \nodiff \nodiff \nodiff \nodiff \nodiff \nodiff \nodiff \nodiff \nodiff \nodiff \nodiff
\nodiff \nodiff \nodiff \nodiff \nodiff \nodiff \nodiff \nodiff \nodiff \nodiff \onediff \nodiff \nodiff \nodiff \nodiff \nodiff\\
\end{tabular}.
\end{center}
%
\item At ($\state_{i + 5},\dstate_{i + 5}$), we must cancel the difference coming from $(\state_{(i + 5) - 4}, \dstate_{(i + 5) - 4}) \circlearrowright 2 = (\state_{i + 1}, \dstate_{i + 1}) \circlearrowright 2$;
this is done by inserting a difference in bit 6 of $\expmess_{i + 4}$:
\begin{center}
\begin{tabular}{c}
$\diff(\expmess_{i+4}, \dexpmess_{i+4}) =$  \nodiff \nodiff \nodiff \nodiff \nodiff \nodiff \nodiff \nodiff \nodiff \nodiff \nodiff \nodiff \nodiff \nodiff \nodiff \nodiff
\nodiff \nodiff \nodiff \nodiff \nodiff \nodiff \nodiff \nodiff \nodiff \nodiff \onediff \nodiff \nodiff \nodiff \nodiff \nodiff\\
\end{tabular}.
\end{center}
%
\item At ($\state_{i + 6},\dstate_{i + 6}$), we must cancel the difference coming from $(\state_{(i + 6) - 5}, \dstate_{(i + 6) - 5}) \circlearrowright 2 = (\state_{i + 1}, \dstate_{i + 1}) \circlearrowright 2$;
this is done by inserting a difference in bit 6 of $\expmess_{i + 5}$:
\begin{center}
\begin{tabular}{c}
$\diff(\expmess_{i+5}, \dexpmess_{i+5}) =$  \nodiff \nodiff \nodiff \nodiff \nodiff \nodiff \nodiff \nodiff \nodiff \nodiff \nodiff \nodiff \nodiff \nodiff \nodiff \nodiff
\nodiff \nodiff \nodiff \nodiff \nodiff \nodiff \nodiff \nodiff \nodiff \nodiff \onediff \nodiff \nodiff \nodiff \nodiff \nodiff\\
\end{tabular}.
\end{center}
\end{itemize}
At this point, we have reached our goal of having no differences in a pair of five consecutive state words.

The pattern formed by the successive message differences of a local collision is commonly seen in various stages of attacks on \sha, and as such it deserves to be shown in its entirety:
\begin{center}
\begin{tabular}{cc}
$\diff(\expmess_{i\ldots i+5}, \dexpmess_{i\ldots i+5}) =$ & \nodiff \nodiff \nodiff \nodiff \nodiff \nodiff \nodiff \nodiff \nodiff \nodiff \nodiff \nodiff \nodiff \nodiff
\nodiff \nodiff \nodiff \nodiff \nodiff \nodiff \nodiff \nodiff \nodiff \nodiff \onediff \nodiff \nodiff \nodiff \nodiff \nodiff \nodiff \nodiff \\
& \nodiff \nodiff \nodiff \nodiff \nodiff \nodiff \nodiff \nodiff \nodiff
\nodiff \nodiff \nodiff \nodiff \nodiff \nodiff \nodiff \nodiff \nodiff \nodiff \onediff \nodiff \nodiff \nodiff \nodiff \nodiff \nodiff \nodiff \nodiff \nodiff \nodiff \nodiff \nodiff \\
& \nodiff \nodiff \nodiff \nodiff \nodiff \nodiff \nodiff \nodiff \nodiff \nodiff \nodiff \nodiff \nodiff \nodiff
\nodiff \nodiff \nodiff \nodiff \nodiff \nodiff \nodiff \nodiff \nodiff \nodiff \onediff \nodiff \nodiff \nodiff \nodiff \nodiff \nodiff \nodiff \\
&  \nodiff \nodiff \nodiff \nodiff \nodiff \nodiff \nodiff \nodiff \nodiff \nodiff \nodiff \nodiff \nodiff \nodiff \nodiff \nodiff
\nodiff \nodiff \nodiff \nodiff \nodiff \nodiff \nodiff \nodiff \nodiff \nodiff \onediff \nodiff \nodiff \nodiff \nodiff \nodiff\\
&  \nodiff \nodiff \nodiff \nodiff \nodiff \nodiff \nodiff \nodiff \nodiff \nodiff \nodiff \nodiff \nodiff \nodiff \nodiff \nodiff
\nodiff \nodiff \nodiff \nodiff \nodiff \nodiff \nodiff \nodiff \nodiff \nodiff \onediff \nodiff \nodiff \nodiff \nodiff \nodiff\\
&  \nodiff \nodiff \nodiff \nodiff \nodiff \nodiff \nodiff \nodiff \nodiff \nodiff \nodiff \nodiff \nodiff \nodiff \nodiff \nodiff
\nodiff \nodiff \nodiff \nodiff \nodiff \nodiff \nodiff \nodiff \nodiff \nodiff \onediff \nodiff \nodiff \nodiff \nodiff \nodiff\\
\end{tabular}.
\end{center}

In the case of \shiun, the probability of obtaining a local collision when following the above pattern is equal to one. However, this is not the case anymore when the true \sha step function is used.
For instance, there is a probability $2^{-1}$ that the introduction of the difference in $(\state_{i+1},\dstate_{i+1})$ with modular addition rather than XOR
leads to a difference in more than one bit because of different
behaviours of the propagation of the carry in the two states (on which we do not assume to have any control). Overall, the probability of obtaining a successful local collision depends on several factors, including which Boolean function is used
and whether several collisions are chained together. We partially address this matter next.

\subsubsection{Difference analysis for impure ARX}
\label{sec:diffs_ana}
We now move away from \shiun and turn back to analysing the behaviour of local collisions for the true \sha function. There are two main points in this analysis: (1)~What conditions \emph{at the bit level}
ensure the highest probability of success for the different types of single local collisions; (2)~Under optimal conditions, what is the probability of a chain of interdependent local collisions?
We focus on the first question here, and defer the answer to the second to \autoref{sec:chain_lc}.

\medskip

In the case of \sha (and more generally ARX primitives), the way of expressing differences between messages is less obvious than for
\eg{} bit or byte-oriented primitives. It is indeed natural to consider both ``XOR differences'' (over $\ftwo^{32}$) and
``modular differences'' (over $\ztt$), as both operations are used in the function.
In practice, the literature on \sha uses several hybrid representations of differences based on \emph{signed XOR differences}.
In its most basic form, such a difference is similar to an XOR difference with the additional information of the value of the differing bits (and of bits equal to each other),
which is a ``sign'' for the difference.

This is an important information when one works with modular addition as the sign impacts the (absence of) propagation of carries in the addition of two differences.
Let us for instance consider the two pairs of words $a = 11011000001_b$, $\tilde{a} = 11011000000_b$ and $b = 10100111000_b$, $\tilde{b} = 10100111001_b$; the XOR
differences $(a \oplus \tilde{a})$ and $(b \oplus \tilde{b})$ are both $00000000001_b$ (\ie \nodiff\nodiff\nodiff\nodiff\nodiff\nodiff\nodiff\nodiff\nodiff\nodiff\onediff),
meaning that $(a \oplus b) = (\tilde{a} \oplus \tilde{b})$. On the other hand, the signed
XOR difference between $a$ and $\tilde{a}$ may be written \nodiff\nodiff\nodiff\nodiff\nodiff\nodiff\nodiff\nodiff\nodiff\nodiff\onediffd to convey the fact that they are different on their lowest bit \emph{and} that
the value of this bit is 1 for $a$ (and thence 0 for $\tilde{a}$), \ie $\tilde{a} = a - 1$ (using modular addition); similarly, the signed difference between $b$ and $\tilde{b}$ may be written
\nodiff\nodiff\nodiff\nodiff\nodiff\nodiff\nodiff\nodiff\nodiff\nodiff\onediffu, which is a difference in the same position but of a different sign, \ie $\tilde{b} = b + 1$. From these differences, we can deduce that $(a + b) = (\tilde{a} + \tilde{b})$
because differences of different signs cancel (while differences of the same sign do not); if we were to swap the values $b$ and $\tilde{b}$, both differences on $a$ and $b$ would have the same sign and
indeed we have $(a + \tilde{b}) \neq (\tilde{a} + b)$ (though $(a \oplus \tilde{b})$ and $(\tilde{a} \oplus b)$ remain equal).
In the case of bits with no differences, we may similarly want to use different notations to express the fact that two bits are equal to zero (\nodiffz) or equal to one (\nodiffo).

It is possible to extend signed differences to account for more generic combinations of possible
values for each message bit; this was for instance done by De~Canni\`ere and Rechberger to aid in the automatic search of differential paths \cite{DBLP:conf/asiacrypt/CanniereR06}.
Another possible extension  is to consider relations between various bits of different (possibly rotated) state words;
this allows to efficiently keep track of the propagation of differences through the step function. Such differences are for instance used by Stevens \cite{DBLP:conf/eurocrypt/Stevens13},
and also later in this work (see \autoref{table:appbitconditions}).

\medskip

Using signed differences, we can express a first simple necessary condition for a local collision to happen: because the initial difference in $\diff\state_{i+1}$ has to be
canceled in $\diff\state_{i+2}$ through the modular addition of $\diff\expmess_{i+1}$, we know that these differences have to be of different sign. As we have some control on the
message, we can ensure that this is always the case for successful introductions of the difference on $\diff\state_{i+1}$ (that do not result in different carry propagations for
$\state_{i+1}$ and $\dstate_{i+1}$) by analysing what may happen in the four possible cases  of an introduction, at the level of one bit (depending on the signs of the involved differences):
\begin{enumerate}
\item \onediffu (the difference in $\diff\expmess_i$, introducing the perturbation) + \nodiffz (the ``difference'' in the same position of the partial sum used to compute $\diff\state_{i+1}$) = \onediffu,
with no different carry propagation in the remainder of $\diff\state_{i+1}$). 
\item \onediffu  + \nodiffo = \onediffd, with carry propagation.
\item \onediffd  + \nodiffz = \onediffd, with no carry propagation.
\item \onediffd  + \nodiffo = \onediffu, with carry propagation.
\end{enumerate}
Of these four cases, (1) and (3) are always favourable to a local collision; (2) and (4) are not considered to be favourable here, except if the differences are on the most significant bit
of the message and state words, as in this case the carry propagation is absorbed by the modular reduction (in that unique case, unsigned differences may be used safely)\footnote{We will
later see in \autoref{sec:chain_lc} that in some cases, a difference in the carry propagation does not actually always result in the absence of a local collision.}.
Thus, except when the difference is introduced on the MSB, we see that
a successful introduction on $\diff\state_{i+1}$ preserves the sign of the difference $\diff\expmess_i$, and thence we must always
choose a different sign for the difference on $\diff\expmess_{i+1}$.

We can analyse the rest of the conditions for a successful local collision in a similar fashion; it is helpful at this point to consider the possible behaviours of the Boolean functions of \sha
for their different signed inputs. We follow here the approach of Joux~\cite[Chapter 5]{algocrypt} and start by analysing the propagation of signed differences \nodiffz, \nodiffo,
\onediffd, \onediffu through $\boolF_\text{IF}$ (\autoref{tbl:diff_if}), $\boolF_\text{XOR}$ (\autoref{tbl:diff_xor}) and $\boolF_\text{MAJ}$ (\autoref{tbl:diff_xor}), before
considering what happens for each remaining correction of the local collision in $\diff\state_{i+3\ldots i+6}$.
An essentially identical analysis can for instance be found in~\cite{phdpeyrin,DBLP:journals/joc/BihamCJ15}.

We should note that for reasons that will be made clear in \autoref{sec:dv_sha0}, the corrections used to obtain a local collision must work with every possible Boolean function\footnote{This is not
true if our goal is to build boomerangs, which is something that we will consider in \autoref{sec:acc_techs_sha0}.}.
A consequence is that we cannot use the fact
that the non-linear functions $\boolF_\text{IF}$ and $\boolF_\text{MAJ}$ may absorb a single difference (as in \eg $\boolF_\text{IF}(\text{\nodiffz, \onediffu, \nodiffz})$),
as there is never such a behaviour with $\boolF_\text{XOR}$; thus, there is always a correction introduced in every message, true to the local collision pattern
of \shiun of \autoref{sec:local_coll}. In general, the following things may then happen in the Boolean functions: the difference is absorbed (this never happens
with $\boolF_\text{XOR}$);
the difference is preserved, but its sign is changed (this never happens with $\fmaj$); the difference is preserved and its sign is not changed.
As we already mentioned, knowing the sign of a difference is important if we want to cancel it with another one; thus, the possibility of
unpredictable changes of signs decreases the success probability of a local collision.
Let us see in general how this latter is impacted by the behaviours of the different Boolean functions. 
\begin{itemize}
\item $\diff\state_{i+3}$: the difference is on the first input ($x$) of the Boolean function. With $\fif$, there is a probability $2^{-1}$ that it is absorbed and $2^{-2}$
that the sign is changed otherwise, for a total success probability of $2^{-2}$ ($2^{-1}$ on the MSB, where a change of sign has no consequence).
With $\fxor$, there is a probability $2^{-1}$ that the sign is changed (total success of $2^{-1}$, 1 on the MSB).
With $\fmaj$, there is a probability $2^{-1}$ that it is absorbed (total success of $2^{-1}$).
\item  $\diff\state_{i+4}$: the difference is on the second input ($y$) of the Boolean function. With $\fif$, there is a probability $2^{-1}$ that it is absorbed (total success of $2^{-1}$).
With $\fxor$ there is a probability $2^{-1}$ that the sign is changed (total success of $2^{-1}$, 1 on the MSB). With $\fmaj$ there is a probability $2^{-1}$ that it is absorbed (total success of
$2^{-1}$).
\item $\diff\state_{i+5}$: the difference is on the third input ($z$) of the Boolean function. This case is the same as for $\diff\state{i+4}$.
\item $\diff\state_{i+6}$: the difference is on a modular addition. If it is on the MSB, nothing needs to be done. Otherwise, it needs to be of sign opposite the one of the introductory difference to
ensure a correction (which will then happen with probability 1).
\end{itemize}

This analysis may be useful in several ways. First, it gives some necessary conditions on the signs of the corrections, possibly conditioned on the Boolean function of the round being considered;
more generally, it may actually be used as a (nearly) exhaustive list of inputs resulting in local collisions, but this is less useful as we do not have control on the values of $\diff\state$ in general. Second,
it helps us to position the local collisions in an ``optimal'' way, by ensuring that many corrections involve bits that are at the most significant position, as we have seen that these may
have higher success probabilities. Finally, it
may be used to check in advance if a given local collisions is going to happen or not, without necessarily computing the two states $\state$ and $\dstate$. For instance, in the first round,
assuming a perturbation on bit $j$ of $\diff\state_{i+1}$, we can predict that there will be no local collision if the bits $j - 2$ of $\diff\state_{i}$ and $\diff\state_{i-1}$ are equal, even if the perturbation is properly
introduced. Indeed, none of $\fif(\text{\onediffu,\nodiffz,\nodiffz})$, $\fif(\text{\onediffu,\nodiffo,\nodiffo})$, $\fif(\text{\onediffd,\nodiffz,\nodiffz})$, $\fif(\text{\onediffd,\nodiffo,\nodiffo})$
results in a difference
\footnote{This can also be trivially deduced from the very definition of the IF function.}, which means that there will be no difference in the Boolean function computation at $\diff\state_{i+3}$ and thus the tentative correction
in $\diff\expmess_{i+2}$ will actually introduce a difference.

\begin{table}[ht]
\caption{Signed difference analysis of $\boolF_\text{IF}(x,y,z)$\label{tbl:diff_if}}
\begin{center}
\begin{tabularx}{\textwidth}{c | c c c c  X  c | c c c c}
\toprule
$x$ = \nodiffz & $y$ = \nodiffz & $y$ = \nodiffo & $y$ = \onediffu & $y$ = \onediffd & & $x$ = \nodiffo & $y$ = \nodiffz & $y$ = \nodiffo & $y$ = \onediffu & $y$ = \onediffd \\
\hline
$z$ = \nodiffz & \nodiffz & \nodiffz & \nodiffz & \nodiffz &                   & $z$ = \nodiffz & \nodiffz & \nodiffo & \onediffu & \onediffd\\
$z$ = \nodiffo & \nodiffo & \nodiffo & \nodiffo & \nodiffo &                   & $z$ = \nodiffo & \nodiffz & \nodiffo & \onediffu & \onediffd\\
$z$ = \onediffu & \onediffu & \onediffu & \onediffu & \onediffu &                   & $z$ = \onediffu & \nodiffz & \nodiffo & \onediffu & \onediffd\\
$z$ = \onediffd & \onediffd & \onediffd & \onediffd & \onediffd &                   & $z$ = \onediffd & \nodiffz & \nodiffo & \onediffu & \onediffd\\ 
\midrule
$x$ = \onediffu & $y$ = \nodiffz & $y$ = \nodiffo & $y$ = \onediffu & $y$ = \onediffd & & $x$ = \onediffd & $y$ = \nodiffz & $y$ = \nodiffo & $y$ = \onediffu & $y$ = \onediffd \\
\hline
$z$ = \nodiffz & \nodiffz & \onediffu & \onediffu & \nodiffz &                 & $z$ = \nodiffz & \nodiffz &  \onediffd & \nodiffz & \onediffd \\
$z$ = \nodiffo & \onediffd & \nodiffo & \nodiffo & \onediffd &                 & $z$ = \nodiffo & \onediffu & \nodiffo & \onediffu & \nodiffo \\
$z$ = \onediffu & \nodiffz & \onediffu & \onediffu & \nodiffz &                & $z$ = \onediffu & \onediffu & \nodiffo & \onediffu & \nodiffo \\
$z$ = \onediffd & \onediffd & \nodiffo & \nodiffo & \onediffd &                & $z$ = \onediffd & \nodiffz & \onediffd & \nodiffz & \onediffd\\
\bottomrule
\end{tabularx}
\end{center}
\end{table}

\begin{table}[ht]
\caption{Signed difference analysis of $\boolF_\text{XOR}(x,y,z)$\label{tbl:diff_xor}}
\begin{center}
\begin{tabularx}{\textwidth}{c | c c c c  X  c | c c c c}
\toprule
$x$ = \nodiffz & $y$ = \nodiffz & $y$ = \nodiffo & $y$ = \onediffu & $y$ = \onediffd & & $x$ = \nodiffo & $y$ = \nodiffz & $y$ = \nodiffo & $y$ = \onediffu & $y$ = \onediffd \\
\hline
$z$ = \nodiffz & \nodiffz & \nodiffo & \onediffu & \onediffd &                   & $z$ = \nodiffz & \nodiffo & \nodiffz & \onediffd & \onediffu\\
$z$ = \nodiffo & \nodiffo & \nodiffz & \onediffu & \onediffd &                   & $z$ = \nodiffo & \nodiffz & \nodiffo & \onediffd & \onediffu\\
$z$ = \onediffu & \onediffu & \onediffd & \nodiffz & \nodiffo &                   & $z$ = \onediffu & \onediffd & \onediffu & \nodiffo & \nodiffz\\
$z$ = \onediffd & \onediffd & \onediffu & \nodiffo & \nodiffz &                   & $z$ = \onediffd & \onediffu & \onediffd & \nodiffz & \nodiffo\\ 
\midrule
$x$ = \onediffu & $y$ = \nodiffz & $y$ = \nodiffo & $y$ = \onediffu & $y$ = \onediffd & & $x$ = \onediffd & $y$ = \nodiffz & $y$ = \nodiffo & $y$ = \onediffu & $y$ = \onediffd \\
\hline
$z$ = \nodiffz & \onediffu & \onediffd & \nodiffz & \nodiffo &                 & $z$ = \nodiffz & \onediffd &  \onediffu & \nodiffo & \nodiffz \\
$z$ = \nodiffo & \onediffd & \onediffu & \nodiffo & \nodiffz &                 & $z$ = \nodiffo & \onediffu & \onediffd & \nodiffz & \nodiffo \\
$z$ = \onediffu & \nodiffz & \nodiffo & \onediffu & \onediffd &                & $z$ = \onediffu & \nodiffo & \nodiffz & \onediffd & \onediffu \\
$z$ = \onediffd & \nodiffo & \nodiffz & \onediffd & \onediffu &                & $z$ = \onediffd & \nodiffz & \nodiffo & \onediffu & \onediffd\\
\bottomrule
\end{tabularx}
\end{center}
\end{table}

\begin{table}[ht]
\caption{Signed difference analysis of $\boolF_\text{MAJ}(x,y,z)$\label{tbl:diff_maj}}
\begin{center}
\begin{tabularx}{\textwidth}{c | c c c c  X  c | c c c c}
\toprule
$x$ = \nodiffz & $y$ = \nodiffz & $y$ = \nodiffo & $y$ = \onediffu & $y$ = \onediffd & & $x$ = \nodiffo & $y$ = \nodiffz & $y$ = \nodiffo & $y$ = \onediffu & $y$ = \onediffd \\
\hline
$z$ = \nodiffz & \nodiffz & \nodiffz & \nodiffz & \nodiffz &                   & $z$ = \nodiffz & \nodiffz & \nodiffo & \onediffu & \onediffd\\
$z$ = \nodiffo & \nodiffz & \nodiffo & \onediffu & \onediffd &                   & $z$ = \nodiffo & \nodiffo & \nodiffo & \nodiffo & \nodiffo\\
$z$ = \onediffu & \nodiffz & \onediffu & \onediffu & \nodiffz &                   & $z$ = \onediffu & \onediffu & \nodiffo & \onediffu & \nodiffo\\
$z$ = \onediffd & \nodiffz & \onediffd & \nodiffz & \onediffd &                   & $z$ = \onediffd & \onediffd & \nodiffo & \nodiffo & \onediffd\\ 
\midrule
$x$ = \onediffu & $y$ = \nodiffz & $y$ = \nodiffo & $y$ = \onediffu & $y$ = \onediffd & & $x$ = \onediffd & $y$ = \nodiffz & $y$ = \nodiffo & $y$ = \onediffu & $y$ = \onediffd \\
\hline
$z$ = \nodiffz & \nodiffz & \onediffu & \onediffu & \nodiffz &                 & $z$ = \nodiffz & \nodiffz &  \onediffd & \nodiffz & \onediffd \\
$z$ = \nodiffo & \onediffu & \nodiffo & \onediffu & \nodiffo &                 & $z$ = \nodiffo & \onediffd & \nodiffo & \nodiffo & \onediffd \\
$z$ = \onediffu & \onediffu & \onediffu & \onediffu & \onediffu &                & $z$ = \onediffu & \nodiffz & \nodiffo & \onediffu & \onediffd \\
$z$ = \onediffd & \nodiffz & \nodiffo & \onediffu & \onediffd &                & $z$ = \onediffd & \onediffd & \onediffd & \onediffd & \onediffd\\
\bottomrule
\end{tabularx}
\end{center}
\end{table}

% two things: 1) optimal conditions; 2) proba (w. interactions) given optimal conditions

\subsubsection{Disturbance vectors}
\label{sec:dv_sha0}

We have seen how using a local collision allows one to create a pair of different messages which may lead to a pair of \sha states that are identical at some point during the
computation of their respective digests. Given enough control on the message, one may obtain such an equality with probability one, and it is quite easy to derive sufficient conditions
for a single local collision to happen, which allows to derive the success probability of uncontrolled independent local collisions. 

In order to mount a complete attack and obtain a collision on the actual digest, we need to ensure that the last five state words are free of differences: this means that no local collision
must have been started after step 75. Of course, we also need the colliding messages to be valid expanded messages, and this is where local collisions become really useful: the idea is to
interleave a series of local collisions together so that: (1)~The resulting message difference follows the message expansion; (2)~The joint probability of all the local collisions being successful
is high (in particular, an attack is obtained if it is higher than $\approx 2^{-80}$).

The first condition is actually easy to meet by defining a \emph{disturbance vector} (\dv). This is a vector of eighty 32-bit words which `1' bits define the positions of the initial perturbations of all the
local collisions of the series. Now it suffices to remark that because the message expansion is linear, a sum of expanded messages is an expanded message itself;
then if the disturbance vector is a valid expanded message word, so is the complete message difference, including the corrections of the local
collisions. For this to be correct, however, two conditions need to be met: (1)~The disturbance vector must have no differences in its first five ``negative'' words $\expmess_{-5,\ldots,-1}$, obtained through
the backward message expansion. Indeed, remembering \autoref{sec:local_coll}, the corrections of the local collisions will be obtained from (possibly rotated) shifted copies of the pattern of initial perturbations introduced by the \dv,
up to five positions down. If we want these copies to be valid expanded message words, they must be equal to the \dv with its last (up to five) words removed and with a few (up to five) words
added at the beginning, which would be obtained from the backward message expansion. As the added words must be zero, lest they create new perturbations that would not be corrected, we obtain the aforementioned
condition. (2)~All corrections need to be performed in the same way, so that they globally conform to the message expansion; this is why one cannot exploit the absorption properties of some Boolean functions as noticed in \autoref{sec:diffs_ana},
as it is impossible to do so consistently across all the rounds.

This simple characterization of series of local collisions allows to define efficient search strategies to find vectors
that achieve a high joint probability.
In the case of \shazero, the message expansion can easily be defined at the bit level, as the bit $i$ of an expanded message word $\expmess_j$, $j > 15$ is entirely determined by the bit $i$
of the message words $\expmess_{j-16\ldots j-1}$. This means that any expanded message (and in particular a disturbance vector) can be seen as the union of 32 ``one-bit expanded messages'' that
do not interact with each other. Thus, a search for good disturbance vectors can focus solely on such one-bit messages. As any consecutive window of sixteen message words entirely
defines the remainder of an expanded message word through the (backward) message expansion, one can see that there is only a small number of $2^{16}$ one-bit disturbance vectors to consider, which
can then be shifted laterally to start on any of the 32 bits of a message word.

Of the $2^{16}$ candidate disturbance vectors, many do not meet the condition of being zero on their five first negative positions and their five last positions. Overall, these requirements give
``10 bits of conditions'', and indeed only $2^6$ vectors meet all of them (one of them being the all-zero vector)~\cite[Chapter 5]{algocrypt}. It now remains to determine which of these lead
to the best attacks.

\medskip

A first (rather crude) way of estimating the cost of a collision associated with a certain \dv is simply to count the number of local collisions it introduces (\ie to consider the Hamming weight
of the vector) and to multiply this by the probability of a local collision being successful. This estimate can immediately be enhanced by discounting the cost of collisions appearing in the
first sixteen state words, as the attacker has a full control on the corresponding message words and can then fulfill all their associated conditions deterministically. Additionally, recalling
\autoref{sec:diffs_ana}, we know that the success probability of a local collision increases if some of the bit differences are located on the MSB. Thus, the probability of a \dv is not
invariant by lateral shift, and each vector should be considered on its best positions only (given the pattern of corrections of a local collision, inserting the perturbations on
bit one (starting from zero) is a good choice).

There is one problem remaining with this first cost function, which is that it assumes that all local collisions are independent. This is actually not the case, and a more detailed analysis of
the success probability of non-independent local collisions is necessary to get an accurate estimate of the overall cost of an attack. We do not detail such an analysis in the present case
of \shazero, and instead refer to \cite[Chapter 5]{algocrypt}, which shows possible interactions between local collisions in a case-by-case study. The main result of this is that
some interactions introduce contradictions that cannot be resolved and which thence entirely disqualify some \dvs (this may happen in particular because of the absorption capabilities
of $\boolF_\text{IF}$; in that case, it disqualifies \dvs with two consecutive perturbations in the first round (the ``IF'' round)),
while the probability of some other interactions being successful can be increased by properly choosing the sign of some specific collisions. We will later discuss
the issue further for \shaone in \autoref{sec:chain_lc}.

With this new cost function in mind, two good disturbance vectors where found for the original attack on \shazero~\cite{DBLP:conf/crypto/ChabaudJ98}; we show the first of them in
\autoref{fig:shaz_dv}, using an unsigned differences notation.

\begin{figure}[ht]
\begin{center}
\nodiff \nodiff \nodiff \onediff \nodiff \nodiff \nodiff \nodiff \nodiff \nodiff \nodiff \onediff \nodiff \nodiff \onediff \nodiff \nodiff \nodiff \nodiff \nodiff
\nodiff \nodiff \onediff \nodiff \nodiff \nodiff \nodiff \onediff \onediff \nodiff \onediff \onediff \nodiff \onediff \onediff \onediff \onediff \onediff \onediff \nodiff\\

\onediff \onediff \nodiff \onediff \nodiff \nodiff \onediff \nodiff \nodiff \nodiff \nodiff \onediff \nodiff \onediff \nodiff \onediff \nodiff \nodiff \onediff \nodiff
\onediff \nodiff \onediff \nodiff \nodiff \nodiff \onediff \nodiff \onediff \onediff \onediff \nodiff \nodiff \onediff \onediff \nodiff \nodiff \nodiff \nodiff \nodiff
\end{center}
\caption{A good (one bit) disturbance vector for \shazero, with a basic cost of $2^{68}$. Bit zero is shown on the top left.\label{fig:shaz_dv}}
\end{figure}

\subsubsection{Accelerating techniques for collision search}
\label{sec:acc_techs_sha0}

A suitable disturbance vector such as the one of \autoref{fig:shaz_dv} defines an explicit attack procedure in a straightforward way: one uses the available freedom in the first
sixteen message words to deterministically fulfill as many conditions for local collisions, while keeping enough free bits to expect being able to fulfill the remaining conditions
probabilistically.

This was the approach taken in the original attack on \shazero, and it results in an attack of complexity $2^{67}$ (measured in the number of message pairs to be tested for a collision)
for the vector of \autoref{fig:shaz_dv}, and $2^{61}$ for another good disturbance vector~\cite{DBLP:conf/crypto/ChabaudJ98}.

Although this is already an attack that is significantly faster than a brute-force approach, it remains fairly expensive, and no explicit collision for \shazero was computed at the time\footnote{Even
nearly twenty years later, such an attack requires considerable ressources; it would roughly be comparable in cost with the full free-start collision for \shaone which is the main topic of
this article.}.
A series of techniques were later developped to make such attacks considerably more efficient, which ultimately led to the first explicit collision on \shazero~\cite{DBLP:conf/eurocrypt/BihamCJCLJ05}.
These improvements were of two kinds: (1) Chaining multiple blocks to enable the use of better \dvs; (2) Using \emph{neutral bits} to make the probabilistic phase of the attack more efficient.
The first improvement as originally used for \shazero was later superseded by the two-block attack structure of Wang \etal, used both for improved attacks on \shazero
and the first full theoretical attack on \shaone~\cite{DBLP:conf/crypto/WangYY05,DBLP:conf/crypto/WangYY05a},
and we will not describe it here. The second improvement,
originally introduced by Biham and Chen~\cite{DBLP:conf/crypto/BihamC04} was much more long-lived, and it is still useful in current attacks, including ours.

The aim of the neutral bits technique is to make a better use of the available freedom in the first sixteen message words, in order to speed up the attack. The idea is to identify bits
of messages that with good probability do not interact with any local collision necessary conditions, say up to step $n$. Thus, if one found a message pair fulfilling all conditions
up to step $n$ (a \emph{partial solution}), flipping a neutral bit will lead to another message pair similarly fulfilling all conditions up to $n$ with good probability. Finding neutral bits then allows to amortize
the cost of finding good message pairs, which makes the attack faster. To make things a bit more formal, we may give the following:

\begin{defi}[Neutral bits]
A bit $b$ of $\expmess_{0 \leq i < 16}$ is a \emph{neutral bit} for a disturbance vector $\mathcal{V}$ at step $n$ with probability $p$ if the probability (taken over the message space) that it does not interact
with any necessary condition for $\mathcal{V}$ up to step $n$ is equal to $p$. 
\end{defi}

This definition can be generalized by considering groups of bits that need to be flipped together. This may be done either because it makes these bits being more effective, or because not doing so would
change the sign of some bits of local collisions later in the expanded message.

In general, one can distinguish two (not mutually excluding) approaches to finding neutral bits: either running a random search across many partial solutions; or using the propagation properties of the step function to identify
good candidates. In particular, the latter approach found a powerful expression with the technique of auxiliary differential paths (or \emph{boomerangs}), first developped for \shaone~\cite{DBLP:conf/crypto/JouxP07}
and which led to the currently fastest attacks on \shazero~\cite{DBLP:conf/fse/ManuelP08}.

A boomerang in that context is a small set of neutral bits that are particularly efficient when activated together (\ie of the first kind mentioned above), in particular because they define a local collision (or possibly
a few of them) for which all necessary conditions have been pre-satisfied. Flipping the bits of the boomerang then only introduces a single local perturbation that is immediately corrected and thus does not propagate.
However, because the local collision defined by the neutral bits is not chained along a disturbance vector, uncorrected perturbations will eventually be introduced by the message expansion and result in interactions
with necessary conditions; this will however happen much later than with ``standard'' neutral bits. One can notice that because the local collisions of a boomerang do not need to be chained, they do not need to
be of a form suitable for every Boolean function. For instance, as it was hinted earlier in \autoref{sec:diffs_ana}, one can obtain better boomerangs by using the absorption properties of the IF function to define
local collisions with fewer corrections necessary.

%TODO give more details => a 3 corr Boomerang for IF

\noindent
We will discuss concrete neutral bits and boomerangs in \autoref{sec:res_76} and \autoref{sec:res_80}.

We conclude this short summary on \shazero by giving an illustration of the different phases of a one-block attack on \shazero in \autoref{fig:struct_shazero}. The two main rectangles represent the state and
expanded messages of a computation of \shazero. The diagonal lines (\,\begin{tikzpicture}[scale=0.25]\draw[thick,pattern=north west lines] (0,0) rectangle (1,1);\end{tikzpicture}\,,
\begin{tikzpicture}[scale=0.25]\draw[thick,pattern=north east lines] (0,0) rectangle (1,1);\end{tikzpicture}\,) represent areas ot the state and messages that are fixed for an attack; any state condition within this zone is deterministically fulfilled.
The non-diagonal lines in the message represent bits that are used to generate many message pairs in the hope that one leads to a collision; horizontal lines
(\,\begin{tikzpicture}[scale=0.25]\draw[thick,pattern=horizontal lines] (0,0) rectangle (1,1);\end{tikzpicture}\,) do so purely probabilistically, and vertical lines
(\,\begin{tikzpicture}[scale=0.25]\draw[thick,pattern=vertical lines] (0,0) rectangle (1,1);\end{tikzpicture}\,)
represent neutral bits which are useful to amortize the cost of finding partial solutions; the (completely determined) remainder of the expanded message is left blank.
Finally, stars
(\,\begin{tikzpicture}[scale=0.25]\draw[thick,pattern=fivepointed stars] (-1,-1) rectangle (0,0);\end{tikzpicture}\,)
represent conditions that need to be fufilled probabilistically; denser zones indicate a higher probability
of satisfying all conditions within. 

\begin{figure}[!htb]
\begin{center}
% defining the new dimensions
\newlength{\starsize}
\newlength{\starspread}
% declaring the keys in tikz
\tikzset{starsize/.code={\setlength{\starsize}{#1}},
         starspread/.code={\setlength{\starspread}{#1}}}
% setting the default values
\tikzset{starsize=1mm,
         starspread=3mm}
% declaring the pattern
\pgfdeclarepatternformonly[\starspread,\starsize]% variables
  {custom fivepointed stars}% name
  {\pgfpointorigin}% lower left corner
  {\pgfqpoint{\starspread}{\starspread}}% upper right corner
  {\pgfqpoint{\starspread}{\starspread}}% tilesize
  {% shape description
   \pgftransformshift{\pgfqpoint{\starsize}{\starsize}}
   \pgfpathmoveto{\pgfqpointpolar{18}{\starsize}}
   \pgfpathlineto{\pgfqpointpolar{162}{\starsize}}
   \pgfpathlineto{\pgfqpointpolar{306}{\starsize}}
   \pgfpathlineto{\pgfqpointpolar{90}{\starsize}}
   \pgfpathlineto{\pgfqpointpolar{234}{\starsize}}
   \pgfpathclose%
   \pgfusepath{fill}
  }



\begin{tikzpicture}[scale=0.22,>=latex']
%\begin{tikzpicture}[scale=0.17,>=latex']
    % teh boxez
    \draw[thick] (0,0) -- (20,0) -- (20,-34);
    \draw[thick,dashed] (20,-34) -- (15,-33) -- (10,-34) -- (5,-33) -- (0,-34);
    \draw[thick] (0,-34) -- (0,0);
    \draw[thick] (25,-4) -- (45,-4) -- (45,-34);
    \draw[thick,dashed] (45,-34) -- (40,-33) -- (35,-34) -- (30,-33) -- (25,-34);
    \draw[thick] (25,-34) -- (25,-4);
    % teh row indices for the state
    \node at (-2,0) {-4};
    \node at (-2,-4) {0};
    \node at (-2,-20) {16};
    \node at (-2,-26) {22};
    \node at (-2,-34) {30};
    % teh row indices for the message
    \node at (23,-4) {0};
    \node at (23,-19) {15};
    % the fillings
    \draw[pattern=crosshatch] (0,0) rectangle (20,-4); % IV
    \fill[pattern=north west lines] (0,-4) rectangle (20,-18); % base solution
    \fill[pattern=north west lines] (0,-18) -- (7,-18) -- (0,-22); % base solution
    \fill[pattern=north west lines] (13,-18) -- (20,-18) -- (20,-22); % base solution
	\draw (0,-22) -- (7,-18) -- (13,-18) -- (20,-22);
    \fill[pattern=custom fivepointed stars,starspread=1.4mm,starsize=0.5mm] (0,-22.2) -- (7,-18) -- (13,-18) -- (20,-22.2); % NB 1
	\fill[pattern=custom fivepointed stars,starspread=1.7mm,starsize=0.5mm] (0,-22.4) rectangle (20,-26); % NB 2
    \fill[pattern=custom fivepointed stars,starspread=2.1mm,starsize=0.5mm] (0,-26.1) rectangle (20,-30); % NB 3
    \fill[pattern=custom fivepointed stars,starspread=2.5mm,starsize=0.5mm] (0,-30) rectangle (20,-33); % NB 6
	\fill[pattern=north east lines] (25,-4) rectangle (45,-15); % message
	\fill[pattern=north east lines] (25,-15) --  (32,-15) -- (29,-19) -- (25,-19) ; % message
	\fill[pattern=north east lines] (38,-15) --  (45,-15) -- (45,-19) -- (41,-19) ; % message
	\draw (25,-19) -- (29,-19) --  (32,-15) -- (38,-15) -- (41,-19) -- (45,-19);
	\draw (25,-19) -- (41,-19);
	\fill[pattern=vertical lines] (32,-15) -- (38,-15) -- (41,-19);
	\fill[pattern=horizontal lines] (29,-19) -- (32,-15) -- (41,-19);

    % legends
    \draw[thick] (50,-8) rectangle (51.5,-9.5);
    \node[anchor=text] at (55,-9.5) {$\iv$};
    \draw[pattern=crosshatch] (50,-8) rectangle (51.5,-9.5);
    %
    \draw[thick] (50,-12) rectangle (51.5,-13.5);
    \node[anchor=text] at (55,-13.5) {Deterministic};
    \node[anchor=text] at (55,-15) {conditions};
    \draw[pattern=north west lines] (50,-12) rectangle (51.5,-13.5);
    %
    \draw[thick] (50,-16) rectangle (51.5,-17.5);
    \node[anchor=text] at (55,-17.5) {Probabilistic};
    \node[anchor=text] at (55,-19) {conditions};
    \draw[pattern=custom fivepointed stars,starspread=1.2mm,starsize=0.5mm] (50,-16) rectangle (51.5,-17.5);
    %
    \draw[thick] (50,-20) rectangle (51.5,-21.5);
    \node[anchor=text] at (55,-21.5) {Fixed};
    \node[anchor=text] at (55,-23) {message};
    \draw[pattern=north east lines] (50,-20) rectangle (51.5,-21.5);
    %
    \draw[thick] (50,-24) rectangle (51.5,-25.5);
    \node[anchor=text] at (55,-25.5) {Randomized};
    \node[anchor=text] at (55,-27) {message};
    \draw[pattern=horizontal lines] (50,-24) rectangle (51.5,-25.5);
    %
    \draw[thick] (50,-28) rectangle (51.5,-29.5);
    \node[anchor=text] at (55,-29.5) {Neutral bits};
    \draw[pattern=vertical lines] (50,-28) rectangle (51.5,-29.5);

    %title
    \node[anchor=text] at (0,3) {\shazero State words ($\state_i$)};
    \node[anchor=text] at (25,3) {\shazero Message words ($\expmess_i$)};

\end{tikzpicture}

\end{center}
\caption{The structure of a one-block attack on \shazero.\label{fig:struct_shazero}}
\end{figure}
 
\subsection{Collision attacks on the full \shaone}
\label{sec:full_sha1}

Having discussed the original attacks on \shazero, we now turn our attention to \shaone for the rest of this article. Although both functions are very similar, their different message expansion makes a direct
application of attacks on \shazero in the form discussed above unapplicable to the full \shaone. We now discuss the new techniques that were developped for that purpose and some of their
subsequent improvements:
\begin{enumerate}
\item The use of \emph{non-linear} differential paths and of a two-block attack structure is the main improvement that makes full attacks on \shaone possible (\autoref{sec:nl_struct}).
\item The different message expansion of \shaone makes the search of good disturbance vectors more complex (\autoref{sec:sha1_dvs}).
\item Better cost function were defined for the disturbance vectors (\autoref{sec:chain_lc}). 
\end{enumerate}
It is important to note that although these improvements were somehow necessary to attack \shaone and led to the first theoretical attack on this function~\cite{DBLP:conf/crypto/WangYY05a},
the same technique can, and were, applied to improve the existing attacks on \shazero as well~\cite{DBLP:conf/crypto/WangYY05}, and also other functions of the \mdsha family.

Let us also mention for completeness that the original attacks of Wang \etal used an accelerating technique named \emph{message modification} instead of neutral bits. Such a technique works
by identifying changes in message words controlled by the attacker that only impact (say) a single necessary condition for a local collision located at a point where no control is directly possible.
This can be used during the probabilistic phase of the attack to carefully fulfill some of the conditions independently of each other, which increases the probability of success. However, it is not
clear whether this technique has a significant advantage over neutral bits (especially when including boomerangs), and it is somehow harder to implement efficiently, especially on a parallel architecture. Thus, we will not detail it
further in this article. 

\subsubsection{The two-block structure and non-linear paths}
\label{sec:nl_struct}
% incl. automated search
With the benefit of hindsight, looking back on the case of \shazero, we can identify two points in particular where the original attack does not seem to be optimal:
\begin{enumerate}
\item The original structure with a single block imposes the \dv to be zero in its last five bits, which may disqualify some otherwise good
\dvs. A way to get around this restriction is to use ``multi-block'' attacks (which culminated in using only two blocks).
\item Using a purely linearized model for the propagation of differences in the round function similarly imposes the condition that the first
five negative bits of the \dv have to be zero, and that there are no consecutive bits during the IF round. A way to remove these conditions
is to switch to a non-linear propagation model for (part of) the first round.
\end{enumerate} 
We now discuss these two improvements in more detail.

\paragraph{The two-block structure.}
In a multi-block attack, one uses \dvs which only result in \emph{near collisions},
\ie final states still containing a few differences, only in controlled locations. The idea is then to chain such \dvs together in a way
that eventually produces a collision, by cancelling in the last block the few differences in the \iv (coming from a near collision in the previous block
through the Merkle-Damg\aa rd domain extension) with appropriate differences in the final state, thanks to the feed-forward of the
Davies-Meyer construction.

This was first done by Biham \etal to obtain the first explicit collision on \shazero, using four blocks of different
\dvs~\cite{DBLP:conf/eurocrypt/BihamCJCLJ05},
and it was improved by Wang \etal in a more systematic fashion using only two blocks with the same \dv, with the second block using a ``negated'' version
of the \dv of the first block (\ie with the differences being changed to their opposite, \eg \onediffu becomes \onediffd, \nodiffo becomes \nodiffz).
This results in a structure with a first near-collision that ends with a difference $+\Delta$, followed by a second block with a final state
of negated difference $-\Delta$. (In fact,
following this exact structure is not strictly necessary, as there are a few admissible differences $\{+\widetilde{\Delta}\}$ for the end of the first block
that can all lead to a collision in the second block.)
%TODO expand??

The main reason why this structure may be used is that a non-linear propagation model is used in the beginning of the attack. Such a model in itself also allows to use
better \dvs.

\paragraph{Non-linear model for the propagation of local collisions.}
We have mentioned some limitations of using a purely linear model to establish the differential path in the state $\diff\state$. In a non-linear model, we do not
try to systematically avoid differences in the carry propagation of $\state$ and $\dstate$, which also implies that not every local collision will be systematically
corrected.

Two advantages of this approach over a purely linear model were already mentioned:
\begin{enumerate}
\item There is no need to ensure the absence of local collisions in the beginning of the negative message anymore.
\item One can keep successive local collisions in the IF round (this point was already partially resolved by Biham \etal for \shazero~\cite{DBLP:conf/eurocrypt/BihamCJCLJ05}).
\end{enumerate}

This allows to select better disturbance vectors than what would otherwise be possible, provided that one is able to find a suitable state difference. It is however not
essential (at least for a basic attack) for the resulting path to be of high probability, as it will be located in (part of) the first round only, where one has entire
control of the message. This amount of freedom is enough to find many solutions to the non-linear part, even when keeping in mind that these need to leave some bits unspecified
so that enough message candidates can later be generated to obtain a collision.

The other main advantage of using a non-linear path is that, as mentioned above, it is the chief reason why an efficient two-block structure can be used for an attack. Indeed,
in the same way as it removes the conditions in the early message differences, it allows to disconnect the presence (for the second block) or absence (for the first)
of incoming \iv differences with the choice of the remaining linear part of the differential path. Thus, the same \dv can be used for two blocks, and choosing opposite signs for the two
is enough to lead to a collision. This new ability of using only one disturbance vector is in particular a consequence of the fact that for a fixed \dv (and even for the same \iv differences), many non-linear paths are possible,
whereas a path following a linear behaviour is unique.

We show the simplified structure of a two-block attack using non-linear paths in \autoref{fig:two_blocks}. The first block (on the left) takes a zero (\textswab{0}) \iv difference,
a message difference $\diff\expmess$, and starts with a non-linear differential path \textswab{NL~1} for which it is easy to find solutions in a deterministic way
(\,\begin{tikzpicture}[scale=0.25]\draw[thick,fill=green!20] (0,0) rectangle (1,1);\end{tikzpicture}\,); this is
followed by a linear path \textswab{L} which is satisfied probabilistically, first using accelerating techniques such as neutral bits (\,\begin{tikzpicture}[scale=0.25]\draw[thick,fill=blue!20] (0,0) rectangle (1,1);\end{tikzpicture}\,),
and then purely randomly (\,\begin{tikzpicture}[scale=0.25]\draw[thick,fill=red!20] (0,0) rectangle (1,1);\end{tikzpicture}\,). This results in a state difference $\diff\state$ which is passed to a second block. This block
uses a different non-linear path \textswab{NL~2} to connect to the negated linear path \textswab{-L} that is obtained by using an oppositely signed message difference $-\diff\expmess$; following this second path leads to
a collision.

\begin{figure}[!htb]
\begin{centering}
\def\mystrutNL{\vrule height 1.5em depth 1em width 0pt}%
\def\mystrutLin{\vrule height 2.5em depth 2em width 0pt}%

\begin{tikzpicture}[
	scale=1.2,
	transform shape,
	node distance=3em and 0em,
]
    \tikzstyle{F} = [
    		draw,
		rectangle split,
		rectangle split horizontal=false, 
		rectangle split parts=3,
		rectangle split draw splits=false, 
		rectangle split part fill={green!20, blue!20, red!20},
		text centered,
		text width=4em,
		minimum width=4em,
		minimum height=12em,
	]
	
	%%
	%% First characteristic
	%%
	\node[F] (c1) {\mystrutNL \textswab{NL 1} \nodepart{two} \nodepart{three}\mystrutLin {\Large\textswab{L}}};
	\draw[dashed] (c1.text split west) -- (c1.text split east);
    \draw[dashed] (c1.two split west) -- (c1.two split east);
	\node[MODADD,scale=1.5,below = 2em of c1] (add1) {};
	\node[above = of c1.north, xshift=-1em] (in11) {};
	\node[above = of c1.north, xshift=1em] (in12) {$\diff\expmess$};

	\draw[line] let \p1=(in11.south),\p2=(c1.north) in (\x1,\y1) -- (\x1,\y2);
	\draw[line] let \p1=(in12.south),\p2=(c1.north) in (\x1,\y1) -- (\x1,\y2);
	\draw[line] (c1) -- node[right] {$\diff\state$} (add1);

	\draw[line] let \p1=(in11),\p2=(c1.north),\p3=(add1) in 
		(\x1,\y2+1.5em) 
		-- node[above left]{$\text{\textswab{0}}$} (\x3-3em,\y2+1.5em) 
		-- (\x3-3em,\y3) 
		-- node[below] {} (add1);
	
	%%
	%% Second characteristic
	%%
	\node[F, right = 8em of c1] (c2) {\mystrutNL \textswab{NL 2}\nodepart{two} \nodepart{three} \mystrutLin {\Large\textswab{-L}}};
	\draw[dashed] (c2.text split west) -- (c2.text split east);
    \draw[dashed] (c2.two split west) -- (c2.two split east);
	\node[MODADD,scale=1.5,below = 2em of c2] (add2) {};
	\node[below = of add2] (out) {};
	\node[above = of c2.north, xshift=-1em] (in21) {};
	\node[above = of c2.north, xshift=1em] (in22) {$-\diff\expmess$};

	\draw[line] (c2) -- node[right] {$-\diff\state$} (add2);
	\draw[line] let \p1=(in22.south),\p2=(c2.north) in (\x1,\y1) -- (\x1,\y2);
	\draw[line] (add2) -- (out) node[below] {\textswab{0}};

	\coordinate (M12) at ($(add1)!0.5!(add2)$);

	\draw[line] let \p1=(c2.north),\p2=(add1),\p3=(in21),\p4=(M12) in 
		(\x2,\y2) 
		-- (\x2,\y2-2em) 
		-- (\x4-1.5em,\y4-2em)
		-- (\x4-1.5em,\y3)
		-- (\x3,\y3)
		-- (\x3,\y1);

	\draw[line] let \p1=(in21),\p2=(c2.north),\p3=(add2) in 
		(\x1,\y2+1.5em) 
		-- node[above left]{$\diff\state$} (\x3-3em,\y2+1.5em) 
		-- (\x3-3em,\y3) 
		-- node[below] {$\diff\state$} (add2);

\end{tikzpicture}

\caption{The structure of a two-block collision attack for \sha.\label{fig:two_blocks}}
\end{centering}
\end{figure}

\medskip

The main difficulty in using non-linear paths is to find the paths themselves, as there is a large number of possible behaviours to take into account.
The search was initially done by hand for the first attack on \shaone, but automated tools were later developped to make this process much
more efficient. One can for instance cite the work of De~Cannière and Rechberger~\cite{DBLP:conf/asiacrypt/CanniereR06}, who used a guess-and-determine
approach to find new non-linear paths, leading in particular to an explicit two-block collision for \shaone reduced to the at-the-time record number of 64 steps.

The idea of the guess-and-determine method is to define a set of \emph{constraints} that encode a differential path, along with an efficient constraint-propagation
algorithm. One then starts from an underdefined initial path with many unconstraint differences (but with the conditions for the \iv and for the connecting
linear path already set) and iteratively chooses an unconstraint difference at random, assigns it a value, and propagates the consequences of this choice.
A backtracking strategy is also used to escape situations where no more valid choices are possible. A path is found when every constraint is either a signed
difference or a (possibly signed) equality.

One can also mention the alternative ``meet-in-the-middle'' approach for the construction of these paths, which was used by Yajima \etal~\cite{DBLP:conf/acisp/YajimaSNISKO07},
and later Stevens~\cite{phdstevens}. This method works by defining two partial differential paths, one expanded forward (\eg starting from the \iv) and one expanded backward
(\eg starting from the purely linear part of the path), that are then connected on a few consecutive steps.

The advantages of an automatic search of the non-linear part of the differential path over a manual one are twofold: (1) It is much faster to create new
attack instances, which allows for instance to experiment quickly with several \dvs. This is particularly useful if one wants both to mount
attacks for the full function (even without running the attack completely) and attacks for a high number of reduced steps, the best \dvs in each case
being likely different. (2) The ability to generate many non-linear paths allows to search for ones that have few constraints (for instance leading
to more available neutral bits) or that can incorporate more preset constraints (that may for instance aid in the use of boomerang neutral bits). 

% advantage is twofold: faster to mount new attacks, to experiment with several dvs; may lead to better paths or paths satisfying predefined constraints

% Imposes L behaviour; (no zero in five back, consecutive IFs (although also partly resolved at EUROCRYPT))
%NL A major idea of Wang \etal in their attack
%allows efficient instantiations of two blocks)
% and moar dvs too (can be dense in beginning of first round)
% also message mod, but bof, NB's fine too


\subsubsection{Classes of disturbance vectors for \shaone}
\label{sec:sha1_dvs}

We recall from \autoref{sec:dv_sha0} that the search for disturbance vectors for \shazero naturally reduces to a search among only $2^{16}$ candidates. Once a cost function
is chosen, it is simple to evaluate it on every potential \dv and one just needs to keep the best of them. Unfortunately, the message expansion of \shaone can no longer be seen as
thirty-two smaller independent message expansions, making the search space for potential \dvs significantly larger.

In the original attack from Wang \etal, the \dv was found when searching through a reduced space of size $2^{38}$, using the Hamming weight of the resulting \dvs
as the primary cost function.
Subsequently, a significant amount of work focused on finding alternate disturbance vectors in the hope of decreasing the cost of the
probabilistic phase of the attack. Manuel then noticed that all \dvs suggested in the literature could actually be concisely described by two simple classes which
lead to the best known vectors~\cite{DBLP:journals/dcc/Manuel11}; we now summarize these results.

\medskip

We start by defining an \emph{extended expanded message} for \shaone as follows:
\begin{defi}[Extended expanded message]
\label{def:eem}
Let  $\mess$  be a \shaone message block made of sixteen 32-bit message words $\mess_0, \ldots, \mess_{15}$. The \emph{extended expanded message} $\eem$ for $\mess$
is made of 144 32-bit words $\eem_{-64}, \ldots, \eem_{79}$ defined by:
\begin{equation}
\label{eq:ext_exp_mess}
\eem_i=
\left\{
\begin{array}{ll}
\mess_i, & \textnormal{ for } 0\leq i\leq 15 \\
(\eem_{i-3} \oplus \eem_{i-8} \oplus \eem_{i-14} \oplus \eem_{i-16}) \circlearrowleft 1, & \textnormal{ for } 16\leq i\leq 79\\
\eem_{i} = (\eem_{i+16} \circlearrowright 1) \oplus \eem_{i+13} \oplus \eem_{i+8} \oplus \eem_{i+2}, & \textnormal{for} -64\leq i\leq -1
\end{array}.
\right.
\end{equation}
\end{defi}
In other words, an extended expanded message expands an initial message both forwards (using \autoref{eq:exp_mess}) by 64 words, but also
backwards (using \autoref{eq:exp_mess_inv}) by a similar amount. By definition, every consecutive 80 words $\eem_i, \ldots, \eem_{i+79}$, $i \in [-64, \ldots, 0]$
of $\eem$ form a valid expanded message ``$(\eem_i)$'' for \shaone. Furthermore, it is easy to check that these 65 expanded messages are exactly the 65 possible such messages
for which sixteen consecutive words are equal to $\mess$.

We also note the following fact:
\begin{fact}[The message expansion of \shaone is a quasi-cyclic code]
If $\expmess = \expmess_0, \ldots, \expmess_{79}$ is a valid expanded message for \shaone, then for every $i \in [0, 31]$, $\expmess^{\circlearrowleft i} =
\expmess_0^{\circlearrowleft i}, \ldots, \expmess_{79}^{\circlearrowleft i}$ is a valid expanded message for \shaone.
\end{fact}
This fact, together with the notion of extended expanded message allows to define equivalence classes for expanded messages:
\begin{defi}[Equivalence class for \shaone expanded messages~\cite{DBLP:journals/dcc/Manuel11}]
Two \shaone expanded messages $\expmess$ and $\expmess'$ are equivalent if there are two pairs $(i,j)$, $(i',j')$ in $[-64, 0] \times [0, 31]$
and an extended expanded message $\eem$ such that $\expmess = (\eem_i)^{\circlearrowleft j}$ and $\expmess' = (\eem_{i'})^{\circlearrowleft j'}$.
\end{defi}
In other words, two expanded messages are equivalent if they can be generated from the same, possibly rotated, extended expanded message. It should be noted
however that a message necessarily belongs to many distinct such equivalence classes.

As the disturbance vectors are expanded messages themselves, one can then use equivalence classes as a natural way to segment the search space for good \dvs. For instance,
Manuel searched for candidates among all classes defined by extended expanded messages associated with messages of Hamming weight up to four, and for some classes defined by
messages of weight five and six. It followed from this search that all good \dvs, including all the ones described in previous work, come from two equivalence classes.
The 16-word messages generating the extended expanded messages of the two classes (up to rotation) are shown in \autoref{fig:dv_types}\footnote{The messages in this
figure are given using an unsigned difference notation. In Definition~\ref{def:eem}, we gave the definition using a ``normal'' message $\in \{\{0,1\}^{32}\}^{16}$. As
a disturbance vector is eventually used to define the difference between two messages, we think that using such a notation is appropriate in this case.}.

\begin{figure}[!ht]
\begin{center}
\begin{tabular}{cc}
\nodiff \nodiff \nodiff \nodiff \nodiff \nodiff \nodiff \nodiff \nodiff \nodiff \nodiff \nodiff \nodiff \nodiff
\nodiff \nodiff \nodiff \nodiff \nodiff \nodiff \nodiff \nodiff \nodiff \nodiff \nodiff \nodiff \nodiff \nodiff \nodiff \nodiff \nodiff \nodiff& 
\nodiff \nodiff \nodiff \nodiff \nodiff \nodiff \nodiff \nodiff \nodiff \nodiff \nodiff \nodiff \nodiff \nodiff
\nodiff \nodiff \nodiff \nodiff \nodiff \nodiff \nodiff \nodiff \nodiff \nodiff \nodiff \nodiff \nodiff \nodiff \nodiff \nodiff \nodiff \nodiff \\
\nodiff \nodiff \nodiff \nodiff \nodiff \nodiff \nodiff \nodiff \nodiff \nodiff \nodiff \nodiff \nodiff \nodiff
\nodiff \nodiff \nodiff \nodiff \nodiff \nodiff \nodiff \nodiff \nodiff \nodiff \nodiff \nodiff \nodiff \nodiff \nodiff \nodiff \nodiff \nodiff& 
\onediff \nodiff \nodiff \nodiff \nodiff \nodiff \nodiff \nodiff \nodiff \nodiff \nodiff \nodiff \nodiff \nodiff
\nodiff \nodiff \nodiff \nodiff \nodiff \nodiff \nodiff \nodiff \nodiff \nodiff \nodiff \nodiff \nodiff \nodiff \nodiff \nodiff \nodiff \nodiff \\
\nodiff \nodiff \nodiff \nodiff \nodiff \nodiff \nodiff \nodiff \nodiff \nodiff \nodiff \nodiff \nodiff \nodiff
\nodiff \nodiff \nodiff \nodiff \nodiff \nodiff \nodiff \nodiff \nodiff \nodiff \nodiff \nodiff \nodiff \nodiff \nodiff \nodiff \nodiff \nodiff& 
\nodiff \nodiff \nodiff \nodiff \nodiff \nodiff \nodiff \nodiff \nodiff \nodiff \nodiff \nodiff \nodiff \nodiff
\nodiff \nodiff \nodiff \nodiff \nodiff \nodiff \nodiff \nodiff \nodiff \nodiff \nodiff \nodiff \nodiff \nodiff \nodiff \nodiff \nodiff \nodiff \\
\nodiff \nodiff \nodiff \nodiff \nodiff \nodiff \nodiff \nodiff \nodiff \nodiff \nodiff \nodiff \nodiff \nodiff
\nodiff \nodiff \nodiff \nodiff \nodiff \nodiff \nodiff \nodiff \nodiff \nodiff \nodiff \nodiff \nodiff \nodiff \nodiff \nodiff \nodiff \nodiff& 
\onediff \nodiff \nodiff \nodiff \nodiff \nodiff \nodiff \nodiff \nodiff \nodiff \nodiff \nodiff \nodiff \nodiff
\nodiff \nodiff \nodiff \nodiff \nodiff \nodiff \nodiff \nodiff \nodiff \nodiff \nodiff \nodiff \nodiff \nodiff \nodiff \nodiff \nodiff \nodiff \\
\nodiff \nodiff \nodiff \nodiff \nodiff \nodiff \nodiff \nodiff \nodiff \nodiff \nodiff \nodiff \nodiff \nodiff
\nodiff \nodiff \nodiff \nodiff \nodiff \nodiff \nodiff \nodiff \nodiff \nodiff \nodiff \nodiff \nodiff \nodiff \nodiff \nodiff \nodiff \nodiff& 
\nodiff \nodiff \nodiff \nodiff \nodiff \nodiff \nodiff \nodiff \nodiff \nodiff \nodiff \nodiff \nodiff \nodiff
\nodiff \nodiff \nodiff \nodiff \nodiff \nodiff \nodiff \nodiff \nodiff \nodiff \nodiff \nodiff \nodiff \nodiff \nodiff \nodiff \nodiff \nodiff \\
\nodiff \nodiff \nodiff \nodiff \nodiff \nodiff \nodiff \nodiff \nodiff \nodiff \nodiff \nodiff \nodiff \nodiff
\nodiff \nodiff \nodiff \nodiff \nodiff \nodiff \nodiff \nodiff \nodiff \nodiff \nodiff \nodiff \nodiff \nodiff \nodiff \nodiff \nodiff \nodiff& 
\nodiff \nodiff \nodiff \nodiff \nodiff \nodiff \nodiff \nodiff \nodiff \nodiff \nodiff \nodiff \nodiff \nodiff
\nodiff \nodiff \nodiff \nodiff \nodiff \nodiff \nodiff \nodiff \nodiff \nodiff \nodiff \nodiff \nodiff \nodiff \nodiff \nodiff \nodiff \nodiff \\
\nodiff \nodiff \nodiff \nodiff \nodiff \nodiff \nodiff \nodiff \nodiff \nodiff \nodiff \nodiff \nodiff \nodiff
\nodiff \nodiff \nodiff \nodiff \nodiff \nodiff \nodiff \nodiff \nodiff \nodiff \nodiff \nodiff \nodiff \nodiff \nodiff \nodiff \nodiff \nodiff& 
\nodiff \nodiff \nodiff \nodiff \nodiff \nodiff \nodiff \nodiff \nodiff \nodiff \nodiff \nodiff \nodiff \nodiff
\nodiff \nodiff \nodiff \nodiff \nodiff \nodiff \nodiff \nodiff \nodiff \nodiff \nodiff \nodiff \nodiff \nodiff \nodiff \nodiff \nodiff \nodiff \\
\nodiff \nodiff \nodiff \nodiff \nodiff \nodiff \nodiff \nodiff \nodiff \nodiff \nodiff \nodiff \nodiff \nodiff
\nodiff \nodiff \nodiff \nodiff \nodiff \nodiff \nodiff \nodiff \nodiff \nodiff \nodiff \nodiff \nodiff \nodiff \nodiff \nodiff \nodiff \nodiff& 
\nodiff \nodiff \nodiff \nodiff \nodiff \nodiff \nodiff \nodiff \nodiff \nodiff \nodiff \nodiff \nodiff \nodiff
\nodiff \nodiff \nodiff \nodiff \nodiff \nodiff \nodiff \nodiff \nodiff \nodiff \nodiff \nodiff \nodiff \nodiff \nodiff \nodiff \nodiff \nodiff \\
\nodiff \nodiff \nodiff \nodiff \nodiff \nodiff \nodiff \nodiff \nodiff \nodiff \nodiff \nodiff \nodiff \nodiff
\nodiff \nodiff \nodiff \nodiff \nodiff \nodiff \nodiff \nodiff \nodiff \nodiff \nodiff \nodiff \nodiff \nodiff \nodiff \nodiff \nodiff \nodiff& 
\nodiff \nodiff \nodiff \nodiff \nodiff \nodiff \nodiff \nodiff \nodiff \nodiff \nodiff \nodiff \nodiff \nodiff
\nodiff \nodiff \nodiff \nodiff \nodiff \nodiff \nodiff \nodiff \nodiff \nodiff \nodiff \nodiff \nodiff \nodiff \nodiff \nodiff \nodiff \nodiff \\
\nodiff \nodiff \nodiff \nodiff \nodiff \nodiff \nodiff \nodiff \nodiff \nodiff \nodiff \nodiff \nodiff \nodiff
\nodiff \nodiff \nodiff \nodiff \nodiff \nodiff \nodiff \nodiff \nodiff \nodiff \nodiff \nodiff \nodiff \nodiff \nodiff \nodiff \nodiff \nodiff& 
\nodiff \nodiff \nodiff \nodiff \nodiff \nodiff \nodiff \nodiff \nodiff \nodiff \nodiff \nodiff \nodiff \nodiff
\nodiff \nodiff \nodiff \nodiff \nodiff \nodiff \nodiff \nodiff \nodiff \nodiff \nodiff \nodiff \nodiff \nodiff \nodiff \nodiff \nodiff \nodiff \\
\nodiff \nodiff \nodiff \nodiff \nodiff \nodiff \nodiff \nodiff \nodiff \nodiff \nodiff \nodiff \nodiff \nodiff
\nodiff \nodiff \nodiff \nodiff \nodiff \nodiff \nodiff \nodiff \nodiff \nodiff \nodiff \nodiff \nodiff \nodiff \nodiff \nodiff \nodiff \nodiff& 
\nodiff \nodiff \nodiff \nodiff \nodiff \nodiff \nodiff \nodiff \nodiff \nodiff \nodiff \nodiff \nodiff \nodiff
\nodiff \nodiff \nodiff \nodiff \nodiff \nodiff \nodiff \nodiff \nodiff \nodiff \nodiff \nodiff \nodiff \nodiff \nodiff \nodiff \nodiff \nodiff \\
\nodiff \nodiff \nodiff \nodiff \nodiff \nodiff \nodiff \nodiff \nodiff \nodiff \nodiff \nodiff \nodiff \nodiff
\nodiff \nodiff \nodiff \nodiff \nodiff \nodiff \nodiff \nodiff \nodiff \nodiff \nodiff \nodiff \nodiff \nodiff \nodiff \nodiff \nodiff \nodiff& 
\nodiff \nodiff \nodiff \nodiff \nodiff \nodiff \nodiff \nodiff \nodiff \nodiff \nodiff \nodiff \nodiff \nodiff
\nodiff \nodiff \nodiff \nodiff \nodiff \nodiff \nodiff \nodiff \nodiff \nodiff \nodiff \nodiff \nodiff \nodiff \nodiff \nodiff \nodiff \nodiff \\
\nodiff \nodiff \nodiff \nodiff \nodiff \nodiff \nodiff \nodiff \nodiff \nodiff \nodiff \nodiff \nodiff \nodiff
\nodiff \nodiff \nodiff \nodiff \nodiff \nodiff \nodiff \nodiff \nodiff \nodiff \nodiff \nodiff \nodiff \nodiff \nodiff \nodiff \nodiff \nodiff& 
\nodiff \nodiff \nodiff \nodiff \nodiff \nodiff \nodiff \nodiff \nodiff \nodiff \nodiff \nodiff \nodiff \nodiff
\nodiff \nodiff \nodiff \nodiff \nodiff \nodiff \nodiff \nodiff \nodiff \nodiff \nodiff \nodiff \nodiff \nodiff \nodiff \nodiff \nodiff \nodiff \\
\nodiff \nodiff \nodiff \nodiff \nodiff \nodiff \nodiff \nodiff \nodiff \nodiff \nodiff \nodiff \nodiff \nodiff
\nodiff \nodiff \nodiff \nodiff \nodiff \nodiff \nodiff \nodiff \nodiff \nodiff \nodiff \nodiff \nodiff \nodiff \nodiff \nodiff \nodiff \nodiff& 
\nodiff \nodiff \nodiff \nodiff \nodiff \nodiff \nodiff \nodiff \nodiff \nodiff \nodiff \nodiff \nodiff \nodiff
\nodiff \nodiff \nodiff \nodiff \nodiff \nodiff \nodiff \nodiff \nodiff \nodiff \nodiff \nodiff \nodiff \nodiff \nodiff \nodiff \nodiff \nodiff \\
\nodiff \nodiff \nodiff \nodiff \nodiff \nodiff \nodiff \nodiff \nodiff \nodiff \nodiff \nodiff \nodiff \nodiff
\nodiff \nodiff \nodiff \nodiff \nodiff \nodiff \nodiff \nodiff \nodiff \nodiff \nodiff \nodiff \nodiff \nodiff \nodiff \nodiff \nodiff \nodiff& 
\nodiff \nodiff \nodiff \nodiff \nodiff \nodiff \nodiff \nodiff \nodiff \nodiff \nodiff \nodiff \nodiff \nodiff
\nodiff \nodiff \nodiff \nodiff \nodiff \nodiff \nodiff \nodiff \nodiff \nodiff \nodiff \nodiff \nodiff \nodiff \nodiff \nodiff \nodiff \nodiff \\
\nodiff \nodiff \nodiff \nodiff \nodiff \nodiff \nodiff \nodiff \nodiff \nodiff \nodiff \nodiff \nodiff \nodiff
\nodiff \nodiff \nodiff \nodiff \nodiff \nodiff \nodiff \nodiff \nodiff \nodiff \nodiff \nodiff \nodiff \nodiff \nodiff \nodiff \nodiff \onediff& 
\nodiff \nodiff \nodiff \nodiff \nodiff \nodiff \nodiff \nodiff \nodiff \nodiff \nodiff \nodiff \nodiff \nodiff
\nodiff \nodiff \nodiff \nodiff \nodiff \nodiff \nodiff \nodiff \nodiff \nodiff \nodiff \nodiff \nodiff \nodiff \nodiff \nodiff \nodiff \onediff \\
\end{tabular}
\end{center}
\caption{The messages defining the class of type I (left) and type II (right) disturbance vectors, given as sixteen 32-bit words $\mess_0, \ldots, \mess_{15}$,
with $\mess_0$ on top.\label{fig:dv_types}}
\end{figure}

\noindent
Following this observation, Manuel termed I$(i,j)$ and II$(i,j)$ the disturbance vectors $(\eem_{-i})^{\circlearrowleft j}$ where
$\eem$ is generated from the messages of type I and II of \autoref{fig:dv_types} respectively.

\subsubsection{Exact cost functions for disturbance vectors}
%\subsubsection{Exact probability of chains of local collisions}
\label{sec:chain_lc}

We have already mentioned the role played by the cost functions when choosing a disturbance vector, both in the case of \shazero in \autoref{sec:dv_sha0} and in the case
of \shaone in the previous \autoref{sec:sha1_dvs}. A first basic such function is simply to take the Hamming weight of a vector (\ie to count the number of local collisions)
over the steps where we expect the attack to be purely probabilistic (\eg starting from step 20). Even in the case of \shazero, we have seen that some additional interactions between
local collisions need to be taken into account to make this function more accurate. The same sort of interactions is also present in the case of \shaone, and some new ones may appear as well, especially
because several local collisions may now be started at the same step.

\medskip

\paragraph{Bit compression}
$\phantom{bouh}$

\medskip

\noindent
A first new kind of interaction between local collisions that is favourable to the cryptanalyst is the effect used in the technique of \emph{bit compression} introduced by Wang \etal~\cite{DBLP:conf/crypto/WangYY05a}
as a \emph{special counting rule} and later named as such by Yajima \etal~\cite{DBLP:conf/ccs/YajimaINSSKO08}. Under certain conditions, this technique allows to significantly
improve the joint probability of two (or more) neighbouring local collisions being successful by making it as high as for a single one. In a nutshell, the idea is to introduce the initial perturbations
using a chain of differences all
having the same sign, except for the last one, and to let the carry propagate from the first
perturbation to the other ones instead of preventing such a propagation everytime.
Let us see how this may work on an example with three neighbouring differences.

\begin{example}[Compression of three differences]
\label{ex:bit_comp}
Consider a chain of differences of the form $\ldots\mnodiff\monediffu\monediffd\monediffd\mnodiff\ldots$ added to a state with no difference $\ldots\mnodiff\mnodiff\mnodiff\mnodiff\mnodiff\ldots$.
It is most useful in this case to first consider the differences as modular ones. If we call $x,\tilde{x}$ and $y,\tilde{y}$ the two states that are added together, we
have $\tilde{x} = x - 2^{i} - 2^{i+1} + 2^{i+2}$ for some value $i$. Now we want to determine what are the probabilities of some of the possible resulting XOR differences between $x + y$ and $\tilde{x} + \tilde{y}$.

Using a traditional view and treating all differences separately (which is what would happen if these were perturbations of local collisions seen independently), we would like to have no
carry propagation at any of the three position to obtain an XOR difference on the same three positions where differences were introduced. It is easy to see that this imposes three conditions
on $x + y$, as it means that we want $\tilde{x} + \tilde{y} = (x + y) \oplus 2^{i} \oplus 2^{i+1} \oplus 2^{i+2}$, which translates to bit $i$, $i+1$ and $i+2$ of $x+y$ being 1, 1 and 0 respectively.
The probability of this happening is thus $2^{-3}$.

However, as we have $2^{i+2} - 2^{i+1} - 2^{i} = 2^{i}$, the alternative XOR difference $\tilde{x} + \tilde{y} = (x + y) \oplus 2^{i}$ only imposes one condition on $x + y$, namely that bit $i$
must be 0; this difference may then happen with probability $2^{-1}$.
\end{example}

We can use the effect showed in the above example to increase the probability of a successful introduction of perturbations in series of local collisions; this is however at the condition that
the ``compressed'' XOR difference obtained as a result is compatible with the following corrections, which are still located on multiple bits. Fortunately, this condition is always fulfilled
provided: (1) That the single difference is not absorbed or flipped in a Boolean function (this is the usual condtion for a successful correction); (2) That the chain of consecutive differences
is not broken through the bit rotations (this is a ``hard'' condition that determines if a series of neighbouring local collisions can indeed be compressed). We illustrate this by continuing
our previous example.

\begin{example}[Correction of compressed differences]
\label{ex:bit_comp2}
Let us use modular differences again. Consider the case of Example~\ref{ex:bit_comp} and assume that the introduction of the perturbation resulted in the modular and XOR difference
$\tilde{x} + \tilde{y} = (x + y) + 2^{i}  = (x + y) \oplus 2^{i}$. Assume that this difference is preserved through the step function, excluding the addition of the message,
resulting in a partial state $z, \tilde{z}$ with $\tilde{z} = z + 2^{i}$. This partial state is rotated to the left by $r \in \{0, 5, 30\}$ in the computation of the new state,
and the differences in the message at this point are at positions $\alpha = i + r\mod32$, $\beta = i+1+r \mod 32$, $\gamma = i+2+r \mod 32$ and of sign opposite the ones of the initial perturbation,
\ie we have $m, \tilde{m}$ with $\tilde{m} = m + 2^{\alpha} + 2^{\beta} - 2^{\gamma}$. If $\alpha < \beta < \gamma$, we then have $\tilde{m} = m - 2^{\alpha}$. In that case,
$(\tilde{z} \circlearrowleft r) + \tilde{m}$ = $(z \circlearrowleft r) + 2^{\alpha} + m - 2^{\alpha}$ = $(z \circlearrowleft r) + m$, and the correction is indeed successful.
\end{example}

It is worth noting that because only a single state difference needs to be preserved through the Boolean functions during the correction, the probability of all corrections of compressed
local collisions being successful is also much higher than compared to the uncompressed case.

To summarize, local collisions starting at the same step and at neighbouring bit positions that remain consecutive through left rotations by 5 and 30 (\ie the considered bit positions
do not include bit 1 or 26 and bits to their left) can be compressed by choosing a proper signing for the initial perturbation. The probability of the resulting compressed collision
to be successful is the same as the success probability of a single local collision.

Finally, let us note that compressing local collisions does not actually hinder in any way their
chance of being successful in the ``traditional'' (independent) way. For instance, two local collisions in no particular position during an XOR round have a ``theoretical'' joint success probability of
$2^{-8}$ when considered independent and of $2^{-4}$ if they are compressed (as per \autoref{sec:diffs_ana}). The two successful events being independent, the total theoretical success probability of these collisions is thence $2^{-3.91}$.
It may thus appear that compressed local collisions actually have a \emph{higher} probability than single ones. This is actually not the case, as we explain next.

\paragraph{Bit decompression}
$\phantom{bouh}$

\medskip

\noindent
We now use the insight gained from the analysis of the bit compression technique to come closer to computing the exact success probability of local collisions.
The observation we make here is that in the same way as neighbouring XOR bit differences of appropriate sign can be compressed into a single modular difference, a single XOR difference
can be ``decompressed'' into a series of multiple modular ones. Similarly as for bit compression, if this difference is the perturbation of a local collision, the corrections may still be effective
if they can themselves be decompressed.

In other words, it is not necessary that the introduction of the perturbation of a local collision does not trigger a difference in carry propagations; even if this is the case, the local
collision may still be successful if the resulting state difference is preserved by the Boolean functions (and if the carry chain is not broken by rotations during the corrections). Thus, the success probability of
a single local collision not on a rotation boundary is strictly higher than the probability obtained from the analysis of \autoref{sec:diffs_ana} (as it is).

We may rephrase this in a slightly more formal way as follows:

\medskip

Consider a local collision started by an initial perturbation on $m_s$ and $\widetilde{m_s}$ of positive sign at position $i < 31$, \ie with a signed bit difference $\ldots\mnodiff\monediffu\mnodiff\ldots$.
This corresponds to a modular difference $\widetilde{m_s} = m_s + 2^i$. Call $x$, $\tilde{x}$ and $y$, $\tilde{y}$ the state to which the message $m_s$, $\widetilde{m_s}$ is added and the result of this addition
respectively. The probability (over the values of $x$) of having a signed difference $\ldots\mnodiff\monediffu\mnodiff\ldots$ for $y$, $\tilde{y}$ is $2^{-1}$.
However, we also have $\widetilde{m_s} = m_s - \sum_{j = i}^{k - 1}2^j + 2^k$ for any $k < 32$. Thus we can write $\tilde{y} = \tilde{x} + \widetilde{m_s} = x + m_s - \sum_{j = i}^{k - 1}2^j + 2^k$. The
probability of having a difference $\ldots\mnodiff\monediffu\monediffd\mnodiff\ldots$ between $\tilde{y}$ and $y$ is thus $2^{-2}$; more generally, the probability of
having a difference $\ldots\mnodiff\monediffu\underbrace{\monediffd\ldots\monediffd}_{u~\text{times}}\mnodiff\ldots$ between $\tilde{y}$ and $y$ (with $i+u+1 < 32$) is $2^{-u-1}$.

For an initial difference of weight $u+1$, the corrections on subsequent message words $m_{s+o}$, $\widetilde{m_{s+o}}$ are of the form $\widetilde{m_{s+o}} = m_{s+o} - 2^{i+r\mod 32}$, $r \in \{0,5,30\}$.
An initial perturbation that resulted in a difference on $y$, $\tilde{y}$ of weight $u+1$ can thus be corrected at every step only if $i+r\mod 32 \leq i+u+r\mod 32$, because the equality
$\widetilde{m_{s+o}} = m_{s+o} + \sum_{j = i+r \mod 32}^{i+r+u-1 \mod 32}2^j - 2^{i+r+u \mod 32}$ must hold.

Now assume that the maximal weight of an initial perturbation that can be corrected is $v$, and that for the sake of simplicity all induced perturbations are in an XOR round in no particular position
(meaning that no correction is on the MSB), then the success probability of having a local collision is $\sum_{i=1}^{v}2^{-4v}$, which is higher than the probability $2^{-4}$ obtained by considering
only the signed difference $\ldots\mnodiff\monediffu\mnodiff\ldots$.

A complete analysis of the impact of carry propagation on the success probability of a single local collision was done
by Mendel \etal for all Boolean functions and positions of the perturbation~\cite{DBLP:conf/fse/MendelPRR06a}. Manuel also performed experiments validating this analysis~\cite{DBLP:journals/dcc/Manuel11}
(in particular, these results seem to show that for a single local collision, no additional effect contributes to the success probability).

\medskip

To conclude this part and the previous one, we have seen that the interaction of XOR and modular differences in \sha leads to a slight \emph{differential} effect for the disturbance vector. A single
message difference actually defines several, not mutually exclusive, local collision patterns. Even though one of these patterns is much more likely than the others (\ie the one with all possible compressions
effectively done, and no decompression), the contribution of the remaining ones is not nil.

\paragraph{Joint local collision analysis}
$\phantom{bouh}$

\medskip

\noindent
We have just considered how the propagation of carries may influence the probability of a single local collision and of neighbouring local collisions started on the same step. We now consider how to account for similar effects
that impact the joint success probability of local collisions sharing some common steps. We have already mentioned in \autoref{sec:dv_sha0} that an anlysis in the spirit of \autoref{sec:diffs_ana} may be done
on closely interacting local collisions (for instance on local collisions that share some correction bits or that have some corrections interacting together through the Boolean functions).
However, this does not take into account the effects of carries, and a more precise study is thus
possible.

We now summarize the \emph{joint local collision analysis} (JLCA) approach of Stevens~\cite{DBLP:conf/eurocrypt/Stevens13,phdstevens}, which allows to compute the exact best probability of a disturbance vector
by considering all interactions between non-disjoint local collisions. This results in a very good ``exact'' cost function for \dvs of \shaone.

\medskip

The objective here is actually slightly more generic: for a given \dv, there is some liberty in the choice of the actual message differences (\eg by specifying their sign), and we have already seen that this
may impact the success probability of local collisions. These variations should be taken into account when comparing \dvs together, and only the message differences resulting in the best probability should
be considered. Going further, for a given range of steps, we may want to determine which initial and final state differences yield the best probability.
Thus we may reformulate our objective as wanting to find the maximum success probability for a given \dv and a prescribed number of steps over the choice of compatible signed message differences and initial
and final state differences.

To fulfill this objective, we may simply try (for a given configuration) to enumerate all differential paths and sum their probabilities. However, although this was feasible for a single local collision
(cf. \cite{DBLP:conf/fse/MendelPRR06a}),
the amount of paths to consider makes this task computationally intractable in general. The idea of Stevens to get around this limitation is to use a notion of \emph{reduced} paths together with
(equivalence) classes of message differences. A reduced path is essentially obtained from its non-reduced analogue by removing differences not interacting with the initial and final differences. Such
paths can easily be enumerated, and most importantly it is possible to compute their associated ``cumulative probabilities'' which is (for a given reduced path and a given message difference) the sum of
the probabilities of all possible complements to the reduced path that result in a valid overall path. Although the number of possible message differences to consider may be big, Stevens also shows how
to find equivalence classes yielding the same probability for all their members. It is then enough to perform the computation for one representative of every class. 

\medskip

We have just mentioned how it is possible to exactly compute the best achievable probability for a given \dv over a given range of steps. However, one should keep in mind that the \dv with the best such
probability (for the range of steps one wishes to consider) is not necessarily the one most suited to an attack. Indeed, somehow in the same way as \dvs were disqualified in the original \shazero
attacks because of incompatibilities in the IF round (for the model used at the time), a \dv with high associated probability may be a worse choice than another with a lower probability if the former
makes it harder to find a good non-linear part for the first round than the latter, or similarly if it makes it harder to use accelerating techniques.

These last criteria are much harder to capture into a cost function, and there were no attempts to do so in the literature. Ultimately, the complexity of the beginning of an attack can only
be precisely determined by evaluating the speed at which an efficient implementation produces partial solutions up to a step where no freedom remains.
The cost functions as presented in this entire section are then a very useful tool to precisely extrapolate the cost of a full attack from this point on.

% maximize sum of proba over start, end, mess diff for a given DV over given steps

% computing this directly is too expensive

% use ``reduced paths'' and ``message classes''

% conclude by recalling that ability to find NL paths and many accelerations is also important



%This seemingly minor difference impacts the function in a significant way
%, as the presence of the rotation greatly increases the minimum Hamming weight of expanded messages. It is easy to check that there are expanded messages
%for \shazero with weight as low as 23 (for instance the expansion of $M_3 = M_7 = M_9 = M_{11} = 1$, all other words being zero), whereas the best knfor \shaone

% Plan:
% History of attacks on MD-family, with references and brief explanation of who introduces what (details later explained on SHA-1)
% SHA-1 attack (CRYPTO 2005), with details
% Developments (DV, NL search, JLCA, tables of nessies)

% Initial SHA-0: 1) linearize, 2) local collisions, 3) disturbance vector, 4) accelerated brute force
% A(SHA-1): uses same techs as improvements on SHA-0 (pour mémoire)
% No pure L part because of constraints from IV and IF round, mostly
% Two blocks for less issues with IV

%\bigskip
%
%
%\textbf{FROM 76-STEP INTRO}
%
%Before the impressive attacks on \mdfive, the NIST had standardized the hash function \shazero~\cite{Nist-SHA0}, designed by the NSA and very similar to \mdfive. This function was quickly very slightly modified and became \shaone~\cite{Nist-SHA}, with no justification provided.
%A plausible explanation came from the pioneering work of Chabaud and Joux~\cite{DBLP:conf/crypto/ChabaudJ98} who found a
%theoretical collision attack that applies to \shazero but not to \shaone.
%Many improvements of this attack were subsequently proposed~\cite{DBLP:conf/crypto/BihamC04}
%and an explicit collision for \shazero was eventually computed~\cite{DBLP:conf/eurocrypt/BihamCJCLJ05}. 
%However, even though \shazero was practically broken, \shaone remained free of attacks until the work of Wang~\etal~\cite{DBLP:conf/crypto/WangYY05a} in 2005, who gave
%the very first theoretical collision attack on \shaone with an expected cost equivalent to $2^{69}$ calls to the compression function.
%This attack has later been improved several times, the most recent improvement being due to Stevens~\cite{DBLP:conf/eurocrypt/Stevens13}, who gave an attack
%with estimated cost $2^{61}$;
%yet no explicit collision has been computed so far.
%With the attacks on the full \shaone remaining impractical, the community focused on computing collisions for reduced versions:
%$64$ steps~\cite{DBLP:conf/asiacrypt/CanniereR06} (with a cost of $2^{35}$ \shaone calls), $70$ steps~\cite{DBLP:conf/sacrypt/CanniereMR07} (cost $2^{44}$ \shaone)%
%%Marc: there seems to be no example 70-step collision in this paper, so removed this citation here: and \cite{JouxP07} (cost ...)
%, $73$ steps~\cite{cryptoeprint:2010:413} (cost $2^{50.7}$ \shaone) and the latest advances reached $75$ steps~\cite{cryptoeprint:2011:641} (cost $2^{57.7}$ \shaone) using extensive GPU computation power. 
%As of today, one is advised to use \eg~\shatwo~\cite{Nist-SHA} or the hash functions of the future \shathree standard~\cite{sha3_draft} when secure hashing is needed.
%
%% \mdfour:  practical collision~\cite{WangLFCY05} 
%% \shazero: practical collision~\cite{WangYY05}, collision in one hour~\cite{ManuelP08}, 
%% \ripemd~\cite{todo}, \ripemdote and \ripemdosz~\cite{DobbertinBP96}
%% \ripemd: practical collision~\cite{WangLFCY05}, 
%% \ripemdote~\cite{LandelleP13} and \ripemdosz still ok
%
%In general, two main points are crucial when dealing with a collision search for a member of the \mdsha family of hash functions (and more generally for almost every hash function): the quality of the differential paths used in the attack and the amount and utilization of the remaining freedom degrees.
%Regarding \shazero or \shaone, the differential paths were originally built by linearizing the step function and by inserting small perturbations and corresponding corrections to avoid the propagation of any difference. These so-called local collisions~\cite{DBLP:conf/crypto/ChabaudJ98} fit nicely with the linear message expansion of \shazero and \shaone and made it easy to generate differential paths and evaluate their quality. However, these linear paths have limitations since not so many different paths can be used as they have to fulfill some constraints
%(for example no difference may be introduced in the input or the output chaining value). In order to relax some of these constraints, Biham~\etal~\cite{DBLP:conf/eurocrypt/BihamCJCLJ05} proposed to use several successive
%\shaone compression function calls to eventually reach a collision. Then, Wang~\etal~\cite{DBLP:conf/crypto/WangYY05a} completely removed these constraints by using only two blocks and by allowing some part of the differential paths to behave non-linearly (\ie not according to a linear behavior of the \shaone step function). Since the non-linear parts have a much lower differential probability than the linear parts,
%to minimize the impact on the final complexity they may only be used where freedom degrees are available, that is
%during the first steps of the compression function. Finding these non-linear parts can in itself be quite hard, and it is remarkable that the first ones were found by hand. 
%Thankfully, to ease the work of the cryptanalysts, generating such non-linear parts can now be done automatically, for instance
%using the guess-and-determine approach of De~Canni\`ere and Rechberger~\cite{DBLP:conf/asiacrypt/CanniereR06},
%or the meet-in-the-middle approach of Stevens~\etal~\cite{phdstevens,hashclash}. 
%In addition, joint local collision analysis~\cite{DBLP:conf/eurocrypt/Stevens13} for the linear part made heuristic analyzes unnecessary and allows to generate optimal differential paths.
%
%% citer les papiers de Manuel ?
%
%Once a differential path has been chosen, the remaining crucial part is the use of the available freedom degrees when searching for the collision.
%Several techniques have been introduced to do so. First, Chabaud and Joux~\cite{DBLP:conf/crypto/ChabaudJ98} noticed that in general the 15 first steps of the differential path can be satisfied for free since the attacker can fix the first 16 message words independently, and thus fulfill these steps one by one.
%Then, Biham and Chen~\cite{DBLP:conf/crypto/BihamC04} introduced the notion of neutral bits, that allows the attacker to save conditions for a few additional steps.
%The technique is simple: when a candidate following the differential path until step $x > 15$ is found, one can amortize the cost for finding this valid candidate by generating many more almost for free. Neutral bits are small modifications in the message that are very likely not to invalidate conditions already fulfilled in the $x$ first steps. In opposition to neutral bits, the aim of message modifications~\cite{DBLP:conf/crypto/WangYY05a} is not
%to multiply valid candidates but to correct the wrong ones: the idea is to make a very specific modification in a message word, so that a condition not verified at a later step eventually becomes valid with very good probability, but without interfering with previously satisfied conditions. Finally, one can cite the tunnel technique from Kl{\'{\i}}ma~\cite{cryptoeprint:2006:105} and the auxiliary paths (or boomerangs) from Joux and Peyrin~\cite{DBLP:conf/crypto/JouxP07}, that basically consist in pre-set, but more powerful neutral bits. Which technique to use, and where and how to use it are complex questions for the attacker and the solution usually greatly depends on the specific case that is being analyzed.
%
%\begin{center}
%\Huge\textinterrobang
%\end{center}
%
%\textbf{FROM 80-STEP PRELIMS}
%
%\subsection{Differential collision attacks on SHA-1}
%
%We now introduce the main notions used in a collision attack on \shaone (and more generally on members of the \mdsha family).
%
%\subsubsection{Background.}
%In a differential collision attack on a (Merkle-Damg\aa rd) hash function, the goal of the attacker is to find a high-probability
%differential path (the differences being on the message, and also on the IV in the case of a freestart attack) which entails a zero difference on the final
%state of the function (\ie{} the hash value). A pair of messages (and optionally IVs) following such a path indeed leads to a collision.
%
%In the case of \shaone (and more generally ARX primitives), the way of expressing differences between messages is less obvious than for
%\eg{} bit or byte-oriented primitives. It is indeed natural to consider both ``XOR differences'' (over $\mathbf{F}_2^n$) and
%``modular differences'' (over $\mathbf{Z}/2^n\mathbf{Z}$) as both operations are used in the function.
%In practice, the literature on \shaone uses several hybrid representations of differences based on \emph{signed XOR differences}.
%In its most basic form, such a difference is similar to an XOR difference with the additional information of the value of the differing bits for each message (and also
%of some bits equal in the two messages),
%which is a ``sign'' for the difference.
%This is an important information when one works with modular addition as the sign impacts the (absence of) propagation of carries in the addition of two differences.
%Let us for instance consider the two pairs of words $a = 11011000001_b$, $\hat{a} = 11011000000_b$ and $b = 10100111000_b$, $\hat{b} = 10100111001_b$; the XOR
%differences $(a \oplus \hat{a})$ and $(b \oplus \hat{b})$ are both $00000000001_b$ (which may be written \texttt{..........x}),
%meaning that $(a \oplus b) = (\hat{a} \oplus \hat{b})$. On the other hand, the signed
%XOR difference between $a$ and $\hat{a}$ may be written \texttt{..........-} to convey the fact that they are different on their lowest bit \emph{and} that
%the value of this bit is 1 for $a$ (and thence 0 for $\hat{a}$); similarly, the signed difference between $b$ and $\hat{b}$ may be written
%\texttt{..........+}, which is a difference in the same position but of a different sign. From these differences, we can deduce that $(a + b) = (\hat{a} + \hat{b})$
%because differences of different signs cancel; if we were to swap the values $b$ and $\hat{b}$, both differences on $a$ and $b$ would have the same sign and
%indeed we would have $(a + b) \neq (\hat{a} + \hat{b})$ (though $(a \oplus b)$ and $(\hat{a} \oplus \hat{b})$ would still be equal). It is possible to extend signed differences to account for more generic combinations of possible
%values for each message bit; this was for instance done by De~Canni\`ere and Rechberger to aid in the automatic search of differential paths \cite{DBLP:conf/asiacrypt/CanniereR06}.
%Another possible extension  is to consider relations between various bits of the (possibly rotated) state words;
%this allows to efficiently keep track of the propagation of differences through the step function. Such differences are for instance used by Stevens \cite{DBLP:conf/eurocrypt/Stevens13},
%and also in this work (see \autoref{table:appbitconditions}).
% 
%\medskip
%
%The structure of differential attacks on \shaone evolved to become quite specific. At a high level, they consist of: 1. a \emph{non-linear} differential path of low probability;
%2. a \emph{linear} differential path of high probability; 3. accelerating techniques.
%
%The terms \emph{non-linear} and \emph{linear} refer to how the paths were obtained: the latter
%is derived from a linear (over $\mathbf{F}_2^{32}$) modelling of the step function. This kind of path is used in the probabilistic phase
%of the attack, where one simply tries many message pairs in order
%to find one that indeed ``behaves'' linearly.
%Computing the exact probability of this event is however not easy, although it is not too hard to find reasonable estimates. This probability is the main
%factor determining the final complexity of the attack.
%
%The role of a non-linear path is to bootstrap the attack by bridging a state with no differences (the IV) with the start of the linear differential path\footnote{For the sake of simplicity,
%we ignore here the fact that a collision attack on \shaone usually uses two blocks, with the second one having differences in its chaining value. The general picture is actually the following:
%once a pair of messages following
%the linear path $\mathcal{P^+}$ is found, the first block ends with a signed difference $+\Delta$; the sign of the linear path is then switched for the second block to become
%$\mathcal{P^-}$ and following this path results in a difference $-\Delta$; the feedforward then cancels both differences and yields a collision.}.
%In a nutshell, this is necessary because these paths
%do not typically lie in the kernel of the linearized \shaone; hence it is impossible to obtain a collision between two messages following a fully linear path.
%This remains true in the present case of a freestart attack, even if the non-linear path now connects the start of the linear path with an IV containing some differences.
%Unlike the linear path, the non-linear one has a very low probability of being followed by random messages. However, the attacker can fully choose the messages
%to guarantee that they do follow the path, as he is free to set the 512 bits of the message. Hence finding conforming message pairs for this path effectively costs nothing in the attack.
%
%Finally, the role of accelerating techniques is to find efficient ways of using the freedom degrees remaining after a pair following the non-linear path has been found, in order to delay
%the effective moment where the probabilistic phase of the attack starts.
%
%\medskip
%
%We conclude this section with a short discussion of how to construct these three main parts of a (freestart) collision attack on \shaone. 
%
%\subsubsection{Linear path; local collisions.}
%The linear differential paths used in collision attacks are built around the concept of \emph{local collision}, introduced by Chabaud and Joux in 1998 to attack \shazero.
%The idea underlying a local collision is first to introduce a difference in one of the intermediate state words of the function, say $A_i$, through a difference in the message word $W_{i-1}$.
%For an internal state made of $j$ words ($j = 5$ in the case of \shazero or \shaone),
%the attacker then uses subsequent differences in (possibly only some of)
%the message words $W_{i\ldots i+(j-1)}$  in order to cancel any contribution of the difference in $A_i$ in the computation of a new internal state $A_{i+1\ldots i+j}$, which will therefore have no differences.
%The positions of these ``correcting'' differences are dictated by the step function, and there may be different options depending on the used Boolean function,
%though originally (and in most subsequent cases) these were chosen according to a linearized model (over $\mathbf{F}_2^{32}$) of the step functions.
%
%Local collisions are a fit basis to generate differential paths of good probability. The main obstacle to do this is that the attacker does not control all of the message words,
%as some are generated by the message expansion. Chabaud and Joux showed how this could be solved by chaining 
%local collisions along a \emph{disturbance vector} (DV) in such a way that the final state of the function contains no difference and that the pattern of the local collisions is compatible with the message expansion.
%The disturbance vector just consists of a sparse message (of sixteen 32-bit words) that has been expanded with the linear message expansion of \shaone. Every ``one'' bit of this expanded message then
%marks the start of a local collision (and expanding all the local collisions thus produces a complete linear path). 
%
%Each local collision in the probabilistic phase of the attack (roughly corresponding to the last three rounds) increases the overall complexity of the attack,
%hence one should use disturbance vectors that are sparse over these rounds.
%Initially, the evaluation of the probability of disturbance vector candidates was done mostly heuristically, using \eg{}
%the Hamming weight of the vector (\cite{DBLP:conf/crypto/BihamC04,DBLP:conf/ima/PramstallerRR05,DBLP:conf/ctrsa/RijmenO05,DBLP:conf/wcc/MatusiewiczP05,cryptoeprint:2005:266}),
%the sum of bit conditions for each local collision independently (not allowing carries) (\cite{DBLP:conf/crypto/WangYY05,DBLP:conf/ccs/YajimaINSSKO08}),
%and the product of independent local collision probabilities (allowing carries) (\cite{DBLP:conf/fse/MendelPRR06a,DBLP:journals/dcc/Manuel11}). 
%Manuel \cite{cryptoeprint:2008:469,DBLP:journals/dcc/Manuel11} noticed that all disturbance vectors used in the literature belong to two classes I$(K,b)$ and II$(K,b)$.
%Within each class all disturbance vectors are forward or backward shifts in the step index (controlled by $K$) and/or bitwise cyclic rotations (controlled by $b$) of the same expanded message.
%We will use this notation through the remainder of this article.
%
%Manuel also showed that success probabilities of local collisions are not always independent, causing biases in the above mentioned heuristic cost functions.
%This was later resolved by Stevens using a technique called joint local-collision analysis (JLCA)\cite{DBLP:conf/eurocrypt/Stevens13,phdstevens},
%which allows to analyze entire sets of differential paths over the last three rounds that conform to the (linear path entailed by the) disturbance vector.
%This is essentially an exhaustive analysis taking into account all local collisions together, using which one can determine the highest possible success probability.
%This analysis also produces a minimal set of sufficient conditions which, when all fulfilled, ensure that a pair of messages follows the linear path;
%the conditions are minimal in the sense that meeting all of them happens with this highest probability that was computed by the analysis.
%Although a direct approach is clearly unfeasible (as it would require dealing with an exponentially growing amount of possible differential paths),
%JLCA can be done practically by exploiting the large amount of redundancy between all the differential paths to a very large extent.
%
%\subsubsection{Non-linear differential path.}
%The construction of non-linear differential paths was initially done by hand by Wang, Yin and Yu in their first attack on the full \shaone \cite{DBLP:conf/crypto/WangYY05a}.
%Efficient algorithmic construction of such differential paths was later proposed
%in 2006 by De~Canni\`ere and Rechberger,
%who introduced a guess-and-determine approach \cite{DBLP:conf/asiacrypt/CanniereR06}.
%A different approach based on a meet-in-the-middle method was also proposed by Stevens~\etal~\cite{phdstevens,hashclash}.
%
%\subsubsection{Accelerating techniques.}
%For a given differential path, one can derive explicit conditions on state and message bits which are sufficient to ensure that a pair of messages follows the path.
%This lets the collision search to be entirely defined over a single compression function computation.
%Furthermore, they also allow detection of ``bad'' message pairs a few steps earlier compared to computing the state and verifying differences, allowing to abort computations earlier in this case.
%
%An important contribution of Wang, Yin and Yu was the introduction of powerful \emph{message modification} techniques,
%which followed an earlier work of Biham and Chen who introduced \emph{neutral bits} to produce better attacks on \shazero~\cite{DBLP:conf/crypto/BihamC04}.
%The goal of both techniques is for the attacker to make a better use of the available freedom in the message words in order to decrease the complexity of the attack. 
%Message modifications try to correct bad message pairs that only slightly deviate from the differential
%path, and neutral bits try to generate several good message pairs out of a single one (by changing the value of a bit which does not invalidate nearby sufficient conditions
%with good probability).
%In essence, both techniques allow to amortize part of the computations, which effectively delays the
%beginning of the purely probabilistic phase of the attack.
%
%Finally, Joux and Peyrin showed how to construct powerful neutral bits and message modifications by using
%auxiliary differential paths akin to \emph{boomerangs} \cite{DBLP:conf/crypto/JouxP07}, which allow more efficient attacks.
%In a nutshell, a boomerang (in collision attacks) is a small set of bits that together form a local collision. Hence flipping
%these bits together ensures that the difference introduced by the first bit of the local collision does not
%propagate to the rest of the state; if the initial difference does not invalidate a sufficient condition, this local collision
%is indeed a neutral bit. Yet, because the boomerang uses a single (or sometimes a few) local collision, more differences will
%actually be introduced when it goes through the message expansion. The essence of boomerangs is thus to properly choose where
%to locate the local collisions so that no differences are introduced for the most steps possible.


\chapter[Collisions à initialisation libres pour \shaone]
        {Freestart collision attacks for \shaone}
\label{cha:shaone_new}

\section{A framework for freestart collisions for SHA-1}


\section{A framework for efficient GPU implementations of collision attacks}

In the previous section, we have described the framework used to mount freestart collision attacks on \shaone. We now turn to
the matter of concrete implementation of the attack procedure. Specifically, we describe implementations on \emph{graphics processing
units} (GPUs).

The use of GPUs is attractive for computation-intensive cryptanalysis, as they offer much more raw computational power than
similarly-priced general-purpose processors (\ie CPUs). The availability of efficient frameworks for general-purpose GPU programming such as CUDA~\cite{cuda} allows
for potentially complex code to be conveniently deployed on GPUs. However, the differences in architecture between CPUs and GPUs
(as highlighted below) need to be taken into account, and not every attack may be suitable for a GPU implementation.

GPUs have already been used with some success in heavy-computation cryptography, notably to aid in integer factorization or in finding discrete
logarithms~\cite{DBLP:conf/asiacrypt/BosK12,DBLP:conf/ches/MieleBKL14,DBLP:conf/eurocrypt/BarbulescuGGM15,DBLP:phd/hal/Jeljeli15}, and also for collision attacks on reduced \shaone~\cite{cryptoeprint:2011:641}.
Even if the latter case in particular is essentially identical to our own, we nonetheless developped our GPU framework for implementing collision attacks from scratch.

\subsection{GPU architecture and programming model}

We start by first recalling a few important points about current GPUs that will help
understanding our design decisions\footnote{We specifically discuss these points
for Nvidia GPUs of the \emph{Maxwell} generation such as the \gtx
used in our attacks.}.

%\paragraph{Number of cores and scheduling.}
A modern GPU can feature more
than a thousand of small cores, that are packed together in a small
number of larger ``multiprocessor'' execution units. Taking the example of the
Nvidia \gtx for concreteness, there are 13 multiprocessors of 128 cores each,
making 1664 cores in total~\cite{gtx970_specs}. The fastest instructions
(such as for instance 32-bit bitwise logical operations or modular addition) have a throughput of 1 per core,
which means that in ideal conditions 1664 instructions may be simultaneously processed by such a GPU in one clock
cycle~\cite{cuda_prog_guide}.

Yet, so many instructions cannot be emitted independently, or to put it in another way, one cannot run an independent
thread of computation for every core. In fact, threads are grouped together by 32 forming a \emph{warp}, and only
warps may be scheduled independently. Threads within a warp may have a diverging control flow, for instance by
taking a different path upon encountering a conditional statement, but their execution in this case is
serialized. At an even higher level, warps executing the same code can be grouped together as \emph{blocks}.

%Furthermore, on each multiprocessor one can run up to 2048 threads simultaneously, which are dynamically scheduled every cycle
%onto the 128 cores at a warp granularity.
%Thus while a warp is waiting for the results of a computation or for a (high latency) memory operation to return,
%another warp can be scheduled.
%Although having more threads does not increase the computational power of the multiprocessor, such overbooking of cores can be used to hide latencies and thus increase efficiency of a GPU program.
%
%In short, to achieve an optimal performance, one must bundle computations by groups of 32 threads executing
%the same instructions most of the time and diverging as little as possible and use as many threads as possible.

Each multiprocessor can host a maximum of 2048 threads regrouped
in at least 2 and at most 32 blocks~\cite{cuda_prog_guide}. If every multiprocessor of the GPU hosts 2048 threads, we say that we have
reached \emph{full occupancy}. While a multiprocessor can only physically run one thread per core (\ie 128) at a given time,
a higher number of resident threads is beneficial to hide computation and memory latencies.
These can have a significant
impact on the performance as a single waiting thread causes its entire warp of 32 to wait with him; it is thus important in
this case for the multiprocessor to be able to schedule another warp in the meantime.

Achieving full occupancy is not however an absolute objective as it may or may not result in optimal performance depending on the
resources needed by every thread. Important factors in that respect are the average amount of memory and the number of registers
needed by a single thread, both being resources shared among the threads.
In our implementation, the threads need to run rather heavy functions and full occupancy is typically
not desirable. One reason why it is so is that we need to allocate 64 registers per thread in order
to prevent register spilling in some of the most expensive functions; a multiprocessor
having ``only'' $2^{16}$ registers, this limits the number of threads to 1024. As a result, we use a layout of
26 blocks of 512 threads each, every multiprocessor being then able to host 2 such blocks.

\bigskip

%\paragraph{Memory architecture and thread synchronization.}
In the same way as they feature many execution units, GPUs also provide
memory of a generous size (\eg 4 GB for the \gtx), which must however be shared among the threads. The amount of memory available to a single thread
is therefore much less than what is typically available on a CPU (it of course highly depends on the number of
running threads, but can be lower than 1\,MB). This, together with the facts that threads of a same warp do not actually execute
independently of each other and that threads of a same block run the same code makes it enticing to organize the memory
structure of a program at the block level. Fortunately, this is made rather easy by the fact that many efficient
synchronization functions are available for the threads, both at the warp and at the block level.

\subsection{High-level structure of the framework}

%\paragraph{Balancing the work between the GPU and the CPU.}
The implementation of our attacks can be broadly decomposed into two phases. The first step consists in computing a certain number
of base solutions and in storing them on disk.
Because the total number of base solutions necessary to find a collision
is rather small (about $2^{25}$ in the 76-step case, for instance) and because they can be computed
quickly, this can be done efficiently in an offline fashion using CPUs.

The second phase then consists in trying to extend probabilistically the base solutions to satisfy path conditions up to a further point by
trying many neutral bit combinations, in the hope of eventually finding a collision.
This is an intensely parallel task that is well suited to running on GPUs. However, as it
was emphasized above, GPUs are most efficient when there is a high coherency between the execution of many threads. For that reason,
we must avoid having idle threads that are waiting because their candidate solutions failed to follow the differential paths, while
others keep on verifying a more successful one. Our approach to this is to fragment the verification into many small functions
(or \emph{snippets})
that are chosen in a way which ensures that coherency is maintained for every thread of a warp when executing a single snippet, except in
only a few small points. This is achieved through a series of intermediary buffers that store inputs and outputs for the snippets;
a warp then only executes a given snippet if enough inputs are available for every of its threads.
One should note that there is no need to entirely decompose the second step of the attack into snippets, and that a final part can again
be run in a more serial fashion, typically on CPU. Indeed, if inputs to such a part are scarce, there is no real advantage in verifying them in
a highly parallel way.


The sort of decomposition used for the GPU phase of our attack as described above is in no way constrained by the specifics of \shaone collision search.
In fact, it is quite general, and we believe that it can be successfully applied to many an implementation of (symmetric) cryptographic
attacks. We conclude this section by giving more details of the application of this approach to the case of \shaone.

%\paragraph{Choice of the snippets.}

\bigskip

As we mentioned in \autoref{sec:framework}, the attack process consists in trying many combinations of neutral bits, with each step in a small window
adding new neutral bits to be tested. It is thus quite natural to reflect this process in the choice of the snippets:
we use intermediary buffers to store partial solutions up to $\state_{17}$ (\ie base solutions), $\state_{18}$, etc.
Then for each step the corresponding snippet consists
in loading one partial solution per thread of a warp and applying every possible combination of neutral bits for this step. Each combination
is tried by every thread at the same time on its own partial solution, thereby maintaining coherency.
Then, each thread assesses if the current combination yields a valid extension (by one step) of its own partial solution, and writes the result to
an output buffer for the snippet (which is the input buffer for the next snippet) if this is the case;
this conditional write is the only part of the code where threads may briefly diverge.

For the
later steps when no neutral bits can be used anymore, the snippets regroup the computation of several steps together.
Eventually the verification that partial solutions up to step 56 (in the 76-step case; 60 in the 80-step one) result in valid collisions is done on a CPU. This is partly because
the amount of available memory makes it hard to use step-by-step snippets until the end, but also because such partial solutions are only
produced very slowly. For instance, a single \gtx produces partial solutions up to step 56 of a 76-step collision at a speed of about 0.017 solution per second, that is about
1 per minute); waiting for enough partial solutions to feed a single complete warp would in this case take a completely unreasonable half hour.

\bigskip


%\paragraph{Complete process of the attack.}
A complete implementation of an attack mostly consists in the snippets and supporting functions, such as buffer management.
Connecting the snippets together is straightforward; every warp tries to work with partial solutions that are up
to the latest step for which enough solutions are available; \ie it visits the buffer of partial solutions in order from the top,
stopping at the first that is able to feed it entirely.
In the worst case where none of the buffers are full enough, it simply resorts to using base solutions.

In practice, warps spend most of their time feeding on partial solutions that are valid up to a rather late step; for instance, in the 76-step attack,
about 90\% of the time is spent on $\state_{24}$ or higher, which is at most two steps away from the latest step where neutral bits are available.
Thus, work on early steps and in particular on base solutions is done only intermittently.

We conclude this high-level description by giving a simplified flow chart
of the GPU part of the 76-step attack in \autoref{fig:attack_diagram}, made slightly incorrect for the sake of clarity (notably omitting the fact that further verification is still done on GPU up to steps 56).

\def\rectanMac{\begin{tikzpicture}[scale=0.2]\draw (0,0) rectangle (1,1);\end{tikzpicture}}
\def\elliMac{\begin{tikzpicture}[scale=0.3,transform shape]\node[draw,ellipse] (e) at (0,0) {\phantom{toto}};\end{tikzpicture}}
\def\plainMac{\begin{tikzpicture}[scale=0.1] \draw[>=latex,->] (0,0) -- (2,2);\end{tikzpicture}}
\def\dottMac{\begin{tikzpicture}[scale=0.1] \draw[dotted,->]   (0,0) -- (2,2);\end{tikzpicture}}

\begin{figure}[htb]
%%\begin{sideways}
%%%%%% EPRINT %%%%%
%% Minipage changed to 6cm from 5cm
%\begin{minipage}{6cm}
  \begin{center}
% Figure scale changed from .68. Set that back for proceedings
  \includegraphics[scale=0.7]{figures/attack_diagram_std.pdf}
%%%%% /EPRINT %%%%
  \end{center}
%\end{minipage}
%\end{sideways}
  \caption{Simplified flow chart for the GPU part of the attack. The start of this infinite loop is
  in the top left corner. Rectangles ``\,\protect\rectanMac\,'' represent
  snippets, ellipses ``\,\protect\elliMac\,''
  represent shared buffers, plain lines ``\,\protect\plainMac~''
  represent control flow, and dotted lines ``\,\protect\dottMac~'' represent data flow.}
  \label{fig:attack_diagram}
\end{figure}


\subsection{Implementation details}

We now give more details about the implementations of the attacks. In particular, we discuss
how partial solutions are represented at various steps in \autoref{sec:part_sol} and
we comment the code of a snippet function in \autoref{sec:snippet}. We very briefly discuss how
to tune GPU setttings to use them more efficiently in \autoref{sec:gpu_tune}.

\subsubsection{Representation of partial solutions}
\label{sec:part_sol}

As the basis of our framework is to store, load and extend many partial solutions for the differential path, we need to be able to work with such objects
in an efficient way. One thus needs to define good representations for partial solutions, such that processing them is computationally simple, and managing them
in memory causes as little overhead as possible.

The representation we use is based on two types of buffers: some are holding enough information to define a \sha computation in its entirety (\ie it
contains the value of at least five (resp. sixteen) consecutive state (resp. message) words), while others only contain the necessary information to
express how the associated partial solution differs from a completely-defined one. Typically, the first kind of buffer may be used to store base solutions,
while the second is used to keep track of which neutral bits are currently active.

It is quite straightforward to define the structure of a base solution buffer, as there is little doubt about the necessary information they need
to include and the way to represent it. For instance, the 76-step base solution buffer contains the value of state words $\state_{13},\ldots,\state_{17}$
and of the message words $\expmess_6,\ldots,\expmess_{21}$. Additionally, it includes the value of $\state_{12}$; while this information is
not strictly necessary, it is useful to speed-up the computation of the activation of some neutral bits. Similar buffers (without an extra state word)
are used to hold partial solutions up to $\state_{36}$ and $\state_{56}$ (resp. $\state_{40}$ and $\state_{60}$ in the 80-step case)\footnote{The choice
of these boundaries to define the late partial solutions (were no more control is possible) simply comes from the fact that a pair of messages following
the differential path of either attack is not expected to have any state differences at these steps. It is thus particularly efficient to screen
solutions that are still valid at these points.}.

The structures of the other buffers are also quite simple, but their instantiations may require some care to make them especially efficient.
In a nutshell, we just want such a buffer to keep a reference to a base solution and to remember the values of the currently active neutral bits.
To make this efficient and convenient to use, we would like to share a similar structure for the input and output buffers of a snippet; this would
allow to extend a partial solution that possibly already has some bits active by simply adding the newly activated bits for this step. This is the
approach we followed in our implementations, with an added refinement.

\medskip

We have already mentioned in the past section that \eg the 76-step attack contains neutral bits acting on a wide range of steps, from $\state_{18}$
to $\state_{26}$; the range for the 80-step attack is even wider, due to the use of boomerangs; the neutral bits themselves are located on many message
words. Thus, it would seem wasteful to recompute the action of paste neutral bits on message words as low as $\expmess_{14}$ while in the snippet corresponding
\eg to $\state_{24}$. Consequently, we have a strong incentive to store intermediary partial solutions so as to save some of this recomputation.

In the 76-step case, there is a natural location that may be used to define what we call \emph{extended base solutions}. As we will detail in
\autoref{sec:res_76}, neutral bits located on words $\expmess_{14}$ to $\expmess_{18}$ are only used up to $\state_{21}$, and the ones located
on words $\expmess_{19}$ to $\expmess_{21}$ are only used in later steps. Thus, it would make sense that once the active bits up to
$\expmess_{19}$ have all been determined, only the modified message words and the corresponding value for the state should be stored. There is
however no need to keep again sixteen message words in such an extended base solution, as most of them are identical to the ones of the corresponding
base solution (and as a base solution is in general extended into many distinct extended base solutions, it would not make sense to \eg add enough
message words to the latter and erasing the former). There is one last subtlety in the definition of the extended base solution, which is that
it also includes the message word $\expmess_{20}$. Although no neutral bits on this word are yet to be used at the point where the solution is
formed, some of its bits may need to be flipped depending on the use of neutral bits on words $\expmess_{15}$ and $\expmess_{16}$, so as to
preserve message bit relations. A convenient way to remember this information is simply to preemptively add the possible contributions of the neutral
bits to $\expmess_{20}$ and to store this modified word in the extended base solution.
All in all, the buffer of extended base solution of the 76-step attack is made of twelve words: five state words $\state_{17}$ to $\state_{21}$,
six message words $\expmess_{14}$ to $\expmess_{18}$ and $\expmess_{20}$, and one word holding an identifier for the base solution from which
it is extended.
The ``inter-snippet'' buffers that refer to changes from a base or extended base solution are only made of two words consisting of concatenated
segments of the message words containing neutral bits and of a reference to the associated solution.

The 80-step attack uses similar representations but with a few variations. As we will detail in \autoref{sec:res_80}, there are considerably
more possible corrections to be performed in the 80-step attack to preserve message bit relations. Consequently, some of the precomputations of
individual neutral bit contributions are stored alongside the actual neutral bits. It is also slightly less immediate to determine where to
start defining an extended base solution, as there is no natural separation between the location of various neutral bits as there were in
the 76-step case. This is not a major issue, however, as the location of the neutral bits of a same word shared between base and extended base solutions
are not overlapping inside the word itself; splitting their representation over multiple buffers thus does not result in significant overhead.
Finally, the additional use of boomerang neutral bits (or somehow equivalently, the use of more neutral bits than in the 76-step attack)
implies that the last inter-snippet buffers contain one more word (\ie three in total) compared to the ones for early snippets and all such buffers
in the 76-step case.

In \autoref{sec:res_76} and \autoref{sec:res_80}, we will describe the full content of the various inter-snippet buffers.

\bigskip

We conclude this part by discussing some implementation aspects of the various buffers.

All of the buffers are cyclic and hold $2^{20}$ elements, regardless of their sizes, except the buffers of partial solutions extended up to
$\state_{36} \sim \state_{40}$ and $\state_{56} \sim \state_{60}$ which only have $2^{10}$ elements as they see a lower production rate due to their purely probabilistic nature.

With the exception of the buffers holding the base solutions and the collision candidates formed by partial solutions
up to $\state_{56} \sim \state_{60}$ (\ie the buffers that are written or read by a CPU),
there is one instance of every buffer per block (\ie 26 buffers per GPU). This allows to use block-wise instead of global
synchronization mechanisms when updating the buffers' content,
thence reducing the overhead inherent to the use of such shared data structures.
Taken together, the buffers thus use a significant portion of the 4\,GB memory available on the \gtx, needing in the neighbourhood of 3\,GB.

We also carefully took into account the presence of a limited amount of
very fast multiprocessor-specific shared memory. While the 96\,KB available per multiprocessor is
hardly enough to store the whole buffers themselves, we take advantage of it by dissociating the storage of
the buffers and of the meta-data used for their control logic, the latter being held in
shared memory. This improves the overall latency of buffer manipulations, especially in case of
heavy contention between different warps. This local shared memory is also very useful to buffer
the writes to the buffers themselves. Indeed, only a fraction (often as low as $2^{-3}$)
of the threads of a warp have a valid solution to write after having tested a single candidate,
and the more unsuccessful threads need to wait while the former write their solution to global memory.
It is therefore beneficial to first write the solutions to a small local warp-specific buffer and to
flush it to the main block-wise buffer as soon as it holds 32 solutions or more, thence
significantly reducing the number of accesses to the slower global memory.


\subsubsection{An example of snippet function}
\label{sec:snippet}

%TODO autoref
We now illustrate the discussion of this section by providing a commented snippet from the 76-step attack given in Listing~\ref{lst:snippet_example} (written in \textsf{CUDA C/C++}), namely a function that is taking partial solutions up to $\state_{22}$  and that is
trying to extend them up to $\state_{23}$ using neutral bits on $\expmess_{19}$. Its structure is a straightforward application of the framework, and is fairly representative of most of the
code of both of our attacks (although some snippets may at first seem more complex due to their use of neutral bits located on several distinct message words).

%\begin{lstlisting}[style=customc]
\begin{listing}[!htb]
\begin{center}
\begin{minted}[linenos,breaklines]{c}
__device__ void stepQ23(uint32_t thread_rd_idx)
{
	uint32_t base_idx = Q22SOLBUF.get<11>(thread_rd_idx);
	uint32_t q17 = Q22SOLBUF.get<0>(thread_rd_idx);
	uint32_t q18 = Q22SOLBUF.get<1>(thread_rd_idx);
	uint32_t q19 = Q22SOLBUF.get<2>(thread_rd_idx);
	uint32_t q20 = Q22SOLBUF.get<3>(thread_rd_idx);
	uint32_t q21 = Q22SOLBUF.get<4>(thread_rd_idx);
	uint32_t m6  = BASESOLBUF.get<6>(base_idx);
	uint32_t m8  = BASESOLBUF.get<8>(base_idx);
	uint32_t m19 = BASESOLBUF.get<19>(base_idx);
	uint32_t m21 = BASESOLBUF.get<21>(base_idx);
	uint32_t m14 = Q22SOLBUF.get<5>(thread_rd_idx);
	uint32_t m22;

	uint32_t q22  = sha1_round2(q21, q20, q19, q18, q17, m21);

	uint32_t w19_q23_nb = 0;
	for (unsigned i = 0; i < 32; i++)
	{
		NEXT_NB(w19_q23_nb, W19NBQ23M);

		m19 &= ~W19NBQ23M;
		m19 |= w19_q23_nb;
		m22 = sha1_mess(m19, m14, m8, m6);

		uint32_t newq20 = q20 + w19_q23_nb;
		uint32_t newq21 = q21 + rotate_left(w19_q23_nb, 5);
		uint32_t newq22 = sha1_round2(newq21, newq20, q19, q18, q17, m21);
		uint32_t newq23 = sha1_round2(newq22, newq21, newq20, q19, q18, m22);

		uint32_t q23nessies = Qset1mask[QOFF + 23] ^ (Qprevrmask [QOFF + 23] & rotate_left(newq22, 30));
		bool valid_sol = (0 == ((newq21 ^ q21) & Qcondmask[QOFF + 21]));
		valid_sol &= (0 == ((newq22 ^ q22) & Qcondmask[QOFF + 22]));
		valid_sol &= (0 == ((newq23 ^ q23nessies) & Qcondmask[QOFF + 23]));

		uint32_t sol_val_0 = pack_update_q23_0(m19);
		uint32_t sol_val_1 = pack_update_q23_1(thread_rd_idx);

		WARP_TMP_BUF.write2(valid_sol, sol_val_0, sol_val_1, Q23SOLBUF, Q23SOLCTL);
	}
	WARP_TMP_BUF.flush2(Q23SOLBUF, Q23SOLCTL);
}
\end{minted}
\end{center}
\caption{The \emph{stepQ23} function from the 76-step attack.}
\label{lst:snippet_example}
\end{listing}
%\end{lstlisting}

\medskip

This function takes as argument a thread-dependent identifier \mintinline{c}{thread_rd_idx} for a partial solution, which is essentially an index for the buffer \mintinline{c}{Q22SOLBUF}
of solutions up to $\state_{22}$.

The partial solution is loaded from the buffer and reconstructed from lines 3 to 16. This buffer being the first one holding an extended base solution,
there is no need to reapply neutral bits and most of the work simply consists in loading the appropriate state and message words either directly from the extended base solution
buffer, or from the matching solution of the base solution buffer \mintinline{c}{BASESOLBUF}, the index of which is recovered on line 3. The only recomputation performed here
is the one of \mintinline{c}{q22} on line 16, using the \shaone round function. This would in fact not be necessary in this function, as the value could have been included in the extended base solution altogether.
This was not done because recomputing \mintinline{c}{q22} is necessary in the snippets of the following steps and causes only minimal overhead in this one, while saving a 32-bit word from the \mintinline{c}{Q22SOLBUF} buffer.

The loop from line 18 to 41 (\ie the remainder of the function) applies every combination of the five neutral bits for this step, all of which being located on $\expmess_{19}$. In more details, line 21 sets the register
\mintinline{c}{w19_q23_nb} to one of the 32 possible combinations. Lines 23 and 24 clear the message word \mintinline{c}{m19} of the previous neutral bit combination and applies the new one, and line 25
computes the expanded message word \mintinline{c}{m22} based on the new value for \mintinline{c}{m19}, using \shaone's message expansion (note that this computation could actually be optimized, \eg by
precomputing the contribution of \mintinline{c}{m14}, \mintinline{c}{m8} and \mintinline{c}{m6}, which are fixed in this function).
Lines 27 to 30 compute the impact of the current neutral bit combination, first by (partially) recomputing the state words \mintinline{c}{newq20} and \mintinline{c}{newq21}, then fully recomputing
\mintinline{c}{newq22}, and finally computing the (as yet unknown) word \mintinline{c}{newq23}.
Line 32 computes a mask of sufficient conditions for the value of \mintinline{c}{newq23} based on the value of \mintinline{c}{newq22}.
Lines 33 to 35 determine if the current combination of neutral bits lead to a valid partial solution for $\state_{23}$, first by comparing the values of \mintinline{c}{q21} and  \mintinline{c}{q22}
(which we know to fulfill the conditions) with the updated values \mintinline{c}{newq21} and \mintinline{c}{newq22} (lines 33 and 34), and then checking \mintinline{c}{newq23} for the previously computed conditions.
Lines 37 and 38 prepare the description of the partial solution.
Finally, line 40 writes the partial solution to the buffer \mintinline{c}{Q23SOLBUF}, at the condition that it is indeed valid. As mentioned at the end of \autoref{sec:part_sol}, this is done through a warp-specific
temporary buffer, which is flushed to the actual buffer on line 42 to ensure that every valid solution is indeed eventually copied to \mintinline{c}{Q23SOLBUF}.



\subsubsection{GPU tuning}
\label{sec:gpu_tune}

After our initial implementation, we did some fine tuning of the GPU BIOS settings in order to try having an optimal performance.
One first objective was to ensure that the GPU fans work at 100\% during the attack, as this was strangely
not the case initially, and was obviously not ideal for cooling.
We also experimented with various temperature limits
(that define when the GPU will start to throttle) and both over-clocking and under-volting.
Taken together, these variations can have a significant impact on the overall performance of the program, as can be seen with
our 76-step attack below.


\section{Freestart collisions for 76-step SHA-1}
\label{sec:res_76}

This section gives the attack parameters for the 76-step freestart collision attack on \shaone.

\subsection{The differential path for most of the two first rounds}
\label{sec:app_diff_path76}

We give a graphical representation of the differential path
used in our attack up to step 29 in \autoref{fig:diff_path76}.
The meanings of the various symbols are defined
in \autoref{table:appbitconditions}.
The remainder of the path up to step 76 can easily
be determined by linerization of the step function, given the differences
in the message.

The message bit relations used in the attack for message words past $\expmess_{35}$ are given in \autoref{fig:msgbitrel76}, together with a
graphical representation in \autoref{fig:msgbitrel76_graph}.

\begin{figure}
\centering
\begin{tabular}{lcc}
$i$ & $\state_i$ & $\mess_i$\\
-4 & \nodiff\nodiff\nodiff\nodiff\nodiff\nodiff\nodiff\nodiff\nodiff\nodiff\nodiff\nodiff\nodiff\nodiff\nodiff\nodiff\nodiff\nodiff\nodiff\nodiff\nodiff\nodiff\nodiff\nodiff\nodiff\nodiff\nodiff\nodiff\nodiff\nodiff\nodiff\nodiff \\
-3 & \nodiff\nodiff\nodiff\nodiff\nodiff\nodiff\nodiff\nodiff\nodiff\nodiff\nodiff\nodiff\nodiff\nodiff\nodiff\nodiff\nodiff\nodiff\nodiff\nodiff\nodiff\nodiff\nodiff\nodiff\nodiff\nodiff\nodiff\nodiff\nodiff\nodiff\nodiff\nodiff \\
-2 & \nodiff\nodiff\nodiff\nodiff\nodiff\nodiff\nodiff\nodiff\nodiff\nodiff\nodiff\nodiff\nodiff\nodiff\nodiff\nodiff\nodiff\nodiff\nodiff\nodiff\nodiff\nodiff\nodiff\nodiff\nodiff\nodiff\nodiff\nodiff\nodiff\equaup$\monediffu$\nodiff \\
-1 & $\mnodiffz$\nodiff\nodiff\nodiff\nodiff\nodiff\nodiff\nodiff\nodiff\nodiff\nodiff\nodiff\nodiff\nodiff\nodiff\nodiff\nodiff\nodiff\nodiff\nodiff\nodiff\nodiff\nodiff\nodiff\nodiff$\mnodiffo$\nodiff\nodiff\nodiff\nodiff\nodiff$\monediffu$ \\
0 & $\mnodiffo$$\mnodiffz$\nodiff$\mnodiffz$\nodiff\nodiff\nodiff\nodiff\nodiff\nodiff\nodiff\nodiff\nodiff\nodiff\nodiff\nodiff\nodiff\nodiff\nodiff\nodiff\nodiff\nodiff\nodiff\nodiff\nodiff$\mnodiffz$\nodiff$\mnodiffz$\nodiff\nodiff\nodiff\nodiff     &  \nodiff\nodiff\nodiff\nodiff\nodiff\nodiff\nodiff\nodiff\nodiff\nodiff\nodiff\nodiff\nodiff\nodiff\nodiff\nodiff\nodiff\nodiff\nodiff\nodiff\nodiff\nodiff\nodiff\nodiff\nodiff\nodiff\nodiff$\monediffu$\nodiff\nodiff\nodiff\nodiff \\
1 & \nodiff$\mnodiffz$\nodiff$\mnodiffo$\nodiff\nodiff\nodiff\nodiff\nodiff\nodiff\nodiff\nodiff\nodiff\nodiff\nodiff\nodiff\nodiff\nodiff\nodiff\nodiff\equaup\nodiff$\mnodiffz$\nodiff\nodiff\nodiff\nodiff$\monediffu$\nodiff\nodiff\nodiff\nodiff     & \nodiff\nodiff\nodiff\nodiff\nodiff$\monediffd$\nodiff\nodiff\nodiff\nodiff\nodiff\nodiff\nodiff\nodiff\nodiff\nodiff\nodiff\nodiff\nodiff\nodiff\nodiff\nodiff\nodiff\nodiff\nodiff\nodiff\nodiff$\monediffd$$\monediffu$$\monediffu$\nodiff\nodiff \\
2 & \nodiff\nodiff\nodiff\equaup\nodiff$\monediffd$\nodiff\nodiff\nodiff\nodiff\nodiff\nodiff\nodiff\nodiff\nodiff\equaup\nodiff$\mnodiffo$\nodiff\nodiff\nodiff\nodiff$\monediffu$\nodiff\nodiff\nodiff\nodiff$\mnodiffo$\nodiff$\monediffd$\nodiff$\mnodiffo$     & $\monediffd$$\monediffu$\nodiff\nodiff$\monediffd$$\monediffu$\nodiff\nodiff\nodiff\nodiff\nodiff\nodiff\nodiff\nodiff\nodiff\nodiff\nodiff\nodiff\nodiff\nodiff\nodiff\nodiff\nodiff\nodiff\nodiff\nodiff\nodiff$\monediffu$\nodiff$\monediffd$\nodiff\nodiff \\
3 & \nodiff\nodiff\nodiff\nodiff\nodiff$\monediffd$\nodiff$\mnodiffo$\nodiff\nodiff\equaup\nodiff$\mnodiffo$\nodiff\nodiff\nodiff\nodiff$\monediffu$\nodiff\nodiff\nodiff\nodiff$\mnodiffo$\nodiff$\monediffd$\nodiff\nodiff\nodiff\equaup$\monediffd$\nodiff$\mnodiffz$     & \nodiff\nodiff\nodiff\nodiff$\monediffu$$\monediffd$\nodiff\nodiff\nodiff\nodiff\nodiff\nodiff\nodiff\nodiff\nodiff\nodiff\nodiff\nodiff\nodiff\nodiff\nodiff\nodiff\nodiff\nodiff\nodiff\nodiff\nodiff\nodiff\nodiff\nodiff$\monediffu$\nodiff \\
4 & \nodiff$\monediffd$\nodiff$\mnodiffo$\nodiff\nodiff\nodiff$\mnodiffo$$\mnodiffo$\nodiff$\mnodiffz$\nodiff$\monediffu$\nodiff\nodiff\nodiff\nodiff$\mnodiffz$$\mnodiffz$$\monediffd$$\mnodiffo$$\mnodiffo$\nodiff\nodiff$\monediffd$\nodiff\equaup\nodiff\nodiff$\mnodiffz$$\monediffu$$\mnodiffo$     & $\monediffu$$\monediffd$\nodiff\nodiff\nodiff\nodiff\nodiff\nodiff\nodiff\nodiff\nodiff\nodiff\nodiff\nodiff\nodiff\nodiff\nodiff\nodiff\nodiff\nodiff\nodiff\nodiff\nodiff\nodiff\nodiff\nodiff\nodiff$\monediffd$\nodiff\nodiff\nodiff\nodiff \\
5 & $\mnodiffo$$\monediffu$\nodiff$\mnodiffz$\nodiff\equaup\nodiff$\mnodiffz$$\mnodiffz$\equaup$\mnodiffo$$\mnodiffo$$\mnodiffo$\nodiff$\monediffd$\nodiff$\mnodiffo$$\mnodiffz$$\mnodiffo$$\monediffd$$\mnodiffz$$\mnodiffz$\nodiff$\mnodiffo$\nodiff\nodiff$\monediffu$\nodiff$\monediffu$$\mnodiffo$\nodiff$\mnodiffz$     & $\monediffu$\nodiff$\monediffu$$\monediffu$\nodiff$\monediffd$\nodiff\nodiff\nodiff\nodiff\nodiff\nodiff\nodiff\nodiff\nodiff\nodiff\nodiff\nodiff\nodiff\nodiff\nodiff\nodiff\nodiff\nodiff\nodiff\nodiff\nodiff$\monediffd$$\monediffu$$\monediffd$\nodiff\nodiff \\
6 & $\mnodiffz$$\mnodiffz$\nodiff$\monediffd$\equaup$\monediffd$$\mnodiffo$$\monediffd$$\mnodiffo$$\monediffd$$\monediffd$$\monediffu$$\monediffu$$\monediffu$$\monediffu$\equaup$\mnodiffz$$\mnodiffo$$\monediffu$$\mnodiffz$$\monediffd$\nodiff$\monediffd$$\monediffd$\nodiff$\mnodiffo$$\mnodiffo$\nodiff$\mnodiffo$\nodiff$\mnodiffz$$\monediffd$     & \nodiff\nodiff$\monediffu$$\monediffd$$\monediffu$$\monediffu$\nodiff\nodiff\nodiff\nodiff\nodiff\nodiff\nodiff\nodiff\nodiff\nodiff\nodiff\nodiff\nodiff\nodiff\nodiff\nodiff\nodiff\nodiff\nodiff\nodiff\nodiff\nodiff\nodiff$\monediffu$\nodiff\nodiff \\
7 & $\mnodiffo$$\monediffu$\nodiff$\monediffd$\equaup$\monediffu$$\monediffd$$\mnodiffz$$\monediffd$$\monediffu$$\mnodiffo$$\mnodiffz$$\monediffu$$\monediffd$$\mnodiffz$\equaup$\monediffu$$\monediffu$$\monediffd$$\monediffu$$\mnodiffo$\equaup$\mnodiffo$\nodiff$\mnodiffz$$\monediffu$\nodiff\nodiff$\mnodiffz$\nodiff$\mnodiffo$$\mnodiffo$     & $\monediffu$\nodiff$\monediffu$$\monediffd$$\monediffd$$\monediffd$\nodiff\nodiff\nodiff\nodiff\nodiff\nodiff\nodiff\nodiff\nodiff\nodiff\nodiff\nodiff\nodiff\nodiff\nodiff\nodiff\nodiff\nodiff\nodiff\nodiff\nodiff$\monediffu$$\monediffd$\nodiff$\monediffd$\nodiff \\
8 & $\monediffu$$\monediffd$\nodiff$\mnodiffo$\nodiff\nodiff$\monediffu$$\monediffd$$\monediffd$$\monediffd$$\monediffd$$\monediffd$$\monediffd$$\monediffd$$\monediffd$$\monediffd$$\monediffd$$\monediffd$$\monediffd$$\monediffd$$\mnodiffo$$\monediffd$$\monediffu$$\monediffu$$\mnodiffo$$\mnodiffo$\nodiff$\monediffd$\nodiff\nodiff$\monediffu$$\mnodiffz$     & \nodiff\nodiff$\monediffd$\nodiff\nodiff\nodiff\nodiff\nodiff\nodiff\nodiff\nodiff\nodiff\nodiff\nodiff\nodiff\nodiff\nodiff\nodiff\nodiff\nodiff\nodiff\nodiff\nodiff\nodiff\nodiff\nodiff\nodiff$\monediffd$\nodiff\nodiff\nodiff\nodiff \\
9 & $\monediffd$$\monediffd$$\monediffu$$\mnodiffz$\nodiff$\mnodiffo$\nodiff$\mnodiffz$$\mnodiffo$$\mnodiffz$\nodiff$\mnodiffo$$\mnodiffz$$\mnodiffz$$\mnodiffz$\nodiff$\mnodiffo$$\mnodiffz$$\monediffu$$\monediffd$\nodiff$\mnodiffz$\nodiff$\mnodiffo$$\mnodiffz$$\mnodiffz$$\monediffd$$\mnodiffo$\nodiff$\mnodiffo$$\mnodiffz$\nodiff     & \nodiff\nodiff$\monediffu$\nodiff\nodiff$\monediffu$\nodiff\nodiff\nodiff\nodiff\nodiff\nodiff\nodiff\nodiff\nodiff\nodiff\nodiff\nodiff\nodiff\nodiff\nodiff\nodiff\nodiff\nodiff\nodiff\nodiff\nodiff$\monediffd$$\monediffu$$\monediffu$\nodiff\nodiff \\
10 & $\monediffd$$\mnodiffo$$\mnodiffz$\nodiff$\mnodiffo$\nodiff\nodiff\nodiff$\mnodiffo$$\mnodiffz$$\mnodiffz$$\mnodiffz$$\mnodiffo$$\mnodiffo$$\mnodiffo$$\mnodiffz$$\mnodiffo$$\mnodiffo$$\mnodiffz$$\mnodiffz$$\mnodiffo$\nodiff\nodiff$\mnodiffo$$\mnodiffo$$\mnodiffo$\nodiff\nodiff$\mnodiffo$$\mnodiffz$\nodiff\nodiff     & $\monediffd$$\monediffd$$\monediffu$\nodiff$\monediffu$$\monediffd$\nodiff\nodiff\nodiff\nodiff\nodiff\nodiff\nodiff\nodiff\nodiff\nodiff\nodiff\nodiff\nodiff\nodiff\nodiff\nodiff\nodiff\nodiff\nodiff\nodiff\nodiff$\monediffd$\nodiff$\monediffu$\nodiff\nodiff \\
11 & $\mnodiffo$\nodiff\nodiff$\monediffd$$\mnodiffo$\nodiff\nodiff\nodiff\nodiff\nodiff\nodiff\nodiff\nodiff\nodiff\nodiff\nodiff\nodiff\nodiff\nodiff\nodiff$\mnodiffo$$\mnodiffz$\nodiff\nodiff\nodiff\nodiff\nodiff\nodiff$\mnodiffo$\nodiff\equaup\nodiff     & \nodiff\nodiff\nodiff\nodiff$\monediffd$$\monediffu$\nodiff\nodiff\nodiff\nodiff\nodiff\nodiff\nodiff\nodiff\nodiff\nodiff\nodiff\nodiff\nodiff\nodiff\nodiff\nodiff\nodiff\nodiff\nodiff\nodiff\nodiff\nodiff\nodiff\nodiff$\monediffd$\nodiff \\
12 & $\monediffu$\nodiff$\monediffu$\nodiff\nodiff$\mnodiffo$\nodiff\nodiff\nodiff\nodiff\nodiff\nodiff\nodiff\nodiff\nodiff\nodiff\nodiff\nodiff\nodiff\nodiff\nodiff\nodiff\nodiff\nodiff\equaup\nodiff\nodiff\nodiff\nodiff\nodiff\nodiff\nodiff     & $\monediffd$$\monediffu$\nodiff\nodiff\nodiff\nodiff\nodiff\nodiff\nodiff\nodiff\nodiff\nodiff\nodiff\nodiff\nodiff\nodiff\nodiff\nodiff\nodiff\nodiff\nodiff\nodiff\nodiff\nodiff\nodiff\nodiff\nodiff$\monediffu$\nodiff\nodiff\nodiff\nodiff \\
13 & $\mnodiffz$\nodiff$\monediffu$\nodiff$\mnodiffo$$\mnodiffz$\nodiff\nodiff\nodiff\nodiff\nodiff\nodiff\nodiff\nodiff\nodiff\nodiff\nodiff\nodiff\nodiff\nodiff\nodiff\nodiff\nodiff\nodiff\nodiff\nodiff$\monediffd$\nodiff\nodiff\nodiff\nodiff$\mnodiffz$     & $\monediffu$\nodiff$\monediffd$$\monediffd$\nodiff$\monediffd$\nodiff\nodiff\nodiff\nodiff\nodiff\nodiff\nodiff\nodiff\nodiff\nodiff\nodiff\nodiff\nodiff\nodiff\nodiff\nodiff\nodiff\nodiff\nodiff\nodiff\nodiff$\monediffu$$\monediffd$$\monediffd$\nodiff\nodiff \\
14 & \nodiff\nodiff$\monediffu$\nodiff\nodiff\nodiff\nodiff\nodiff\nodiff\nodiff\nodiff\nodiff\nodiff\nodiff\nodiff\nodiff\nodiff\nodiff\nodiff\nodiff\nodiff\nodiff\nodiff\nodiff\nodiff\nodiff\nodiff\nodiff$\mnodiffz$\nodiff\nodiff$\mnodiffo$     & \nodiff\nodiff$\monediffd$\nodiff$\monediffd$$\monediffu$\nodiff\nodiff\nodiff\nodiff\nodiff\nodiff\nodiff\nodiff\nodiff\nodiff\nodiff\nodiff\nodiff\nodiff\nodiff\nodiff\nodiff\nodiff\nodiff\nodiff\nodiff\nodiff\nodiff$\monediffd$\nodiff\nodiff \\
15 & $\mnodiffo$$\monediffd$\nodiff\nodiff\nodiff\nodiff\nodiff\nodiff\nodiff\nodiff\nodiff\nodiff\nodiff\nodiff\nodiff\nodiff\nodiff\nodiff\nodiff\nodiff\nodiff\nodiff\nodiff\nodiff\nodiff\nodiff\nodiff\nodiff$\mnodiffo$\nodiff\diffup\nodiff     & $\monediffu$\nodiff$\monediffd$$\monediffu$$\monediffu$$\monediffu$\nodiff\nodiff\nodiff\nodiff\nodiff\nodiff\nodiff\nodiff\nodiff\nodiff\nodiff\nodiff\nodiff\nodiff\nodiff\nodiff\nodiff\nodiff\nodiff\nodiff\nodiff$\monediffu$$\monediffd$\nodiff\nodiff\nodiff \\
16 & $\monediffu$\nodiff\nodiff$\mnodiffo$$\mnodiffz$\nodiff\nodiff\nodiff\nodiff\nodiff\nodiff\nodiff\nodiff\nodiff\nodiff\nodiff\nodiff\nodiff\nodiff\nodiff\nodiff\nodiff\nodiff\nodiff\nodiff\nodiff\nodiff\nodiff\nodiff\nodiff\equaup\nodiff     & $\monediffd$\nodiff$\monediffd$$\monediffu$\nodiff\nodiff\nodiff\nodiff\nodiff\nodiff\nodiff\nodiff\nodiff\nodiff\nodiff\nodiff\nodiff\nodiff\nodiff\nodiff\nodiff\nodiff\nodiff\nodiff\nodiff\nodiff\nodiff$\monediffd$\nodiff\nodiff\nodiff\nodiff \\
17 & $\monediffu$\nodiff$\monediffd$$\mnodiffo$\nodiff\nodiff\nodiff\nodiff\nodiff\nodiff\nodiff\nodiff\nodiff\nodiff\nodiff\nodiff\nodiff\nodiff\nodiff\nodiff\nodiff\nodiff\nodiff\nodiff\nodiff\nodiff\nodiff\nodiff\nodiff\nodiff\diffup\nodiff     & \nodiff\nodiff\nodiff\nodiff\nodiff\nodiff\nodiff\nodiff\nodiff\nodiff\nodiff\nodiff\nodiff\nodiff\nodiff\nodiff\nodiff\nodiff\nodiff\nodiff\nodiff\nodiff\nodiff\nodiff\nodiff\nodiff\nodiff\nodiff$\monediffd$$\monediffu$\nodiff\nodiff \\
18 & $\monediffu$\nodiff\nodiff\nodiff$\mnodiffo$\nodiff\nodiff\nodiff\nodiff\nodiff\nodiff\nodiff\nodiff\nodiff\nodiff\nodiff\nodiff\nodiff\nodiff\nodiff\nodiff\nodiff\nodiff\nodiff\nodiff\nodiff\nodiff\nodiff\nodiff\nodiff\nodiff\equaup     & $\monediffu$\nodiff$\monediffu$$\monediffd$$\monediffu$\nodiff\nodiff\nodiff\nodiff\nodiff\nodiff\nodiff\nodiff\nodiff\nodiff\nodiff\nodiff\nodiff\nodiff\nodiff\nodiff\nodiff\nodiff\nodiff\nodiff\nodiff\nodiff$\monediffd$\nodiff\nodiff\nodiff\nodiff \\
19 & \nodiff$\monediffd$\nodiff\nodiff\diffrightup\nodiff\nodiff\nodiff\nodiff\nodiff\nodiff\nodiff\nodiff\nodiff\nodiff\nodiff\nodiff\nodiff\nodiff\nodiff\nodiff\nodiff\nodiff\nodiff\nodiff\nodiff\nodiff\nodiff\nodiff\nodiff\nodiff\nodiff     & \nodiff\nodiff\nodiff\nodiff$\monediffd$\nodiff\nodiff\nodiff\nodiff\nodiff\nodiff\nodiff\nodiff\nodiff\nodiff\nodiff\nodiff\nodiff\nodiff\nodiff\nodiff\nodiff\nodiff\nodiff\nodiff\nodiff\nodiff$\monediffu$$\monediffd$\nodiff\nodiff\nodiff \\
20 & $\monediffd$\nodiff\diffrightup\diffrightupup\nodiff\nodiff\nodiff\nodiff\nodiff\nodiff\nodiff\nodiff\nodiff\nodiff\nodiff\nodiff\nodiff\nodiff\nodiff\nodiff\nodiff\nodiff\nodiff\nodiff\nodiff\nodiff\nodiff\nodiff\nodiff\nodiff\nodiff\nodiff     & \nodiff$\monediffu$$\monediffu$$\monediffu$$\monediffu$\nodiff\nodiff\nodiff\nodiff\nodiff\nodiff\nodiff\nodiff\nodiff\nodiff\nodiff\nodiff\nodiff\nodiff\nodiff\nodiff\nodiff\nodiff\nodiff\nodiff\nodiff\nodiff$\monediffu$\nodiff\nodiff\nodiff\nodiff \\
21 & $\monediffd$\nodiff$\monediffd$\diffrightup\nodiff\nodiff\nodiff\nodiff\nodiff\nodiff\nodiff\nodiff\nodiff\nodiff\nodiff\nodiff\nodiff\nodiff\nodiff\nodiff\nodiff\nodiff\nodiff\nodiff\nodiff\nodiff\nodiff\nodiff\nodiff\nodiff\nodiff\nodiff     & \nodiff\nodiff\nodiff\nodiff$\monediffd$\nodiff\nodiff\nodiff\nodiff\nodiff\nodiff\nodiff\nodiff\nodiff\nodiff\nodiff\nodiff\nodiff\nodiff\nodiff\nodiff\nodiff\nodiff\nodiff\nodiff\nodiff\nodiff$\monediffu$\nodiff$\monediffu$\nodiff\nodiff \\
22 & $\monediffd$\nodiff\nodiff\nodiff\equarightupup\nodiff\nodiff\nodiff\nodiff\nodiff\nodiff\nodiff\nodiff\nodiff\nodiff\nodiff\nodiff\nodiff\nodiff\nodiff\nodiff\nodiff\nodiff\nodiff\nodiff\nodiff\nodiff\nodiff\nodiff\nodiff\nodiff\nodiff     & \nodiff$\monediffd$$\monediffu$$\monediffu$\nodiff\nodiff\nodiff\nodiff\nodiff\nodiff\nodiff\nodiff\nodiff\nodiff\nodiff\nodiff\nodiff\nodiff\nodiff\nodiff\nodiff\nodiff\nodiff\nodiff\nodiff\nodiff\nodiff$\monediffu$\nodiff\nodiff\nodiff\nodiff \\
23 & $\monediffd$\nodiff$\monediffu$\nodiff\equarightup\nodiff\nodiff\nodiff\nodiff\nodiff\nodiff\nodiff\nodiff\nodiff\nodiff\nodiff\nodiff\nodiff\nodiff\nodiff\nodiff\nodiff\nodiff\nodiff\nodiff\nodiff\nodiff\nodiff\nodiff\nodiff\nodiff\nodiff     & $\monediffd$\nodiff$\monediffu$$\monediffd$$\monediffu$\nodiff\nodiff\nodiff\nodiff\nodiff\nodiff\nodiff\nodiff\nodiff\nodiff\nodiff\nodiff\nodiff\nodiff\nodiff\nodiff\nodiff\nodiff\nodiff\nodiff\nodiff\nodiff$\monediffu$$\monediffd$$\monediffu$\nodiff\nodiff \\
24 & \nodiff\nodiff\nodiff\nodiff\diffrightupup\nodiff\nodiff\nodiff\nodiff\nodiff\nodiff\nodiff\nodiff\nodiff\nodiff\nodiff\nodiff\nodiff\nodiff\nodiff\nodiff\nodiff\nodiff\nodiff\nodiff\nodiff\nodiff\nodiff\nodiff\nodiff\nodiff\nodiff     & $\monediffd$$\monediffd$$\monediffu$\nodiff$\monediffu$\nodiff\nodiff\nodiff\nodiff\nodiff\nodiff\nodiff\nodiff\nodiff\nodiff\nodiff\nodiff\nodiff\nodiff\nodiff\nodiff\nodiff\nodiff\nodiff\nodiff\nodiff\nodiff\nodiff\nodiff\nodiff\nodiff\nodiff \\
25 & \nodiff\nodiff$\monediffd$\nodiff\diffrightup\nodiff\nodiff\nodiff\nodiff\nodiff\nodiff\nodiff\nodiff\nodiff\nodiff\nodiff\nodiff\nodiff\nodiff\nodiff\nodiff\nodiff\nodiff\nodiff\nodiff\nodiff\nodiff\nodiff\nodiff\nodiff\nodiff\nodiff     & $\monediffd$\nodiff$\monediffu$$\monediffu$\nodiff\nodiff\nodiff\nodiff\nodiff\nodiff\nodiff\nodiff\nodiff\nodiff\nodiff\nodiff\nodiff\nodiff\nodiff\nodiff\nodiff\nodiff\nodiff\nodiff\nodiff\nodiff\nodiff\nodiff\nodiff$\monediffu$\nodiff\nodiff \\
26 & $\monediffd$\nodiff\nodiff\nodiff\diffrightupup\nodiff\nodiff\nodiff\nodiff\nodiff\nodiff\nodiff\nodiff\nodiff\nodiff\nodiff\nodiff\nodiff\nodiff\nodiff\nodiff\nodiff\nodiff\nodiff\nodiff\nodiff\nodiff\nodiff\nodiff\nodiff\nodiff\equaup     & \nodiff$\monediffd$\nodiff$\monediffu$$\monediffd$\nodiff\nodiff\nodiff\nodiff\nodiff\nodiff\nodiff\nodiff\nodiff\nodiff\nodiff\nodiff\nodiff\nodiff\nodiff\nodiff\nodiff\nodiff\nodiff\nodiff\nodiff\nodiff$\monediffu$\nodiff\nodiff\nodiff\nodiff \\
27 & \nodiff$\monediffd$$\monediffd$\nodiff\equarightup\nodiff\nodiff\nodiff\nodiff\nodiff\nodiff\nodiff\nodiff\nodiff\nodiff\nodiff\nodiff\nodiff\nodiff\nodiff\nodiff\nodiff\nodiff\nodiff\nodiff\nodiff\nodiff\nodiff\nodiff\nodiff\nodiff\nodiff     & $\monediffu$\nodiff$\monediffu$$\monediffd$\nodiff\nodiff\nodiff\nodiff\nodiff\nodiff\nodiff\nodiff\nodiff\nodiff\nodiff\nodiff\nodiff\nodiff\nodiff\nodiff\nodiff\nodiff\nodiff\nodiff\nodiff\nodiff\nodiff\nodiff$\monediffu$$\monediffu$\nodiff\nodiff \\
28 & \nodiff\nodiff\equarightup\diffrightupup\diffrightupup\nodiff\nodiff\nodiff\nodiff\nodiff\nodiff\nodiff\nodiff\nodiff\nodiff\nodiff\nodiff\nodiff\nodiff\nodiff\nodiff\nodiff\nodiff\nodiff\nodiff\nodiff\nodiff\nodiff\nodiff\nodiff\nodiff\nodiff     & \nodiff$\monediffu$\nodiff\nodiff$\monediffu$\nodiff\nodiff\nodiff\nodiff\nodiff\nodiff\nodiff\nodiff\nodiff\nodiff\nodiff\nodiff\nodiff\nodiff\nodiff\nodiff\nodiff\nodiff\nodiff\nodiff\nodiff\nodiff\nodiff\nodiff\nodiff\nodiff\nodiff \\
29 & \nodiff\nodiff\nodiff\equarightup\equarightup\nodiff\nodiff\nodiff\nodiff\nodiff\nodiff\nodiff\nodiff\nodiff\nodiff\nodiff\nodiff\nodiff\nodiff\nodiff\nodiff\nodiff\nodiff\nodiff\nodiff\nodiff\nodiff\nodiff\nodiff\nodiff\nodiff\nodiff    & %$\monediffu$\nodiff$\monediffu$$\monediffd$\nodiff\nodiff\nodiff\nodiff\nodiff\nodiff\nodiff\nodiff\nodiff\nodiff\nodiff\nodiff\nodiff\nodiff\nodiff\nodiff\nodiff\nodiff\nodiff\nodiff\nodiff\nodiff\nodiff\nodiff\nodiff\nodiff\nodiff\nodiff \\
%30 & $\monediffd$\nodiff\nodiff\nodiff\nodiff\nodiff\nodiff\nodiff\nodiff\nodiff\nodiff\nodiff\nodiff\nodiff\nodiff\nodiff\nodiff\nodiff\nodiff\nodiff\nodiff\nodiff\nodiff\nodiff\nodiff\nodiff\nodiff\nodiff\nodiff\nodiff\nodiff\nodiff     & $\monediffd$\nodiff$\monediffu$$\monediffu$$\monediffu$\nodiff\nodiff\nodiff\nodiff\nodiff\nodiff\nodiff\nodiff\nodiff\nodiff\nodiff\nodiff\nodiff\nodiff\nodiff\nodiff\nodiff\nodiff\nodiff\nodiff\nodiff\nodiff$\monediffu$\nodiff\nodiff\nodiff\nodiff \\
%31 & $\monediffd$\nodiff\equarightupup\nodiff\nodiff\nodiff\nodiff\nodiff\nodiff\nodiff\nodiff\nodiff\nodiff\nodiff\nodiff\nodiff\nodiff\nodiff\nodiff\nodiff\nodiff\nodiff\nodiff\nodiff\nodiff\nodiff\nodiff\nodiff\nodiff\nodiff\nodiff\nodiff     & $\monediffd$\nodiff\nodiff$\monediffu$$\monediffu$\nodiff\nodiff\nodiff\nodiff\nodiff\nodiff\nodiff\nodiff\nodiff\nodiff\nodiff\nodiff\nodiff\nodiff\nodiff\nodiff\nodiff\nodiff\nodiff\nodiff\nodiff\nodiff$\monediffu$\nodiff\nodiff\nodiff\nodiff \\
%32 & \nodiff\nodiff\nodiff\nodiff\nodiff\nodiff\nodiff\nodiff\nodiff\nodiff\nodiff\nodiff\nodiff\nodiff\nodiff\nodiff\nodiff\nodiff\nodiff\nodiff\nodiff\nodiff\nodiff\nodiff\nodiff\nodiff\nodiff\nodiff\nodiff\nodiff\nodiff\nodiff     & $\monediffd$\nodiff$\monediffu$\nodiff\nodiff\nodiff\nodiff\nodiff\nodiff\nodiff\nodiff\nodiff\nodiff\nodiff\nodiff\nodiff\nodiff\nodiff\nodiff\nodiff\nodiff\nodiff\nodiff\nodiff\nodiff\nodiff\nodiff\nodiff\nodiff\nodiff\nodiff\nodiff \\
%33 & \nodiff\nodiff\diffrightup\nodiff\nodiff\nodiff\nodiff\nodiff\nodiff\nodiff\nodiff\nodiff\nodiff\nodiff\nodiff\nodiff\nodiff\nodiff\nodiff\nodiff\nodiff\nodiff\nodiff\nodiff\nodiff\nodiff\nodiff\nodiff\nodiff\nodiff\nodiff\nodiff     & \nodiff\nodiff\nodiff\nodiff\nodiff\nodiff\nodiff\nodiff\nodiff\nodiff\nodiff\nodiff\nodiff\nodiff\nodiff\nodiff\nodiff\nodiff\nodiff\nodiff\nodiff\nodiff\nodiff\nodiff\nodiff\nodiff\nodiff\nodiff\nodiff\nodiff\nodiff\nodiff \\
%34 & \nodiff\nodiff\nodiff\nodiff\nodiff\nodiff\nodiff\nodiff\nodiff\nodiff\nodiff\nodiff\nodiff\nodiff\nodiff\nodiff\nodiff\nodiff\nodiff\nodiff\nodiff\nodiff\nodiff\nodiff\nodiff\nodiff\nodiff\nodiff\nodiff\nodiff\nodiff\nodiff     & \nodiff\nodiff\nodiff\nodiff\nodiff\nodiff\nodiff\nodiff\nodiff\nodiff\nodiff\nodiff\nodiff\nodiff\nodiff\nodiff\nodiff\nodiff\nodiff\nodiff\nodiff\nodiff\nodiff\nodiff\nodiff\nodiff\nodiff\nodiff\nodiff\nodiff\nodiff\nodiff \\
%35 & \nodiff\nodiff\nodiff\nodiff\nodiff\nodiff\nodiff\nodiff\nodiff\nodiff\nodiff\nodiff\nodiff\nodiff\nodiff\nodiff\nodiff\nodiff\nodiff\nodiff\nodiff\nodiff\nodiff\nodiff\nodiff\nodiff\nodiff\nodiff\nodiff\nodiff\nodiff\nodiff     & \nodiff\nodiff$\monediffu$\nodiff\nodiff\nodiff\nodiff\nodiff\nodiff\nodiff\nodiff\nodiff\nodiff\nodiff\nodiff\nodiff\nodiff\nodiff\nodiff\nodiff\nodiff\nodiff\nodiff\nodiff\nodiff\nodiff\nodiff\nodiff\nodiff\nodiff\nodiff\nodiff \\
%36 & \nodiff\nodiff\nodiff\nodiff\nodiff\nodiff\nodiff\nodiff\nodiff\nodiff\nodiff\nodiff\nodiff\nodiff\nodiff\nodiff\nodiff\nodiff\nodiff\nodiff\nodiff\nodiff\nodiff\nodiff\nodiff\nodiff\nodiff\nodiff\nodiff\nodiff\nodiff\nodiff \\
\end{tabular}
\caption{The differential path of the 76-step attack up to step 29.\label{fig:diff_path76}}
\end{figure}

\begingroup
\fontsize{8pt}{9pt}\selectfont
\begin{verbbox}
- W54[29] ^ W55[29] = 0
- W53[29] ^ W55[29] = 0
- W51[4] ^ W55[29] = 0
- W49[29] ^ W50[29] = 0
- W48[29] ^ W50[29] = 0
- W47[28] ^ W47[29] = 1
- W46[4] ^ W50[29] = 0
- W46[28] ^ W47[28] = 0
- W46[29] ^ W47[28] = 1
- W45[28] ^ W47[28] = 0
- W44[29] ^ W44[30] = 0
- W43[3] ^ W47[28] = 0
- W43[4] ^ W47[28] = 1
- W42[29] ^ W47[28] = 1
- W41[4] ^ W47[28] = 0
- W40[29] ^ W47[28] = 0
- W39[4] ^ W47[28] = 1
- W37[4] ^ W47[28] = 0
- W59[4] ^ W63[29] = 0
- W57[4] ^ W59[29] = 0
- W74[0] = 1
- W75[5] = 0
- W73[1] ^ W74[6] = 1
- W71[5] ^ W75[30] = 0
- W70[0] ^ W75[30] = 1
\end{verbbox}
\endgroup
\begin{figure}
\begin{center}
\begin{tabular}{l l l}
 $\expmess_{54}[29] \oplus \expmess_{55}[29] = 0$ & $\expmess_{53}[29] \oplus \expmess_{55}[29] = 0$ & $\expmess_{51}[4]  \oplus \expmess_{55}[29] = 0$\\
 $\expmess_{49}[29] \oplus \expmess_{50}[29] = 0$ & $\expmess_{48}[29] \oplus \expmess_{50}[29] = 0$ & $\expmess_{47}[28] \oplus \expmess_{47}[29] = 1$\\
 $\expmess_{46}[4]  \oplus \expmess_{50}[29] = 0$ & $\expmess_{46}[28] \oplus \expmess_{47}[28] = 0$ & $\expmess_{46}[29] \oplus \expmess_{47}[28] = 1$\\
 $\expmess_{45}[28] \oplus \expmess_{47}[28] = 0$ & $\expmess_{44}[29] \oplus \expmess_{44}[30] = 0$ & $\expmess_{43}[3]  \oplus \expmess_{47}[28] = 0$\\
 $\expmess_{43}[4]  \oplus \expmess_{47}[28] = 1$ & $\expmess_{42}[29] \oplus \expmess_{47}[28] = 1$ & $\expmess_{41}[4]  \oplus \expmess_{47}[28] = 0$\\
 $\expmess_{40}[29] \oplus \expmess_{47}[28] = 0$ & $\expmess_{39}[4]  \oplus \expmess_{47}[28] = 1$ & $\expmess_{37}[4]  \oplus \expmess_{47}[28] = 0$\\
 $\expmess_{59}[4]  \oplus \expmess_{63}[29] = 0$ & $\expmess_{57}[4]  \oplus \expmess_{59}[29] = 0$ & $\expmess_{74}[0] = 1$\\ 
 $\expmess_{75}[5] = 0$ & $\expmess_{73}[1]  \oplus \expmess_{74}[6] = 1$ & $\expmess_{71}[5]  \oplus \expmess_{75}[30] = 0$\\
 $\expmess_{70}[0]  \oplus \expmess_{75}[30] = 1$\\
\end{tabular}
\end{center}
  \caption{The message bit relations of the 76-step attack for message words $\expmess_{36}$ to $\expmess_{75}$.
  \label{fig:msgbitrel76}}
\end{figure}

\begingroup
\fontsize{8pt}{9pt}\selectfont
\begin{verbbox}
W36:	 . . . . . . . . . . . . . . . . . . . . . . . . . . . . . . . .
W37:	 . . . . . . . . . . . . . . . . . . . . . . . . . . . a . . . .
W38:	 . . . . . . . . . . . . . . . . . . . . . . . . . . . . . . . .
W39:	 . . . . . . . . . . . . . . . . . . . . . . . . . . . A . . . .
W40:	 . . a . . . . . . . . . . . . . . . . . . . . . . . . . . . . .
W41:	 . . . . . . . . . . . . . . . . . . . . . . . . . . . a . . . .
W42:	 . . A . . . . . . . . . . . . . . . . . . . . . . . . . . . . .
W43:	 . . . . . . . . . . . . . . . . . . . . . . . . . . . A a . . .
W44:	 . b b . . . . . . . . . . . . . . . . . . . . . . . . . . . . .
W45:	 . . . a . . . . . . . . . . . . . . . . . . . . . . . . . . . .
W46:	 . . A a . . . . . . . . . . . . . . . . . . . . . . . c . . . .
W47:	 . . A a . . . . . . . . . . . . . . . . . . . . . . . . . . . .
W48:	 . . c . . . . . . . . . . . . . . . . . . . . . . . . . . . . .
W49:	 . . c . . . . . . . . . . . . . . . . . . . . . . . . . . . . .
W50:	 . . c . . . . . . . . . . . . . . . . . . . . . . . . . . . . .
W51:	 . . . . . . . . . . . . . . . . . . . . . . . . . . . d . . . .
W52:	 . . . . . . . . . . . . . . . . . . . . . . . . . . . . . . . .
W53:	 . . d . . . . . . . . . . . . . . . . . . . . . . . . . . . . .
W54:	 . . d . . . . . . . . . . . . . . . . . . . . . . . . . . . . .
W55:	 . . d . . . . . . . . . . . . . . . . . . . . . . . . . . . . .
W56:	 . . . . . . . . . . . . . . . . . . . . . . . . . . . . . . . .
W57:	 . . . . . . . . . . . . . . . . . . . . . . . . . . . e . . . .
W58:	 . . . . . . . . . . . . . . . . . . . . . . . . . . . . . . . .
W59:	 . . e . . . . . . . . . . . . . . . . . . . . . . . . f . . . .
W60:	 . . . . . . . . . . . . . . . . . . . . . . . . . . . . . . . .
W61:	 . . . . . . . . . . . . . . . . . . . . . . . . . . . . . . . .
W62:	 . . . . . . . . . . . . . . . . . . . . . . . . . . . . . . . .
W63:	 . . f . . . . . . . . . . . . . . . . . . . . . . . . . . . . .
W64:	 . . . . . . . . . . . . . . . . . . . . . . . . . . . . . . . .
W65:	 . . . . . . . . . . . . . . . . . . . . . . . . . . . . . . . .
W66:	 . . . . . . . . . . . . . . . . . . . . . . . . . . . . . . . .
W67:	 . . . . . . . . . . . . . . . . . . . . . . . . . . . . . . . .
W68:	 . . . . . . . . . . . . . . . . . . . . . . . . . . . . . . . .
W69:	 . . . . . . . . . . . . . . . . . . . . . . . . . . . . . . . .
W70:	 . . . . . . . . . . . . . . . . . . . . . . . . . . . . . . . g
W71:	 . . . . . . . . . . . . . . . . . . . . . . . . . . G . . . . .
W72:	 . . . . . . . . . . . . . . . . . . . . . . . . . . . . . . . .
W73:	 . . . . . . . . . . . . . . . . . . . . . . . . . . . . . . h .
W74:	 . . . . . . . . . . . . . . . . . . . . . . . . . H . . . . . 1
W75:	 . G . . . . . . . . . . . . . . . . . . . . . . . . 0 . . . . .
\end{verbbox}
\endgroup

\begin{figure}[ht]
\centering
  \theverbbox
  \caption[The message bit-relations used in the attack for words $\expmess_{36}$ to $\expmess_{75}$ (graphical representation).]{The message bit-relations used in the attack for words $\expmess_{36}$ to $\expmess_{75}$ (graphical representation).
  A dot (``\texttt{.}'') means an absence of condition. A zero (``\texttt{0}'') or one (``\texttt{1}'') character represents a bit unconditionally set to 0 or 1.
  A pair of two identical letters $x$ means that the two bits have the same value. A pair of two
  letters $x$ and $X$ means that the two bits have different values.
  \label{fig:msgbitrel76_graph}}
\end{figure}

\subsection{The neutral bits}
\label{sec:neutral_bits76}
We give here the list of the neutral bits used in the attack.
There are fifty-one of them over the eight message words
$\expmess_{14}$ to $\expmess_{21}$, distributed as
follows:
\begin{itemize}
\item $\expmess_{14}$: 9 neutral bits at  bit positions (starting with the least significant bit (\emph{LSB}) at zero) 5,6,7,8,9,10,11,12,13
\item $\expmess_{15}$: 11 neutral bits at positions 5,6,7,8,9,10,11,12,13,14,16
\item $\expmess_{16}$: 8 neutral bits at positions 6,7,8,9,10,11,13,16
\item $\expmess_{17}$: 5 neutral bits at positions 10,11,12,13,19
\item $\expmess_{18}$: 2 neutral bits at positions 15,16
\item $\expmess_{19}$: 8 neutral bits at positions 6,7,8,9,10,11,12,14
\item $\expmess_{20}$: 5 neutral bits at positions 0,6,11,12,13
\item $\expmess_{21}$: 3 neutral bits at positions 11,16,17
\end{itemize}
We give a graphical representation of the repartition of these neutral bits in \autoref{fig:neutbits76}.

\begin{figure}
\centering
\begin{tabular}{l c}
$\expmess_{14}$: & \nodiff\nodiff\nodiff\nodiff\nodiff\nodiff\nodiff\nodiff\nodiff\nodiff\nodiff\nodiff\nodiff\nodiff\nodiff\nodiff\nodiff\nodiff\onediff\onediff\onediff\onediff\onediff\onediff\onediff\onediff\onediff\nodiff\nodiff\nodiff\nodiff\nodiff \\
$\expmess_{15}$: & \nodiff\nodiff\nodiff\nodiff\nodiff\nodiff\nodiff\nodiff\nodiff\nodiff\nodiff\nodiff\nodiff\nodiff\nodiff\onediff\nodiff\onediff\onediff\onediff\onediff\onediff\onediff\onediff\onediff\onediff\onediff\nodiff\nodiff\nodiff\nodiff\nodiff \\
$\expmess_{16}$: & \nodiff\nodiff\nodiff\nodiff\nodiff\nodiff\nodiff\nodiff\nodiff\nodiff\nodiff\nodiff\nodiff\nodiff\nodiff\onediff\nodiff\nodiff\onediff\nodiff\onediff\onediff\onediff\onediff\onediff\onediff\nodiff\nodiff\nodiff\nodiff\nodiff\nodiff \\
$\expmess_{17}$: & \nodiff\nodiff\nodiff\nodiff\nodiff\nodiff\nodiff\nodiff\nodiff\nodiff\nodiff\nodiff\onediff\nodiff\nodiff\nodiff\nodiff\nodiff\onediff\onediff\onediff\onediff\nodiff\nodiff\nodiff\nodiff\nodiff\nodiff\nodiff\nodiff\nodiff\nodiff \\
$\expmess_{18}$: & \nodiff\nodiff\nodiff\nodiff\nodiff\nodiff\nodiff\nodiff\nodiff\nodiff\nodiff\nodiff\nodiff\nodiff\nodiff\onediff\onediff\nodiff\nodiff\nodiff\nodiff\nodiff\nodiff\nodiff\nodiff\nodiff\nodiff\nodiff\nodiff\nodiff\nodiff\nodiff \\
$\expmess_{19}$: & \nodiff\nodiff\nodiff\nodiff\nodiff\nodiff\nodiff\nodiff\nodiff\nodiff\nodiff\nodiff\nodiff\nodiff\nodiff\nodiff\nodiff\onediff\nodiff\onediff\onediff\onediff\onediff\onediff\onediff\onediff\nodiff\nodiff\nodiff\nodiff\nodiff\nodiff \\
$\expmess_{20}$: & \nodiff\nodiff\nodiff\nodiff\nodiff\nodiff\nodiff\nodiff\nodiff\nodiff\nodiff\nodiff\nodiff\nodiff\nodiff\nodiff\nodiff\nodiff\onediff\onediff\onediff\nodiff\nodiff\nodiff\nodiff\onediff\nodiff\nodiff\nodiff\nodiff\nodiff\onediff \\
$\expmess_{21}$: & \nodiff\nodiff\nodiff\nodiff\nodiff\nodiff\nodiff\nodiff\nodiff\nodiff\nodiff\nodiff\nodiff\nodiff\onediff\onediff\nodiff\nodiff\nodiff\nodiff\onediff\nodiff\nodiff\nodiff\nodiff\nodiff\nodiff\nodiff\nodiff\nodiff\nodiff\nodiff \\
\end{tabular}
  \caption[The fifty-one neutral bits of the 76-step attack.]{The fifty-one neutral bits of the 76-step attack, using (with some abuse) a ``difference'' notation.
  A ``\nodiff'' (resp. ``\onediff'') symbol means the absence (resp. presence) of a neutral bit on the corresponding bit.
  The message words are (as usual) written left to right from MSB to LSB.
  \label{fig:neutbits76}}
\end{figure}

Not all of the neutral bits located on the same word (say $\expmess_{14}$) are neutral up to the same state word. Their repartition
in that respect is as follows
\begin{itemize}
	\item Bits neutral up to step 18 (excluded): $\expmess_{14}$[9,10,11,12,13], $\expmess_{15}$[14,16]
	\item Bits neutral up to step 19 (excluded): $\expmess_{14}$[5,6,7,8], $\expmess_{15}$[8,9,10,11,12,13], $\expmess_{16}$[13,16], $\expmess_{17}$[19]
	\item Bits neutral up to step 20 (excluded): $\expmess_{15}$[5,6,7], $\expmess_{16}$[9,10,11]
	\item Bits neutral up to step 21 (excluded): $\expmess_{16}$[6,7,8], $\expmess_{17}$[10,11,12,13], $\expmess_{18}$[15,16]
	\item Bits neutral up to step 23 (excluded): $\expmess_{19}$[9,10,11,12,14]
	\item Bits neutral up to step 24 (excluded): $\expmess_{19}$[6,7], $\expmess_{20}$[11,12], $\expmess_{21}$[16,17]
	\item Bits neutral up to step 25 (excluded): $\expmess_{19}$[8], $\expmess_{20}$[6,13], $\expmess_{21}$[11]
	\item Bits neutral up to step 26 (excluded): $\expmess_{20}$[0]
\end{itemize}
We also give a graphical representation of this repartition in \autoref{fig:neutbits76_2}.

\begin{figure}
\centering
\begin{tabular}{l c}
$\state_{18}$ \\
$\expmess_{14}$: & \nodiff\nodiff\nodiff\nodiff\nodiff\nodiff\nodiff\nodiff\nodiff\nodiff\nodiff\nodiff\nodiff\nodiff\nodiff\nodiff\nodiff\nodiff\onediff\onediff\onediff\onediff\onediff\nodiff\nodiff\nodiff\nodiff\nodiff\nodiff\nodiff\nodiff\nodiff \\
$\expmess_{15}$: & \nodiff\nodiff\nodiff\nodiff\nodiff\nodiff\nodiff\nodiff\nodiff\nodiff\nodiff\nodiff\nodiff\nodiff\nodiff\onediff\nodiff\onediff\nodiff\nodiff\nodiff\nodiff\nodiff\nodiff\nodiff\nodiff\nodiff\nodiff\nodiff\nodiff\nodiff\nodiff \\

$\state_{19}$ \\
$\expmess_{14}$: & \nodiff\nodiff\nodiff\nodiff\nodiff\nodiff\nodiff\nodiff\nodiff\nodiff\nodiff\nodiff\nodiff\nodiff\nodiff\nodiff\nodiff\nodiff\nodiff\nodiff\nodiff\nodiff\nodiff\onediff\onediff\onediff\onediff\nodiff\nodiff\nodiff\nodiff\nodiff \\
$\expmess_{15}$: & \nodiff\nodiff\nodiff\nodiff\nodiff\nodiff\nodiff\nodiff\nodiff\nodiff\nodiff\nodiff\nodiff\nodiff\nodiff\nodiff\nodiff\nodiff\onediff\onediff\onediff\onediff\onediff\onediff\nodiff\nodiff\nodiff\nodiff\nodiff\nodiff\nodiff\nodiff \\
$\expmess_{16}$: & \nodiff\nodiff\nodiff\nodiff\nodiff\nodiff\nodiff\nodiff\nodiff\nodiff\nodiff\nodiff\nodiff\nodiff\nodiff\onediff\nodiff\nodiff\onediff\nodiff\nodiff\nodiff\nodiff\nodiff\nodiff\nodiff\nodiff\nodiff\nodiff\nodiff\nodiff\nodiff \\
$\expmess_{17}$: & \nodiff\nodiff\nodiff\nodiff\nodiff\nodiff\nodiff\nodiff\nodiff\nodiff\nodiff\nodiff\onediff\nodiff\nodiff\nodiff\nodiff\nodiff\nodiff\nodiff\nodiff\nodiff\nodiff\nodiff\nodiff\nodiff\nodiff\nodiff\nodiff\nodiff\nodiff\nodiff \\

$\state_{20}$ \\
$\expmess_{15}$: & \nodiff\nodiff\nodiff\nodiff\nodiff\nodiff\nodiff\nodiff\nodiff\nodiff\nodiff\nodiff\nodiff\nodiff\nodiff\nodiff\nodiff\nodiff\nodiff\nodiff\nodiff\nodiff\nodiff\nodiff\onediff\onediff\onediff\nodiff\nodiff\nodiff\nodiff\nodiff \\
$\expmess_{16}$: & \nodiff\nodiff\nodiff\nodiff\nodiff\nodiff\nodiff\nodiff\nodiff\nodiff\nodiff\nodiff\nodiff\nodiff\nodiff\nodiff\nodiff\nodiff\nodiff\nodiff\onediff\onediff\onediff\nodiff\nodiff\nodiff\nodiff\nodiff\nodiff\nodiff\nodiff\nodiff \\

$\state_{21}$\\
$\expmess_{16}$: & \nodiff\nodiff\nodiff\nodiff\nodiff\nodiff\nodiff\nodiff\nodiff\nodiff\nodiff\nodiff\nodiff\nodiff\nodiff\nodiff\nodiff\nodiff\nodiff\nodiff\nodiff\nodiff\nodiff\onediff\onediff\onediff\nodiff\nodiff\nodiff\nodiff\nodiff\nodiff \\
$\expmess_{17}$: & \nodiff\nodiff\nodiff\nodiff\nodiff\nodiff\nodiff\nodiff\nodiff\nodiff\nodiff\nodiff\nodiff\nodiff\nodiff\nodiff\nodiff\nodiff\onediff\onediff\onediff\onediff\nodiff\nodiff\nodiff\nodiff\nodiff\nodiff\nodiff\nodiff\nodiff\nodiff \\
$\expmess_{18}$: & \nodiff\nodiff\nodiff\nodiff\nodiff\nodiff\nodiff\nodiff\nodiff\nodiff\nodiff\nodiff\nodiff\nodiff\nodiff\onediff\onediff\nodiff\nodiff\nodiff\nodiff\nodiff\nodiff\nodiff\nodiff\nodiff\nodiff\nodiff\nodiff\nodiff\nodiff\nodiff \\

$\state_{23}$\\
$\expmess_{19}$: & \nodiff\nodiff\nodiff\nodiff\nodiff\nodiff\nodiff\nodiff\nodiff\nodiff\nodiff\nodiff\nodiff\nodiff\nodiff\nodiff\nodiff\onediff\nodiff\onediff\onediff\onediff\onediff\nodiff\nodiff\nodiff\nodiff\nodiff\nodiff\nodiff\nodiff\nodiff \\

$\state_{24}$\\
$\expmess_{19}$: & \nodiff\nodiff\nodiff\nodiff\nodiff\nodiff\nodiff\nodiff\nodiff\nodiff\nodiff\nodiff\nodiff\nodiff\nodiff\nodiff\nodiff\nodiff\nodiff\nodiff\nodiff\nodiff\nodiff\nodiff\onediff\onediff\nodiff\nodiff\nodiff\nodiff\nodiff\nodiff \\
$\expmess_{20}$: & \nodiff\nodiff\nodiff\nodiff\nodiff\nodiff\nodiff\nodiff\nodiff\nodiff\nodiff\nodiff\nodiff\nodiff\nodiff\nodiff\nodiff\nodiff\nodiff\onediff\onediff\nodiff\nodiff\nodiff\nodiff\nodiff\nodiff\nodiff\nodiff\nodiff\nodiff\nodiff \\
$\expmess_{21}$: & \nodiff\nodiff\nodiff\nodiff\nodiff\nodiff\nodiff\nodiff\nodiff\nodiff\nodiff\nodiff\nodiff\nodiff\onediff\onediff\nodiff\nodiff\nodiff\nodiff\nodiff\nodiff\nodiff\nodiff\nodiff\nodiff\nodiff\nodiff\nodiff\nodiff\nodiff\nodiff \\

$\state_{25}$\\
$\expmess_{19}$: & \nodiff\nodiff\nodiff\nodiff\nodiff\nodiff\nodiff\nodiff\nodiff\nodiff\nodiff\nodiff\nodiff\nodiff\nodiff\nodiff\nodiff\nodiff\nodiff\nodiff\nodiff\nodiff\nodiff\onediff\nodiff\nodiff\nodiff\nodiff\nodiff\nodiff\nodiff\nodiff \\
$\expmess_{20}$: & \nodiff\nodiff\nodiff\nodiff\nodiff\nodiff\nodiff\nodiff\nodiff\nodiff\nodiff\nodiff\nodiff\nodiff\nodiff\nodiff\nodiff\nodiff\onediff\nodiff\nodiff\nodiff\nodiff\nodiff\nodiff\onediff\nodiff\nodiff\nodiff\nodiff\nodiff\nodiff \\
$\expmess_{21}$: & \nodiff\nodiff\nodiff\nodiff\nodiff\nodiff\nodiff\nodiff\nodiff\nodiff\nodiff\nodiff\nodiff\nodiff\nodiff\nodiff\nodiff\nodiff\nodiff\nodiff\onediff\nodiff\nodiff\nodiff\nodiff\nodiff\nodiff\nodiff\nodiff\nodiff\nodiff\nodiff \\

$\state_{26}$\\
$\expmess_{20}$: & \nodiff\nodiff\nodiff\nodiff\nodiff\nodiff\nodiff\nodiff\nodiff\nodiff\nodiff\nodiff\nodiff\nodiff\nodiff\nodiff\nodiff\nodiff\nodiff\nodiff\nodiff\nodiff\nodiff\nodiff\nodiff\nodiff\nodiff\nodiff\nodiff\nodiff\nodiff\onediff \\
\end{tabular}
  \caption[The fifty-one neutral bits regrouped by the first state where they start to interact.]{The fifty-one neutral bits regrouped by the first state where they start to interact. A ``\onediff'' represents the presence
  of a neutral bit, and a ``\nodiff'' the absence thereof. The LSB position is the rightmost one.
  \label{fig:neutbits76_2}}
\end{figure}

Finally, on the implementation side, we show how the neutral bits are packed together inside the ``inter-snippet'' buffers (mentioned in \autoref{sec:part_sol}) in \autoref{fig:nb_packing76} and
\autoref{fig:nb_packing76_2}. These figures represent each of the two words of the buffers as thirty-two numbered coloured circles, one for each bit. The colours represent the original message word from which the
bit comes from, and the number is the bit position in this original word. For instance, in \autoref{fig:nb_packing76}, \begin{tikzpicture}[scale=0.67,transform shape]\draw[fill=Cerulean!85] (0,0) circle (.3) node{13};\end{tikzpicture}
is the value of the thirteenth bit (starting from zero) in some instance of $\expmess_{14}$ (\ie $\expmess_{14}[13]$); the fact that this circle is the leftmost one in the top sequence means that in the buffer, this information is stored as the MSB
of the first word.
One should note that not all consecutive bits of a message (for a certain window) are neutral; non-neutral bits do not need to be stored, but it is nonetheless useful to maintain the relative distance between the actual neutral bits inside
the packing.
For instance, there is no neutral bit on $\expmess_{15}[15]$, thus this bit is not included \emph{per se} in \autoref{fig:nb_packing76}, but a padding bit is added instead.
This is shown as \begin{tikzpicture}[scale=0.67,transform shape]\draw[fill=Rhodamine!30] (0,0) circle (.3) node{.};\end{tikzpicture}.

\begin{figure}[!htb]
\begin{center}
\begin{tikzpicture}[scale=0.67,transform shape]
\draw[fill=Cerulean!85] (0,0) circle (.3) node{13};
\draw[fill=Cerulean!85] (0.7,0) circle (.3) node{12};
\draw[fill=Cerulean!85] (1.4,0) circle (.3) node{11};
\draw[fill=Cerulean!85] (2.1,0) circle (.3) node{10};
\draw[fill=Cerulean!85] (2.8,0) circle (.3) node{ 9};
\draw[fill=Cerulean!85] (3.5,0) circle (.3) node{ 8};
\draw[fill=Cerulean!85] (4.2,0) circle (.3) node{ 7};
\draw[fill=Cerulean!85] (4.9,0) circle (.3) node{ 6};
\draw[fill=Cerulean!85] (5.6,0) circle (.3) node{ 5};
\draw[fill=Rhodamine!30] (6.3,0) circle (.3) node{16};
\draw[fill=Rhodamine!30] (7.0,0) circle (.3) node{.};
\draw[fill=Rhodamine!30] (7.7,0) circle (.3) node{14};
\draw[fill=Rhodamine!30] (8.4,0) circle (.3) node{13};
\draw[fill=Rhodamine!30] (9.1,0) circle (.3) node{12};
\draw[fill=Rhodamine!30] (9.8,0) circle (.3) node{11};
\draw[fill=Rhodamine!30] (10.5,0) circle (.3) node{10};
\draw[fill=Rhodamine!30] (11.2,0) circle (.3) node{ 9};
\draw[fill=Rhodamine!30] (11.9,0) circle (.3) node{ 8};
\draw[fill=Rhodamine!30] (12.6,0) circle (.3) node{ 7};
\draw[fill=Rhodamine!30] (13.3,0) circle (.3) node{ 6};
\draw[fill=Rhodamine!30] (14.0,0) circle (.3) node{ 5};
\draw[fill=BurntOrange!85] (14.7,0) circle (.3) node{16};
\draw[fill=BurntOrange!85] (15.4,0) circle (.3) node{.};
\draw[fill=BurntOrange!85] (16.1,0) circle (.3) node{.};
\draw[fill=BurntOrange!85] (16.8,0) circle (.3) node{13};
\draw[fill=BurntOrange!85] (17.5,0) circle (.3) node{.};
\draw[fill=BurntOrange!85] (18.2,0) circle (.3) node{11};
\draw[fill=BurntOrange!85] (18.9,0) circle (.3) node{10};
\draw[fill=BurntOrange!85] (19.6,0) circle (.3) node{ 9};
\draw[fill=BurntOrange!85] (20.3,0) circle (.3) node{ 8};
\draw[fill=BurntOrange!85] (21.0,0) circle (.3) node{ 7};
\draw[fill=BurntOrange!85] (21.7,0) circle (.3) node{ 6};
\draw[fill=YellowOrange!45] (0,-1) circle (.3) node{19};
\draw[fill=YellowOrange!45] (0.7,-1) circle (.3) node{.};
\draw[fill=YellowOrange!45] (1.4,-1) circle (.3) node{.};
\draw[fill=YellowOrange!45] (2.1,-1) circle (.3) node{.};
\draw[fill=YellowOrange!45] (2.8,-1) circle (.3) node{.};
\draw[fill=YellowOrange!45] (3.5,-1) circle (.3) node{.};
\draw[fill=YellowOrange!45] (4.2,-1) circle (.3) node{13};
\draw[fill=YellowOrange!45] (4.9,-1) circle (.3) node{12};
\draw[fill=YellowOrange!45] (5.6,-1) circle (.3) node{11};
\draw[fill=YellowOrange!45] (6.3,-1) circle (.3) node{10};
\draw[fill=Fuchsia!70] (7.0,-1) circle (.3) node{16};
\draw[fill=Fuchsia!70] (7.7,-1) circle (.3) node{15};
\draw[fill=LimeGreen!40] (8.4,-1) circle (.3) node{$\star$};
\draw[fill=LimeGreen!40] (9.1,-1) circle (.3) node{$\star$};
\draw[fill=LimeGreen!40] (9.8,-1) circle (.3) node{$\star$};
\draw[fill=LimeGreen!40] (10.5,-1) circle (.3) node{$\star$};
\draw[fill=LimeGreen!40] (11.2,-1) circle (.3) node{$\star$};
\draw[fill=LimeGreen!40] (11.9,-1) circle (.3) node{$\star$};
\draw[fill=LimeGreen!40] (12.6,-1) circle (.3) node{$\star$};
\draw[fill=LimeGreen!40] (13.3,-1) circle (.3) node{$\star$};
\draw[fill=LimeGreen!40] (14.0,-1) circle (.3) node{$\star$};
\draw[fill=LimeGreen!40] (14.7,-1) circle (.3) node{$\star$};
\draw[fill=LimeGreen!40] (15.4,-1) circle (.3) node{$\star$};
\draw[fill=LimeGreen!40] (16.1,-1) circle (.3) node{$\star$};
\draw[fill=LimeGreen!40] (16.8,-1) circle (.3) node{$\star$};
\draw[fill=LimeGreen!40] (17.5,-1) circle (.3) node{$\star$};
\draw[fill=LimeGreen!40] (18.2,-1) circle (.3) node{$\star$};
\draw[fill=LimeGreen!40] (18.9,-1) circle (.3) node{$\star$};
\draw[fill=LimeGreen!40] (19.6,-1) circle (.3) node{$\star$};
\draw[fill=LimeGreen!40] (20.3,-1) circle (.3) node{$\star$};
\draw[fill=LimeGreen!40] (21.0,-1) circle (.3) node{$\star$};
\draw[fill=LimeGreen!40] (21.7,-1) circle (.3) node{$\star$};
\end{tikzpicture}
\end{center}
\caption[The inter-snippet buffer for steps $\state_{18}$ to $\state_{21}$.]{The inter-snippet buffer for steps $\state_{18}$ to $\state_{21}$.
From top to bottom, left to right, bright cerulean (\protect\tikz{\protect\draw[fill=Cerulean!85] (0,0) circle (.15);})
refers to bits from $\expmess_{14}$; light rhodamine (\protect\tikz{\protect\draw[fill=Rhodamine!30] (0,0) circle (.15);})
refers to bits from $\expmess_{15}$; bright (burnt) orange (\protect\tikz{\protect\draw[fill=BurntOrange!85] (0,0) circle (.15);})
bits come from $\expmess_{16}$, while light (yellow) orange (\protect\tikz{\protect\draw[fill=YellowOrange!45] (0,0) circle (.15);})
means bits from $\expmess_{17}$. Finally, bright fuchsia (\protect\tikz{\protect\draw[fill=Fuchsia!70] (0,0) circle (.15);})
and light lime (\protect\tikz{\protect\draw[fill=LimeGreen!40] (0,0) circle (.15);}) refer to bits of $\expmess_{18}$
and bits holding the index of a base solution respectively.}
\label{fig:nb_packing76}
\end{figure}

\begin{figure}[!htb]
\begin{center}
\begin{tikzpicture}[scale=0.67,transform shape]
\draw[fill=LimeGreen!40] (0,0) circle (.3) node{.};
\draw[fill=LimeGreen!40] (0.7,0) circle (.3) node{.};
\draw[fill=Cerulean!85] (1.4,0) circle (.3) node{17};
\draw[fill=Cerulean!85] (2.1,0) circle (.3) node{16};
\draw[fill=Cerulean!85] (2.8,0) circle (.3) node{.};
\draw[fill=Cerulean!85] (3.5,0) circle (.3) node{.};
\draw[fill=Cerulean!85] (4.2,0) circle (.3) node{.};
\draw[fill=Cerulean!85] (4.9,0) circle (.3) node{.};
\draw[fill=Cerulean!85] (5.6,0) circle (.3) node{11};
\draw[fill=Rhodamine!30] (6.3,0) circle (.3) node{14};
\draw[fill=Rhodamine!30] (7.0,0) circle (.3) node{.};
\draw[fill=Rhodamine!30] (7.7,0) circle (.3) node{12};
\draw[fill=Rhodamine!30] (8.4,0) circle (.3) node{11};
\draw[fill=Rhodamine!30] (9.1,0) circle (.3) node{10};
\draw[fill=Rhodamine!30] (9.8,0) circle (.3) node{ 9};
\draw[fill=Rhodamine!30] (10.5,0) circle (.3) node{ 8};
\draw[fill=Rhodamine!30] (11.2,0) circle (.3) node{ 7};
\draw[fill=Rhodamine!30] (11.9,0) circle (.3) node{ 6};
\draw[fill=BurntOrange!85] (12.6,0) circle (.3) node{13};
\draw[fill=BurntOrange!85] (13.3,0) circle (.3) node{12};
\draw[fill=BurntOrange!85] (14.0,0) circle (.3) node{11};
\draw[fill=BurntOrange!85] (14.7,0) circle (.3) node{.};
\draw[fill=BurntOrange!85] (15.4,0) circle (.3) node{.};
\draw[fill=BurntOrange!85] (16.1,0) circle (.3) node{.};
\draw[fill=BurntOrange!85] (16.8,0) circle (.3) node{.};
\draw[fill=BurntOrange!85] (17.5,0) circle (.3) node{ 6};
\draw[fill=BurntOrange!85] (18.2,0) circle (.3) node{.};
\draw[fill=BurntOrange!85] (18.9,0) circle (.3) node{.};
\draw[fill=BurntOrange!85] (19.6,0) circle (.3) node{.};
\draw[fill=BurntOrange!85] (20.3,0) circle (.3) node{.};
\draw[fill=BurntOrange!85] (21.0,0) circle (.3) node{.};
\draw[fill=BurntOrange!85] (21.7,0) circle (.3) node{ 0};
\draw[fill=YellowOrange!45] (0,-1) circle (.3) node{.};
\draw[fill=YellowOrange!45] (0.7,-1) circle (.3) node{.};
\draw[fill=YellowOrange!45] (1.4,-1) circle (.3) node{.};
\draw[fill=YellowOrange!45] (2.1,-1) circle (.3) node{.};
\draw[fill=YellowOrange!45] (2.8,-1) circle (.3) node{.};
\draw[fill=YellowOrange!45] (3.5,-1) circle (.3) node{.};
\draw[fill=YellowOrange!45] (4.2,-1) circle (.3) node{.};
\draw[fill=YellowOrange!45] (4.9,-1) circle (.3) node{.};
\draw[fill=YellowOrange!45] (5.6,-1) circle (.3) node{.};
\draw[fill=YellowOrange!45] (6.3,-1) circle (.3) node{.};
\draw[fill=YellowOrange!45] (7.0,-1) circle (.3) node{.};
\draw[fill=YellowOrange!45] (7.7,-1) circle (.3) node{.};
\draw[fill=Fuchsia!70] (8.4,-1) circle (.3) node{$\star$};
\draw[fill=Fuchsia!70] (9.1,-1) circle (.3) node{$\star$};
\draw[fill=Fuchsia!70] (9.8,-1) circle (.3) node{$\star$};
\draw[fill=Fuchsia!70] (10.5,-1) circle (.3) node{$\star$};
\draw[fill=Fuchsia!70] (11.2,-1) circle (.3) node{$\star$};
\draw[fill=Fuchsia!70] (11.9,-1) circle (.3) node{$\star$};
\draw[fill=Fuchsia!70] (12.6,-1) circle (.3) node{$\star$};
\draw[fill=Fuchsia!70] (13.3,-1) circle (.3) node{$\star$};
\draw[fill=Fuchsia!70] (14.0,-1) circle (.3) node{$\star$};
\draw[fill=Fuchsia!70] (14.7,-1) circle (.3) node{$\star$};
\draw[fill=Fuchsia!70] (15.4,-1) circle (.3) node{$\star$};
\draw[fill=Fuchsia!70] (16.1,-1) circle (.3) node{$\star$};
\draw[fill=Fuchsia!70] (16.8,-1) circle (.3) node{$\star$};
\draw[fill=Fuchsia!70] (17.5,-1) circle (.3) node{$\star$};
\draw[fill=Fuchsia!70] (18.2,-1) circle (.3) node{$\star$};
\draw[fill=Fuchsia!70] (18.9,-1) circle (.3) node{$\star$};
\draw[fill=Fuchsia!70] (19.6,-1) circle (.3) node{$\star$};
\draw[fill=Fuchsia!70] (20.3,-1) circle (.3) node{$\star$};
\draw[fill=Fuchsia!70] (21.0,-1) circle (.3) node{$\star$};
\draw[fill=Fuchsia!70] (21.7,-1) circle (.3) node{$\star$};
\end{tikzpicture}
\end{center}
\caption[The inter-snippet buffer for steps $\state_{23}$ to $\state_{26}$.]{The inter-snippet buffer for steps $\state_{23}$ to $\state_{26}$.
From top to bottom, left to right, the two lime bits (\protect\tikz{\protect\draw[fill=LimeGreen!40] (0,0) circle (.15);})
are bits of padding that do not hold any meaningful data;
bright cerulean (\protect\tikz{\protect\draw[fill=Cerulean!85] (0,0) circle (.15);})
refers to bits from $\expmess_{21}$ and light rhodamine (\protect\tikz{\protect\draw[fill=Rhodamine!30] (0,0) circle (.15);})
to bits from $\expmess_{19}$; bright (burnt) orange (\protect\tikz{\protect\draw[fill=BurntOrange!85] (0,0) circle (.15);})
bits come from $\expmess_{20}$. Light (yellow) orange (\protect\tikz{\protect\draw[fill=YellowOrange!45] (0,0) circle (.15);})
bits are also padding bits, and bright fuchsia (\protect\tikz{\protect\draw[fill=Fuchsia!70] (0,0) circle (.15);})
bits finally hold the index of an extended base solution.}
\label{fig:nb_packing76_2}
\end{figure}

\subsection{An example of colliding message pair}
\label{sec:colli_ex76}
We give an example of collision in \autoref{tbl:fscoll76}.
This shows the two (message, \iv) pairs with their (identical) resulting digest. The \ivs and the digests' words are ordered similarly
as in \autoref{tab:sha_iv}; the messages' words are ordered as $\mess_0,\ldots,\mess_{15}$ from top to bottom, left to right.
Although the separations between the 32-bit words are not materialized, they should be taken into account, and
the values should not be interpreted as binary strings.
Note that unlike most inputs discussed in this chapter, the \iv values are compatible with the \emph{original} description
of \shaone, \ie they are not (all) equal to the corresponding state values $\state_{0},\ldots,\state_{-4}$; compared to
$\state_{-2},\ldots,\state_{-4}$, the last three \iv words are rotated by two to the right.

\begin{table}[!htb]
\caption[A freestart collision for 76-step \shaone.]{A freestart collision for 76-step \shaone. Message and \iv bytes with differences are highlighted with \framebox{\color{LimeGreen}coloured boxes}.}\label{tbl:fscoll76}
\centering
\begin{tabular}{l l}
\toprule
 & Message 1\\
\midrule
\iv &  \hspace{-8mm}\texttt{81 bf 23 06  41 b8 3b \framebox{\color{Cerulean}5c  03} e9 a7 8f  ba 50 28 d5  fc 50 87 88} \\
\midrule
$\mess$ & \texttt{\hspace{1.15mm}46\hspace{1.25mm} fa 5a \framebox{\color{Cerulean}88  f4} f0 c7 \framebox{\color{Cerulean}f0  b8} de db \framebox{\color{Cerulean}ec  95} 1e 25 \framebox{\color{Cerulean}88}}\\
        & \texttt{\framebox{\color{Cerulean}77} 34 fd \framebox{\color{Cerulean}f5 4c} 42 c4 \framebox{\color{Cerulean}97 52} d7 d8 \framebox{\color{Cerulean}f9 5f} 14 52 \framebox{\color{Cerulean}ea}} \\
		& \texttt{\framebox{\color{Cerulean}b4} 9e 93 \framebox{\color{Cerulean}b2 91} c2 30 \framebox{\color{Cerulean}71 c7} 0f 35 \framebox{\color{Cerulean}9b 8a} ba cf \framebox{\color{Cerulean}af}} \\
		& \texttt{\framebox{\color{Cerulean}b3} 7f fb \framebox{\color{Cerulean}27 3d} fe 7f \framebox{\color{Cerulean}ad 7a} de 56 \framebox{\color{Cerulean}95 20} fd 7c \framebox{\color{Cerulean}ea}} \\
\midrule
$\compress(\iv,\mess)$ & \texttt{af 49 5d 10  52 82 35 03  e4 9e 46 78}\\
& \texttt{dc e7 f3 b3  d6 da a3 24} \\
\bottomrule\\

\toprule
 & Message 2 \\
\midrule
$\diff$\iv & \hspace{-8mm}\texttt{81 bf 23 06  41 b8 3b \framebox{\color{RubineRed}5d  83} e9 a7 8f  ba 50 28 d5  fc 50 87 88}\\
\midrule
$\diff\mess$ & \texttt{\hspace{1.15mm}46\hspace{1.25mm} fa 5a \framebox{\color{RubineRed}98  f0} f0 c7 \framebox{\color{RubineRed}ec  7a} de db \framebox{\color{RubineRed}f8  99} 1e 25 \framebox{\color{RubineRed}8a}}\\
      		 & \texttt{\framebox{\color{RubineRed}b7} 34 fd \framebox{\color{RubineRed}e5 f8} 42 c4 \framebox{\color{RubineRed}8b 6e} d7 d8 \framebox{\color{RubineRed}fd e3} 14 52 \framebox{\color{RubineRed}f0}} \\
			 & \texttt{\framebox{\color{RubineRed}94} 9e 93 \framebox{\color{RubineRed}a2 b5} c2 30 \framebox{\color{RubineRed}6d 2b} 0f 35 \framebox{\color{RubineRed}8f 86} ba cf \framebox{\color{RubineRed}ad}} \\
			 & \texttt{\framebox{\color{RubineRed}73} 7f fb \framebox{\color{RubineRed}37 89} fe 7f \framebox{\color{RubineRed}b1 56} de 56 \framebox{\color{RubineRed}91 9c} fd 7c \framebox{\color{RubineRed}f2}} \\
\midrule
$\compress(\diff\iv,\diff\mess)$ & \texttt{af 49 5d 10  52 82 35 03  e4 9e 46 78}\\
 & \texttt{dc e7 f3 b3  d6 da a3 24}\\
\bottomrule
\end{tabular}
\end{table}

\subsection{Complexity of the attack}
\label{sec:comp76}

As we already mentioned in \autoref{sec:gpu_tune}, the expected time to find a collision on a single average-performing \gtx is 4.42 days.
This figure is obtained by considering the rate at which partial solutions up to $\state_{56}$ are produced together with the probability that such a partial solution leads to a collision.
In our experiments, partial solutions were obtained at a rate of 0.0171 per second on average, while JLCA gives a probability of $2^{-12.67}$, leading to the above expected time to collision.
As we mentioned in \autoref{sec:effi_fw}, this translates to a complexity of $2^{50.34}$ compression function calls on a \gtx.


\section{Freestart collisions for 80-step SHA-1}
\label{sec:res_80}

This section gives the attack parameters for the 80-step (full) freestart collision attack on \shaone.

\subsection{The differential path for part of the two first rounds}
\label{sec:app_diff_path80}

A graphical representation of the differential path
used in our attack up to step 28 is given in \autoref{fig:diff_path80}.
It consists of sufficient conditions for the state, and of the associated message signed bit differences.
The meaning of the bit condition symbols were defined in \autoref{table:appbitconditions}.
Note that the signs of message bit differences are enforced through message bit relations.
The message bit relations
used in the attack past $\expmess_{28}$ are given in \autoref{fig:msgbitrel80}, and a graphical
representation thereof in
\autoref{fig:msgbitrel80_graph}.
The remainder of the path can easily be determined by linearization of the step function given the differences
in the message.

\begingroup
\fontsize{8pt}{9pt}\selectfont
\begin{verbbox}
A-4: ........ ........ ........ ........
A-3: ........ ........ ........ ........
A-2: ........ ........ ........ .....^-.
A-1: 1...1... ........ ........ .0.....+
A0 : 01..0... ........ ........ .1......  W0 : x.+...+. ........ ........ ...+....
A1 : 11+^..+. ........ ....^... ...+....  W1 : ..-..-.. ........ ........ ...-++..
A2 : ..-11-1. 1......^ .....1+1 10.1.0..  W2 : ..+..--. ........ ........ ...-.+..
A3 : .0.0-001 1.^.10.. .+01.011 11^0.1.1  W3 : ..-..--. ........ ........ ...-+.-.
A4 : .1.11+-1 +^^^+1^^ ^011^^.- +++++-.+  W4 : ........ ........ ........ ...+....
A5 : .+.+.-++ ++++++++ ++++++++ .+0-1111  W5 : .....-.. ........ ........ ...+++..
A6 : .0.0.1.0 11.111.1 1110-010 0-1.10-+  W6 : x+..++.. ........ ........ ...-.+..
A7 : 1-.+.1.0 10100010 00000011 1+.-.0.+  W7 : ....-+.. ........ ........ ......+.
A8 : 0+.0.0.. ........ ......0. .+.-.0.1  W8 : x-...... ........ ........ ...+....
A9 : .+.0.0.. ........ ........ .0.+...^  W9 : x.-+.-.. ........ ........ ...-++..
A10: .+...... ........ ........ ...+.0..  W10: ..-+++.. ........ ........ .....-..
A11: ...-.... ........ ........ ........  W11: x.++++.. ........ ........ ...-+.+.
A12: ...0.1.. ........ ........ .....1..  W12: ..-..... ........ ........ ...-....
A13: .1...0.. ........ ........ ......!^  W13: ..+..+.. ........ ........ ...-++..
A14: +-...... ........ ........ ........  W14: x++.+-.. ........ ........ ...-.+..
A15: 1.1-.... ........ ........ ......!.  W15: ....+-.. ........ ........ ......+.
A16: +.10.1.. ........ ........ ........  W16: x+...... ........ ........ ...-....
A17: 1.-..0.. ........ ........ .......^  W17: x.++.+.. ........ ........ ...+--..
A18: .+-.0... ........ ........ .......!  W18: ..+.--.. ........ ........ .....-..
A19: .+.s.... ........ ........ ........  W19: x.+---.. ........ ........ ...-+...
A20: -...R... ........ ........ ........  W20: x.++.... ........ ........ ...+....
A21: -.+R.... ........ ........ ........  W21: ........ ........ ........ ....++..
A22: -...S... ........ ........ .......^  W22: x.---... ........ ........ ...+....
A23: .-..R... ........ ........ ........  W23: ....-... ........ ........ ...+-...
A24: -.rs.... ........ ........ ........  W24: .-+--... ........ ........ ...+....
A25: -.-r.... ........ ........ ........  W25: ....+... ........ ........ ...+.+..
A26: -...s... ........ ........ ........  W26: .+--.... ........ ........ ...+....
A27: -.-.r... ........ ........ ........  W27: x.+-+... ........ ........ ...++-..
A28: ........ ........ ........ ........  W28: x+-.-... ........ ........ ........
A29: ..-..... ........ ........ ........
\end{verbbox}
\endgroup

\begin{figure}[!htb]
\centering
\begin{tabular}{lcc}
$i$ & $\state_i$ & $\mess_i$\\
-4 & \nodiff\nodiff\nodiff\nodiff\nodiff\nodiff\nodiff\nodiff\nodiff\nodiff\nodiff\nodiff\nodiff\nodiff\nodiff\nodiff\nodiff\nodiff\nodiff\nodiff\nodiff\nodiff\nodiff\nodiff\nodiff\nodiff\nodiff\nodiff\nodiff\nodiff\nodiff\nodiff \\
-3 & \nodiff\nodiff\nodiff\nodiff\nodiff\nodiff\nodiff\nodiff\nodiff\nodiff\nodiff\nodiff\nodiff\nodiff\nodiff\nodiff\nodiff\nodiff\nodiff\nodiff\nodiff\nodiff\nodiff\nodiff\nodiff\nodiff\nodiff\nodiff\nodiff\nodiff\nodiff\nodiff \\
-2 & \nodiff\nodiff\nodiff\nodiff\nodiff\nodiff\nodiff\nodiff\nodiff\nodiff\nodiff\nodiff\nodiff\nodiff\nodiff\nodiff\nodiff\nodiff\nodiff\nodiff\nodiff\nodiff\nodiff\nodiff\nodiff\nodiff\nodiff\nodiff\nodiff\equaup$\monediffd$\nodiff \\
-1 & $\mnodiffo$\nodiff\nodiff\nodiff$\mnodiffo$\nodiff\nodiff\nodiff\nodiff\nodiff\nodiff\nodiff\nodiff\nodiff\nodiff\nodiff\nodiff\nodiff\nodiff\nodiff\nodiff\nodiff\nodiff\nodiff\nodiff$\mnodiffz$\nodiff\nodiff\nodiff\nodiff\nodiff$\monediffu$ \\
0  & $\mnodiffz$$\mnodiffo$\nodiff\nodiff$\mnodiffz$\nodiff\nodiff\nodiff\nodiff\nodiff\nodiff\nodiff\nodiff\nodiff\nodiff\nodiff\nodiff\nodiff\nodiff\nodiff\nodiff\nodiff\nodiff\nodiff\nodiff$\mnodiffo$\nodiff\nodiff\nodiff\nodiff\nodiff\nodiff  &  \onediff\nodiff$\monediffu$\nodiff\nodiff\nodiff$\monediffu$\nodiff\nodiff\nodiff\nodiff\nodiff\nodiff\nodiff\nodiff\nodiff\nodiff\nodiff\nodiff\nodiff\nodiff\nodiff\nodiff\nodiff\nodiff\nodiff\nodiff$\monediffu$\nodiff\nodiff\nodiff\nodiff \\
1  & $\mnodiffo$$\mnodiffo$$\monediffu$\equaup\nodiff\nodiff$\monediffu$\nodiff\nodiff\nodiff\nodiff\nodiff\nodiff\nodiff\nodiff\nodiff\nodiff\nodiff\nodiff\nodiff\equaup\nodiff\nodiff\nodiff\nodiff\nodiff\nodiff$\monediffu$\nodiff\nodiff\nodiff\nodiff  &  \nodiff\nodiff$\monediffd$\nodiff\nodiff$\monediffd$\nodiff\nodiff\nodiff\nodiff\nodiff\nodiff\nodiff\nodiff\nodiff\nodiff\nodiff\nodiff\nodiff\nodiff\nodiff\nodiff\nodiff\nodiff\nodiff\nodiff\nodiff$\monediffd$$\monediffu$$\monediffu$\nodiff\nodiff \\
2  & \nodiff\nodiff$\monediffd$$\mnodiffo$$\mnodiffo$$\monediffd$$\mnodiffo$\nodiff$\mnodiffo$\nodiff\nodiff\nodiff\nodiff\nodiff\nodiff\equaup\nodiff\nodiff\nodiff\nodiff\nodiff$\mnodiffo$$\monediffu$$\mnodiffo$$\mnodiffo$$\mnodiffz$\nodiff$\mnodiffo$\nodiff$\mnodiffz$\nodiff\nodiff  &  \nodiff\nodiff$\monediffu$\nodiff\nodiff$\monediffd$$\monediffd$\nodiff\nodiff\nodiff\nodiff\nodiff\nodiff\nodiff\nodiff\nodiff\nodiff\nodiff\nodiff\nodiff\nodiff\nodiff\nodiff\nodiff\nodiff\nodiff\nodiff$\monediffd$\nodiff$\monediffu$\nodiff\nodiff \\
3  & \nodiff$\mnodiffz$\nodiff$\mnodiffz$$\monediffd$$\mnodiffz$$\mnodiffz$$\mnodiffo$$\mnodiffo$\nodiff\equaup\nodiff$\mnodiffo$$\mnodiffz$\nodiff\nodiff\nodiff$\monediffu$$\mnodiffz$$\mnodiffo$\nodiff$\mnodiffz$$\mnodiffo$$\mnodiffo$$\mnodiffo$$\mnodiffo$\equaup$\mnodiffz$\nodiff$\mnodiffo$\nodiff$\mnodiffo$  &  \nodiff\nodiff$\monediffd$\nodiff\nodiff$\monediffd$$\monediffd$\nodiff\nodiff\nodiff\nodiff\nodiff\nodiff\nodiff\nodiff\nodiff\nodiff\nodiff\nodiff\nodiff\nodiff\nodiff\nodiff\nodiff\nodiff\nodiff\nodiff$\monediffd$$\monediffu$\nodiff$\monediffd$\nodiff \\
4  & \nodiff$\mnodiffo$\nodiff$\mnodiffo$$\mnodiffo$$\monediffu$$\monediffd$$\mnodiffo$$\monediffu$\equaup\equaup\equaup$\monediffu$$\mnodiffo$\equaup\equaup\equaup$\mnodiffz$$\mnodiffo$$\mnodiffo$\equaup\equaup\nodiff$\monediffd$$\monediffu$$\monediffu$$\monediffu$$\monediffu$$\monediffu$$\monediffd$\nodiff$\monediffu$  &  \nodiff\nodiff\nodiff\nodiff\nodiff\nodiff\nodiff\nodiff\nodiff\nodiff\nodiff\nodiff\nodiff\nodiff\nodiff\nodiff\nodiff\nodiff\nodiff\nodiff\nodiff\nodiff\nodiff\nodiff\nodiff\nodiff\nodiff$\monediffu$\nodiff\nodiff\nodiff\nodiff \\
5  & \nodiff$\monediffu$\nodiff$\monediffu$\nodiff$\monediffd$$\monediffu$$\monediffu$$\monediffu$$\monediffu$$\monediffu$$\monediffu$$\monediffu$$\monediffu$$\monediffu$$\monediffu$$\monediffu$$\monediffu$$\monediffu$$\monediffu$$\monediffu$$\monediffu$$\monediffu$$\monediffu$\nodiff$\monediffu$$\mnodiffz$$\monediffd$$\mnodiffo$$\mnodiffo$$\mnodiffo$$\mnodiffo$  &  \nodiff\nodiff\nodiff\nodiff\nodiff$\monediffd$\nodiff\nodiff\nodiff\nodiff\nodiff\nodiff\nodiff\nodiff\nodiff\nodiff\nodiff\nodiff\nodiff\nodiff\nodiff\nodiff\nodiff\nodiff\nodiff\nodiff\nodiff$\monediffu$$\monediffu$$\monediffu$\nodiff\nodiff \\
6  & \nodiff$\mnodiffz$\nodiff$\mnodiffz$\nodiff$\mnodiffo$\nodiff$\mnodiffz$$\mnodiffo$$\mnodiffo$\nodiff$\mnodiffo$$\mnodiffo$$\mnodiffo$\nodiff$\mnodiffo$$\mnodiffo$$\mnodiffo$$\mnodiffo$$\mnodiffz$$\monediffd$$\mnodiffz$$\mnodiffo$$\mnodiffz$$\mnodiffz$$\monediffd$$\mnodiffo$\nodiff$\mnodiffo$$\mnodiffz$$\monediffd$$\monediffu$  &  \onediff$\monediffu$\nodiff\nodiff$\monediffu$$\monediffu$\nodiff\nodiff\nodiff\nodiff\nodiff\nodiff\nodiff\nodiff\nodiff\nodiff\nodiff\nodiff\nodiff\nodiff\nodiff\nodiff\nodiff\nodiff\nodiff\nodiff\nodiff$\monediffd$\nodiff$\monediffu$\nodiff\nodiff \\
7  & $\mnodiffo$$\monediffd$\nodiff$\monediffu$\nodiff$\mnodiffo$\nodiff$\mnodiffz$$\mnodiffo$$\mnodiffz$$\mnodiffo$$\mnodiffz$$\mnodiffz$$\mnodiffz$$\mnodiffo$$\mnodiffz$$\mnodiffz$$\mnodiffz$$\mnodiffz$$\mnodiffz$$\mnodiffz$$\mnodiffz$$\mnodiffo$$\mnodiffo$$\mnodiffo$$\monediffu$\nodiff$\monediffd$\nodiff$\mnodiffz$\nodiff$\monediffu$  &  \nodiff\nodiff\nodiff\nodiff$\monediffd$$\monediffu$\nodiff\nodiff\nodiff\nodiff\nodiff\nodiff\nodiff\nodiff\nodiff\nodiff\nodiff\nodiff\nodiff\nodiff\nodiff\nodiff\nodiff\nodiff\nodiff\nodiff\nodiff\nodiff\nodiff\nodiff$\monediffu$\nodiff \\
8  & $\mnodiffz$$\monediffu$\nodiff$\mnodiffz$\nodiff$\mnodiffz$\nodiff\nodiff\nodiff\nodiff\nodiff\nodiff\nodiff\nodiff\nodiff\nodiff\nodiff\nodiff\nodiff\nodiff\nodiff\nodiff$\mnodiffz$\nodiff\nodiff$\monediffu$\nodiff$\monediffd$\nodiff$\mnodiffz$\nodiff$\mnodiffo$  &  \onediff$\monediffd$\nodiff\nodiff\nodiff\nodiff\nodiff\nodiff\nodiff\nodiff\nodiff\nodiff\nodiff\nodiff\nodiff\nodiff\nodiff\nodiff\nodiff\nodiff\nodiff\nodiff\nodiff\nodiff\nodiff\nodiff\nodiff$\monediffu$\nodiff\nodiff\nodiff\nodiff \\
9  & \nodiff$\monediffu$\nodiff$\mnodiffz$\nodiff$\mnodiffz$\nodiff\nodiff\nodiff\nodiff\nodiff\nodiff\nodiff\nodiff\nodiff\nodiff\nodiff\nodiff\nodiff\nodiff\nodiff\nodiff\nodiff\nodiff\nodiff$\mnodiffz$\nodiff$\monediffu$\nodiff\nodiff\nodiff\equaup  &  \onediff\nodiff$\monediffd$$\monediffu$\nodiff$\monediffd$\nodiff\nodiff\nodiff\nodiff\nodiff\nodiff\nodiff\nodiff\nodiff\nodiff\nodiff\nodiff\nodiff\nodiff\nodiff\nodiff\nodiff\nodiff\nodiff\nodiff\nodiff$\monediffd$$\monediffu$$\monediffu$\nodiff\nodiff \\
10 & \nodiff$\monediffu$\nodiff\nodiff\nodiff\nodiff\nodiff\nodiff\nodiff\nodiff\nodiff\nodiff\nodiff\nodiff\nodiff\nodiff\nodiff\nodiff\nodiff\nodiff\nodiff\nodiff\nodiff\nodiff\nodiff\nodiff\nodiff$\monediffu$\nodiff$\mnodiffz$\nodiff\nodiff  &  \nodiff\nodiff$\monediffd$$\monediffu$$\monediffu$$\monediffu$\nodiff\nodiff\nodiff\nodiff\nodiff\nodiff\nodiff\nodiff\nodiff\nodiff\nodiff\nodiff\nodiff\nodiff\nodiff\nodiff\nodiff\nodiff\nodiff\nodiff\nodiff\nodiff\nodiff$\monediffd$\nodiff\nodiff \\
11 & \nodiff\nodiff\nodiff$\monediffd$\nodiff\nodiff\nodiff\nodiff\nodiff\nodiff\nodiff\nodiff\nodiff\nodiff\nodiff\nodiff\nodiff\nodiff\nodiff\nodiff\nodiff\nodiff\nodiff\nodiff\nodiff\nodiff\nodiff\nodiff\nodiff\nodiff\nodiff\nodiff  &  \onediff\nodiff$\monediffu$$\monediffu$$\monediffu$$\monediffu$\nodiff\nodiff\nodiff\nodiff\nodiff\nodiff\nodiff\nodiff\nodiff\nodiff\nodiff\nodiff\nodiff\nodiff\nodiff\nodiff\nodiff\nodiff\nodiff\nodiff\nodiff$\monediffd$$\monediffu$\nodiff$\monediffu$\nodiff \\
12 & \nodiff\nodiff\nodiff$\mnodiffz$\nodiff$\mnodiffo$\nodiff\nodiff\nodiff\nodiff\nodiff\nodiff\nodiff\nodiff\nodiff\nodiff\nodiff\nodiff\nodiff\nodiff\nodiff\nodiff\nodiff\nodiff\nodiff\nodiff\nodiff\nodiff\nodiff$\mnodiffo$\nodiff\nodiff  &  \nodiff\nodiff$\monediffd$\nodiff\nodiff\nodiff\nodiff\nodiff\nodiff\nodiff\nodiff\nodiff\nodiff\nodiff\nodiff\nodiff\nodiff\nodiff\nodiff\nodiff\nodiff\nodiff\nodiff\nodiff\nodiff\nodiff\nodiff$\monediffd$\nodiff\nodiff\nodiff\nodiff \\
13 & \nodiff$\mnodiffo$\nodiff\nodiff\nodiff$\mnodiffz$\nodiff\nodiff\nodiff\nodiff\nodiff\nodiff\nodiff\nodiff\nodiff\nodiff\nodiff\nodiff\nodiff\nodiff\nodiff\nodiff\nodiff\nodiff\nodiff\nodiff\nodiff\nodiff\nodiff\nodiff\diffup\equaup  &  \nodiff\nodiff$\monediffu$\nodiff\nodiff$\monediffu$\nodiff\nodiff\nodiff\nodiff\nodiff\nodiff\nodiff\nodiff\nodiff\nodiff\nodiff\nodiff\nodiff\nodiff\nodiff\nodiff\nodiff\nodiff\nodiff\nodiff\nodiff$\monediffd$$\monediffu$$\monediffu$\nodiff\nodiff \\
14 & $\monediffu$$\monediffd$\nodiff\nodiff\nodiff\nodiff\nodiff\nodiff\nodiff\nodiff\nodiff\nodiff\nodiff\nodiff\nodiff\nodiff\nodiff\nodiff\nodiff\nodiff\nodiff\nodiff\nodiff\nodiff\nodiff\nodiff\nodiff\nodiff\nodiff\nodiff\nodiff\nodiff  &  \onediff$\monediffu$$\monediffu$\nodiff$\monediffu$$\monediffd$\nodiff\nodiff\nodiff\nodiff\nodiff\nodiff\nodiff\nodiff\nodiff\nodiff\nodiff\nodiff\nodiff\nodiff\nodiff\nodiff\nodiff\nodiff\nodiff\nodiff\nodiff$\monediffd$\nodiff$\monediffu$\nodiff\nodiff \\
15 & $\mnodiffo$\nodiff$\mnodiffo$$\monediffd$\nodiff\nodiff\nodiff\nodiff\nodiff\nodiff\nodiff\nodiff\nodiff\nodiff\nodiff\nodiff\nodiff\nodiff\nodiff\nodiff\nodiff\nodiff\nodiff\nodiff\nodiff\nodiff\nodiff\nodiff\nodiff\nodiff\diffup\nodiff  &  \nodiff\nodiff\nodiff\nodiff$\monediffu$$\monediffd$\nodiff\nodiff\nodiff\nodiff\nodiff\nodiff\nodiff\nodiff\nodiff\nodiff\nodiff\nodiff\nodiff\nodiff\nodiff\nodiff\nodiff\nodiff\nodiff\nodiff\nodiff\nodiff\nodiff\nodiff$\monediffu$\nodiff \\
16 & $\monediffu$\nodiff$\mnodiffo$$\mnodiffz$\nodiff$\mnodiffo$\nodiff\nodiff\nodiff\nodiff\nodiff\nodiff\nodiff\nodiff\nodiff\nodiff\nodiff\nodiff\nodiff\nodiff\nodiff\nodiff\nodiff\nodiff\nodiff\nodiff\nodiff\nodiff\nodiff\nodiff\nodiff\nodiff  &  \onediff$\monediffu$\nodiff\nodiff\nodiff\nodiff\nodiff\nodiff\nodiff\nodiff\nodiff\nodiff\nodiff\nodiff\nodiff\nodiff\nodiff\nodiff\nodiff\nodiff\nodiff\nodiff\nodiff\nodiff\nodiff\nodiff\nodiff$\monediffd$\nodiff\nodiff\nodiff\nodiff \\
17 & $\mnodiffo$\nodiff$\monediffd$\nodiff\nodiff$\mnodiffz$\nodiff\nodiff\nodiff\nodiff\nodiff\nodiff\nodiff\nodiff\nodiff\nodiff\nodiff\nodiff\nodiff\nodiff\nodiff\nodiff\nodiff\nodiff\nodiff\nodiff\nodiff\nodiff\nodiff\nodiff\nodiff\equaup  &  \onediff\nodiff$\monediffu$$\monediffu$\nodiff$\monediffu$\nodiff\nodiff\nodiff\nodiff\nodiff\nodiff\nodiff\nodiff\nodiff\nodiff\nodiff\nodiff\nodiff\nodiff\nodiff\nodiff\nodiff\nodiff\nodiff\nodiff\nodiff$\monediffu$$\monediffd$$\monediffd$\nodiff\nodiff \\
18 & \nodiff$\monediffu$$\monediffd$\nodiff$\mnodiffz$\nodiff\nodiff\nodiff\nodiff\nodiff\nodiff\nodiff\nodiff\nodiff\nodiff\nodiff\nodiff\nodiff\nodiff\nodiff\nodiff\nodiff\nodiff\nodiff\nodiff\nodiff\nodiff\nodiff\nodiff\nodiff\nodiff\diffup  &  \nodiff\nodiff$\monediffu$\nodiff$\monediffd$$\monediffd$\nodiff\nodiff\nodiff\nodiff\nodiff\nodiff\nodiff\nodiff\nodiff\nodiff\nodiff\nodiff\nodiff\nodiff\nodiff\nodiff\nodiff\nodiff\nodiff\nodiff\nodiff\nodiff\nodiff$\monediffd$\nodiff\nodiff \\
19 & \nodiff$\monediffu$\nodiff\equarightupup\nodiff\nodiff\nodiff\nodiff\nodiff\nodiff\nodiff\nodiff\nodiff\nodiff\nodiff\nodiff\nodiff\nodiff\nodiff\nodiff\nodiff\nodiff\nodiff\nodiff\nodiff\nodiff\nodiff\nodiff\nodiff\nodiff\nodiff\nodiff  &  \onediff\nodiff$\monediffu$$\monediffd$$\monediffd$$\monediffd$\nodiff\nodiff\nodiff\nodiff\nodiff\nodiff\nodiff\nodiff\nodiff\nodiff\nodiff\nodiff\nodiff\nodiff\nodiff\nodiff\nodiff\nodiff\nodiff\nodiff\nodiff$\monediffd$$\monediffu$\nodiff\nodiff\nodiff \\
20 & $\monediffd$\nodiff\nodiff\nodiff\diffrightup\nodiff\nodiff\nodiff\nodiff\nodiff\nodiff\nodiff\nodiff\nodiff\nodiff\nodiff\nodiff\nodiff\nodiff\nodiff\nodiff\nodiff\nodiff\nodiff\nodiff\nodiff\nodiff\nodiff\nodiff\nodiff\nodiff\nodiff  &  \onediff\nodiff$\monediffu$$\monediffu$\nodiff\nodiff\nodiff\nodiff\nodiff\nodiff\nodiff\nodiff\nodiff\nodiff\nodiff\nodiff\nodiff\nodiff\nodiff\nodiff\nodiff\nodiff\nodiff\nodiff\nodiff\nodiff\nodiff$\monediffu$\nodiff\nodiff\nodiff\nodiff \\
21 & $\monediffd$\nodiff$\monediffu$\diffrightup\nodiff\nodiff\nodiff\nodiff\nodiff\nodiff\nodiff\nodiff\nodiff\nodiff\nodiff\nodiff\nodiff\nodiff\nodiff\nodiff\nodiff\nodiff\nodiff\nodiff\nodiff\nodiff\nodiff\nodiff\nodiff\nodiff\nodiff\nodiff  &  \nodiff\nodiff\nodiff\nodiff\nodiff\nodiff\nodiff\nodiff\nodiff\nodiff\nodiff\nodiff\nodiff\nodiff\nodiff\nodiff\nodiff\nodiff\nodiff\nodiff\nodiff\nodiff\nodiff\nodiff\nodiff\nodiff\nodiff\nodiff$\monediffu$$\monediffu$\nodiff\nodiff \\
22 & $\monediffd$\nodiff\nodiff\nodiff\diffrightupup\nodiff\nodiff\nodiff\nodiff\nodiff\nodiff\nodiff\nodiff\nodiff\nodiff\nodiff\nodiff\nodiff\nodiff\nodiff\nodiff\nodiff\nodiff\nodiff\nodiff\nodiff\nodiff\nodiff\nodiff\nodiff\nodiff\equaup  &  \onediff\nodiff$\monediffd$$\monediffd$$\monediffd$\nodiff\nodiff\nodiff\nodiff\nodiff\nodiff\nodiff\nodiff\nodiff\nodiff\nodiff\nodiff\nodiff\nodiff\nodiff\nodiff\nodiff\nodiff\nodiff\nodiff\nodiff\nodiff$\monediffu$\nodiff\nodiff\nodiff\nodiff \\
23 & \nodiff$\monediffd$\nodiff\nodiff\diffrightup\nodiff\nodiff\nodiff\nodiff\nodiff\nodiff\nodiff\nodiff\nodiff\nodiff\nodiff\nodiff\nodiff\nodiff\nodiff\nodiff\nodiff\nodiff\nodiff\nodiff\nodiff\nodiff\nodiff\nodiff\nodiff\nodiff\nodiff  &  \nodiff\nodiff\nodiff\nodiff$\monediffd$\nodiff\nodiff\nodiff\nodiff\nodiff\nodiff\nodiff\nodiff\nodiff\nodiff\nodiff\nodiff\nodiff\nodiff\nodiff\nodiff\nodiff\nodiff\nodiff\nodiff\nodiff\nodiff$\monediffu$$\monediffd$\nodiff\nodiff\nodiff \\
24 & $\monediffd$\nodiff\equarightup\equarightupup\nodiff\nodiff\nodiff\nodiff\nodiff\nodiff\nodiff\nodiff\nodiff\nodiff\nodiff\nodiff\nodiff\nodiff\nodiff\nodiff\nodiff\nodiff\nodiff\nodiff\nodiff\nodiff\nodiff\nodiff\nodiff\nodiff\nodiff\nodiff  &  \nodiff$\monediffd$$\monediffu$$\monediffd$$\monediffd$\nodiff\nodiff\nodiff\nodiff\nodiff\nodiff\nodiff\nodiff\nodiff\nodiff\nodiff\nodiff\nodiff\nodiff\nodiff\nodiff\nodiff\nodiff\nodiff\nodiff\nodiff\nodiff$\monediffu$\nodiff\nodiff\nodiff\nodiff \\
25 & $\monediffd$\nodiff$\monediffd$\equarightup\nodiff\nodiff\nodiff\nodiff\nodiff\nodiff\nodiff\nodiff\nodiff\nodiff\nodiff\nodiff\nodiff\nodiff\nodiff\nodiff\nodiff\nodiff\nodiff\nodiff\nodiff\nodiff\nodiff\nodiff\nodiff\nodiff\nodiff\nodiff  &  \nodiff\nodiff\nodiff\nodiff$\monediffu$\nodiff\nodiff\nodiff\nodiff\nodiff\nodiff\nodiff\nodiff\nodiff\nodiff\nodiff\nodiff\nodiff\nodiff\nodiff\nodiff\nodiff\nodiff\nodiff\nodiff\nodiff\nodiff$\monediffu$\nodiff$\monediffu$\nodiff\nodiff \\
26 & $\monediffd$\nodiff\nodiff\nodiff\equarightupup\nodiff\nodiff\nodiff\nodiff\nodiff\nodiff\nodiff\nodiff\nodiff\nodiff\nodiff\nodiff\nodiff\nodiff\nodiff\nodiff\nodiff\nodiff\nodiff\nodiff\nodiff\nodiff\nodiff\nodiff\nodiff\nodiff\nodiff  &  \nodiff$\monediffu$$\monediffd$$\monediffd$\nodiff\nodiff\nodiff\nodiff\nodiff\nodiff\nodiff\nodiff\nodiff\nodiff\nodiff\nodiff\nodiff\nodiff\nodiff\nodiff\nodiff\nodiff\nodiff\nodiff\nodiff\nodiff\nodiff$\monediffu$\nodiff\nodiff\nodiff\nodiff \\
27 & $\monediffd$\nodiff$\monediffd$\nodiff\equarightup\nodiff\nodiff\nodiff\nodiff\nodiff\nodiff\nodiff\nodiff\nodiff\nodiff\nodiff\nodiff\nodiff\nodiff\nodiff\nodiff\nodiff\nodiff\nodiff\nodiff\nodiff\nodiff\nodiff\nodiff\nodiff\nodiff\nodiff  &  \onediff\nodiff$\monediffu$$\monediffd$$\monediffu$\nodiff\nodiff\nodiff\nodiff\nodiff\nodiff\nodiff\nodiff\nodiff\nodiff\nodiff\nodiff\nodiff\nodiff\nodiff\nodiff\nodiff\nodiff\nodiff\nodiff\nodiff\nodiff$\monediffu$$\monediffu$$\monediffd$\nodiff\nodiff \\
28 & \nodiff\nodiff\nodiff\nodiff\nodiff\nodiff\nodiff\nodiff\nodiff\nodiff\nodiff\nodiff\nodiff\nodiff\nodiff\nodiff\nodiff\nodiff\nodiff\nodiff\nodiff\nodiff\nodiff\nodiff\nodiff\nodiff\nodiff\nodiff\nodiff\nodiff\nodiff\nodiff  &  \onediff$\monediffu$$\monediffd$\nodiff$\monediffd$\nodiff\nodiff\nodiff\nodiff\nodiff\nodiff\nodiff\nodiff\nodiff\nodiff\nodiff\nodiff\nodiff\nodiff\nodiff\nodiff\nodiff\nodiff\nodiff\nodiff\nodiff\nodiff\nodiff\nodiff\nodiff\nodiff\nodiff \\
29 & \nodiff\nodiff$\monediffd$\nodiff\nodiff\nodiff\nodiff\nodiff\nodiff\nodiff\nodiff\nodiff\nodiff\nodiff\nodiff\nodiff\nodiff\nodiff\nodiff\nodiff\nodiff\nodiff\nodiff\nodiff\nodiff\nodiff\nodiff\nodiff\nodiff\nodiff\nodiff\nodiff \\
  \end{tabular}
  \caption{The differential path used in the 80-step attack up to step 29.
  \label{fig:diff_path80}}
\end{figure}

\begin{figure}[!htb]
\centering
  \begin{tabular}{l l l}
$\expmess_{29}[2] = 0$ & $\expmess_{29}[28] = 0$ & $\expmess_{29}[29] = 0$ \\
$\expmess_{30}[27] \oplus \expmess_{30}[28] = 1$ & $\expmess_{30}[30] = 1$ & $\expmess_{31}[2] = 0$ \\
$\expmess_{31}[3] = 0$ & $\expmess_{31}[28] = 0$ & $\expmess_{31}[29] = 0$ \\
$\expmess_{33}[28] \oplus \expmess_{33}[29] = 1$ & $\expmess_{30}[4] \oplus \expmess_{34}[29] = 0$ & $\expmess_{35}[27] = 0$ \\
$\expmess_{35}[28] = 0$ & $\expmess_{35}[4]  \oplus \expmess_{39}[29] = 0$ & $\expmess_{58}[29] \oplus \expmess_{59}[29] = 0$ \\
$\expmess_{57}[29] \oplus \expmess_{59}[29] = 0$ & $\expmess_{55}[4]  \oplus \expmess_{59}[29] = 0$ & $\expmess_{53}[29] \oplus \expmess_{54}[29] = 0$ \\ 
$\expmess_{52}[29] \oplus \expmess_{54}[29] = 0$ & $\expmess_{51}[28] \oplus \expmess_{51}[29] = 1$ & $\expmess_{50}[4]  \oplus \expmess_{54}[29] = 0$ \\ 
$\expmess_{50}[28] \oplus \expmess_{51}[28] = 0$ & $\expmess_{50}[29] \oplus \expmess_{51}[28] = 1$ & $\expmess_{49}[28] \oplus \expmess_{51}[28] = 0$ \\ 
$\expmess_{48}[29] \oplus \expmess_{48}[30] = 0$ & $\expmess_{47}[3]  \oplus \expmess_{51}[28] = 0$ & $\expmess_{47}[4]  \oplus \expmess_{51}[28] = 1$ \\ 
$\expmess_{46}[29] \oplus \expmess_{51}[28] = 1$ & $\expmess_{45}[4]  \oplus \expmess_{51}[28] = 0$ & $\expmess_{44}[29] \oplus \expmess_{51}[28] = 0$ \\ 
$\expmess_{43}[4]  \oplus \expmess_{51}[28] = 1$ & $\expmess_{43}[29] \oplus \expmess_{51}[28] = 0$ & $\expmess_{41}[4]  \oplus \expmess_{51}[28] = 0$ \\ 
$\expmess_{63}[4]  \oplus \expmess_{67}[29] = 0$ & $\expmess_{79}[5] = 0$ & $\expmess_{78}[0] = 1$ \\
$\expmess_{77}[1] \oplus \expmess_{78}[6] = 1$ & $\expmess_{75}[5] \oplus \expmess_{79}[30] = 0$ & $\expmess_{74}[0] \oplus \expmess_{79}[30] = 1$ \\
  \end{tabular}
  \caption{The message bit relations of the 80-step attack for message words $\expmess_{29}$ to $\expmess_{79}$.
  \label{fig:msgbitrel80}}
\end{figure}

\begingroup
\fontsize{8pt}{9pt}\selectfont
\begin{verbbox}
W29:	 . . 0 0 . . . . . . . . . . . . . . . . . . . . . . . . . 0 . .
W30:	 . 1 . A a . . . . . . . . . . . . . . . . . . . . . . c . . . .
W31:	 . . 0 0 . . . . . . . . . . . . . . . . . . . . . . . . 0 0 . .
W32:	 . . . . . . . . . . . . . . . . . . . . . . . . . . . . . . . .
W33:	 . . B b . . . . . . . . . . . . . . . . . . . . . . . . . . . .
W34:	 . . c . . . . . . . . . . . . . . . . . . . . . . . . . . . . .
W35:	 . . . 0 0 . . . . . . . . . . . . . . . . . . . . . . d . . . .
W36:	 . . . . . . . . . . . . . . . . . . . . . . . . . . . . . . . .
W37:	 . . . . . . . . . . . . . . . . . . . . . . . . . . . . . . . .
W38:	 . . . . . . . . . . . . . . . . . . . . . . . . . . . . . . . .
W39:	 . . d . . . . . . . . . . . . . . . . . . . . . . . . . . . . .
W40:	 . . . . . . . . . . . . . . . . . . . . . . . . . . . . . . . .
W41:	 . . . . . . . . . . . . . . . . . . . . . . . . . . . e . . . .
W42:	 . . . . . . . . . . . . . . . . . . . . . . . . . . . . . . . .
W43:	 . . e . . . . . . . . . . . . . . . . . . . . . . . . E . . . .
W44:	 . . e . . . . . . . . . . . . . . . . . . . . . . . . . . . . .
W45:	 . . . . . . . . . . . . . . . . . . . . . . . . . . . e . . . .
W46:	 . . E . . . . . . . . . . . . . . . . . . . . . . . . . . . . .
W47:	 . . . . . . . . . . . . . . . . . . . . . . . . . . . E e . . .
W48:	 . f f . . . . . . . . . . . . . . . . . . . . . . . . . . . . .
W49:	 . . . e . . . . . . . . . . . . . . . . . . . . . . . . . . . .
W50:	 . . E e . . . . . . . . . . . . . . . . . . . . . . . g . . . .
W51:	 . . E e . . . . . . . . . . . . . . . . . . . . . . . . . . . .
W52:	 . . g . . . . . . . . . . . . . . . . . . . . . . . . . . . . .
W53:	 . . g . . . . . . . . . . . . . . . . . . . . . . . . . . . . .
W54:	 . . g . . . . . . . . . . . . . . . . . . . . . . . . . . . . .
W55:	 . . . . . . . . . . . . . . . . . . . . . . . . . . . h . . . .
W56:	 . . . . . . . . . . . . . . . . . . . . . . . . . . . . . . . .
W57:	 . . h . . . . . . . . . . . . . . . . . . . . . . . . . . . . .
W58:	 . . h . . . . . . . . . . . . . . . . . . . . . . . . . . . . .
W59:	 . . h . . . . . . . . . . . . . . . . . . . . . . . . . . . . .
W60:	 . . . . . . . . . . . . . . . . . . . . . . . . . . . . . . . .
W61:	 . . . . . . . . . . . . . . . . . . . . . . . . . . . . . . . .
W62:	 . . . . . . . . . . . . . . . . . . . . . . . . . . . . . . . .
W63:	 . . . . . . . . . . . . . . . . . . . . . . . . . . . i . . . .
W64:	 . . . . . . . . . . . . . . . . . . . . . . . . . . . . . . . .
W65:	 . . . . . . . . . . . . . . . . . . . . . . . . . . . . . . . .
W66:	 . . . . . . . . . . . . . . . . . . . . . . . . . . . . . . . .
W67:	 . . i . . . . . . . . . . . . . . . . . . . . . . . . . . . . .
W68:	 . . . . . . . . . . . . . . . . . . . . . . . . . . . . . . . .
W69:	 . . . . . . . . . . . . . . . . . . . . . . . . . . . . . . . .
W70:	 . . . . . . . . . . . . . . . . . . . . . . . . . . . . . . . .
W71:	 . . . . . . . . . . . . . . . . . . . . . . . . . . . . . . . .
W72:	 . . . . . . . . . . . . . . . . . . . . . . . . . . . . . . . .
W73:	 . . . . . . . . . . . . . . . . . . . . . . . . . . . . . . . .
W74:	 . . . . . . . . . . . . . . . . . . . . . . . . . . . . . . . j
W75:	 . . . . . . . . . . . . . . . . . . . . . . . . . . J . . . . .
W76:	 . . . . . . . . . . . . . . . . . . . . . . . . . . . . . . . .
W77:	 . . . . . . . . . . . . . . . . . . . . . . . . . . . . . . k .
W78:	 . . . . . . . . . . . . . . . . . . . . . . . . . K . . . . . 1
W79:	 . J . . . . . . . . . . . . . . . . . . . . . . . . 0 . . . . .
\end{verbbox}
\endgroup

\begin{figure}[!htb]
\centering
  \theverbbox
  \caption{The message bit-relations used in the attack for words $\expmess_{29}$ to $\expmess_{79}$ (graphical representation).
  A dot (``\texttt{.}'') means an absence of condition. A zero (``\texttt{0}'') or a one (``\texttt{1}'') character represents a bit unconditionally set to 0 or 1.
  A pair of two letters $x$ means that the two bits have the same value. A pair of two
  letters $x$ and $X$ means that the two bits have different values.
  \label{fig:msgbitrel80_graph}}
\end{figure}

\subsection{The neutral bits and boomerangs}
\label{sec:neutral_bits80}

We give here the list of the neutral bits used in the 80-step attack.
There are sixty of them over the seven message words
$\expmess_{14}$ to $\expmess_{20}$, distributed as
follows:
\begin{itemize}
\item $\expmess_{14}$: 6 neutral bits at  bit positions (starting with the least significant bit (\emph{LSB}) at zero) 5,7,8,9,10,11
\item $\expmess_{15}$: 11 neutral bits at positions 4,7,8,9,10,11,12,13,14,15,16
\item $\expmess_{16}$: 9 neutral bits at positions 8,9,10,11,12,13,14,15,16
\item $\expmess_{17}$: 10 neutral bits at positions 10,11,12,13,14,15,16,17,18,19 
\item $\expmess_{18}$: 11 neutral bits at positions 4,6,7,8,9,10,11,12,13,14,15
\item $\expmess_{19}$: 8 neutral bits at positions 6,7,8,9,10,11,12,14
\item $\expmess_{20}$: 5 neutral bits at positions 6,11,12,13,15 
\end{itemize}
We give a graphical representation of the position of these neutral bits in \autoref{fig:neutbits80}.

\begin{figure}[!htb]
\centering
\begin{tabular}{l c}
$\expmess_{14}$: & \nodiff\nodiff\nodiff\nodiff\nodiff\nodiff\nodiff\nodiff\nodiff\nodiff\nodiff\nodiff\nodiff\nodiff\nodiff\nodiff\nodiff\nodiff\nodiff\nodiff\onediff\onediff\onediff\onediff\onediff\nodiff\onediff\nodiff\nodiff\nodiff\nodiff\nodiff \\
$\expmess_{15}$: & \nodiff\nodiff\nodiff\nodiff\nodiff\nodiff\nodiff\nodiff\nodiff\nodiff\nodiff\nodiff\nodiff\nodiff\nodiff\onediff\onediff\onediff\onediff\onediff\onediff\onediff\onediff\onediff\onediff\nodiff\nodiff\onediff\nodiff\nodiff\nodiff\nodiff \\
$\expmess_{16}$: & \nodiff\nodiff\nodiff\nodiff\nodiff\nodiff\nodiff\nodiff\nodiff\nodiff\nodiff\nodiff\nodiff\nodiff\nodiff\onediff\onediff\onediff\onediff\onediff\onediff\onediff\onediff\onediff\nodiff\nodiff\nodiff\nodiff\nodiff\nodiff\nodiff\nodiff \\
$\expmess_{17}$: & \nodiff\nodiff\nodiff\nodiff\nodiff\nodiff\nodiff\nodiff\nodiff\nodiff\nodiff\nodiff\onediff\onediff\onediff\onediff\onediff\onediff\onediff\onediff\onediff\onediff\nodiff\nodiff\nodiff\nodiff\nodiff\nodiff\nodiff\nodiff\nodiff\nodiff \\
$\expmess_{18}$: & \nodiff\nodiff\nodiff\nodiff\nodiff\nodiff\nodiff\nodiff\nodiff\nodiff\nodiff\nodiff\nodiff\nodiff\nodiff\nodiff\onediff\onediff\onediff\onediff\onediff\onediff\onediff\onediff\onediff\onediff\nodiff\onediff\nodiff\nodiff\nodiff\nodiff \\
$\expmess_{19}$: & \nodiff\nodiff\nodiff\nodiff\nodiff\nodiff\nodiff\nodiff\nodiff\nodiff\nodiff\nodiff\nodiff\nodiff\nodiff\nodiff\nodiff\onediff\nodiff\onediff\onediff\onediff\onediff\onediff\onediff\onediff\nodiff\nodiff\nodiff\nodiff\nodiff\nodiff \\
$\expmess_{20}$: & \nodiff\nodiff\nodiff\nodiff\nodiff\nodiff\nodiff\nodiff\nodiff\nodiff\nodiff\nodiff\nodiff\nodiff\nodiff\nodiff\onediff\nodiff\onediff\onediff\onediff\nodiff\nodiff\nodiff\nodiff\onediff\nodiff\nodiff\nodiff\nodiff\nodiff\nodiff \\
\end{tabular}
  \caption{The sixty neutral bits of the 80-step attack, using (with some abuse) a ``difference'' notation.
  A ``\nodiff'' (resp. ``\onediff'') symbol means the absence (resp. presence) of a neutral bit on the corresponding bit.
  The message words are (as usual) written left to right from MSB to LSB.
  \label{fig:neutbits80}}
\end{figure}

Not all of the neutral bits of the same word (say $\expmess_{14}$) are used at the same step during the attack. Their repartition
in that respect is as follows
\begin{itemize}
	\item Bits neutral up to step 18 (excluded): $\expmess_{14}$[8,9,10,11], $\expmess_{15}$[13,14,15,16]
	\item Bits neutral up to step 19 (excluded): $\expmess_{14}$[5,7], $\expmess_{15}$[8,9,10,11,12], $\expmess_{16}$[12,13,14,15,16]
	\item Bits neutral up to step 20 (excluded): $\expmess_{15}$[4,7,8,9], $\expmess_{16}$[8,9,10,11,12], $\expmess_{17}$[14,15,16,\linebreak17,18,19]
	\item Bits neutral up to step 21 (excluded): $\expmess_{17}$[10,11,12,13], $\expmess_{18}$[15]
	\item Bits neutral up to step 22 (excluded): $\expmess_{18}$[9,10,11,12,13,14], $\expmess_{19}$[10,14]
	\item Bits neutral up to step 23 (excluded): $\expmess_{18}$[4,6,7,8], $\expmess_{19}$[9,11,12], $\expmess_{20}$[15]
	\item Bits neutral up to step 24 (excluded): $\expmess_{19}$[6,7,8], $\expmess_{20}$[11,12,13]
	\item Bit neutral up to step 25 (excluded): $\expmess_{20}$[7]
\end{itemize}
One should note that this list only includes a single bit per neutral bit group. As we mentioned in the previous section, some
additional flips may be needed in order to preserve message bit relations.

We also give a graphical representation of this repartition in \autoref{fig:neutbits80_2}.

\begin{figure}[ht]
\centering
\begin{tabular}{l c}
$\state_{18}$ \\
$\expmess_{14}$:  & \nodiff\nodiff\nodiff\nodiff\nodiff\nodiff\nodiff\nodiff\nodiff\nodiff\nodiff\nodiff\nodiff\nodiff\nodiff\nodiff\nodiff\nodiff\nodiff\nodiff\onediff\onediff\onediff\onediff\nodiff\nodiff\nodiff\nodiff\nodiff\nodiff\nodiff\nodiff \\ 
$\expmess_{15}$:  & \nodiff\nodiff\nodiff\nodiff\nodiff\nodiff\nodiff\nodiff\nodiff\nodiff\nodiff\nodiff\nodiff\nodiff\nodiff\onediff\onediff\onediff\onediff\nodiff\nodiff\nodiff\nodiff\nodiff\nodiff\nodiff\nodiff\nodiff\nodiff\nodiff\nodiff\nodiff \\ 
$\state_{19}$ \\
$\expmess_{14}$:  & \nodiff\nodiff\nodiff\nodiff\nodiff\nodiff\nodiff\nodiff\nodiff\nodiff\nodiff\nodiff\nodiff\nodiff\nodiff\nodiff\nodiff\nodiff\nodiff\nodiff\nodiff\nodiff\nodiff\nodiff\onediff\nodiff\onediff\nodiff\nodiff\nodiff\nodiff\nodiff \\ 
$\expmess_{15}$:  & \nodiff\nodiff\nodiff\nodiff\nodiff\nodiff\nodiff\nodiff\nodiff\nodiff\nodiff\nodiff\nodiff\nodiff\nodiff\nodiff\nodiff\nodiff\nodiff\onediff\onediff\onediff\onediff\onediff\nodiff\nodiff\nodiff\nodiff\nodiff\nodiff\nodiff\nodiff \\ 
$\expmess_{16}$:  & \nodiff\nodiff\nodiff\nodiff\nodiff\nodiff\nodiff\nodiff\nodiff\nodiff\nodiff\nodiff\nodiff\nodiff\nodiff\onediff\onediff\onediff\onediff\onediff\nodiff\nodiff\nodiff\nodiff\nodiff\nodiff\nodiff\nodiff\nodiff\nodiff\nodiff\nodiff \\ 
$\state_{20}$ \\
$\expmess_{15}$:  & \nodiff\nodiff\nodiff\nodiff\nodiff\nodiff\nodiff\nodiff\nodiff\nodiff\nodiff\nodiff\nodiff\nodiff\nodiff\nodiff\nodiff\nodiff\nodiff\nodiff\nodiff\nodiff\nodiff\nodiff\onediff\nodiff\nodiff\onediff\nodiff\nodiff\nodiff\nodiff \\ 
$\expmess_{16}$:  & \nodiff\nodiff\nodiff\nodiff\nodiff\nodiff\nodiff\nodiff\nodiff\nodiff\nodiff\nodiff\nodiff\nodiff\nodiff\nodiff\nodiff\nodiff\nodiff\nodiff\onediff\onediff\onediff\onediff\nodiff\nodiff\nodiff\nodiff\nodiff\nodiff\nodiff\nodiff \\ 
$\expmess_{17}$:  & \nodiff\nodiff\nodiff\nodiff\nodiff\nodiff\nodiff\nodiff\nodiff\nodiff\nodiff\nodiff\onediff\onediff\onediff\onediff\onediff\onediff\nodiff\nodiff\nodiff\nodiff\nodiff\nodiff\nodiff\nodiff\nodiff\nodiff\nodiff\nodiff\nodiff\nodiff \\ 
$\state_{21}$ \\
$\expmess_{17}$:  & \nodiff\nodiff\nodiff\nodiff\nodiff\nodiff\nodiff\nodiff\nodiff\nodiff\nodiff\nodiff\nodiff\nodiff\nodiff\nodiff\nodiff\nodiff\onediff\onediff\onediff\onediff\nodiff\nodiff\nodiff\nodiff\nodiff\nodiff\nodiff\nodiff\nodiff\nodiff \\ 
$\expmess_{18}$:  & \nodiff\nodiff\nodiff\nodiff\nodiff\nodiff\nodiff\nodiff\nodiff\nodiff\nodiff\nodiff\nodiff\nodiff\nodiff\nodiff\onediff\nodiff\nodiff\nodiff\nodiff\nodiff\nodiff\nodiff\nodiff\nodiff\nodiff\nodiff\nodiff\nodiff\nodiff\nodiff \\ 
$\state_{22}$ \\
$\expmess_{18}$:  & \nodiff\nodiff\nodiff\nodiff\nodiff\nodiff\nodiff\nodiff\nodiff\nodiff\nodiff\nodiff\nodiff\nodiff\nodiff\nodiff\nodiff\onediff\onediff\onediff\onediff\onediff\onediff\nodiff\nodiff\nodiff\nodiff\nodiff\nodiff\nodiff\nodiff\nodiff \\ 
$\expmess_{19}$:  & \nodiff\nodiff\nodiff\nodiff\nodiff\nodiff\nodiff\nodiff\nodiff\nodiff\nodiff\nodiff\nodiff\nodiff\nodiff\nodiff\nodiff\onediff\nodiff\nodiff\nodiff\onediff\nodiff\nodiff\nodiff\nodiff\nodiff\nodiff\nodiff\nodiff\nodiff\nodiff \\ 
$\state_{23}$ \\
$\expmess_{18}$:  & \nodiff\nodiff\nodiff\nodiff\nodiff\nodiff\nodiff\nodiff\nodiff\nodiff\nodiff\nodiff\nodiff\nodiff\nodiff\nodiff\nodiff\nodiff\nodiff\nodiff\nodiff\nodiff\nodiff\onediff\onediff\onediff\nodiff\onediff\nodiff\nodiff\nodiff\nodiff \\ 
$\expmess_{19}$:  & \nodiff\nodiff\nodiff\nodiff\nodiff\nodiff\nodiff\nodiff\nodiff\nodiff\nodiff\nodiff\nodiff\nodiff\nodiff\nodiff\nodiff\nodiff\nodiff\onediff\onediff\nodiff\onediff\nodiff\nodiff\nodiff\nodiff\nodiff\nodiff\nodiff\nodiff\nodiff \\ 
$\expmess_{20}$:  & \nodiff\nodiff\nodiff\nodiff\nodiff\nodiff\nodiff\nodiff\nodiff\nodiff\nodiff\nodiff\nodiff\nodiff\nodiff\nodiff\onediff\nodiff\nodiff\nodiff\nodiff\nodiff\nodiff\nodiff\nodiff\nodiff\nodiff\nodiff\nodiff\nodiff\nodiff\nodiff \\ 
$\state_{24}$ \\
$\expmess_{19}$:  & \nodiff\nodiff\nodiff\nodiff\nodiff\nodiff\nodiff\nodiff\nodiff\nodiff\nodiff\nodiff\nodiff\nodiff\nodiff\nodiff\nodiff\nodiff\nodiff\nodiff\nodiff\nodiff\nodiff\onediff\onediff\onediff\nodiff\nodiff\nodiff\nodiff\nodiff\nodiff \\ 
$\expmess_{20}$:  & \nodiff\nodiff\nodiff\nodiff\nodiff\nodiff\nodiff\nodiff\nodiff\nodiff\nodiff\nodiff\nodiff\nodiff\nodiff\nodiff\nodiff\nodiff\onediff\onediff\onediff\nodiff\nodiff\nodiff\nodiff\nodiff\nodiff\nodiff\nodiff\nodiff\nodiff\nodiff \\ 
$\state_{25}$ \\
$\expmess_{20}$:  & \nodiff\nodiff\nodiff\nodiff\nodiff\nodiff\nodiff\nodiff\nodiff\nodiff\nodiff\nodiff\nodiff\nodiff\nodiff\nodiff\nodiff\nodiff\nodiff\nodiff\nodiff\nodiff\nodiff\nodiff\nodiff\onediff\nodiff\nodiff\nodiff\nodiff\nodiff\nodiff \\ 
	\end{tabular}
  \caption{The sixty neutral bits regrouped by the first state where they start to interact. A ``\onediff'' represents the presence
  of a neutral bit, and a ``\nodiff'' the absence thereof. The LSB position is the rightmost one.
  \label{fig:neutbits80_2}}
\end{figure}

In addition to the ``single'' neutral bits, the 80-step attack also uses boomerangs. These are regrouped in two sets of two.
The first one first introduces a difference in the message at word $\expmess_{10}$;
as it does not significantly impact conditions up to step 27, it is used to increase the number of partial solutions for $\state_{28}$ that are generated.
The second set first introduces a difference at word $\expmess_{11}$, and is used to generate partial solutions up to $\state_{30}$.
More precisely, the four boomerangs have their first differences on bits 7,8 of $\expmess_{10}$ and 8,9 of $\expmess_{11}$.
In \autoref{fig:boom_coll}, we give a graphical representation of the complete set of message bits to be flipped for each
boomerang. One can see that these follow the pattern of a local collisions, with some ``linear'' corrections omitted thanks to the absorption
properties of the $\fif$ Boolean function. 

\begin{figure}[!htb]
\centering
\begin{tabular}{l c}
$\expmess_{10}$: & \nodiff\nodiff\nodiff\nodiff\nodiff\nodiff\nodiff\nodiff\nodiff\nodiff\nodiff\nodiff\nodiff\nodiff\nodiff\nodiff\nodiff\nodiff\nodiff\nodiff\nodiff\nodiff$\monediffu$\diffup\nodiff\nodiff\nodiff\nodiff\nodiff\nodiff\nodiff\nodiff \\ 
$\expmess_{11}$: & \nodiff\nodiff\nodiff\nodiff\nodiff\nodiff\nodiff\nodiff\nodiff\nodiff\nodiff\nodiff\nodiff\nodiff\nodiff\nodiff\nodiff$\mnodiffo$\equaup\nodiff\nodiff\nodiff\nodiff$\monediffd$\diffrightup\nodiff\nodiff\nodiff\nodiff\nodiff\nodiff\nodiff \\
$\expmess_{12}$: & \nodiff\nodiff\nodiff\nodiff\nodiff\nodiff\nodiff\nodiff\nodiff\nodiff\nodiff\nodiff\nodiff\nodiff\nodiff\nodiff\nodiff\nodiff$\mnodiffz$\equarightup\nodiff\nodiff\nodiff\nodiff\nodiff\nodiff\nodiff\nodiff\nodiff\nodiff\nodiff\nodiff \\
$\expmess_{13}$: & \nodiff\nodiff\nodiff\nodiff\nodiff\nodiff\nodiff\nodiff\nodiff\nodiff\nodiff\nodiff\nodiff\nodiff\nodiff\nodiff\nodiff\nodiff\nodiff\nodiff\nodiff\nodiff\nodiff\nodiff\nodiff\nodiff\nodiff\nodiff\nodiff\nodiff\nodiff\nodiff \\
$\expmess_{14}$: & \nodiff\nodiff\nodiff\nodiff\nodiff\nodiff\nodiff\nodiff\nodiff\nodiff\nodiff\nodiff\nodiff\nodiff\nodiff\nodiff\nodiff\nodiff\nodiff\nodiff\nodiff\nodiff\nodiff\nodiff\nodiff\equaup\nodiff\nodiff\nodiff\nodiff\nodiff\nodiff \\
$\expmess_{15}$: & \nodiff\nodiff\nodiff\nodiff\nodiff\nodiff\nodiff\nodiff\nodiff\nodiff\nodiff\nodiff\nodiff\nodiff\nodiff\nodiff\nodiff\nodiff\nodiff\nodiff\nodiff\nodiff\nodiff\nodiff$\mnodiffo$\equaup\nodiff\nodiff\nodiff\nodiff\nodiff\nodiff \\
$\expmess_{16}$: & \nodiff\nodiff\nodiff\nodiff\nodiff\nodiff\nodiff\nodiff\nodiff\nodiff\nodiff\nodiff\nodiff\nodiff\nodiff\nodiff\nodiff\nodiff\nodiff\nodiff\nodiff\nodiff\nodiff\nodiff\nodiff$\mnodiffz$\equarightup\nodiff\nodiff\nodiff\nodiff\nodiff \\
\end{tabular}
  \caption{The local collision patterns for each of the four boomerangs, using ``difference'' symbols by an abuse of notation. The position of the first difference to be introduced is highlighted
  with a difference (black) symbol; the associated correcting differences (identified by the corresponding white symbols) must then have a sign different from this one. Note that boomerang ``\diffup'' uses one more difference than the others.
  \label{fig:boom_coll}}
\end{figure}

We conclude by showing how the neutral bits are packed together with the index of an (extended) base solution in \autoref{fig:nb_packing80_1} and \autoref{fig:nb_packing80_1}.
Note that neutral bits on $\expmess_{17}$ are split between the buffers for steps 18--20 and 21--25. Furthermore, the packing of steps 21--25 also includes
some ``flip'' values, which are partial sums of some selected neutral bits that aid in determining if additionnal bits need to be flipped so as to preserve message bit relations.
The representation of the packing is done similarly as in \autoref{sec:neutral_bits76}.

\begin{verbbox}[\vspace{1mm}]
For steps A18--20
W14:a W15:b W16:c W17:d basesol:i
ccccccccc...bbbbbbbbbb..baaaaa.a
dddddd......iiiiiiiiiiiiiiiiiiii

For steps A21--25
W17:a W18:b W19:c W20:d extsol:i
Additional flips:F
bbbbbbbbbb.b..F.d.ddd....d.aaaaF
F.Fc.ccccccciiiiiiiiiiiiiiiiiiii
\end{verbbox}

\begin{figure}[ht]
\centering
  \theverbbox
  \caption{The packing of the neutral bits and the (extended) base solution index.
  A letter \{\texttt{a,b,c,d}\} represents a bit on a given word, an ``\texttt{F}'' a flip bit, an ``\texttt{i}'' a bit
  of an (extended) solution index, and  a ``\texttt{.}'' an unused bit (only present to maintain a proper alignment).
  \label{fig:nb_packing80_1}}
\end{figure}



\subsection{An example of colliding message pair}
\label{sec:colli_ex80}

We give an example of 80-step collision in \autoref{tbl:fscoll80}.
This shows the two (message, \iv) pairs with their (identical) resulting digest.
This table is formatted in the same way as the one of \autoref{sec:colli_ex76}.

\begin{table}[!htb]
\caption{A freestart collision for 80-step \shaone. Message and \iv bytes with differences are highlighted with \framebox{\color{LimeGreen}coloured boxes}.}\label{tbl:fscoll80}
\centering
\begin{tabular}{c c}
\toprule
 & Message 1\\
\midrule
\iv &  \hspace{-1.95mm}\tt 50 6b 01 78 ff 6d 18 \framebox{\color{Cerulean}90 20} 22 91 fd 3a de 38 71 b2 c6 65 ea \\
\midrule
$\mess$ & \tt \framebox{\color{Cerulean}9d} 44 38 \framebox{\color{Cerulean}28 a5} ea 3d \framebox{\color{Cerulean}f0 86} ea a0 \framebox{\color{Cerulean}fa 77} 83 a7 \framebox{\color{Cerulean}36}\\
      & \tt \hspace{1.15mm}33\hspace{1.25mm} 24 48 \framebox{\color{Cerulean}4d af} 70 2a \framebox{\color{Cerulean}aa a3} da b6 \framebox{\color{Cerulean}79 d8} a6 9e \framebox{\color{Cerulean}2d} \\
			& \tt \framebox{\color{Cerulean}54} 38 20 \framebox{\color{Cerulean}ed a7} ff fb \framebox{\color{Cerulean}52 d3} ff 49 \framebox{\color{Cerulean}3f c3} ff 55 \framebox{\color{Cerulean}1e} \\
			& \tt \framebox{\color{Cerulean}fb} ff d9 \framebox{\color{Cerulean}7f 55} fe ee \framebox{\color{Cerulean}f2 08} 5a f3 \framebox{\color{Cerulean}12 08} 86 88 \framebox{\color{Cerulean}a9} \\
\midrule
$\compress(\iv,\mess)$ & \tt f0 20 48 6f 07 1b f1 10 53 54 7a 86 f4 a7 15 3b 3c 95 0f 4b \\
\bottomrule\\

\toprule
 & Message 2 \\
\midrule
$\diff\iv$ & \hspace{-1.95mm}\tt 50 6b 01 78 ff 6d 18 \framebox{\color{RubineRed}91 a0} 22 91 fd 3a de 38 71 b2 c6 65 ea \\
\midrule
$\diff\mess$ & \tt \framebox{\color{RubineRed}3f} 44 38 \framebox{\color{RubineRed}38 81} ea 3d \framebox{\color{RubineRed}ec a0} ea a0 \framebox{\color{RubineRed}ee 51} 83 a7 \framebox{\color{RubineRed}2c} \\
      & \tt \hspace{1.15mm}33\hspace{1.25mm} 24 48 \framebox{\color{RubineRed}5d ab} 70 2a \framebox{\color{RubineRed}b6 6f} da b6 \framebox{\color{RubineRed}6d d4} a6 9e \framebox{\color{RubineRed}2f} \\
			& \tt \framebox{\color{RubineRed}94} 38 20 \framebox{\color{RubineRed}fd 13} ff fb \framebox{\color{RubineRed}4e ef} ff 49 \framebox{\color{RubineRed}3b 7f} ff 55 \framebox{\color{RubineRed}04} \\
			& \tt \framebox{\color{RubineRed}db} ff d9 \framebox{\color{RubineRed}6f 71} fe ee \framebox{\color{RubineRed}ee e4} 5a f3 \framebox{\color{RubineRed}06 04} 86 88 \framebox{\color{RubineRed}ab} \\
\midrule
$\compress(\diff\iv,\diff\mess)$ & \tt f0 20 48 6f 07 1b f1 10 53 54 7a 86 f4 a7 15 3b 3c 95 0f 4b \\
\bottomrule
\end{tabular}
\end{table}


\FloatBarrier

\section{Conclusion}
\label{sec:conclusion}

The work described in this chapter culminated in the computation of an explicit freestart collision for the full \shaone. Although it remains the case that no collision for the entire hash algorithm is known,
the progress we have made does allow us to precisely estimate and update what would be the computational and financial cost to generate such a collision, using the latest cryptanalysis results \cite{DBLP:conf/eurocrypt/Stevens13};
the computational cost required to generate such a collision is actually a recurrent debate in the academic community since the first theoretical attack from Wang~\etal~\cite{DBLP:conf/crypto/WangYY05a}.

Schneier's projections~\cite{schneierSHA1} on the cost of \shaone collisions, made in 2012 (on Amazon EC2: $\approx$700K US\$ by 2015, $\approx$173K US\$ by 2018 and $\approx$43K US\$ by 2021) were based on
(an early announcement of) \cite{DBLP:conf/eurocrypt/Stevens13}. These projections have been used to establish the timeline of migrating away from \shaone-based signatures for secure Internet websites,
resulting in a migration by January 2017 ---one year before Schneier estimated that a \shaone collision would be within the resources of criminal syndicates. 

This work demonstrated that GPUs are much faster for this type of attacks (compared to CPUs)
and we could estimate that at the time where thiw work was done, in the autumn 2015, a full \shaone collision would not cost more than between 75K and 120K US\$ by renting computational power on Amazon EC2.
Our new GPU-based projections are more accurate and significantly below Schneier's estimations. What could be considered more worrying
is that they were theoretically already within Schneier's estimated financial resources of criminal syndicates as of the end of 2015,
almost two years earlier than previously expected, and one year before \shaone would start being marked as unsafe in modern Internet browsers.
This led us to recommend that migration from \shaone to the secure \shatwo or \shathree hash algorithms should be done sooner than previously planned.

It had previously been shown that a more advanced \emph{chosen-prefix collision} attack on \mdfive allowed the creation of a rogue Certification Authority undermining the security of all secure websites \cite{DBLP:conf/crypto/StevensSALMOW09}. 
Collisions on \shaone can result in \eg{} signature forgeries, but do not directly undermine the security of the Internet at large. Chosen-prefix collisions
are significantly more threatening, but currently much costlier to mount for \shaone. Yet, given the lessons learned with the \mdfive full collision break, it is not advisable to wait until these become practically possible
to move away from using \shaone.

\medskip

At the time of the find of the 80-step freestart collision, in October 2015, we learned that in an ironic turn of events the CA/Browser Forum (which is the main association of industries regulating the use of digital certificates on the Internet)
was planning to hold a ballot to decide whether to extend issuance of \shaone certificates through the year 2016 \cite{cabforum}.
With our new cost projections in mind, we strongly recommended against this extension and the ballot was also subsequently withdrawn \cite{cabforum2}.
Further action was subsequently considered by major browser providers such as Microsoft \cite{MS_sha} and Mozilla \cite{Moz_sha} to speed up the deprectation of \shaone certificates.


\chapter[Attaques en préimages sur \shaone]
        {Preimage attacks for \shaone}
\label{cha:shaone_pre}

\section{Introduction}

This short chapter presents new preimage attacks for the \shaone hash function.
The results are obtained by extending the differential view of \mitm preimage attacks with higher-order differentials.

\medskip

The starting point of the work described here is the \mitm technique, which
was first used in cryptography by Diffie and Hellman in 1977 to attack double-encryption~\cite{DH77}.
Its use for preimage attack is much more recent and is due to Aoki and Sasaki, who used it as a framework to
attack various hash functions, including for instance SHA-0 and \shaone~\cite{AS09}.
The basic principle behind a \mitm technique is to exploit the fact that some intermediate value of
a function's computation can be expressed
in two different ways involving different parts of a secret, which can then be sampled independently of
each other. In the case of hash function cryptanalysis, there is no actual secret to consider, but a
similar technique can nonetheless be exploited in certain cases; we show in more details how to do
so in the preliminaries of~\autoref{KKFramework}.

At CRYPTO~2012, Knellwolf and Khovratovich introduced a differential formulation of the \mitm framework
of Aoki and Sasaki, which they used to improve the best attacks on \shaone~\cite{DBLP:conf/crypto/KnellwolfK12}. One of the main interests
of their approach is that it simplifies the formulation of several advanced extensions of
the \mitm technique, and thereby facilitates the search for attack parameters, which in the case
of \mitm attacks roughly correspond to good partitions for the ``secret''.

In this chapter, we further exploit this differential formulation and generalise it to use higher-order
differentials, which were introduced in cryptography by Lai in 1994~\cite{L94}. The essence
of this technique is to consider ``standard'' differential cryptanalysis as exploiting properties
of the first-order derivative of the function one wishes to analyse; it is then somehow natural
to generalise the idea and to consider higher-order derivatives as well. We illustrate this
with a small example using XOR differences: consider a function $\Fs$ and
assume the equality $\Delta_\alpha\Fs(x) \defas \Fs(x) \oplus \Fs(x \oplus \alpha) = A$ holds with a good probability
over the values of $x$; this
is the same as saying that the derivative of $\Fs$ in $\alpha$ is biased towards $A$. In an extreme case,
if $\Fs$ is linear, then $\Delta_\alpha\Fs$ is constantly equal to $\Fs(\alpha)$.
%, though this is obviously not true in general.
Now if we consider the expression $\Delta_\alpha\Fs(x) \oplus \Delta_\alpha\Fs(x \oplus \beta) =
\Fs(x) \oplus \Fs(x \oplus \alpha) \oplus \Fs(x \oplus \beta)
\oplus \Fs(x \oplus \alpha \oplus \beta)$, this corresponds to taking the derivative of $\Fs$ twice,
first
in $\alpha$, and then in $\beta$. A possible advantage of doing this is that the resulting function may be more
biased than $\Delta_\alpha\Fs$ was, for instance by being constant when $\Delta_\alpha\Fs$ is linear.
This process can be iterated at will, each time decreasing the algebraic degree of the resulting
function until it reaches zero.

As higher-order differentials are obviously best formulated in differential form,
they combine neatly with the differential view of the framework of Knellwolf and Khovratovich, whereas
using a similar technique in a \mitm attack independently of any differential formulation would probably prove to be
much more difficult.
As a final motivation for this generalisation, we show a small application to
the analysis of the \mdfour hash function~\cite{Rivest-md4}. This does not improve the best known
preimage attacks~\cite{md4p2,md4p3}, but gives a good illustration of the potential of the technique.

\paragraph{Higher-order differentials for the compression function of \mdfour.}

The inverse of the state update function inside the compression function of \mdfour is of
the form: $q_{i-4} \leftarrow (q_i \circlearrowleft s_i) - \boolF(q_{i-1}, q_{i-2}, q_{i-3}) - m_j$,
with
the subtraction being done modulo $2^{32}$.
Four consecutive steps of this inverse function can thus be written as:
\[
q_3 \leftarrow (q_7 \circlearrowleft s_7) - \boolF(q_6, q_5, q_4) - m_7
\]
\[
q_2 \leftarrow (q_6 \circlearrowleft s_6) - \boolF(q_5, q_4, q_3) - m_6
\]
\[
q_1 \leftarrow (q_6 \circlearrowleft s_5) - \boolF(q_4, q_3, q_2) - m_5
\]
\[
q_0 \leftarrow (q_4 \circlearrowleft s_4) - \boolF(q_3, q_2, q_1) - m_4
\]

If we consider order-2 differentials on $m_6$ and $m_7$, with 
additive differences modulo $2^{32}$ concentrated around
the most significant bit, that is of the form $\onediff\onediff\onediff\onediff\nodiff\nodiff\nodiff\nodiff$,
computing the state update from above on these differences results in
a state $q_{0\ldots3}$ with differences of the form:

\medskip

\begin{tabular}{l l l l l l}
\phantom{toto}$q_3^\ddagger$: & \dunnodiff\dunnodiff\dunnodiff\dunnodiff\nodiff\nodiff\nodiff\nodiff & \phantom{toto} $q_3^\star$: &
\nodiff\nodiff\nodiff\nodiff\nodiff\nodiff\nodiff\nodiff & \phantom{toto} $q_3^\dagger$:  & equal to $q_3^\ddagger$ \\
\phantom{toto}$q_2^\ddagger$: & \dunnodiff\dunnodiff\dunnodiff\dunnodiff\nodiff\nodiff\nodiff\nodiff & \phantom{toto} $q_2^\star$: &
\dunnodiff\dunnodiff\dunnodiff\dunnodiff\nodiff\nodiff\nodiff\nodiff     & \phantom{toto} $q_2^\dagger$:  & \dunnodiff\dunnodiff\dunnodiff\dunnodiff\nodiff\nodiff\nodiff\nodiff \\
\phantom{toto}$q_1^\ddagger$: & \dunnodiff\dunnodiff\dunnodiff\dunnodiff\nodiff\nodiff\nodiff\nodiff & \phantom{toto} $q_1^\star$: &
\dunnodiff\dunnodiff\dunnodiff\dunnodiff\nodiff\nodiff\nodiff\nodiff     & \phantom{toto} $q_1^\dagger$:  & \dunnodiff\dunnodiff\dunnodiff\dunnodiff\nodiff\nodiff\nodiff\nodiff \\
\phantom{toto}$q_0^\ddagger$: & \dunnodiff\dunnodiff\dunnodiff\dunnodiff\nodiff\nodiff\nodiff\nodiff & \phantom{toto} $q_0^\star$: &
\dunnodiff\dunnodiff\dunnodiff\dunnodiff\nodiff\nodiff\nodiff\nodiff     & \phantom{toto} $q_0^\dagger$:  & \dunnodiff\dunnodiff\dunnodiff\dunnodiff\nodiff\nodiff\nodiff\nodiff \\
\end{tabular}

\smallskip

\begin{tabular}{l l l}
\phantom{toto} For diff. on $m_7$ & \phantom{toto} For diff. on $m_6$ & \phantom{toto} For diff. on $m_6$ \& $m_7$\\
\end{tabular}

Thus there is no difference on the value $q_3^\dagger - q_3^\star - q_3^\ddagger + q_3$, and
the differences on the remaining words also have a high probability. This good differential behaviour
can then be exploited in a later attack.

It is worth noting that in the very case of \mdfour, one could also use very
good local collisions from order-1 message differences, and higher-order differentials do not typically outperform these; we just
gave them for illustration.

\subsection{Previous and new results on \shaone}
The first preimage attacks on \shaone were due to De~Canni\`ere and Rechberger~\cite{DBLP:conf/crypto/CanniereR08},
who used a system-based approach that in particular allows to compute practical preimages for a non-trivial
number of steps. In order to attack more steps, Aoki and Sasaki later used a \mitm approach~\cite{AS09}.
This was subsequently improved by Knellwolf and Khovratovich~\cite{DBLP:conf/crypto/KnellwolfK12}, who attacked the highest number
of rounds so far. To be more precise,
they attack reduced versions of the function up to 52 steps for \emph{one-block preimages with padding},
57 steps for \emph{one-block preimages without padding}, and 60 steps for \emph{one-block pseudo-preimages
with padding}, \ie freestart preimages with padding. The latter two attacks can be combined to give 57 steps \emph{two-block preimages with padding}.
In this work, we present \emph{one-block preimages with padding} up to 56 steps,
\emph{one-block preimages without padding} up to 62 steps, \emph{one-block pseudo preimages with padding} up
to 64 steps, resulting in \emph{two-block preimages with padding} up to 62 steps.

We give a summary of these results in \autoref{tbl:res}.

\begin{table}[!htb]
\caption[Existing and new preimage attacks on \shaone.]{Existing and new preimage attacks on \shaone, with complexity given in base-2 logarithm.\label{tbl:res}}
\begin{center}
\begin{tabularx}{\textwidth}{@{\extracolsep{2mm} } l c c X  X}
\toprule
Function & \# blocks & \# rounds &  complexity &  ref.\\
\toprule
 \multirow{6}{*}{\shaone} & 1 & 52 & 158.4 & \cite{DBLP:conf/crypto/KnellwolfK12} \\
 & 1 & 52 & 156.7 & \autoref{sec:one_wi_pad} \\
 & 1 & 56 & 159.4 & \autoref{sec:one_wi_pad} \\
 & 2 & 57 & 158.8 & \cite{DBLP:conf/crypto/KnellwolfK12} \\
 & 2 & 58 & 157.9 & \autoref{sec:one_two} \\
 & 2 & 62 & 159.3 & \autoref{sec:one_two} \\
\midrule
\multirow{3}{*}{\shaone, without padding} & 1 & 57 & 158.7 & \cite{DBLP:conf/crypto/KnellwolfK12} \\
 & 1 & 58 & 157.4 & \autoref{sec:one_wo_pad} \\
 & 1 & 62 & 159 & \autoref{sec:one_wo_pad} \\
\midrule
\multirow{3}{*}{\shaone, pseudo-preimage} & 1 & 60 & 157.4 & \cite{DBLP:conf/crypto/KnellwolfK12} \\
 & 1 & 61 & 156.4 & \autoref{sec:one_two} \\
 & 1 & 64 & 158.7 & \autoref{sec:one_two} \\
\bottomrule
\end{tabularx}

\end{center}
\end{table}


\section{Meet-in-The-Middle Attacks and the Differential Framework from CRYPTO 2012}
\label{KKFramework}

As a preliminary, we give a description of the \mitm framework for preimage attacks on hash functions,
and in particular of the differential formulation of Knellwolf and Khovratovich from CRYPTO~2012~\cite{DBLP:conf/crypto/KnellwolfK12}.

The relevance of \mitm for preimage attacks comes from the fact that many hash functions are built from a compression
function which is an \ah block cipher used in one of the ``PGV'' modes~\cite{PGV93}.
The \shaone function itself follows this design strategy and uses the particular Davies-Meyer mode, already described
in \autoref{chap:hashfun}.
We recall that in this mode, the compression function $\Hc~: \{0, 1\}^v \times \{0, 1\}^n \rightarrow \{0, 1\}^v$ compressing a chaining value $c$ with
a message $m$ to form the updated chaining value $c' \defas \Hc(c, m)$ is defined as $\Hc(c, m) = \Fs(m, c) + c$, with
$\Fs~: \{0, 1\}^n \times \{0, 1\}^v \rightarrow \{0, 1\}^v$ a block cipher of key-length and message-length $n$ and $v$.
Given a compression function $\Hc$,
the problem of finding a preimage of $t$ for $\Hc$ is then equivalent to finding a key $m$ for
$\Fs$
such that $\Fs(m, p) = c$ for a pair $(p, c)$, with $c = t - p$. Additional constraints can also be put on $p$, such as prescribing
it to a fixed initialization value $\iv$.

In its most basic form, a \mitm attack can speed-up the search for a preimage if the block cipher $\Fs$ can equivalently be
described as the composition $\Fd \circ \Fuu$ of two block ciphers $\Fuu~: \ksu \times \{0, 1\}^v
\rightarrow \{0, 1\}^v$ and $\Fd~: \ksd \times \{0, 1\}^v\rightarrow \{0, 1\}^v$ with independent key spaces $\ksu, \ksd \subset \{0,1\}^n$.
Indeed, if such a decomposition is possible, an attacker can select a subset $\{k^1_i, i = 1\ldots N_1\}$ (resp. $\{k^2_j, j = 1\ldots N_2\}$)
of keys of $\ksu$ (resp. $\ksd$), which
together suggest $N \defas N_1 \cdot N_2$ candidate keys $k^{12}_{ij} \defas (k^1_i, k^2_j)$ for $\Fs$ by setting
$\Fs(k^{12}_{ij},\cdot) = \Fd(k^2_j,\cdot) \circ \Fuu(k^1_i,\cdot)$.

Since the two sets $\{\Fuu(k^1_i, p), i = 1\ldots N_1\}$ and $\{\Fd^{-1}(k^2_j, c), j = 1\ldots N_2\}$ can be computed  independently, the complexity
of testing $\Fs(k^{12}_{ij},p) = c$ for
$N$ keys is only of $\bigo(\max(N_1, N_2))$ time and $\bigo(\min(N_1, N_2))$ memory, which is less than $N$
and can be as low as $\sqrt{N}$ when $N_1 = N_2$.

\subsection{Formalizing \mitm attacks with related-key differentials}

Let us denote by $(\alpha,  \beta) \overset{\Fs}{\underset{p}{\longrightarrow}} \gamma$ the fact that 
$\underset{(x,y)}{\Pr}\big[\Fs(x \oplus \alpha, y \oplus \beta) = \Fs(x, y) \oplus \gamma\big] = p$, meaning
that $(\alpha, \beta)$ is a related-key differential for $\Fs$ that holds with probability $p$.
The goal of an attacker is now to find two linear sub-spaces $D_1$ and $D_2$ of $\{0,1\}^m$ such that:
\begin{equation}
D_1 \cap D_2 = \{0\}
\end{equation}
\vspace{-4mm}
\begin{equation}
\forall \delta_1 \in D_1~\exists~ \Delta_1 \in \{0,1\}^v \text{ s.t. } (\delta_1, 0) \overset{\Fuu}{\underset{1}{\longrightarrow}} \Delta_1
\label{diff1}
\end{equation}
\vspace{-4mm}
\begin{equation}
\forall \delta_2 \in D_2~ \exists~ \Delta_2 \in \{0,1\}^v \text{ s.t. } (\delta_2, 0) \overset{\Fd^{-1}}{\underset{1}{\longrightarrow}} \Delta_2.
\label{diff2}
\end{equation}
Let $d_1$ and $d_2$ be the dimension of $D_1$ and $D_2$ respectively. Then for a set $M$ of messages $\mu_i \in \{0,1\}^m$, or more
precisely the quotient space of $\{0,1\}^m$ by $D_1 \oplus D_2$,
one can define $\#M$ distinct sets $\mu_i \oplus D_1 \oplus D_2$ of dimension $d_1 + d_2$ (and size $2^{d_1 + d_2}$),
which can be tested for a preimage with a complexity of only $\bigo(\max(2^{d_1}, 2^{d_2}))$ time and $\bigo(\min(2^{d_1}, 2^{d_2}))$ memory.
We recall the procedure to do so in \autoref{affine_testing}.
 
\begin{algorithm}[ht]
\LinesNumbered
\KwIn{
	$D_1, D_2 \subset \{0,1\}^m$, $\mu \in \{0,1\}^m$, $p$, $c$
}
\KwOut{
	A preimage of $c + p$ if there is one in $\mu \oplus D_1 \oplus D_2$, $\bot$ otherwise
}
\KwData{ Two lists $L_1, L_2$ indexed by $\delta_2, \delta_1$ respectively
}

\ForAll{$\delta_2 \in D_2$
}
{
	$L_1[\delta_2] \leftarrow \Fuu(\mu \oplus \delta_2, p) \oplus \Delta_2$
}
\ForAll{$\delta_1 \in D_1$
}
{
	$L_2[\delta_1] \leftarrow \Fd^{-1}(\mu \oplus \delta_1, c) \oplus \Delta_1$
}
\ForAll{$(\delta_1, \delta_2) \in D_1 \times D_2$
}
{
\If{$L_1[\delta_2] = L_2[\delta_1]$}
{
	\Return{$\mu \oplus \delta_1 \oplus \delta_2$}	
}
}
\Return{$\bot$}
\caption{\label{affine_testing}Testing $\mu \oplus D_1 \oplus D_2$ for a preimage~\cite{DBLP:conf/crypto/KnellwolfK12}}

\end{algorithm}

\subsubsection{Analysis of \autoref{affine_testing}.}
For the sake of simplicity we assume that $d_1$ and $d_2$ are equal to a certain value $d < \frac{v}{2}$.
The running time of every loop of \autoref{affine_testing} is therefore $\bigo(2^d)$, assuming efficient
data structures and equality testing for the lists,
and $\bigo(2^d)$ memory is necessary for storing $L_1$ and $L_2$. It is also
clear that if the condition $L_1[\delta_2] = L_2[\delta_1]$ is met, then $\mu \oplus \delta_1 \oplus \delta_2$
is a preimage of $c + p$. Indeed, this translates to $\Fuu(\mu \oplus \delta_2, p) \oplus \Delta_2
= \Fd^{-1}(\mu \oplus \delta_1, c) \oplus \Delta_1$, and using the differential properties
of $D_1$ and $D_2$ for $\Fuu$ and $\Fd$, we
have that $\Fuu(\mu \oplus \delta_1 \oplus \delta_2, p) = \Fuu(\mu \oplus \delta_2, p) \oplus \Delta_1$
and $\Fd^{-1}(\mu \oplus \delta_1 \oplus \delta_2, c) = \Fd^{-1}(\mu \oplus \delta_1, c) \oplus \Delta_2$.
Hence, $\Fuu(\mu \oplus \delta_1 \oplus \delta_2, p) = \Fd^{-1}(\mu \oplus \delta_1 \oplus \delta_2$), and
$\Fs(\mu \oplus \delta_1 \oplus \delta_2, p) = c$.
This algorithm therefore allows to search through $2^{2d}$ candidate preimages with a complexity
of $\bigo(2^d)$, and thus gives a speed-up of $2^d$. The complexity of a full attack is hence $\bigo(2^{v - d})$.

\subsubsection{Comparison with the basic \mitm attack.}
When setting $\Delta_1 = \Delta_2 = 0$, this differential variant of the \mitm technique becomes a special case
of the general formulation of the basic technique given at the beginning of this section: the key spaces $\ksu$ and $\ksd$ now possess a structure
of affine spaces.
Although this is a restriction on the shape of the key partition, the additional structure is useful when one is intent on finding actual attacks.

The advantage of the differential view of the \mitm attack then comes from the fact that it gives a practical way of searching for the key spaces,
as differential path search is a well-studied area of symmetric cryptanalysis. Another major advantage is that it makes the formulation of
several extensions to the basic attack very natural, without compromising the ease of the search for the key spaces. One such immediate
extension is obviously to consider non-zero values for $\Delta_1$ and $\Delta_2$. As noted by Knellwolf and Khovratovich,
this already corresponds to an advanced technique of \emph{indirect matching} in the original framework of Aoki and Sasaki.
Further extensions are detailed next.

\subsection{Probabilistic truncated differential \mitm}
\label{probtrunc}

There are two natural ways to generalise the differential formulation of the \mitm, that each correspond to relaxing one condition.
First, one can consider differentials of probability less than one ---although a high probability is still usually needed;
second, one can consider truncated differentials by using an
equivalence relation ``$\equiv$'' instead of the equality, with one usually taking $\equiv$ to be a truncated equality:
$a \equiv b~[m] \Leftrightarrow a \wedge m = b \wedge m$ for $a, b, m
\in \{0,1\}^v$.
One can then use the notation
$(\alpha,  \beta) \overset{\Fs}{\underset{p}{\rightsquigarrow}} \gamma$ to denote the fact that 
$\underset{(x,y)}{\Pr}\big[\Fs(x \oplus \alpha, y \oplus \beta) \equiv \Fs(x, y) \oplus \gamma\big] = p$. Hence \autoref{diff1} becomes:
\begin{equation}
\forall \delta_1 \in D_1~\exists~ \Delta_1 \in \{0,1\}^v \text{ s.t. } (\delta_1, 0) \overset{\Fuu}{\underset{p_1}{\rightsquigarrow}} \Delta_1,
\end{equation}
for some probability $p_1$ and relation $\equiv$, and the same changes apply to \autoref{diff2}.

Again, these generalisations correspond to advanced techniques of Aoki and Sasaki's framework, which find here a concise and efficient description.

\medskip

The only change to \autoref{affine_testing} needed to accommodate these extensions is to replace the equality by the appropriate equivalence
relation on line 6. However, the fact that this equivalence holds no longer ensures that $x \defas \mu \oplus \delta_1 \oplus \delta_2$ is a preimage,
which implies an increased complexity for the attack.
This increase comes from to factors:  first, even when $x$ \emph{is} a preimage, the relation on line 6 might not
hold with probability $1 - p_1p_2$, meaning that on average one needs to test $1/p_1p_2$ times more candidates in order to account for
the false negatives; secondly, if we denote by $s$ the average size of the equivalence classes under $\equiv$ (when using truncation as
above, this is equal to $2^{v - r}$ with $r$ the Hamming weight of $m$), then one needs to check $s$ potential preimages
as returned on line 6 on average before finding a valid one, in order to account for the false positives.
The total complexity of an attack with the modified algorithm is therefore $\bigo((2^{v-d} + s)/\tilde{p_1}\tilde{p_2})$, where $\tilde{p_1}$ and $\tilde{p_2}$ are the respective
average probabilities for $p_1$ and $p_2$ over the spaces $D_1$ and $D_2$.

\subsection{Splice-and-cut, initial structures and bicliques}

The two techniques we present now are older than the framework of~\cite{DBLP:conf/crypto/KnellwolfK12}, but are fully compatible with its differential approach.

Splice-and-cut was introduced by Aoki and Sasaki in 2008~\cite{AS08}. Its idea is to use the feedforward of the compression
function so as to
be able to start the computation of $\Fuu$ and $\Fd^{-1}$ not from $p$ and $c$ but from an intermediate value from the middle of the computation
of $\Fs$. If one sets $\Fs = \Ft \circ \Fd \circ \Fuu$ and calls $s$ the intermediate value $\Ft^{-1}(c)$ (or equivalently $\Fd \circ \Fuu(p)$),
an attacker may now sample the functions $\Fuu(t - \Ft(s))$ and $\Fd^{-1}(s)$ on their respective key-spaces, which are as always taken to
be independent, when searching
a preimage for $t$.
By giving more possible choices for the decomposition of $\Fs$, one can hope for better attacks. This however comes at the cost that they are now
pseudo-preimage attacks, as one does not control the value of the IV anymore, which becomes equal to $t - \Ft(s)$.

A possible improvement to a splice-and-cut decomposition is the use of \emph{initial structures}~\cite{SA09}, which were later reformulated as \emph{bicliques}~\cite{KRS12}.
Instead of starting the computations in the middle from an intermediate value $s$, the idea is now to start from a set of multiple values
possessing a special structure that spans several rounds. If the cost of constructing such sets is negligible w.r.t the rest of the computations,
the rounds spanned by the structure actually come for free. In more details, a biclique, say for $\Ft$ in the above
decomposition of $\Fs$, is a set $\{m,D_1,D_2,Q_1,Q_2\}$ where $m$ is a message,
$D_1$ and $D_2$ are linear spaces of dimension $d$, and $Q_1$ (resp. $Q_2$) is a list of $2^d$ values indexed by the differences $\delta_1$
of $D_1$ (resp. $D_2$) s.t. $\forall (\delta_1, \delta_2) \in D_1 \times D_2\quad Q_2[\delta_2] = \Ft(m \oplus \delta_1 \oplus \delta_2, Q_1[\delta_1])$.
This allows to search the message space $m \oplus D_1 \oplus D_2$ in $\bigo(2^d)$
with a meet-in-the-middle approach that does not need any call to $\Ft$,
essentially bypassing this part of the decomposition.


\section{Higher-Order Differential Meet-in-The-Middle}
\label{NewFramework}

We now describe how to modify the framework of \autoref{KKFramework} to use higher-order differentials.
Let us denote by 
$(\{\alpha_1,\alpha_2\}, \{\beta_1,\beta_2\}) \overset{\Fs}{\underset{p}{\longrightarrow}} \gamma$
the fact that
  $\underset{(x,y)}{\Pr}\big[\Fs(x \oplus \alpha_1 \oplus \alpha_2, y \oplus \beta_1 \oplus \beta_2) \oplus
  \Fs(x \oplus \alpha_1, y \oplus \beta_1) \oplus  \Fs(x \oplus \alpha_2 , y \oplus \beta_2) =
  \Fs(x,y )\oplus \gamma \big] = p$
, meaning
that $(\{\alpha_1,\alpha_2\}, \{\beta_1,\beta_2\})$ is a related-key order-2 differential for $\Fs$ that holds with probability $p$.

Similarly as in \autoref{KKFramework}, the goal of the attacker is to find four linear subspaces
$D_{1,1}, D_{1,2}, D_{2,1}, D_{2,2}$ of $\{0,1\}^m$ in direct sum such that:
%\begin{equation}
%\label{eq:Dir_sum}
%  D_{1,1} \bigoplus D_{1,2} \bigoplus D_{2,1} \bigoplus D_{2,2} 
%\end{equation}
%\vspace{-4mm}
\begin{equation}
\forall \delta_{1,1} , \delta_{1,2} \in D_{1,1} \times D_{1,2}~\exists~ \Delta_1 \in \{0,1\}^v \text{ s.t. }
(\{\delta_{1,1}, \delta_{1,2}\}, \{0,0\})
\overset{\Fuu}{\underset{1}{\longrightarrow}} \Delta_1
\label{Differential_def_1}
\end{equation}
\vspace{-4mm}
\begin{equation}
\forall \delta_{2,1} , \delta_{2,2} \in D_{2,1} \times D_{2,2}~\exists~ \Delta_2 \in \{0,1\}^v \text{ s.t. }
(\{\delta_{2,1}, \delta_{2,2}\}, \{0,0\})
\overset{\Fd^{-1}}{\underset{1}{\longrightarrow}} \Delta_2.
\label{Differential_def_2}
\end{equation}
Then $M \oplus \delta_{1,1} \oplus \delta_{1,2} \oplus \delta_{2,1} \oplus \delta_{2,2}$ 
is a \emph{preimage} of $c+p$ if and only if $
    \Fuu(\mu \oplus \delta_{1,1} \oplus \delta_{1,2} \oplus 
    \delta_{2,1} \oplus \delta_{2,2}, c) = 
    \Fd^{-1}(\mu \oplus \delta_{1,1} \oplus \delta_{1,2} \oplus 
    \delta_{2,1} \oplus \delta_{2,2}, p) $
which is equivalent by the \autoref{Differential_def_1} 
and \autoref{Differential_def_2} to the equality: 
\begin{equation}
  \label{eq:Hmitm_eq}
  \begin{split}
  \Fuu(\mu \oplus \delta_{1,1} \oplus \delta_{2,1} \oplus \delta_{2,2}, p) & 
    \oplus \quad \qquad \quad 
    \Fd^{-1}(\mu \oplus \delta_{2,1},\oplus \delta_{1,1} \oplus \delta_{1,2}, c) \oplus \\
    \Fuu(\mu \oplus \delta_{1,2} \oplus \delta_{2,1} \oplus \delta_{2,2}, p) & 
    \oplus\quad  = \quad \quad   
    \Fd^{-1}(\mu \oplus \delta_{2,2}, \oplus \delta_{1,1} \oplus \delta_{1,2}, c) \oplus \\ 
    \Fuu(\mu \oplus \delta_{2,1} \oplus \delta_{2,2}, p) \oplus \Delta_1 & 
    \qquad \quad \quad \quad   
    \Fd^{-1}(\mu \oplus \delta_{1,1}  \oplus \delta_{1,2}, c) \oplus \Delta_2.
    \end{split}
\end{equation}

We denote by $d_{i,j}$ the dimension of the 
sub-space $D_{i,j}$ for $i,j=1,2$. Then for a set $M$ of messages $\mu \in \{0,1\}^m)$ one can define $\# M$ affine sub-sets 
$\mu_i \oplus D_{1,1} \oplus D_{1,2} \oplus D_{2,1} \oplus D_{2,2}$ of dimension $d_{1,1} + d_{1,2} + d_{2,1} +d_{2,2}$
(since the sub-spaces $D_{i,j}$ are in direct sum by hypothesis), which can be tested for a preimage using \autoref{eq:Hmitm_eq}.
This can be done efficiently by a modification of \autoref{affine_testing} into the following \autoref{affine_testing_II}.

\begin{algorithm}[ht]
\LinesNumbered 
\KwIn{
	$D_{1,1}, D_{1,2},D_{2,1},D_{2,2} \subset \{0,1\}^m$, $\mu \in \{0,1\}^m$, $p$, $c$
}
\KwOut{
  A preimage of $c + p$ if there is one in $\mu \oplus D_{1,1} \oplus D_{1,2}, \oplus D_{2,1} \oplus D_{2,2} $, $\bot$ otherwise
}
\KwData{ Six lists:\\
$L_{1,1}$ indexed by $\delta_{1,2} , \delta_{2,1} , \delta_{2,2}$\\
$L_{1,2}$ indexed by $\delta_{1,1},  \delta_{2,1},  \delta_{2,2}$\\
$L_{2,1}$ indexed by $\delta_{1,1},  \delta_{1,2},  \delta_{2,2}$\\
$L_{2,2}$ indexed by $\delta_{1,1} , \delta_{1,2},  \delta_{2,1}$\\
$L_{1}$ indexed by $\delta_{2,2},  \delta_{2,1}$\\
$L_{2}$ indexed by $\delta_{1,1} , \delta_{1,2}$
}
        \ForAll{
          $ \delta_{1,2}, \delta_{2,1}, \delta_{2,2} \in
            D_{1,2} \times D_{2,1} \times D_{2,2} $
        }
        {
          $ L_{1,1}[\delta_{1,2}, \delta_{2,1}, \delta_{2,2} ] \leftarrow 
          \Fd^{-1}(\mu \oplus \delta_{1,2} \oplus \delta_{2,1} \oplus \delta_{2,2}, c)$ \;
        }

        \ForAll{
          $ \delta_{1,1}, \delta_{2,1}, \delta_{2,2}\in 
          D_{1,1} \times D_{2,1} \times D_{2,2}$
        }
        {
          $ L_{1,2}[\delta_{1,1}, \delta_{2,1}, \delta_{2,2}] \leftarrow 
          \Fd^{-1}(\mu \oplus \delta_{1,1} \oplus \delta_{2,1} \oplus \delta_{2,2}, c)$ \;
        }

        \ForAll{
          $ \delta_{1,1}, \delta_{1,2}, \delta_{2,2}\in 
          D_{1,1} \times D_{1,2} \times D_{2,2} \times$
        }
        {
          $ L_{2,1}[\delta_{1,1}, \delta_{1,2}, \delta_{2,2}] \leftarrow 
          \Fuu(\mu \oplus \delta_{1,1} \oplus \delta_{1,2} \oplus \delta_{2,2},p)$ \;
        }

        \ForAll{
          $ \delta_{1,1}, \delta_{1,2}, \delta_{2,1} \in 
          D_{1,1} \times D_{1,2} \times D_{2,1}$
        }
        {
          $ L_{2,2}[\delta_{1,1}, \delta_{1,2}, \delta_{2,1}] \leftarrow
         \Fuu(\mu \oplus \delta_{1,1} \oplus \delta_{1,2} \oplus \delta_{2,1}, p)$ \;
        }

        \ForAll{
          $ \delta_{1,1}, \delta_{1,2} \in D_{1,1} \times D_{1,2}$
        }
        {
          $ L_{2}[\delta_{1,1}, \delta_{1,2}] \leftarrow
          \Fd^{-1}(\mu \oplus \delta_{1,1} \oplus \delta_{1,2}, c) \oplus \Delta_1$ \;
        }

        \ForAll{
          $ \delta_{2,1}, \delta_{2,2}\in D_{2,1} \times D_{2,2} $
        }
        {
          $ L_{1}[\delta_{2,1}, \delta_{2,2}] \leftarrow
          \Fuu(\mu \oplus \delta_{2,1} \oplus \delta_{2,2}, 
          p  ) \oplus \Delta_2$ \;
        }

        \ForAll{
          $ \delta_{1,1},\delta_{1,2}, \delta_{2,1}, \delta_{2,2} \in 
            D_{1,1} \times D_{1,2} \times D_{2,1} \times D_{2,2} $
        }
        { 
          \If {$L_{1,1}[\delta_{1,2}, \delta_{2,1}, \delta_{2,2} ] \oplus 
            L_{1,2}[\delta_{1,1}, \delta_{2,1}, \delta_{2,2} ] \oplus 
            L_1[\delta_{2,1}, \delta_{2,2}] = 
            L_{2,1}[\delta_{1,1}, \delta_{1,2}, \delta_{2,2} ] \oplus 
            L_{2,2}[\delta_{1,1}, \delta_{1,2}, \delta_{2,1} ] \oplus 
          L_2[\delta_{1,1}, \delta_{1,2}] $ 
        }
          {
            \Return $\mu  \oplus \delta_{1,1} \oplus \delta_{1,2} 
                        \oplus \delta_{2,1} \oplus \delta_{2,2}$ 
          }
        }
        \Return {$\bot$}
        \caption{\label{affine_testing_II} Testing $\mu \oplus D_{1,1} \oplus D_{1,2} \oplus D_{2,1} \oplus D_{2,2}$ for a preimage}

      \end{algorithm}

    \subsection{Analysis of \autoref{affine_testing_II}}
      If we denote by $\Gamma_1$ and $\Gamma_2$ the cost of evaluating
      of $\Fuu$ and $\Fd^{-1}$ and $\Gamma_{match}$ the cost of the test on line~14,
      then the algorithm allows to test
      $2^{d_{1,1}+d_{1,2}+d_{2,1}+d_{2,2}}$ messages with a complexity of 
          $2^{d_{1,2}+d_{2,1}+d_{2,2}} \Gamma_2 + 2^{d_{1,1}+d_{2,1}+d_{2,2}} \Gamma_2 +
          2^{d_{1,1}+d_{1,2}+d_{2,1}} \Gamma_1 +2^{d_{1,1}+d_{1,2}+d_{2,1}} \Gamma_1 +
          2^{d_{1,1}+d_{1,2}} \Gamma_2 + 2^{d_{2,1}+d_{2,2}} \Gamma_1 + \Gamma_{\text{match}}$. 
      The algorithm must then be run $2^{n - (d_{1,1}+d_{1,2}+d_{2,1}+d_{2,2})}$ times in order to test
      $2^n$ messages. 
      In the special case where all the linear spaces have the same 
      dimension $d$ and if we consider that $\Gamma_{\text{match}}$ is negligible
      with respect to the total complexity, the total complexity of an attack is
      then of : 
      $2^{n-4d}\cdot(2^{3d}\cdot(2\Gamma_1+2\Gamma_2)+2^{2d}\cdot(\Gamma_1+\Gamma_2)) = 
      2^{n-d+1} \Gamma + 2^{n-2d} \Gamma = \bigo(2^{n-d})$ where $\Gamma$ is the cost of the 
      evaluation of the total compression function $\Fs$. We think that the assumption on
      the cost of $\Gamma_\text{match}$ is
      reasonable given the small size of $d$ in actual attacks and the fact that performing
      a single match is much faster than computing $\Fs$.

      The factor that is gained from a brute-force search of complexity $\bigo(2^n)$ is hence of
      $2^d$, which is the same as for \autoref{affine_testing}. However, one now needs
      four spaces of differences of size $2^d$ instead of only two,
      which might look like a setback. Indeed the real
      interest of this method does not lie in a simpler attack but in the fact that
      using higher-order differentials may now allow to attack functions for which no
      good-enough order-1 differentials are available.


    \subsection{Using probabilistic truncated differentials}
    Similarly as in  \autoref{KKFramework}, \autoref{affine_testing_II} can be modified in order
    to use probabilistic truncated differentials instead of probability-1 differentials
    on the full state. The changes to the algorithm and the complexity evaluation are
    identical to the ones described in \autoref{probtrunc}.


\section{Applications to \shaone}
\label{sec:sha1}

%\subsection{Description of SHA-1}
%
%SHA-1 is an NSA-designed hash function standardized by the NIST~\cite{sha1}. It combines a compression
%function which is a block cipher with 512-bit keys and 160-bit messages used in Davies-Meyer mode
%with a Merkle-Damg\aa rd mode of operation~\cite[Chap. 9]{HAC96}. Thus, the initial vector (IV)
%as well as the final hash
%are 160-bit values, and messages are processed in 512-bit blocks.
%The underlying block cipher of the compression function can be described as follows:
%let us denote by $m_0, \ldots m_{15}$ the 512-bit key as 16 32-bit words. The \emph{expanded} key
%$w_0, \ldots w_{79}$ is defined as:
%\[
%  w_i = \left\{\begin{array}{ll}
%      m_i & \text{if } i < 16\\
%      (w_{i - 3} \oplus w_{i - 8} \oplus w_{i - 14} \oplus w_{i - 16}) \lll 1 & \text{otherwise}.
%    \end{array}
%    \right.
%  \]
%  Then, if we denote by $a, b, c, d, e$ a 160-bit state made of 5 32-bit words and initialized
%  with the plaintext, the ciphertext is the value held in this state
%  after iterating the following procedure (parametered by the round number $i$) 80 times:
%  \[
%    \begin{array}{l}
%      t \leftarrow (a \lll 5) + \bool_{i \div 20}(b, c, d) + e + k_{i \div 20} + w_i  \\
%      e \leftarrow d\\
%      d \leftarrow c\\
%      c \leftarrow b \lll 30\\
%      b \leftarrow a\\
%      a \leftarrow t,\\
%    \end{array}\quad
%  \]
%  where `$\div$' denotes the integer division, $\Phi_{0\ldots3}$ are four bitwise Boolean functions,
%  and $k_{0\ldots3}$ are four constants (we refer to~\cite{sha1} for their definition).
%
%  Importantly, before being hashed, a message is \emph{always} padded with
%  at least 65 bits, made of a `1' bit, a (possibly zero) number of `0' bits,
%  and the length of the message in bits as a 64-bit integer. This padding places
%  an additional constraint on the attacker as it means that even a preimage for the compression function with
%  a valid IV is not necessarily a preimage for the hash function.


\subsection{One-block preimages without padding}
\label{sec:one_wo_pad}
  We apply the framework of \autoref{NewFramework} to mount attacks on \shaone
  for \emph{one-block preimages without padding}. These are rather direct applications
  of the framework, the only difference being the fact that we use \emph{sets} of
  differentials instead of \emph{linear spaces}. This has no impact
  on \autoref{affine_testing_II}, but makes the description of the attack parameters
  less compact.

  As was noted in~\cite{DBLP:conf/crypto/KnellwolfK12}, the message expansion of \shaone being linear, it is possible to attack 15 steps both in the forward and backward 
  direction, for a total of 30, without advanced matching techniques: it is sufficient to use a message difference
  in the kernel of the 15 first steps of
  the message expansion. 
  When applying our framework to attack more steps, say 55 to 62, we have observed experimentally
  that splitting the forward and backward parts around steps 22 to 27 seems to give the best results.
  A similar behaviour was observed by Knellwolf and Khovratovich in their attacks, and this can be explained by the fact
  that the \shaone step function has a somewhat weaker diffusion when computed backward compared to forward.

  We use \autoref{difference-getter} to
  construct a suitable set of differences in the preparation of an attack.
  This algorithm was run on input differences of low Hamming weight; these
  are kept only when they result in output differences
  with truncation masks that are long enough and with good overall
  probabilities. The sampling parameter $Q$ that we used was $2^{15}$;
  the threshold value $t$ was subjected to a tradeoff: the larger it is,
  the less bits are chosen in the truncation mask, but the better the probability of
  the resulting differential. In practice, we used values between 2 and 5, 
  depending on the differential considered. 

  \begin{algorithm}[ht]
        \LinesNumbered
    \KwIn{
      A chunk $\Fs_i$ of the compression function, $\delta_{i,1}, \delta_{i,2} \in \{0,1\}^m$, a threshold value $t$,
      a sample size $Q$, an internal state $c$.
    }
    \KwOut{
      An output difference $\mathcal{S}$ , and a mask $T_\mathcal{S}$ for the differential 
      $((\delta_{i,1}, \delta_{i,2}), 0) \overset{\Fs_i}{{\rightsquigarrow}}~\mathcal{S}$
    }
    \KwData{ 
      An array $d$ of $n$ counters initially set to $0$.
    }
    \For{$ q = 0$ to $Q$}
    {
      Choose $\mu \in \{0,1\}^m$ at random\;
      $\Delta \leftarrow \Fs_i (\mu \oplus \delta_{i,1} \oplus \delta_{i,2} , c) \oplus \Fs_i (\mu \oplus \delta_{i,1} , c) \oplus
      \Fs_i(\mu \oplus \delta_{i,2},c) \oplus \Fs_i (\mu,c)$\;
      \For{$i = 0$ to $ n-1$ }
      {
        \If{ the i$^\text{th}$ bit of $\Delta$ is 1}
        {
          $d[i] \leftarrow d[i] + 1$\;
        }
      }
    }
    \For{$i = 0$ to $ n-1$ }
    {
      \If{ $d[i] \geq  t $}
      {
        Set the $i$-th bit of the output difference $\mathcal{S}$ to $1$\;
      }
    }
    \caption{\label{difference-getter}Computing a suitable output difference for a given input difference}
  \end{algorithm}

  Once input and output differences have been chosen, we use an adapted version of \cite[Algorithm~2]{DBLP:conf/crypto/KnellwolfK12} given in
  \autoref{mask-getter} to compute suitable truncation masks. 

  \begin{algorithm}[ht]
        \LinesNumbered
    \KwIn{
      $D_{1,1}, D_{1,2}, D_{2,1}, D_{2,2} \subset \{0,1\}^m$, a sample size $Q$, a mask size $d$.
    }
    \KwOut{
      A truncation mask $T \in \{0,1\}^n$ of Hamming weight $d$.
    }
    \KwData{ 
      An array $k$ of $n$ counters initially set to $0$.
    }

    \For{$ q = 0$ to $Q$}
    {
      Choose $\mu \in \{0,1\}^m$ at random \;
      $c \leftarrow \Fs(\mu, \iv)$\;
      Choose $(\delta_{1,1},\delta_{1,2},\delta_{2,1},\delta_{2,2}) \in D_{1,1} \times D_{1,2} \times D_{2,1} \times D_{2,2} $ at random\; 
      $\Delta \leftarrow 
      \Fuu (\mu \oplus \delta_{1,1} \oplus \delta_{1,2} , c) \oplus 
      \Fuu (\mu \oplus \delta_{1,1} , c) \oplus 
      \Fuu(\mu \oplus \delta_{1,2},c)$\;
      $\Delta \leftarrow \Delta \oplus
      \Fd^{-1} (\mu \oplus \delta_{2,1} \oplus \delta_{2,2}, c) \oplus 
      \Fd^{-1}(\mu \oplus \delta_{2,2} , c) \oplus 
      \Fd^{-1} (\mu \oplus \delta_{2,2} , c) $\;
      \For{$i = 0$ to $ n-1$ }
      {
        \If{ the i$^\text{th}$ bit of $\Delta$ is 1}
        {
          $k[i] \leftarrow k[i] + 1$\;
        }
      }
    }
    Set the d bits of lowest counter value in $k$ to 1 in T.
    \caption{\label{mask-getter}Find truncation mask T for matching}
  \end{algorithm}

  The choice of the size of the truncation mask $d$ in this algorithm
  offers a tradeoff between the probability one can hope to achieve for the resulting truncated differential
  and how efficient a filtering of ``\,bad\,'' messages it will offer.
  In our applications to \shaone, we chose masks of size about
  $\min(\log_2(|D_{1,1}|), \log_2(|D_{1,2}|)$, $\log_2(|D_{2,1}|), \log_2(|D_{2,2}|))$,
  which is consistent with taking masks of size the dimension of the affine
  spaces as is done in~\cite{DBLP:conf/crypto/KnellwolfK12}.

  \medskip

  We similarly adapt \cite[Algorithm 3]{DBLP:conf/crypto/KnellwolfK12} as \autoref{typeI-getter} in order to estimate the average
  false negative probability associated with the final truncated differential.

  \begin{algorithm}[ht]
        \LinesNumbered
    \KwIn{
      $D_{1,1}, D_{1,2}, D_{2,1}, D_{2,2} \subset \{0,1\}^m, T \in \{0,1\}^n$, a sample size $Q$
    }
    \KwOut{
      Average false negative probability $\alpha$.
    }
    \KwData{ 
      A counter $k$ initially set to $0$.
    }

    \For{$q = 0$ to $Q$}
    {
      Choose $\mu \in \{0,1\}^m$ at random \;
      $c \leftarrow \Fs(\mu, \iv)$\;
      Choose $(\delta_{1,1},\delta_{1,2},\delta_{2,1},\delta_{2,2}) \in D_{1,1} \times D_{1,2} \times D_{2,1} \times D_{2,2} $ at random\; 
      $\Delta \leftarrow 
      \Fuu (\mu \oplus \delta_{1,1} \oplus \delta_{1,2} , c) \oplus 
      \Fuu (\mu \oplus \delta_{1,1} , c) \oplus 
      \Fuu(\mu \oplus \delta_{1,2},c)$\;
      $\Delta \leftarrow \Delta \oplus
      \Fd^{-1} (\mu \oplus \delta_{2,1} \oplus \delta_{2,2}, c) \oplus 
      \Fd^{-1}(\mu \oplus \delta_{2,2} , c) \oplus 
      \Fd^{-1} (\mu \oplus \delta_{2,2} , c) $\;
      \For{$i = 0$ to $ n-1$ }
      {
        \If{$\Delta \not\equiv_T 0^n$}
        {
          $k \leftarrow k+1$\;
        }
      }
    }
    \Return {$k/Q$}
    \caption{\label{typeI-getter}Estimate the average false negative probability}
  \end{algorithm}

  We conclude this section by giving the statistics for the best attacks that we found for
  various reduced versions of \shaone in \autoref{SHA11}, the highest number of
  attacked rounds being 62. Because the difference spaces
  are no longer affine, they do not lend themselves to a compact description and their size
  is not necessarily a power of 2 anymore. The ones we use do not have many elements, however,
  which makes them easy to enumerate.

  \begin{table}[t]
    \caption[One block preimage without padding.]{One block preimage without padding. $N$ is the 
              number of attacked steps, \emph{Split} is the separation step between the
              forward and the backward chunk, $d_{i,j}$ is the $\log_2$ of the cardinal
              of $D_{i,j}$ and $\alpha$ is the estimate for the false negative probability. The complexity
              is computed as described in  \autoref{NewFramework}.\label{SHA11}}
    \begin{center}
      \begin{tabular}{l c c c c c c r @{}} \toprule
        $N\qquad$ &  \emph{Split} & $d_{1,1}$ &  $d_{1,2}$ & $d_{2,1}$ & $d_{2,2}$ & $\alpha $ & Complexity \\\midrule
        58    & 25  & 7.6  & 9.0 & 9.2 & 9.0 & 0.73  & 157.4\\ 
        59    & 25  & 7.6  & 9.0 & 6.7 & 6.7  & 0.69  & 157.7\\ 
        60    & 26  & 6.5 & 6.0 & 6.7 & 6.0  & 0.60  & 158.0\\ 
        61    & 27  & 4.7 & 4.8 & 5.7  & 5.8  & 0.51  & 158.5\\ 
        62    & 27  & 4.7 & 4.8 & 4.3  & 4.6  & 0.63 & 159.0 \\ 
        \bottomrule
        \hline
      \end{tabular}
    \end{center}
  \end{table}

  \subsection{One-block preimages with padding}
\label{sec:one_wi_pad}


  If we want the message to be properly padded, 65 out of the 512 bits of the last message
  blocks need to be fixed according to the padding rules, and this naturally restricts the positions
  of where one can now use message differences. This has in particular an adverse effect on
  the differences in the backward step, which Hamming weight increases because of some
  features of \shaone's message expansion algorithm. The overall process of finding attack
  parameters is otherwise unchanged from the non-padded case. We give statistics for
  the best attacks that we found in \autoref{SHA12}. One will note that the highest number of attacked
  rounds dropped from 62 to 56 when compared to \autoref{SHA11}.
 
  \begin{table}[t]
    \caption[One block preimage with padding.]{One block preimage with padding. $N$ is the 
              number of attacked steps, \emph{Split} is the separation step between the
              forward and the backward chunk, $d_{i,j}$ is the $\log_2$ of the cardinal
              of $D_{i,j}$ and $\alpha$ is the estimation for false negative probability. The complexity
              is computed as described in \autoref{NewFramework}.\label{SHA12}}
    \begin{center}
      \begin{tabular}{l c c c c c c r @{}} \toprule
        $N\qquad$ &  \emph{Split} & $d_{1,1}$ &  $d_{1,2}$ & $d_{2,1}$ & $d_{2,2}$ & $\alpha $ & Complexity \\\midrule
      51    & 23  & 8.7  & 8.7 & 8.7 & 8.7  & 0.72  & 155.6\\ 
      52    & 23  & 9.1  & 9.1 & 8.2 & 8.2  & 0.61  & 156.7\\ 
      53    & 23  & 9.1  & 9.1 & 3.5 & 3.5  & 0.61  & 157.7\\ 
      55    & 25  & 6.5  & 6.5 & 5.9 & 5.7  & 0.52  & 158.2\\ 
      56    & 25  & 6  & 6.2 & 7.2 & 7.2  & 0.6  & 159.4\\ 
        \bottomrule
        \hline
      \end{tabular}
    \end{center}
  \end{table}



  \subsection{Two-block preimages with padding}
\label{sec:one_two}


  We can increase the number of rounds for which we can find a preimage with a properly padded
  message at the cost of using a slightly longer message of two blocks: if we are able to find
  \emph{one-block pseudo preimages with padding} on enough rounds, we can then use the
  \emph{one-block preimage without padding} to bridge the former to the IV. Indeed,
  in a pseudo-preimage
  setting, the additional freedom degrees gained from removing any constraint on the IV
  more than compensate for the ones added by the padding. We first describe
  how to compute such pseudo-preimages.

  \subsubsection{One-block pseudo-preimages.}

  If we relax the conditions on the IV and do not impose anymore that it is fixed
  to the one of the specifications, it becomes possible to use a splice-and-cut
  decomposition of the function, as well as short properly padded bicliques.

  The reduced compression function of \shaone $\Fs$ is now decomposed into three smaller
  functions as $\Fs = \Fd^t \circ \Fuu \circ \Ft \circ \Fd^b$, $\Ft$ being the rounds covered by the
  biclique. The function $\Fuu$ covers the steps $s_1$ to $e$, $\Fd = \Fd^t \circ \Fd^b$ covers
  $s_2$ to $e + 1$ through the feedforward, and $\Ft$ covers $s_2 + 1$ to $s_1 - 1$, as shown in
  \autoref{biclique}.

    \begin{figure}
	\center
      \includegraphics[scale=0.6]{../III_SHA-1/preimages/SHA.pdf}
      \caption{A splice-and-cut decomposition with biclique.\label{biclique}}
    \end{figure}

    Finding the parameters is done in the exact same way as for the one-block preimage attacks.
    Since the bicliques only cover $7$ steps, one can generate many of them
    from a single one by modifying some of the remaining message words outside of the biclique proper.
    Therefore, the amortized cost of their construction is small and considered negligible w.r.t. the
    rest of the attack. The resulting attacks are shown in \autoref{SHA13}.

  \begin{table}[ht]
    \caption[One block pseudo-preimage with padding.]{One block pseudo-preimage with padding. $N$ is the 
              number of attacked steps, $d_{i,j}$ is the $\log_2$ of the cardinal
              of the set $D_{1,2}$ and $\alpha$ is the estimation for false negative probability.
              The various splits are labeled as in \autoref{biclique}.
            The complexity is computed as described in \autoref{NewFramework}.\label{SHA13}}
    \begin{center}
      \begin{tabular}{l c c c c c  c c c r @{}} \toprule
        $N\qquad$ & $s_1$ & $e$ & $s_2$ & $d_{1,1}$ &  $d_{1,2}$ & $d_{2,1}$ & $d_{2,2}$ & $\alpha $ & Complexity \\\midrule
        61    & 27 & 49 & 20   & 7.0  & 7.0 & 7.5 & 7.5  & 0.56  & 156.4\\ 
        62    & 27 & 50 & 20   & 5.8  & 5.7 & 7.2 & 7.2  & 0.57  & 157.0\\ 
        63    & 27 & 50 & 20   & 6.7  & 6.7 & 7.7 & 7.7  & 0.57  & 157.6\\ 
        64    & 27 & 50 & 20   & 3 & 3 & 4.5 & 4.7  & 0.69  & 158.7\\ 
        \bottomrule
        \hline
      \end{tabular}
    \end{center}
  \end{table}

  \subsubsection{Complexity of the two-block attacks.}
  Using both one-block attacks, it is simple to mount a two-block attack at the combined cost of each of them.
  For a given target $c$, the process is the followinf:
    \begin{enumerate}
      \item The attacker uses a properly-padded pseudo-preimage attack, yielding the second message block $\mu_2$ and an IV $\iv'$;
      \item The attacker then uses a non-padded preimage attack with target $\iv'$, yielding a first message block $\mu_2$.
    \end{enumerate}
From the Merkle-Damg\aa rd structure of \shaone, it follows that the two-block message $(\mu_1, \mu_2)$ 
    is a preimage of $c$.

    For attacks up to 60 rounds, we can use the pseudo-preimage attacks of~\cite{DBLP:conf/crypto/KnellwolfK12}; for 61 and 62 rounds,
    we use the ones of this section. This results in attacks of complexities as shown in \autoref{SHA14}.

  \begin{table}[t]
    \caption[Two-block preimage.]{Two-block preimage attacks on \shaone reduced to $N$ steps. The pseudo-preimage attacks followed by
    `$\star$' come from \cite{DBLP:conf/crypto/KnellwolfK12}.\label{SHA14}}
    \begin{center}
      \begin{tabular}{l c c c } \toprule
        $N\qquad$ & Second block complexity\phantom{bla} & First block complexity\phantom{bla}  & Total Complexity \\\midrule
        58    & $156.3^\star$ & 157.4& 157.9\\ 
        59    & $156.7^\star$ & 157.7& 158.3\\ 
        60    & $157.5^\star$ & 158.0& 158.7\\ 
        61    & 156.4    & 158.5& 158.8\\ 
        62    & 157.0    & 159.0& 159.3\\ 
        \bottomrule
        \hline
      \end{tabular}
    \end{center}
  \end{table}

\section{Conclusion}
This chapter showed how to extend the \mitm framework for hash function preimage attacks with higher-order differentials,
and how this could be applied to find better attacks on \shaone.
The source of the improvements come from the fact that higher-order differentials lead to better message partitions, which
then mechanically allow to mount attacks up to a larger number of steps than the previous best
results. However, this comes at the cost of a more complex attack procedure.



\FloatBarrier
