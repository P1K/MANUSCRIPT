\makeatletter

\newcommand{\dedication}[1]{%
{\cleardoublepage\thispagestyle{empty}\mbox{}\vspace{\stretch{1}}\flushright {#1} \par\vspace{\stretch{2}}\clearpage}}

\newsavebox{\mybox}
\newcolumntype{h}{>{\begin{lrbox}{\mybox}\sffamily}{r}<{\end{lrbox}\eqparbox{md4tabh}{\hfill\sffamily\usebox{\mybox}}}}

\makechapterstyle{test}{
   \renewcommand\chapnamefont{\normalfont\Large\scshape\raggedleft}
   \renewcommand\chaptitlefont{\normalfont\Huge\bfseries\boldmath\sffamily}
   \renewcommand\chapternamenum{}
   \renewcommand\printchapternum{%
     \makebox[0pt][l]{%
       \hspace{0.4em}
       \resizebox{!}{4ex}{\chapnamefont\bfseries\sffamily\thechapter}
     }
   }
   \renewcommand\afterchapternum{\par\hspace{1.5cm}\hrule\vskip\midchapskip}
   \renewcommand{\printchaptertitle}[1]{%
     \chaptitlefont \textbf{##1}} 
%   \renewcommand{\printchapternonum}{\raggedleft}
}

\makeheadstyles{test}{%
  \chapterstyle{test}%
  \setbeforesecskip{-3.5ex \@plus -1ex \@minus -.2ex}%
  \setaftersecskip{2.3ex \@plus .2ex}%
  \setsecheadstyle{\normalfont\Large\sffamily\bfseries\boldmath\raggedright}%
  \setbeforesubsecskip{-3.25ex \@plus -1ex \@minus -.2ex}%
  \setaftersubsecskip{1.5ex \@plus .2ex}%
  \setsubsecheadstyle{\normalfont\large\sffamily\bfseries\boldmath\raggedright}%
%   \setbeforesubsubsecskip{-3.25ex \@plus -1ex \@minus -.2ex}%
%   \setaftersubsubsecskip{1.5ex \@plus .2ex}%
%   \setsubsubsecheadstyle{\normalfont\normalsize\sffamily\bfseries\raggedright}%
%   \setbeforeparaskip{3.25ex \@plus 1ex \@minus .2ex}%
%   \setafterparaskip{-1em}%
%   \setparaheadstyle{\normalfont\normalsize\sffamily\bfseries}%
  \setbeforesubsubsecskip{-3ex \@plus -1ex \@minus -.2ex}%
  \setaftersubsubsecskip{1ex \@plus .2ex}%
%  \setbeforesubsubsecskip{3.25ex \@plus 1ex \@minus .2ex}%
%  \setaftersubsubsecskip{-1em}%
  \setsubsubsecheadstyle{\normalfont\normalsize\sffamily\bfseries\boldmath}%
  %\setparaindent{\parindent}%
  \setbeforeparaskip{3.25ex \@plus 1ex \@minus .2ex}%
  \setafterparaskip{-1em}%
  \setparaheadstyle{\normalfont\normalsize\sffamily\itshape}
%  \setparaheadstyle{\normalfont\normalsize\sffamily\scshape}
  \setsubparaindent{\parindent}%
  \setbeforesubparaskip{3.25ex \@plus 1ex \@minus .2ex}%
  \setaftersubparaskip{-1em}%
  \setsubparaheadstyle{\normalfont\normalsize\itshape\addperiod}%
%
  \let\beforeparttoc\empty%
  \let\beforepartlof\empty%
  \let\beforepartlot\empty%
%
  \let\afterparttoc\empty%
  \let\afterpartlof\empty%
  \let\afterpartlot\empty%
%  
  \noptcrule%
  \mtcsetdepth{parttoc}{2}%
%
  \renewcommand{\printpartname}{\hrulefill ~ \partnamefont \partname}%
  \renewcommand{\partnamenum}{\space}%
  \renewcommand{\printpartnum}{\partnumfont \numtoName{\c@part} \hrulefill}%
  \renewcommand{\parttitlefont}{\Huge\bfseries\boldmath\sffamily}%
  \renewcommand{\partnamefont}{\LARGE\scshape\sffamily}%
  \renewcommand{\partnumfont}{\LARGE\scshape\sffamily}%
  \renewcommand*{\beforepartskip}{\null}%
  \renewcommand*{\afterpartskip}{\vskip 2\onelineskip \hrule \vskip
    4\onelineskip}%\incrementptc \section*{Contents}\parttoc\decrementptc}%
  \renewcommand{\midpartskip}{\par\vskip 2\onelineskip}%
  \renewcommand{\afterchaptertitle}{\par\nobreak\vskip \afterchapskip}%
}

\setlength{\cftpartnumwidth}{2em}

%\newlistof{tabledesmatieres}{tdm}{Table des matières}
\newlistof{tabledesmatieres}{tdm}{Sommaire}
\newlistentry[chapter]{fsection}{tdm}{0}
\cftsetindents{fsection}{2em}{1.5em}
%\setcounter{tdmdepth}{200}

\newlistof{tablecontents}{tct}{Contents}
\newlistentry[chapter]{tchapter}{tdm}{0}
\newlistentry[section]{tsection}{tdm}{0}
\cftsetindents{tsection}{3em}{2.5em}

% \newlistentry[chapter]{section}{tdm}{0}
% \newlistentry[section]{subsection}{tdm}{1}
% \newlistentry[subsection]{subsubsection}{tdm}{2}
% \newlistentry[subsubsection]{paragraph}{tdm}{3}
% \newlistentry[paragraph]{subparagraph}{tdm}{4}

% \newlistentry[chapter]{section}{tct}{0}
% \newlistentry[section]{subsection}{tct}{1}
% \newlistentry[subsection]{subsubsection}{tct}{2}
% \newlistentry[subsubsection]{paragraph}{tct}{3}
% \newlistentry[paragraph]{subparagraph}{tct}{4}

% \let\old@part=\part
% \let\old@chapter=\chapter

\newcommand\mem@chapter{%
  \@ifstar{\@m@mschapter}{\@m@mchapter}}

\newcommand{\mem@part}{%
  \secdef\@part\@spart}

% \renewcommand*{\float@listhead}[1]{%
%   \def\@tempa{\section*}%
%   \@tempa{#1\@mkboth{\MakeUppercase{#1}}{\MakeUppercase{#1}}}}%

% \renewcommand{\chapter}[1][]{%
%   \@ifstar{}{%
%     \addtocounter{chapter}{1}%
%     \addcontentsline{tdm}{fsection}{%
%       \protect\numberline{\thechapter}#1}%
%     \addtocounter{chapter}{-1}%
%   }%
%   \iflanguage{french}{\frenchchapter{#1}}{\old@chapter}%
% }

\newif\ifthesis@french

\newcommand{\setfrench}{
  \global\thesis@frenchtrue
  \selectlanguage{french}
  \selectbiblanguage{french}
  \global\let\myoldsection\section
  \newcommand{\frsection}[2][]{\myoldsection[##2][##1]{##1}}
  \newcommand{\fssection}[2][]{\myoldsection*{##1}}
  \renewcommand{\section}{\@ifstar{\fssection}{\frsection}}
}

\newcommand{\setenglish}{
  \global\thesis@frenchfalse
  \selectlanguage{english}
  \selectbiblanguage{english}
  \global\let\section\myoldsection
}


\let\old@chapter\chapter

\def\chapter{%
  \@ifstar{\old@chapter*}{\my@chapter}}

\def\my@chapter[#1]#2{%
  \ifthesis@french%
  \old@chapter[#1][#1]{#1}%
  \else%
  \old@chapter[#2][#2]{#2}%
  \fi%
  \addcontentsline{tdm}{fsection}{\protect\numberline{\thechapter}#1}%
  \addcontentsline{tct}{tchapter}{\protect\numberline{\thechapter}#2}%
}

\def\chapterabbr[#1][#2]#3{%
  \ifthesis@french%
  \old@chapter[#1][#1]{#1}%
  \else%
  \old@chapter[#2][#2]{#3}%
  \fi%
  \addcontentsline{tdm}{fsection}{\protect\numberline{\thechapter}#1}%
  \addcontentsline{tct}{tchapter}{\protect\numberline{\thechapter}#2}%
}

\let\old@part\part

\def\part[#1]#2{%
  \ifthesis@french%
  \old@part{#1}%
  \else%
  \old@part{#2}%
  \fi%
  \addcontentsline{tdm}{part}{\protect\partnumberline{\thepart}#1}%
  \addcontentsline{tct}{part}{\protect\partnumberline{\thepart}#2}%
}

\let\old@section\section

\def\section{%
  \@ifstar{\my@star@section}{\my@section}}

\def\my@star@section{%
  \old@section*
}

\def\my@section{%
  \@ifnextchar[{\my@@@section}{\my@@section}}

\def\my@@@section[#1]#2{%
  \ifthesis@french%
  \old@section{#1}%
  \else%
  \old@section{#2}%
  \fi%
%  \addcontentsline{tdm}{section}{\protect\numberline{\thesection}#1}%
  \addcontentsline{tct}{tsection}{\protect\numberline{\thesection}#2}%
}

\def\my@@section#1{%
  \old@section{#1}%
  \ifthesis@french%
%  \addcontentsline{tdm}{section}{\protect\numberline{\thesection}#1}%
  \else%
  \addcontentsline{tct}{tsection}{\protect\numberline{\thesection}#1}%
  \fi%
}

\renewcommand*{\descriptionlabel}[1]{\hspace\labelsep
                                     \normalfont\sffamily #1}

\newcommand{\property}[1]{\statement{#1:}}
\newcommand{\statement}[1]{{\par\noindent\sffamily\bfseries\boldmath#1}}

\headstyles{test}
\pagestyle{ruled}
\nopartblankpage

% \renewcommand*{\l@chapter}{\@dottedtocline{0}{2em}{2.3em}}
% \renewcommand*{\l@chapter}[2]{\framebox{#1}\hfill#2}
% \newcommand{\chapternumberline}[1]{~~#1~~}
%\let\chapternumberline=\numberline

%\cftpagenumberson{chapter}
%\mtcsetpagenumbers{*}{on}

\newif\ifdraft
\newif\if@fr@chaptername


\newsavebox{\fmbox}
\newenvironment{figbox}%
  {\begin{lrbox}{\fmbox}\begin{minipage}{.7\textwidth}}%
  {\end{minipage}\end{lrbox}%
   \setlength{\fboxsep}{1.5ex}%
   \centerline{\fbox{\usebox{\fmbox}}}}

\makeatother

\newcommand{\sof}{\shorthandoff{;:?!}}
\newcommand{\son}{\shorthandon{;:?!}}

\setlist{leftmargin=*}

%%%%%%%%%%%%%%%%%%%%%%%%%%%%%%%%%%%%%%%%%%%%%%%%%%%%%%%%%%%%%%%%%%%%%%%%
%% Theorem styles

%\theoremstyle{plain}
%\newtheorem{theorem}{Theorem}[chapter]
%\newtheorem{lemma}{Lemma}[chapter]
%\newtheorem{corollary}{Corollary}[chapter]
%\newtheorem{proposition}{Proposition}[chapter]
%\newtheorem{conjecture}{Conjecture}[chapter]
%
%\theoremstyle{definition}
%\newtheorem{definition}{Definition}[chapter]
%\newtheorem{example}{Example}[chapter]
%
%\theoremstyle{remark}
%\newtheorem{remark}{Remark}[chapter]
\newtheoremstyle{newdef}{}{}{\normalfont}{}{\normalfont\sffamily\bfseries\boldmath}{\addperiod}{ }{}
\newtheoremstyle{newthm}{}{}{\normalfont\itshape}{}{\normalfont\sffamily\bfseries\boldmath}{\addperiod}{ }{}

\theoremstyle{newthm}
\newtheorem{fact}{Fact}[chapter]
\newtheorem{prop}{Proposition}[chapter]
\newtheorem{thm}{Theorem}[chapter]
\newtheorem{cor}{Corollary}[chapter]
\newtheorem{conj}{Conjecture}[chapter]
\newtheorem{obs}{Observation}
%\theoremstyle{definition}
\theoremstyle{newdef}
\newtheorem{defi}{Definition}[chapter]
\newtheorem{example}{Example}[chapter]
%
\newtheorem{fdefi}{Définition}[chapter]

%%%%%%%%%%%%%%%%%%%%%%%%%%%%%%%%%%%%%%%%%%%%%%%%%%%%%%%%%%%%%%%%%%%%%%%%
%% Global Macros

\renewcommand{\labelitemi}{--}
\renewcommand{\labelitemii}{--}
\renewcommand{\labelitemiii}{--}
\renewcommand{\labelitemiv}{--}

%\renewcommand{\algorithmicrequire}{\textbf{Input:}}
%\renewcommand{\algorithmicensure}{\textbf{Output:}}

\setcounter{MaxMatrixCols}{20}

\newcommand\Ftwo{\ensuremath{\mathbf{F}_{2}}}
\newcommand\Ftwom{\ensuremath{\mathbf{F}_{2^m}}}
\newcommand\Fst{\mathbf{F}_{2^4}}
\newcommand\Fth{\mathbf{F}_{2^8}}
\newcommand\Fu{\mathbf{F}_{2^m}}
\newcommand\Fq{\ensuremath{\mathbf{F}_{q}}}

% Primitives
\newcommand\AES{AES\xspace}
\newcommand\whirlpool{\textsc{Whirlpool}\xspace}
\newcommand\mc{MixColumn\xspace}
\newcommand\shark{SHARK\xspace}
\newcommand\khazad{\textsc{Khazad}\xspace}
\newcommand\rijndael{Rijndael\xspace}
\newcommand\photon{PHOTON\xspace}
\newcommand\piccolo{Piccolo\xspace}
\newcommand\led{LED\xspace}
\newcommand{\shazero}              {SHA-0\xspace}
\newcommand{\shaone}               {SHA-1\xspace}
\newcommand{\shatwo}               {SHA-2\xspace}
\newcommand{\shathree}             {SHA-3\xspace}
\newcommand{\mdfive} 			   {MD5\xspace}
\newcommand{\mdfour} 			   {MD4\xspace}
\newcommand{\mdsha} 			   {MD-SHA\xspace}
\newcommand{\md} 			   	   {MD\xspace}
\newcommand{\sha} 			 	   {SHA\xspace}
\newcommand{\ripemd} 			   {RIPEMD\xspace}
\newcommand{\hmac}			       {HMAC\xspace}
\newcommand{\shiun}			       {SHI-1\xspace}
\newcommand{\groestl}			       {Gr\o stl\xspace}
\newcommand{\keccak}			       {\textsc{Keccak}\xspace}
\newcommand{\samneric} {\textsc{Samneric}\xspace}
\newcommand{\sam} {\textsc{Sam}\xspace}
\newcommand{\eric} {\textsc{Eric}\xspace}
\newcommand{\lex} {LEX\xspace}
\newcommand\proest{\textsc{Pr\o st}\xspace}
\newcommand\proestotr{\textsc{Pr\o st-OTR}\xspace}
\newcommand\proestcopa{\textsc{Pr\o st-COPA}\xspace}
\newcommand\proestape{\textsc{Pr\o st-APE}\xspace}
\newcommand\proestem{\textsc{Pr\o st/SEM}\xspace}
\newcommand\minalpher{Minalpher\xspace}
\newcommand\caesar{\textsc{Caesar}\xspace}
\newcommand{\asasa}               {ASASA\xspace}
\newcommand{\spacehard}               {SPACE\xspace}
\newcommand{\sasas}               {SASAS\xspace}
\newcommand{\pc}               {\textsc{PuppyCipher}\xspace}
\newcommand{\cdb}               {\textsc{CoureurDesBois}\xspace}
\newcommand{\blake}               {BLAKE\xspace}

%%%%%%%%%%%%%%%%%%%%%%%%%%%%%%%%%%%%%%%%%%%%%%%%%%%%%%%%%%%%%%%%%%%%%%%%
%% SHA-1

\newcommand{\gtx}              {GTX\,970\xspace}

\newcommand\merkdam{Merkle-Damg{\aa}rd\xspace}
\newcommand\iv{\textit{IV}\xspace}
\newcommand\ivs{\textit{IVs}\xspace}
\newcommand\dv{\textit{DV}\xspace}
\newcommand\dvs{\textit{DVs}\xspace}

\newcommand{\etal}{\mbox{\emph{et al.}}\xspace}
\newcommand\ie{\emph{i.e.}\xspace}
\newcommand\eg{\emph{e.g.}\xspace}

\DeclareMathOperator\bigo{\mathcal{O}}
\DeclareMathOperator\oracle{\mathit{O}}

\newcommand\messblock{\mathsf{m}}
\newcommand\freeiv{\mathsf{i}}
\newcommand\mess{\mathcal{M}}
\newcommand\eem{\overline{\expmess}}
\newcommand\chain{\mathsf{c}}
\newcommand\mainmess{\mathcal{M}}
\newcommand\expmess{\mathcal{W}}
\newcommand\dexpmess{\mathcal{\widetilde W}}
\newcommand\state{\mathcal{A}}
\newcommand\dstate{\mathcal{\widetilde A}}
\DeclareMathOperator\compress{H}
\DeclareMathOperator\hash{\mathcal{H}}
\DeclareMathOperator\blockE{\mathcal{E}}
\DeclareMathOperator\boolF{\varphi}
\newcommand\fif{\boolF_\text{IF}}
\newcommand\fxor{\boolF_\text{XOR}}
\newcommand\fmaj{\boolF_\text{MAJ}}
\newcommand\tagg{t}
\DeclareMathOperator\ro{\mathcal{R}}
\DeclareMathOperator\sign{\mathcal{S}}
\DeclareMathOperator\mac{\mathcal{T}}
\DeclareMathOperator\perm{P}

\DeclareMathOperator\diff{\Delta}

\newcommand\nodiff{\ensuremath{\circ}}
\newcommand\nodiffz{\ensuremath{\smalltriangledown\xspace}}
\newcommand\nodiffo{$\smalltriangleup$\xspace}
\newcommand\onediff{\ensuremath{\bullet}}
\newcommand\onediffd{$\filledtriangledown$\xspace}
\newcommand\onediffu{$\filledtriangleup$\xspace}
\newcommand\monediff{\bullet}
\newcommand\mnodiff{\circ}
\newcommand\monediffd{\filledtriangledown\hspace{-0.32mm}}
\newcommand\monediffu{\filledtriangleup\hspace{-0.32mm}}
\newcommand\mnodiffz{\smalltriangledown\hspace{-0.32mm}}
\newcommand\mnodiffo{\smalltriangleup\hspace{-0.32mm}}
%\newcommand\dunnodiff{{\fontsize{2.34mm}{2.34mm}$\circledast$}}
%\newcommand\dunnodiff{{\fontsize{3.34mm}{3.34mm}$\ast$}}
%\newcommand\dunnodiff{{\fontsize{3.84mm}{3.84mm}$\ast$}}
\newcommand\dunnodiff{{\fontsize{3.23mm}{3.23mm}$\ast$}}
\newcommand\equaup{$\smallstar$\hspace{-.32mm}}
\newcommand\diffup{$\filledstar$\hspace{-.32mm}}
\newcommand\equarightup{$\smalldiamond$}
\newcommand\diffrightup{$\filleddiamond$}
\newcommand\equarightupup{$\smallsquare$\hspace{.1mm}}
\newcommand\diffrightupup{$\filledsquare$\hspace{.1mm}}

\newcommand\ftwo{\Ftwo}
\newcommand\ztt{\mathbf{Z}/2^{32}\mathbf{Z}}

\newcommand\defas{:=}


% SHA-1 F!A
\newcommand\mitm{meet-in-the-middle\xspace}
\newcommand\Mitm{Meet-in-the-middle\xspace}
\newcommand\wlo{w.l.o.g.\xspace}
\newcommand\al{\emph{et al.}\xspace}
\newcommand\ah{\emph{ad~hoc}\xspace}
\newcommand{\D}{\mathcal{D}}
\DeclareMathOperator{\Hf}{\hash}
\DeclareMathOperator{\Hc}{\compress}
\DeclareMathOperator{\Fs}{\mathit{f}}
\DeclareMathOperator{\Fuu}{\mathit{f_1}}
\DeclareMathOperator{\Fd}{\mathit{f_2}}
\DeclareMathOperator{\Ft}{\mathit{f_3}}
\DeclareMathOperator{\bool}{\boolF}
\DeclareMathOperator{\hw}{\wt}
\DeclareMathOperator{\Gm}{G}
\DeclareMathOperator{\permu}{\sigma}
\newcommand{\ksu}{\mathcal{K}_1}
\newcommand{\ksd}{\mathcal{K}_2}



%\newcolumntype{P}[1]{>{\centering\arraybackslash}p{#1}}

%%%%%%%%%%%%%%%%%%%%%%%%%%%%%%%%%%%%%%%%%%%%%%%%%%%%%%%%%%%%%%%%%%%%%%%%
%% EMRKA
\newcommand\msbtwo{{\kappa/2 - 1\xspace}}
\newcommand\msbtwoone{{\kappa/2\xspace}}
\newcommand\MSB{{\mathsf{MSB}\xspace}}

\DeclareMathOperator\rka{\phi}
\DeclareMathOperator\xrka{\phi^{\oplus}}
\DeclareMathOperator\arka{\phi^{+}}
\DeclareMathOperator\E{\mathcal{E}}
\DeclareMathOperator\R{\mathcal{R}}
\DeclareMathOperator\F{\mathcal{F}}
\DeclareMathOperator\Perm{\mathcal{P}}
\DeclareMathOperator\permi{\pi}
\DeclareMathOperator\IEM{\mathsf{IEM}}
\DeclareMathOperator\Adv{\mathbf{Adv}}
\DeclareMathOperator\EDP{EDP}
\DeclareMathOperator\DP{DP}
\DeclareMathOperator\poi{Poi}

\newcommand\EM{\textsf{EM}\xspace}

%%%%%%%%%%%%%%%%%%%%%%%%%%%%%%%%%%%%%%%%%%%%%%%%%%%%%%%%%%%%%%%%%%%%%%%%
%% Matrices


\newcommand\affi{\mathbf{A}}
\newcommand\projec{\mathbf{P}}
\newcommand\curve{\mathcal{X}}
\newcommand\nscurve{\widetilde{\curve}}
\newcommand\code{\mathcal{C}}
\newcommand\evcode{\mathcal{C}_\mathit{ev}}
\newcommand\mat{M}
\newcommand\matdiff{{\mathcal{M}}}
\newcommand\veci{\mathbf{x}}
\newcommand\veco{\mathbf{y}}
\newcommand\vect{\mathbf{z}}
\newcommand\nullvec{\mathbf{0}}
\newcommand\cmess{\mathbf{m}}
\newcommand\matvec{\mathbf{m}}
\newcommand\amatvec{\mathbf{a}}
\newcommand\pshufb{\mintinline{c}{pshufb}\xspace}
\newcommand\xmm{\mintinline{c}{xmm}\xspace}
\newcommand\sapp{{\mathcal{S}}}
\newcommand\adkey{{\mathcal{K}}}

\newcommand\frob{\ensuremath{F}}
\DeclareMathOperator{\broad}{\text{\emph{broadcast}}}
\DeclareMathOperator{\bn}{\text{\emph{branch-number}}}
\DeclareMathOperator{\shuff}{\mathcal{S}}
\DeclareMathOperator{\cost}{\text{\emph{cost2}}}
\DeclareMathOperator{\indic}{\mathbb{I}}
\DeclareMathOperator{\crs}{\mathcal{C}_{\text{RS}}}
\DeclareMathOperator{\cag}{\mathcal{C}_{\text{AG}}}
\DeclareMathOperator{\cagh}{\mathcal{C}_2}
\DeclareMathOperator{\rrs}{\mathcal{L}}
\DeclareMathOperator{\ord}{ord}
\DeclareMathOperator{\val}{\mathit{v}}
\DeclareMathOperator{\cdiv}{Div}
\DeclareMathOperator{\aut}{\text{Aut}}
\DeclareMathOperator{\sub}{\text{Sub}}
%\DeclareMathOperator{\round}{R_\sharp}
\DeclareMathOperator{\mappi}{\mathcal{M}}
\DeclareMathOperator{\matspace}{\mathcal{M}}
\DeclareMathOperator{\weight}{wt}
\DeclareMathOperator{\hdist}{d}
\DeclareMathOperator{\fun}{\mathit{f}}
\DeclareMathOperator{\evmap}{\mathrm{Ev}}

\newcommand\funcspace{\ensuremath{\mathcal{F}}}
\newcommand\dom{\ensuremath{\mathcal{D}}}
\newcommand\points{\ensuremath{\mathcal{P}}}
\newcommand\locring{\ensuremath{\mathcal{O}}}
\newcommand\maxi{\ensuremath{\mathfrak{m}}}

\setcounter{MaxMatrixCols}{20}


%%%%%%%%%%%%%%%%%%%%%%%%%%%%%%%%%%%%%%%%%%%%%%%%%%%%%%%%%%%%%%%%%%%%%%%%
%% Littlun

\newcommand\littlun{\textsc{Littlun}\xspace}
\newcommand\littlunOne{\textsc{Littlun-1}\xspace}
\newcommand\littlunS{\textsc{Littlun-S4}\xspace}
\newcommand\pride{\textsc{Pride}\xspace}
\newcommand\littlunpride{\textsc{Fly}\xspace}
\newcommand\fly{\littlunpride}
\newcommand\flyrk{\fly_{RK}}
\newcommand\idea{\textsc{IDEA}\xspace}
\newcommand\aes{\textsc{AES}\xspace}
\newcommand\serpent{\textsc{Serpent}\xspace}
\newcommand\present{\textsc{Present}\xspace}
\newcommand\robin{\textsc{Robin}\xspace}
\newcommand\fantomas{\textsc{Fantomas}\xspace}
\newcommand\shiftrow{\textsc{Shiftrow}\xspace}
\newcommand\noekeon{\textsc{Noekeon}\xspace}
\newcommand\rectangle{\textsc{Rectangle}\xspace}
\newcommand\fox{\textsc{Fox}\xspace}
\newcommand\simon{\textsc{Simon}\xspace}
\newcommand\speck{\textsc{Speck}\xspace}
\newcommand\simonC{\textsc{Simon}64-128\xspace}
\newcommand\speckC{\textsc{Speck}64-128\xspace}

\newcommand\C{\texttt{C}\xspace}

%\DeclareMathOperator\fun{\mathcal{F}}
\DeclareMathOperator\sbo{\mathcal{S}}
\DeclareMathOperator\wt{wt}
\DeclareMathOperator\Ldiff{\delta}
\DeclareMathOperator\LdiffU{\Delta}
\DeclareMathOperator\lin{\mathcal{L}}
\DeclareMathOperator\linU{\ell}
\DeclareMathOperator\rfly{\mathcal{R}_\fly}
\DeclareMathOperator\blit{\textsc{Bls}}
\DeclareMathOperator\ark{\textsc{Ark}}
\DeclareMathOperator\rot{\textsc{Rot}}
\DeclareMathOperator\ksOne{\textsc{KS1}}
\DeclareMathOperator\ksTwo{\textsc{KS2}}
\DeclareMathOperator\Lperm{\pi}
\DeclareMathOperator\rott{\circlearrowleft}
